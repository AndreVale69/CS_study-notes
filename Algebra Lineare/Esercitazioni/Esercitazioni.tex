\documentclass[a4paper]{article}
\usepackage[T1]{fontenc}			% pacchetto per \chapter
\usepackage[italian]{babel}
\usepackage[italian]{isodate}  		% formato delle date in italiano
\usepackage{graphicx}				% gestione delle immagini
\usepackage{amsfonts}
\usepackage{booktabs}				% tabelle di qualità superiore
\usepackage{amsmath}				% pacchetto matematica
\usepackage{mathtools}				% per sottolineare sotto le equazioni
\usepackage{stmaryrd} 				% per '\llbracket' e '\rrbracket'
\usepackage{amsthm}					% teoremi migliorati
\usepackage{enumitem}				% gestione delle liste
\usepackage{pifont}					% pacchetto con elenchi carini
\usepackage{enumitem}				% pacchetto per elenchi con lettere dell'alfabeto
\usepackage{cancel}					% per cancellare delle espressioni matematiche


\usepackage[x11names]{xcolor}		% pacchetto colori RGB
% Link ipertestuali per l'indice
\usepackage{xcolor}
\usepackage[linkcolor=black, citecolor=blue, urlcolor=cyan]{hyperref}
\hypersetup{
	colorlinks=true
}

%\usepackage{showframe}				% visualizzazione bordi
%\usepackage{showkeys}				% visualizzazione etichetta

\newtheorem{theorem}{\textcolor{Red3}{\underline{Teorema}}}
\newtheorem{lemma}{Lemma}
\renewcommand{\qedsymbol}{QED}
\newcommand{\exec}[1]{\llbracket #1\:\rrbracket}
\newcommand{\dquotes}[1]{``#1''}
\newcommand{\longline}{\noindent\rule{\textwidth}{0.4pt}}

\begin{document}
	\author{VR443470}
	\title{Esercitazioni Algebra Lineare}
	\date{\printdayoff\today}
	\maketitle
	
	\newpage
	
	% indice
	\tableofcontents
	
	\newpage
	
	\section{Basi}
	
	\subsection{Somma e trasposte}
	
	I classici esercizi di Algebra Lineare prevedono varie operazioni sulle matrici. Partendo dalle basi, si introducono le operazioni di somma e trasposizione.\newline
	
	\noindent
	Date $3$ matrici $A, B, C$:
	\begin{equation*}
		A = \begin{bmatrix}
			   1 & -i & 3 \\
			-2+i &  5 & i2
		\end{bmatrix} \hspace{2em}
		B = \begin{bmatrix}
			3i	& 2		\\
			4	& -i 	\\
			2-i	& -1
		\end{bmatrix} \hspace{2em}
		C = \begin{bmatrix}
			-2	& 3i	& 2-i	\\
			4i	& 1		& 0
		\end{bmatrix}
	\end{equation*}
	La \textcolor{Red3}{prima operazione} da eseguire è la classificazione delle matrici. In questo caso, le matrici hanno le seguenti dimensioni:
	\begin{equation*}
		A\in\mathbb{M}_{2\times3} \hspace{2em} B\in\mathbb{M}_{3\times2} \hspace{2em} C\in\mathbb{M}_{2\times3}
	\end{equation*}
	La \textcolor{Red3}{seconda operazione} da eseguire è controllare se è possibile eseguire l'operazione richiesta dall'esercizio. In questo caso, viene chiesta la somma. Per eseguire quest'ultima (vale lo stesso per la sottrazione), le dimensioni delle matrici devono essere tutte \textbf{identiche}. Dato che in questo caso la matrice $B \left(3\times2\right)$ differisce di dimensione rispetto alle due matrici $A,C \left(2\times3\right)$, è necessario fare qualcosa per eseguire l'operazione di somma.\newline
	
	\noindent
	Dato che è ancora l'inizio, non verranno effettuate manipolazioni complesse. Quindi, si supponga di eseguire questa operazione di somma/sottrazione:
	\begin{equation*}
		2A^{T} - 4\overline{B} + 3C^{T}
	\end{equation*}
	Prima di eseguire l'operazione, si ottengono le relative matrici coniugate e trasposte. L'operazione di \textbf{coniugazione} è eseguibile cambiando i segni ai valori complessi (quindi alle $i$). Invece, l'operazione di \textbf{trasposizione} ($T$) inverte le colonne e le righe di una matrice. I risultati sono:
	\begin{equation*}
		\begin{array}{lllllllll}
			A & = \begin{bmatrix}
				1 	& -i	& 3	\\
				-2+i&  5 	& i2
			\end{bmatrix} &
			A^{T} & = \begin{bmatrix}
				1	& -2+i	\\
				-i	& 5		\\
				3	& i2
			\end{bmatrix} &
			2A^{T} & = \begin{bmatrix}
				2	& -4+2i	\\
				-2i	& 10	\\
				6	& 4i
			\end{bmatrix} \\
			%%%%%%%%%%%%%%%%%%%%%%%%%%%%%%%%%%%%%%%%
			&&&&& \\
			%%%%%%%%%%%%%%%%%%%%%%%%%%%%%%%%%%%%%%%%
			B & = \begin{bmatrix}
				3i	& 2		\\
				4	& -i 	\\
				2-i	& -1
			\end{bmatrix} &
			\overline{B} & = \begin{bmatrix}
				-3i	& 2		\\
				4	& i		\\
				2+i	& -1
			\end{bmatrix} &
			4\overline{B} & = \begin{bmatrix}
				-12i	& 8		\\
				16		& 4i	\\
				8+4i	& -4
			\end{bmatrix} \\
			%%%%%%%%%%%%%%%%%%%%%%%%%%%%%%%%%%%%%%%%
			&&&&& \\
			%%%%%%%%%%%%%%%%%%%%%%%%%%%%%%%%%%%%%%%%
			C & = \begin{bmatrix}
				-2	& 3i	& 2-i	\\
				4i	& 1		& 0
			\end{bmatrix} &
			C^{T} & = \begin{bmatrix}
				-2	& 4i	\\
				3i	& 1		\\
				2-i	& 0
			\end{bmatrix} &
			3C^{T} & = \begin{bmatrix}
				-6	& 12i	\\
				9i	& 3		\\
				6-3i& 0
			\end{bmatrix}
		\end{array}
	\end{equation*}\newpage
	Adesso è possibile eseguire la sottrazione tra $\alpha = 2A^{T} - 4\overline{B}$ e successivamente la somma tra $\alpha + 3C^{T}$:
	\begin{equation*}
		\alpha = 2A^{T} - 4\overline{B} = \begin{bmatrix}
			2	& -4+2i	\\
			-2i	& 10	\\
			6	& 4i
		\end{bmatrix} - \begin{bmatrix}
			-12i	& 8		\\
			16		& 4i	\\
			8+4i	& -4
		\end{bmatrix} = \begin{bmatrix}
			2+12i	& 12+2i		\\
			-16-2i	& 10-4i		\\
			1-4i	& 4+4i
		\end{bmatrix}
	\end{equation*}
	Si esegue la somma:
	\begin{equation*}
		\alpha + 3C^{T} = \begin{bmatrix}
			2+12i	& 12+2i		\\
			-16-2i	& 10-4i		\\
			1-4i	& 4+4i
		\end{bmatrix} + \begin{bmatrix}
			-6	& 12i	\\
			9i	& 3		\\
			6-3i& 0
		\end{bmatrix} = \begin{bmatrix}
			-4+12i	& -12+14i	\\
			-16+7i	& 13-4i		\\
			7-7i	& 4+4i
		\end{bmatrix}
	\end{equation*}\newpage

	\subsection{(Anti-)Hermitiane e (anti-)simmetriche}
	
	Diamo alcune definizioni per capire come fare gli esercizi:
	\begin{itemize}
		\item È possibile abbreviare letteralmente le operazioni di trasposizione e coniugazione scrivendo \textbf{trasposta-coniugata};
		
		\item Una matrice viene detta \textcolor{Red3}{\textbf{hermitiana}} quando la matrice originaria è uguale alla sua trasposta-coniugata:
		\begin{equation*}
			\overline{\left(A^{T}\right)} = \left(\overline{A}\right)^{T} = A \Longrightarrow A^{H}
		\end{equation*}
		
		\item Una matrice viene detta \textcolor{Red3}{\textbf{anti-hermitiana}} quando la matrice trasposta-coniugata corrisponde alla matrice originaria ma cambiata di segno:
		\begin{equation*}
			\overline{\left(A^{T}\right)} = \left(\overline{A}\right)^{T} = -A \Longrightarrow \text{ anti-hermitiana}
		\end{equation*}
		
		\item Una matrice viene detta \textcolor{Red3}{\textbf{simmetrica}} quando la matrice originaria è uguale alla sua trasposta:
		\begin{equation*}
			A = A^{T} \Longrightarrow \text{ simmetrica}
		\end{equation*}
		
		\item Una matrice viene detta \textcolor{Red3}{\textbf{anti-simmetrica}} quando la sua trasposta corrisponde alla matrice originaria ma cambiata di segno:
		\begin{equation*}
			-A = A^{T} \Longrightarrow \text{ anti-simmetrica}
		\end{equation*}
	\end{itemize}
	Prendendo come esempio le tre matrici $A,B,C$:
	\begin{equation*}
		A = \begin{bmatrix}
			2i 	& 3		\\
			-3	& i
		\end{bmatrix} \hspace{2em}
		B = \begin{bmatrix}
			i 	& 2		\\
			2	& i
		\end{bmatrix} \hspace{2em}
		C = \begin{bmatrix}
			-1 	& i3	\\
			-i3	& 1
		\end{bmatrix}
	\end{equation*}
	Si eseguono le rispettive operazioni di trasposizione e coniugazione:
	\begin{equation*}
		\begin{array}{lllllllll}
			A & = \begin{bmatrix}
				2i 	& 3		\\
				-3	& i
			\end{bmatrix} &
			A^{T} & = \begin{bmatrix}
				2i	& -3	\\
				3	& i
			\end{bmatrix} &
			\overline{A^{T}} & = \begin{bmatrix}
				-2i	& -3	\\
				3	& -i
			\end{bmatrix} \\
			%%%%%%%%%%%%%%%%%%%%%%%%%%%%%%%%%%%%%%%%
			&&&&& \\
			%%%%%%%%%%%%%%%%%%%%%%%%%%%%%%%%%%%%%%%%
			B & = \begin{bmatrix}
				i 	& 2		\\
				2	& i
			\end{bmatrix} &
			B^{T} & = \begin{bmatrix}
				i	& 2		\\
				2	& i
			\end{bmatrix} &
			\overline{B^{T}} & = \begin{bmatrix}
				-i	& 2		\\
				2	& -i
			\end{bmatrix} \\
			%%%%%%%%%%%%%%%%%%%%%%%%%%%%%%%%%%%%%%%%
			&&&&& \\
			%%%%%%%%%%%%%%%%%%%%%%%%%%%%%%%%%%%%%%%%
			C & = \begin{bmatrix}
				-1 	& i3	\\
				-i3	& 1
			\end{bmatrix} &
			C^{T} & = \begin{bmatrix}
				-1	& -i3	\\
				i3	& 1
			\end{bmatrix} &
			\overline{C^{T}} & = \begin{bmatrix}
				-1	& i3	\\
				-i3	& 1
			\end{bmatrix}
		\end{array}
	\end{equation*}
	Da questi risultati è possibile notare come $A, B$ \underline{non} siano hermitiane, mentre $C$ lo sia. Inoltre, dalle trasposte è possibile osservare come $A, C$ \underline{non} siano simmetriche, mentre $B$ lo sia. Invece, per verificare l'anti-hermitiana e l'anti-simmetrica, è necessario negare le matrici originarie:
	\begin{equation*}
		-A = \begin{bmatrix}
			-2i	& -3	\\
			3	& -i
		\end{bmatrix} \hspace{2em}
		-B = \begin{bmatrix}
			-i 	& -2	\\
			-2	& -i
		\end{bmatrix} \hspace{2em}
		-C = \begin{bmatrix}
			1 	& -i3	\\
			i3	& -1
		\end{bmatrix}
	\end{equation*}
	Da questi risultati è possibile notare come $B,C$ \underline{non} siano anti-hermitiane, mentre $A$ lo sia. Inoltre, osservando nuovamente le trasposte, è possibile osservare come $A, B$ e $C$ \underline{non} siano anti-simmetriche
\end{document}
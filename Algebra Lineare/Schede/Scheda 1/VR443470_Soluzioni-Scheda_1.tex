\documentclass[a4paper]{article}
\usepackage[T1]{fontenc}			% pacchetto per \chapter
\usepackage[italian]{babel}
\usepackage[italian]{isodate}  		% formato delle date in italiano
\usepackage{graphicx}				% gestione delle immagini
\usepackage{amsfonts}
\usepackage{booktabs}				% tabelle di qualità superiore
\usepackage{amsmath}				% pacchetto matematica
\usepackage{mathtools}				% per sottolineare sotto le equazioni
\usepackage{stmaryrd} 				% per '\llbracket' e '\rrbracket'
\usepackage{amsthm}					% teoremi migliorati
\usepackage{enumitem}				% gestione delle liste
\usepackage{pifont}					% pacchetto con elenchi carini
\usepackage{enumitem}				% pacchetto per elenchi con lettere dell'alfabeto
\usepackage{cancel}					% per cancellare delle espressioni matematiche



\usepackage[x11names]{xcolor}		% pacchetto colori RGB
% Link ipertestuali per l'indice
\usepackage{xcolor}
\usepackage[linkcolor=black, citecolor=blue, urlcolor=cyan]{hyperref}
\hypersetup{
	colorlinks=true
}

\usepackage{tikz}
\newcommand{\MyTikzmark}[2]{%
	\tikz[overlay,remember picture,baseline] \node [anchor=base] (#1) {#2};%
}
\newcommand{\DrawVLine}[3][]{%
	\begin{tikzpicture}[overlay,remember picture]
		\draw[shorten <=0.3ex, #1] (#2.north) -- (#3.south);
	\end{tikzpicture}
}
\newcommand{\DrawHLine}[3][]{%
	\begin{tikzpicture}[overlay,remember picture]
		\draw[shorten <=0.2em, #1] (#2.west) -- (#3.east);
	\end{tikzpicture}
}


%\usepackage{showframe}				% visualizzazione bordi
%\usepackage{showkeys}				% visualizzazione etichetta

\newtheorem{theorem}{\textcolor{Red3}{\underline{Teorema}}}
\newtheorem{lemma}{Lemma}
\renewcommand{\qedsymbol}{QED}
\newcommand{\exec}[1]{\llbracket #1\:\rrbracket}
\newcommand{\dquotes}[1]{``#1''}
\newcommand{\longline}{\noindent\rule{\textwidth}{0.4pt}}
\newcommand{\circledtext}[1]{\raisebox{.5pt}{\textcircled{\raisebox{-.9pt}{#1}}}}

\newenvironment{rowequmat}[1]{\left(\array{@{}#1@{}}}{\endarray\right)}
\newenvironment{rowequmatbra}[1]{\left[\array{@{}#1@{}}}{\endarray\right]}

\begin{document}
	\author{VR443470 - Andrea Valentini}
	\title{Soluzioni scheda 1}
	\date{\printdayoff\today}
	\maketitle
	
	\newpage
	
	% indice
	\tableofcontents
	
	\newpage
	
	\section{Soluzione esercizio 1}
	
	Si considerino le seguenti matrici su $\mathbb{C}$:
	\begin{equation*}
		A = \begin{rowequmat}{cc}
			1 & -1 \\ [0.3em]
			i & \frac{1}{2}
		\end{rowequmat}, \hspace{1em}
		B = \begin{rowequmat}{cc}
			0 & 3i \\
			-1 & 0
		\end{rowequmat}, \hspace{1em}
		C = \begin{rowequmat}{cc}
			3 & 4 \\
			-2 & -2
		\end{rowequmat}, \hspace{1em}
		D = \begin{rowequmat}{cc}
			0 & -1 \\
			1+i & 2
		\end{rowequmat}
	\end{equation*}
	
	\subsection{Soluzione \emph{a}}
	
	Si determini la matrice risultante ottenibile eseguendo l'operazione $\left(CD\right)A$:
	\begin{equation*}
		\begin{array}{lll}
			\left(CD\right)A & = &
			\left[ \begin{rowequmat}{cc}
				3 & 4 \\
				-2 & -2
			\end{rowequmat} \cdot \begin{rowequmat}{cc}
				0 & -1 \\
				1+i & 2
			\end{rowequmat} \right] \cdot \begin{rowequmat}{cc}
				1 & -1 \\ [0.3em]
				i & \frac{1}{2}
			\end{rowequmat} \\
			\\
			&=& \begin{rowequmat}{cc}
				4+4i & 5 \\
				-2-2i & -2
			\end{rowequmat} \cdot \begin{rowequmat}{cc}
				1 & -1 \\ [0.3em]
				i & \frac{1}{2}
			\end{rowequmat} \\
			\\
			&=& \begin{rowequmat}{cc}
				4+9i  & -\frac{3}{2}-4i \\ [0.3em]
				-2-4i & 1+2i
			\end{rowequmat}
		\end{array}
	\end{equation*}
	
	\subsection{Soluzione \emph{b}}
	
	Si determini la matrice risultante ottenibile eseguendo l'operazione $A^{T} B$:
	\begin{gather*}
		A = \begin{rowequmat}{cc}
			1 & -1 \\ [0.3em]
			i & \frac{1}{2}
		\end{rowequmat} \longrightarrow
		A^{T} = \begin{rowequmat}{cc}
			1 & i \\ [0.3em]
			-1 & \frac{1}{2}
		\end{rowequmat}\\
		\\
		A^{T} B = \begin{rowequmat}{cc}
			1 & i \\ [0.3em]
			-1 & \frac{1}{2}
		\end{rowequmat} \cdot \begin{rowequmat}{cc}
			0 & 3i \\
			-1 & 0
		\end{rowequmat} =
		\begin{rowequmat}{cc}
			-i & 3i \\[0.3em]
			-\frac{1}{2} & -3i
		\end{rowequmat}
	\end{gather*}
	
	\subsection{Soluzione \emph{c}}
	
	Si determini la matrice risultante ottenibile eseguendo l'operazione $3A\left(B-D^{T}\right)$:
	\begin{gather*}
		D = \begin{rowequmat}{cc}
			0 	& -1 \\
			1+i & 2
		\end{rowequmat} \longrightarrow
		D^{T} = \begin{rowequmat}{cc}
			0 	& 1+i \\
			-1 	& 2
		\end{rowequmat} \longrightarrow
		-D^{T} = \begin{rowequmat}{cc}
			0 	& -1-i \\
			1 	& -2
		\end{rowequmat} \\
		\\
		\begin{array}{lll}
			3A\left(B-D^{T}\right) & = & 
			3\cdot\begin{rowequmat}{cc}
				1 & -1 \\ [0.3em]
				i & \frac{1}{2}
			\end{rowequmat} \cdot \left[
			\begin{rowequmat}{cc}
				0 & 3i \\
				-1 & 0
			\end{rowequmat} + \begin{rowequmat}{cc}
				0 	& -1-i \\
				1 	& -2
			\end{rowequmat}
			\right] \\
			\\
			&=& 3\cdot\begin{rowequmat}{cc}
				1 & -1 \\ [0.3em]
				i & \frac{1}{2}
			\end{rowequmat} \cdot \begin{rowequmat}{cc}
				0 & -1+2i \\
				0 & -2
			\end{rowequmat} \\
			\\
			&=& \begin{rowequmat}{cc}
				3 & -3 \\ [0.3em]
				3i & \frac{3}{2}
			\end{rowequmat} \cdot \begin{rowequmat}{cc}
				0 & -1+2i \\
				0 & -2
			\end{rowequmat} = \begin{rowequmat}{cc}
				0 & 3+6i \\
				0 & -9-3i
			\end{rowequmat}
		\end{array}
	\end{gather*}\newpage
	
	\subsection{Soluzione \emph{d}}
	
	Si determini la matrice risultante ottenibile eseguendo l'operazione $\left(4B-C\right)^{T}~-~DC$:
	\begin{equation*}
		\begin{array}{lll}
			\left(4B-C\right)^{T}-DC &=&
			\left[
			\begin{rowequmat}{cc}
				0  & 12i \\
				-4 & 0
			\end{rowequmat} - \begin{rowequmat}{cc}
				3 & 4 \\
				-2 & -2
			\end{rowequmat}
			\right]^{T} -
			\begin{rowequmat}{cc}
				0 & -1 \\
				1+i & 2
			\end{rowequmat}	\cdot
			\begin{rowequmat}{cc}
				3 & 4 \\
				-2 & -2
			\end{rowequmat} \\
			\\ %%%
			&=& \left[
			\begin{rowequmat}{cc}
				-3 & -4+12i \\
				-2 & 2
			\end{rowequmat}
			\right]^{T} - 
			\begin{rowequmat}{cc}
				2 & 2 \\
				-1+3i & 4i
			\end{rowequmat} \\
			\\ %%%
			&=& \begin{rowequmat}{cc}
				-3 & -2 \\
				-4+12i & 2
			\end{rowequmat} - \begin{rowequmat}{cc}
				2 & 2 \\
				-1+3i & 4i
			\end{rowequmat} \\
			\\ %%%
			&=& \begin{rowequmat}{cc}
				-5 & -4 \\
				-3+9i & 2-4i
			\end{rowequmat}
		\end{array}
	\end{equation*}\newpage
	
	\section{Soluzione esercizio 2}
	
	Si considerino le seguenti matrici su $\mathbb{R}$:
	\begin{gather*}
		A = \begin{rowequmat}{ccccc}
			0 & 0 & -1 & 4 & 0 \\
			0 & 2 & 0 & -2 & 0 \\
			0 & 0 & 3 & 0 & 3 \\
			0 & 0 & 0 & 0 & 5
		\end{rowequmat} \hspace{1em}
		B = \begin{rowequmat}{ccccccc}
			0 & 0 & 0 & 1 & 0 & 0 & 0 \\
			0 & 0 & 1 & 0 & 1 & 0 & 0 \\
			0 & 1 & 0 & 1 & 0 & 1 & 0 \\
			1 & 0 & 1 & 0 & 1 & 0 & 1
		\end{rowequmat} \\
		C = \begin{rowequmat}{ccc}
			2 & 0 & 1 \\
			-4 & 0 & -2 \\
			0 & 3 & 2 \\
			2 & -2 & 3 \\
			4 & 3 & 4
		\end{rowequmat} \hspace{1em}
		D = \begin{rowequmat}{ccc}
			2 & 0 & 1 \\
			-1 & 2 & 2 \\
			2 & 0 & 2
		\end{rowequmat}
	\end{gather*}
	
	\subsection{Soluzione \emph{a}}
	
	Si utilizza l'algoritmo di Eliminazione di Gauss per determinare una forma ridotta di ognuna delle matrici elencate.
\end{document}
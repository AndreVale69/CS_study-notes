\documentclass[a4paper]{article}
\usepackage[T1]{fontenc}			% pacchetto per \chapter
\usepackage[italian]{babel}
\usepackage[italian]{isodate}  		% formato delle date in italiano
\usepackage{graphicx}				% gestione delle immagini
\usepackage{amsfonts}
\usepackage{booktabs}				% tabelle di qualità superiore
\usepackage{amsmath}				% pacchetto matematica
\usepackage{mathtools}				% per sottolineare sotto le equazioni
\usepackage{stmaryrd} 				% per '\llbracket' e '\rrbracket'
\usepackage{amsthm}					% teoremi migliorati
\usepackage{enumitem}				% gestione delle liste
\usepackage{pifont}					% pacchetto con elenchi carini
\usepackage{enumitem}				% pacchetto per elenchi con lettere dell'alfabeto
\usepackage{cancel}					% per cancellare delle espressioni matematiche

\usepackage{mathrsfs}

\usepackage[x11names]{xcolor}		% pacchetto colori RGB
% Link ipertestuali per l'indice
\usepackage{xcolor}
\usepackage[linkcolor=black, citecolor=blue, urlcolor=cyan]{hyperref}
\hypersetup{
	colorlinks=true
}

\usepackage{tikz}
\newcommand{\MyTikzmark}[2]{%
	\tikz[overlay,remember picture,baseline] \node [anchor=base] (#1) {#2};%
}
\newcommand{\DrawVLine}[3][]{%
	\begin{tikzpicture}[overlay,remember picture]
		\draw[shorten <=0.3ex, #1] (#2.north) -- (#3.south);
	\end{tikzpicture}
}
\newcommand{\DrawHLine}[3][]{%
	\begin{tikzpicture}[overlay,remember picture]
		\draw[shorten <=0.2em, #1] (#2.west) -- (#3.east);
	\end{tikzpicture}
}


%\usepackage{showframe}				% visualizzazione bordi
%\usepackage{showkeys}				% visualizzazione etichetta

\newtheorem{theorem}{\textcolor{Red3}{\underline{Teorema}}}
\newtheorem{lemma}{Lemma}
\renewcommand{\qedsymbol}{QED}
\newcommand{\exec}[1]{\llbracket #1\:\rrbracket}
\newcommand{\dquotes}[1]{``#1''}
\newcommand{\longline}{\noindent\rule{\textwidth}{0.4pt}}
\newcommand{\circledtext}[1]{\raisebox{.5pt}{\textcircled{\raisebox{-.9pt}{#1}}}}
\DeclareMathOperator{\rk}{rk}

\newenvironment{rowequmat}[1]{\left(\array{@{}#1@{}}}{\endarray\right)}
\newenvironment{rowequmatbra}[1]{\left[\array{@{}#1@{}}}{\endarray\right]}

\begin{document}
	\author{\begin{tabular}{ll}
			VR455961 & Davide Bragantini \\
			VR443470 & Andrea Valentini
		\end{tabular}
		   }
	\title{Università degli studi di Verona \\
			\:\\
			Soluzioni scheda 2}
	\date{\printdayoff\today}
	\maketitle
	
	\newpage
	
	% indice
	\tableofcontents
	
	\newpage
	
	\section{Soluzione esercizio 1}
	
	Sia $A_{k}$ la seguente matrice reale:
	\begin{equation*}
		A_{k} = \begin{pmatrix}
			2 & 2 & 2k \\
			k-1 & k & k^{2} \\
			-k & -k & 0
		\end{pmatrix}
	\end{equation*}
	
	\subsection{Soluzione \emph{a}}
	
	Si determini per quali valori di $k \in \mathbb{R}$ la matrice $A_{k}$ ammette inversa.\newline
	
	\noindent
	Una matrice quadrata a coefficienti in un campo dell'insieme $\mathbb{K}$ è invertibile se e solo se il suo determinante è diverso da zero. Quindi, si procede con il calcolo del determinante della matrice $A_{k}$. Visto che si tratta di una matrice di ordine $3$, si utilizza la regola di Sarrus per risolvere il determinante. Si duplica la matrice:
	\begin{equation*}
		\begin{matrix}
			2 	& 2		& 2k 	& 2 	& 2 	& 2k 		\\
			k-1 & k 	& k^{2} & k-1 	& k 	& k^{2} 	\\
			-k 	& -k	& 0		& -k 	& -k 	& 0
		\end{matrix}
	\end{equation*}
	Si sommano i prodotti lungo le prime tre diagonali principali:
	\begin{equation*}
		\begin{array}{lll}
			\text{1° diag} & : & \left(2 \cdot k \cdot 0\right) = 0 \\ [0.5em]
			\text{2° diag} & : & \left(2 \cdot k^{2} \cdot -k\right) = 2k^{2} \cdot -k = -2k^{3} \\[0.5em]
			\text{3° diag} & : & \left[2k \cdot \left(k-1\right) \cdot -k\right] = 2k \cdot \left(-k^{2} + k\right) = -2k^{3} + 2k^{2} \\ [0.5em]
			\text{Somma}   & : & 0 + \left(-2k^{3}\right) + \left(-2k^{3} + 2k^{2}\right) = -4k^{3} + 2k^{2}
		\end{array}
	\end{equation*}
	E si esegue lo stesso calcolo considerando le tre diagonali opposte:
	\begin{equation*}
		\begin{array}{lll}
			\text{1° diag opp} & : & \left(2k \cdot k \cdot -k\right) = -2k^{3} \\ [0.5em]
			\text{2° diag opp} & : & \left[2 \cdot \left(k-1\right) \cdot 0\right] = 0  \\ [0.5em]
			\text{3° diag opp} & : & \left(2 \cdot k^{2} \cdot -k\right) = -2k^{3} \\ [0.5em]
			\text{Somma}   & : & -2k^{3} + 0 + \left(-2k^{3}\right) = -4k^{3}
		\end{array}
	\end{equation*}
	Si esegue la sottrazione dei due risultati ottenuti mantenendo a sinistra quello della diagonale principale:
	\begin{equation*}
		\left(-4k^{3} + 2k^{2}\right) - \left(-4k^{3}\right) = 2k^{2}
	\end{equation*}
	Quindi il determinante è:
	\begin{equation*}
		\det\left(A_{k}\right) = 2k^{2}
	\end{equation*}
	La matrice $A_{k}$ ammette inversa per qualsiasi valore reale di $k$, poiché non esiste nessun valore (in $\mathbb{R}$) in grado di annullare l'espressione $2k^{2}$.\newpage
	
	\subsection{Soluzione \emph{b}}
	
	Sia $k \in \mathbb{R}$ tale che $A_{k}$ ammette inversa. Si calcoli $A_{k}^{-1}$ usando la formula $A_{k}^{-1} = \frac{1}{\det\left(A_{k}\right)} A_{k}^{*}$.\newline
	
	\noindent
	Per calcolare la matrice inversa si calcolano prima i complementi algebrici $\mathrm{Com}$:
	\begin{equation*}
		\begin{array}{ll}
			\mathrm{Com}\left(A_{11}\right) = & \left(-1\right)^{1+1} \cdot C_{11} = \left(-1\right)^{2} \cdot \det\begin{pmatrix}
				k & k^{2} \\
				-k & 0
			\end{pmatrix} = 1 \cdot \left[0 - \left(-k^{3}\right)\right] = k^{3} \\ [1.2em]
			
			\mathrm{Com}\left(A_{21}\right) = & \left(-1\right)^{2+1} \cdot C_{21} = \left(-1\right)^{3} \cdot \det\begin{pmatrix}
				2  & 2k \\
				-k & 0
			\end{pmatrix} = -1 \cdot \left[0 - \left(-2k^{2}\right)\right] = -2k^{2} \\ [1.2em]
			
			\mathrm{Com}\left(A_{31}\right) = & \left(-1\right)^{3+1} \cdot C_{31} = \left(-1\right)^{4} \cdot \det\begin{pmatrix}
				2 & 2k \\
				k & k^{2}
			\end{pmatrix} = 1 \cdot \left(2k^{2} - 2k^{2}\right) = 0 \\ [1.2em]
			
			\mathrm{Com}\left(A_{12}\right) = & \left(-1\right)^{1+2} \cdot C_{12} = \left(-1\right)^{3} \cdot \det\begin{pmatrix}
				k-1 & k^{2} \\
				-k  & 0
			\end{pmatrix} = -1 \cdot \left[0 - \left(-k^{3}\right)\right] = -k^{3} \\ [1.2em]
			
			\mathrm{Com}\left(A_{22}\right) = & \left(-1\right)^{2+2} \cdot C_{22} = \left(-1\right)^{4} \cdot \det\begin{pmatrix}
				2  & 2k \\
				-k & 0
			\end{pmatrix} = 1 \cdot \left[0 - \left(-2k^{2}\right)\right] = 2k^{2} \\ [1.2em]
			
			\mathrm{Com}\left(A_{32}\right) = & \left(-1\right)^{3+2} \cdot C_{32} = \left(-1\right)^{5} \cdot \det\begin{pmatrix}
				2   & 2k	\\
				k-1 & k^{2}
			\end{pmatrix} = -1 \cdot \left[2k^{2} - \left(2k^{2} - 2k\right)\right] = -2k\\ [1.2em]
			
			\mathrm{Com}\left(A_{13}\right) = & \left(-1\right)^{1+3} \cdot C_{13} = \left(-1\right)^{4} \cdot \det\begin{pmatrix}
				k-1 & k \\
				-k  & -k
			\end{pmatrix} = 1 \cdot \left[-k^{2} + k - \left(-k^{2}\right)\right] = k \\ [1.2em]
			
			\mathrm{Com}\left(A_{23}\right) = & \left(-1\right)^{2+3} \cdot C_{23} = \left(-1\right)^{5} \cdot \det\begin{pmatrix}
				2  & 2 \\
				-k & -k
			\end{pmatrix} = -1 \cdot \left[-2k - \left(-2k\right)\right] = 0 \\ [1.2em]
			
			\mathrm{Com}\left(A_{33}\right) = & \left(-1\right)^{3+3} \cdot C_{33} = \left(-1\right)^{6} \cdot \det\begin{pmatrix}
				2   & 2 \\
				k-1 & k
			\end{pmatrix} = 1 \cdot \left[2k - \left(2k - 2\right)\right] = -2
		\end{array}
	\end{equation*}
	Con i complementi algebrici si costruisce la matrice e si esegue la trasposta per ottenere $A_{k}^{*}$:
	\begin{equation*}
		A_{k}^{*} = \begin{pmatrix}
			k^{3}	& -k^{3}	& k \\
			-2k^{2} & 2k^{2}	& 0 \\
			0		& -2k		& -2
		\end{pmatrix}^{T} = \begin{pmatrix}
			k^{3} 	& -2k^{2} 	& 0 \\
			-k^{3}	& 2k^{2}	& -2k \\
			k		& 0			& -2
		\end{pmatrix}
	\end{equation*}
	Infine, si applica la formula:
	\begin{equation*}
		A_{k}^{-1} = \dfrac{1}{\det\left(A_{k}\right)} A_{k}^{*} = \dfrac{1}{2k^{2}} \begin{pmatrix}
			k^{3} 	& -2k^{2} 	& 0 \\
			-k^{3}	& 2k^{2}	& -2k \\
			k		& 0			& -2
		\end{pmatrix} = \begin{rowequmat}{ccc}
			\frac{k}{2}		& -1	&	0 			\\ [0.3em]
			-\frac{k}{2}	& 1		& -\frac{1}{k} 	\\ [0.3em]
			\frac{1}{2k}	& 0		& -\frac{1}{k^{2}}
		\end{rowequmat}
	\end{equation*}\newpage
	
	\section{Soluzione esercizio 2}
	
	Nello spazio vettoriale $\mathbb{R^{R}}$ definito nell'Esempio 5.2(2), si consideri il seguente sottoinsieme per ogni $t \in \mathbb{R}$:
	\begin{equation*}
		\mathscr{S}_{t} = \left\{f \in \mathbb{R^{R}} \: | \: f\left(0\right) = t\right\}
	\end{equation*}
	
	\subsection{Soluzione \emph{a}}
	
	Si trovino i valori di $t$ per cui l'insieme $\mathscr{S}_{t}$ è un sottospazio di $\mathbb{R^{R}}$.\newline
	
	\noindent
	Per verificare quali valori di $t$ rispettano il teorema di caratterizzazione dei sottospazi vettoriali, si possono effettuare alcune prove banali:
	\begin{itemize}
		\item Assumendo che $t = 1$:
		\begin{equation*}
			\mathscr{S}_{1} = \left\{f \in \mathbb{R^{R}} \: | \: f\left(0\right) = 1\right\}
		\end{equation*}
		E trovando due funzioni che appartengano all'insieme:
		\begin{gather*}
			f : \mathbb{R} \rightarrow \mathbb{R} \hspace{2em} f\left(x\right) = \begin{cases}
				1 & x = 0 \\
				4 & x \ne 0
			\end{cases} \\
			\\
			g : \mathbb{R} \rightarrow \mathbb{R} \hspace{2em} g\left(x\right) = \begin{cases}
				1 & x = 0 \\
				5 & x \ne 0
			\end{cases}
		\end{gather*}
		Si esegue la verifica delle due proprietà:
		\begin{equation*}
			\begin{array}{lll}
				\text{Proprietà }a & : & \left(f + g\right)\left(0\right) = f\left(0\right) + g\left(0\right) = 1+1 = 2 \ne 0 \\ [0.5em]
				\text{Proprietà }b & : & \lambda f\left(x\right) = \lambda f\left(0\right) = \lambda 1 \ne 0
			\end{array}
		\end{equation*}
		Le proprietà non sono rispettate, quindi con il valore $t = 1$ l'insieme $\mathscr{S}_{1}$ non è un sottospazio di $\mathbb{R^{R}}$.
		
		\item Assumendo che $t = -1$:
		\begin{equation*}
			\mathscr{S}_{-1} = \left\{f \in \mathbb{R^{R}} \: | \: f\left(0\right) = -1\right\}
		\end{equation*}
		E trovando due funzioni che appartengano all'insieme:
		\begin{gather*}
			f : \mathbb{R} \rightarrow \mathbb{R} \hspace{2em} f\left(x\right) = \begin{cases}
				-1 & x = 0 \\
				8 & x \ne 0
			\end{cases} \\
			\\
			g : \mathbb{R} \rightarrow \mathbb{R} \hspace{2em} g\left(x\right) = \begin{cases}
				-1 & x = 0 \\
				10 & x \ne 0
			\end{cases}
		\end{gather*}
		Si esegue la verifica delle due proprietà:
		\begin{equation*}
			\begin{array}{lll}
				\text{Proprietà }a & : & \left(f + g\right)\left(0\right) = f\left(0\right) + g\left(0\right) = -1-1 = -2 \ne 0 \\ [0.5em]
				\text{Proprietà }b & : & \lambda f\left(x\right) = \lambda f\left(0\right) = \lambda -1 \ne 0
			\end{array}
		\end{equation*}
		Le proprietà non sono rispettate, quindi con il valore $t = -1$ l'insieme $\mathscr{S}_{-1}$ non è un sottospazio di $\mathbb{R^{R}}$.\newpage
		
		\item Assumendo che $t = 0$:
		\begin{equation*}
			\mathscr{S}_{0} = \left\{f \in \mathbb{R^{R}} \: | \: f\left(0\right) = 0\right\}
		\end{equation*}
		E trovando due funzioni che appartengano all'insieme:
		\begin{gather*}
			f : \mathbb{R} \rightarrow \mathbb{R} \hspace{2em} f\left(x\right) = \begin{cases}
				0 & x = 0 \\
				1 & x \ne 0
			\end{cases} \\
			\\
			g : \mathbb{R} \rightarrow \mathbb{R} \hspace{2em} g\left(x\right) = \begin{cases}
				0 & x = 0 \\
				2 & x \ne 0
			\end{cases}
		\end{gather*}
		Si esegue la verifica delle due proprietà:
		\begin{equation*}
			\begin{array}{lll}
				\text{Proprietà }a & : & \left(f + g\right)\left(0\right) = f\left(0\right) + g\left(0\right) = 0+0 = 0 = 0 \\ [0.5em]
				\text{Proprietà }b & : & \lambda f\left(x\right) = \lambda f\left(0\right) = \lambda 0 = 0
			\end{array}
		\end{equation*}
		Le proprietà sono rispettate, quindi con il valore $t = 0$ l'insieme $\mathscr{S}_{0}$ è un sottospazio di $\mathbb{R^{R}}$.
	\end{itemize}
	È possibile concludere le prove con i valori $t$ e giungere ad una conclusione. L'insieme $\mathscr{S}_{t}$ è un sottospazio di $\mathbb{R^{R}}$ se e solo se $t$ ha valore $0$. Negli altri casi l'insieme non rispetta il teorema di caratterizzazione dei sottospazi vettoriali. Infatti, andando ad aumentare positivamente o negativamente la $t$, la proprietà \emph{a}, e anche \emph{b}, avrà come risultato un valore sempre diverso da zero.
	
	\subsection{Soluzione \emph{b}}
	
	Sia $\mathscr{U}$ il sottospazio di $\mathbb{R^{R}}$ generato da $f$ e $g$ dove $f\left(x\right) = \sin\left(x\right)$ e $g\left(x\right) = \cos\left(x\right)$ per ogni $x \in \mathbb{R}$. Si trovi una base dell'intersezione $\mathscr{U} \cap \mathscr{S}_{0}$.\newline
	
	\noindent
	L'insieme $\mathscr{S}_{0}$ è così definito:
	\begin{equation*}
		\mathscr{S}_{0} = \left\{f \in \mathbb{R^{R}} \: | \: f\left(0\right) = 0\right\}
	\end{equation*}
	Per evitare errori, una funzione che appartiene a questo insieme verrà indicata con l'apice, quindi $f'\in \mathscr{S}_{0}$. Si prende una qualsiasi funzione dall'insieme $\mathscr{S}_{0}$ così definita:
	\begin{equation*}
		f' \in \mathscr{S}_{0} \hspace{2em} f':\mathbb{R} \rightarrow \mathbb{R} \hspace{2em} f'\left(x\right) = \begin{cases}
			0 & x = 0 \\
			4 & x \ne 0
		\end{cases}
	\end{equation*}
	Le funzioni generatori del sottospazio $\mathscr{U}$ sono:
	\begin{equation*}
		\begin{array}{l}
			f \in \mathscr{U} \hspace{2em} f\left(x\right) = \sin\left(x\right) \\ [0.5em]
			g \in \mathscr{U} \hspace{2em} g\left(x\right) = \cos\left(x\right)
		\end{array}
	\end{equation*}
	Si sceglie un valore comodo e si formano due insiemi:
	\begin{equation*}
		\begin{array}{lll}
			x = 0 & \Rightarrow & v_{1} = \left\{f'\left(0\right), f\left(0\right), g\left(0\right)\right\} = \left\{0, 0, 1\right\} \\
			\\
			x = 90 & \Rightarrow & v_{2} = \left\{f'\left(90\right), f\left(90\right), g\left(90\right)\right\} = \left\{4, 1, 0\right\}
		\end{array}
	\end{equation*}
	Adesso è necessario dimostrare che $\left\{v_{1}, v_{2}\right\}$ è un sistema di generatori, cioè è necessario stabilire se per ogni $\mathbf{w} = \left(w_{1}, w_{2}, w_{3}\right) \in \mathscr{U} \cap \mathscr{S}_{0}$ esistono due scalari $a_ {1}, a_{2} \in \mathbb{R}$ tali che:
	\begin{equation*}
		\begin{array}{llrll}
			a_{1}v_{1} + a_{2}v_{2} = \mathbf{w} & \xlongrightarrow{\text{sostituzione}} & a_{1}\left(0,0,1\right) + a_{2}\left(4,1,0\right) &=& \left(w_{1}, w_{2}, w_{3}\right) \\ [0.5em]
			&& \left(0,0,a_{1}\right) + \left(4a_{2}, a_{2}, 0\right) &=& \left(w_{1}, w_{2}, w_{3}\right) \\ [0.5em]
			&& \left(4a_{2}, a_{2}, a_{1}\right) &=& \left(w_{1}, w_{2}, w_{3}\right)
		\end{array}
	\end{equation*}
	Il relativo sistema lineare:
	\begin{equation*}
		\begin{cases}
			4a_{2} = w_{1} \\
			a_{2} = w_{2} \\
			a_{1} = w_{3}
		\end{cases}
	\end{equation*}
	I vettori $v_{1},v_{2}$ sono un sistema di generatori se e solo se il sistema ammette soluzione. Per farlo, si utilizza il teorema di Rouché Capelli e per calcolare il rango, necessario per il teorema, si utilizza l'Eliminazione di Gauss così da ottenere una forma matriciale ridotta:
	\begin{gather*}
		\begin{rowequmat}{c|c}
			4 & w_{1} \\
			1 & w_{2} \\
			1 & w_{3}
		\end{rowequmat}
		\xlongrightarrow{E_{1,3}\left(-\frac{1}{4}\right)}
		\begin{rowequmat}{c|c}
			4 & w_{1} \\ [0.3em]
			1 & w_{2} \\ [0.3em]
			0 & w_{3} - \frac{w_{1}}{4}
		\end{rowequmat}
		\xlongrightarrow{E_{1,2}\left(-\frac{1}{4}\right)}
		\begin{rowequmat}{c|c}
			4 & w_{1} \\ [0.3em]
			0 & w_{2} - \frac{w_{1}}{4} \\ [0.3em]
			0 & w_{3} - \frac{w_{1}}{4}
		\end{rowequmat}
		\longrightarrow
		\begin{rowequmat}{c|c}
			4 & w_{1} \\ [0.3em]
			0 & \frac{4w_{2} - w_{1}}{4} \\ [0.3em]
			0 & \frac{4w_{3} - w_{1}}{4}
		\end{rowequmat} \\
		\\
		\xlongrightarrow{E_{2,3} \left( \frac{-w_{1} + 4w_{3}}{w_{1} - 4w_{2}} \right) }
		\begin{rowequmat}{c|c}
			4 & w_{1} \\ [0.3em]
			0 & \frac{4w_{2} - w_{1}}{4} \\ [0.3em]
			0 & 0
		\end{rowequmat}
	\end{gather*}
	Il rango della matrice aumentata è 2. Tuttavia, nel caso in cui $w_{2}, w_{1}$ siano pari a zero, il rango è 1. Dunque, esiste una e un'unica soluzione al sistema in questo caso. Quindi è possibile concludere che $v_{1},v_{2}$ sono un sistema di generatori. Per affermare che siano anche una base è necessario dimostrare che siano anche linearmente indipendenti:
	\begin{equation*}
		\begin{array}{llrll}
			b_{1}v_{1} + b_{2}v_{2} = 0 & \xlongrightarrow{\text{sostituzione}} & b_{1}\left(0,0,1\right) + b_{2}\left(4,1,0\right) &=& \left(0, 0, 0\right) \\ [0.5em]
			&& \left(0,0,b_{1}\right) + \left(4b_{2}, b_{2}, 0\right) &=& \left(0,0,0\right) \\ [0.5em]
			&& \left(4b_{2}, b_{2}, b_{1}\right) &=& \left(0,0,0\right)
		\end{array}
	\end{equation*}
	Il sistema relativo:
	\begin{equation*}
		\begin{cases}
			4b_{2} = 0 \\
			b_{2} = 0 \\
			b_{1} = 0
		\end{cases}
	\end{equation*}
	E con il metodo di sostituzione si trova subito che l'unica soluzione possibile è $b_{1} = b_{2} = 0$. Dunque i vettori sono linearmente indipendenti e l'insieme:
	\begin{equation*}
		\left\{v_{1}, v_{2}\right\} = \left\{ \left(f'\left(0\right), f\left(0\right), f\left(0\right)\right), \left(f'\left(90\right), f\left(90\right), g\left(90\right)\right) \right\} = \left\{ \left(0,0,1\right), \left(4,1,0\right)\right\}
	\end{equation*}
	È una base di $\mathscr{U} \cap \mathscr{S}_{0}$.\newpage
	
	\section{Soluzione esercizio 3}
	
	Sia $f: \mathbb{C}^{3} \rightarrow \mathbb{C}^{2}$ l'applicazione data da:
	\begin{equation*}
		f \left(
		\begin{pmatrix}
			 x \\ y \\ z
		\end{pmatrix}
		\right) = \begin{pmatrix}
			x-y+z \\
			3x-3y+3z
		\end{pmatrix}
	\end{equation*}
	Per ogni $\begin{pmatrix}
		x \\ y \\ z
	\end{pmatrix} \in \mathbb{C}^{3}$.
	\subsection{Soluzione \emph{a}}
	
	Si verifichi che $f$ è lineare.\newline
	
	\noindent
	Per verificare la linearità si prova rapidamente a \dquotes{mandare lo zero nello zero}:
	\begin{equation*}
		f\left( \begin{pmatrix}
			0 \\ 0 \\ 0
		\end{pmatrix} \right) = \begin{pmatrix}
			0 - 0 + 0 \\
			3\cdot 0 - 3 \cdot 0 + 3 \cdot 0
		\end{pmatrix} = \begin{pmatrix}
			0 \\ 0 \\ 0
		\end{pmatrix}
	\end{equation*}
	La condizione necessaria, ma non sufficiente, di linearità è stata dimostrata. Adesso, si prosegue con la verifica della proprietà di condizione di linearità dividendola nella dimostrazione della condizione di additività e omogeneità.
	\begin{proof}[\textbf{Dimostrazione condizione di additività}]
		Dati due vettori generici $v_{1},v_{2}$ dello spazio vettoriale:
		\begin{equation*}
			v_{1} = \begin{pmatrix}
				x_{1} \\ y_{1} \\ z_{1}
			\end{pmatrix}, v_{2} = \begin{pmatrix}
				x_{2} \\ y_{2} \\ z_{2}
			\end{pmatrix} \in \mathbb{C}^{3}
		\end{equation*}
		Si dimostra che $f$ è una funzione additiva:
		\begin{equation*}
			f\begin{pmatrix}
				v_{1} + v_{2}
			\end{pmatrix} =
			f\left(v_{1}\right) + f\left(v_{2}\right) =
			f\begin{pmatrix}
				x_{1} + x_{2} \\
				y_{1} + y_{2} \\
				z_{1} + z_{2}
			\end{pmatrix} =
			f\begin{pmatrix}
				x_{1} \\ y_{1} \\ z_{1}
			\end{pmatrix} +
			f\begin{pmatrix}
				x_{2} \\ y_{2} \\ z_{2}
			\end{pmatrix}
		\end{equation*}
		Si esegue prima il calcolo del blocco dell'unica funzione:
		\begin{equation*}
			\begin{array}{rll}
				f\begin{pmatrix}
					x_{1} + x_{2} \\
					y_{1} + y_{2} \\
					z_{1} + z_{2}
				\end{pmatrix} &=&
				\begin{pmatrix}
					\left(x_{1} + x_{2}\right) - \left(y_{1} + y_{2}\right) + \left(z_{1} + z_{2}\right) \\
					3\left(x_{1}+x_{2}\right) - 3\left(y_{1}+y_{2}\right) + 3\left(z_{1}+z_{2}\right)
				\end{pmatrix}
			\end{array}
		\end{equation*}
		E successivamente si esegue il calcolo delle due funzioni distinte:
		\begin{equation*}
			\begin{array}{lll}
				f\begin{pmatrix}
					x_{1} \\ y_{1} \\ z_{1}
				\end{pmatrix} +
				f\begin{pmatrix}
					x_{2} \\ y_{2} \\ z_{2}
				\end{pmatrix} &=&
				\begin{pmatrix}
					x_{1} - y_{1} + z_{1} \\ 3x_{1} - 3y_{1} + 3z_{1}
				\end{pmatrix} +
				\begin{pmatrix}
					x_{2} - y_{2} + z_{2} \\ 3x_{2} - 3y_{2} + 3z_{2}
				\end{pmatrix} \\ [1.8em]
				&=&
				\begin{pmatrix}
					\left(x_{1} + x_{2}\right) - \left(y_{1} + y_{2}\right) + \left(z_{1} + z_{2}\right) \\
					3\left(x_{1}+x_{2}\right) - 3\left(y_{1}+y_{2}\right) + 3\left(z_{1}+z_{2}\right)
				\end{pmatrix}
			\end{array}
		\end{equation*}
		I due risultati coincidono, per cui la funzione $f$ è additiva.
	\end{proof}\newpage
	
	\begin{proof}[\textbf{Dimostrazione condizione di omogeneità}]
		Dato un vettore generico $v_{1}$ dello spazio vettoriale e dato uno scalare $\lambda$:
		\begin{equation*}
			v_{1} = \begin{pmatrix}
				x_{1} \\ y_{1} \\ z_{1}
			\end{pmatrix} \in \mathbb{C}^{3}, \hspace{2em} \lambda \in \mathbb{K}
		\end{equation*}
		Si dimostra che $f$ è una funzione omogenea:
		\begin{equation*}
			f\left(\lambda v_{1}\right) = \lambda f\left(v_{1}\right) =
			f\begin{pmatrix}
				\lambda \cdot x_{1} \\ \lambda \cdot y_{1} \\ \lambda \cdot z_{1}
			\end{pmatrix} = \lambda \cdot f\begin{pmatrix}
				x_{1} \\ y_{1} \\ z_{1}
			\end{pmatrix}
		\end{equation*}
		Si esegue prima il calcolo della funzione con lo scalare internamente:
		\begin{equation*}
			f\begin{pmatrix}
				\lambda \cdot x_{1} \\ \lambda \cdot y_{1} \\ \lambda \cdot z_{1}
			\end{pmatrix} =
			\begin{pmatrix}
				\lambda x - \lambda y + \lambda z \\
				3\lambda x - 3\lambda y + 3\lambda z
			\end{pmatrix} =
			\begin{pmatrix}
				\lambda \left(x - y + z\right) \\
				\lambda \left(3x - 3y + 3z\right)
			\end{pmatrix}
		\end{equation*}
		E successivamente si esegue il calcolo della funzione con lo scalare esternamente:
		\begin{equation*}
			\lambda \cdot f\begin{pmatrix}
				x_{1} \\ y_{1} \\ z_{1}
			\end{pmatrix} = \lambda \cdot
			\begin{pmatrix}
				x-y+z \\
				3x-3y+3z
			\end{pmatrix} =
			\begin{pmatrix}
				\lambda \left(x - y + z\right) \\
				\lambda \left(3x - 3y + 3z\right)
			\end{pmatrix}
		\end{equation*}
		I due risultati coincidono, per cui la funzione è anche omogenea.
	\end{proof}\:\newline

	\noindent
	Entrambe le condizioni, additività e omogeneità, sono state dimostrate e confermate. Per cui, la funzione $f$ è un'applicazione lineare.\newpage
	
	\subsection{Soluzione \emph{b}}
	
	Si determini la matrice $A$ associata a $f$ rispetto alla base canonica e si dica se $f$ è un isomorfismo.\newline
	
	\noindent
	La base canonica del dominio è:
	\begin{equation*}
		\mathcal{C}_{\mathbb{C}^{3}} = \left\{
		\begin{pmatrix}
			1 \\ 0 \\ 0
		\end{pmatrix}, 
		\begin{pmatrix}
			0 \\ 1 \\ 0
		\end{pmatrix}, 
		\begin{pmatrix}
			0 \\ 0 \\ 1
		\end{pmatrix}
		\right\}
	\end{equation*}
	E i rispettivi vettori immagine mediante $f$ sono:
	\begin{equation*}
		\begin{array}{lllll}
			f\begin{pmatrix}
				1 \\ 0 \\ 0
			\end{pmatrix} &=&
			\begin{pmatrix}
				1 - 0 + 0 \\
				3 \cdot 1 - 3 \cdot 0 + 3 \cdot 0
			\end{pmatrix} &=&
			\begin{pmatrix}
				1 \\ 3
			\end{pmatrix} \\ [1.8em]
			
			f\begin{pmatrix}
				0 \\ 1 \\ 0
			\end{pmatrix} &=&
			\begin{pmatrix}
				0 - 1 + 0 \\
				3 \cdot 0 - 3 \cdot 1 + 3 \cdot 0
			\end{pmatrix} &=&
			\begin{pmatrix}
				-1 \\ -3
			\end{pmatrix} \\ [1.8em]
			
			f\begin{pmatrix}
				0 \\ 0 \\ 1
			\end{pmatrix} &=&
			\begin{pmatrix}
				0 - 0 + 1 \\
				3 \cdot 0 - 3 \cdot 0 + 3 \cdot 1
			\end{pmatrix} &=&
			\begin{pmatrix}
				1 \\ 3
			\end{pmatrix}
		\end{array}
	\end{equation*}
	La matrice associata $A$ rispetto alla base canonica è:
	\begin{equation*}
		A = \begin{pmatrix}
			1 & -1 & 1 \\
			3 & -3 & 3
		\end{pmatrix}
	\end{equation*}
	Per verificare se $f$ è un isomorfismo, è possibile verificare se l'applicazione lineare è invertibile. In caso affermativo, allora la funzione è isomorfa. L'applicazione lineare inversa si calcola eseguendo l'inversione della matrice associata $A$, che in questo caso \underline{non} è quadrata. Per cui, è possibile già concludere che l'applicazione lineare $f$ non è un isomorfismo.\newpage
	
	\subsection{Soluzione \emph{c}}
	
	Si calcolino le dimensioni degli spazi vettoriali $\mathrm{Im}\left(f\right) \subseteq \mathbb{C}^{2}$ e $\mathrm{N}\left(f\right) \subseteq \mathbb{C}^{3}$.\newline
	
	\noindent
	La dimensione dell'immagine dell'applicazione lineare corrisponde alla dimensione della matrice associata nel caso in cui quest'ultima sia stata costruita utilizzando le basi canoniche. Inoltre, la dimensione della matrice associata corrisponde al rango di essa:
	\begin{equation*}
		\dim\left(\mathrm{Im}\left(f\right)\right) = \dim\left(A\right) = \rk\left(A\right)
	\end{equation*}
	Si procede con il calcolo applicando l'eliminazione di Gauss:
	\begin{equation*}
		\begin{pmatrix}
			1 & -1 & 1 \\
			3 & -3 & 3
		\end{pmatrix} \xlongrightarrow{E_{1,2}\left(-3\right)}
		\begin{pmatrix}
			1 & -1 & 1 \\
			0 &  0 & 0
		\end{pmatrix}
	\end{equation*}
	Il rango corrisponde a $1$, quindi anche la sua dimensione è $1$:
	\begin{equation*}
		\rk\left(A\right) = 1 = \dim\left(A\right)
	\end{equation*}
	Come viene enunciato nella lezione 17, la dimensione $\dim\left(f\right)$ viene chiamata nullità di $f$. Essa è possibile calcolarla trovando la dimensione del nucleo dell'applicazione lineare $f$:
	\begin{equation*}
		N\left(f\right) = \dim\left(\ker\left(f\right)\right)
	\end{equation*}
	Sia $v \in \mathbb{C}^{3}$ e siano $\left(\alpha, \beta, \gamma\right)$ le sue coordinate riferite alla base $\mathcal{C}_{\mathbb{C}^{3}}$. Si imposta l'uguaglianza e il sistema relativo:
	\begin{equation*}
		Av = 0_{\mathbb{C}^{2}} \longrightarrow
		\begin{pmatrix}
			1 & -1 & 1 \\
			3 & -3 & 3
		\end{pmatrix}
		\begin{pmatrix}
			\alpha \\ \beta \\ \gamma
		\end{pmatrix} =
		\begin{pmatrix}
			0 \\ 0
		\end{pmatrix} \longrightarrow
		\begin{cases}
			\alpha -\beta + \gamma = 0 \\
			3\alpha -3\beta +3 \gamma = 0
		\end{cases}
	\end{equation*}
	Il rango della matrice è pari 1. Per applicare il teorema di Rouché Capelli, è necessario calcolare il rango anche della matrice aumentata tramite EG:
	\begin{equation*}
		\begin{rowequmat}{ccc|c}
			1 & -1 & 1 & 0 \\
			3 & -3 & 3 & 0
		\end{rowequmat} \xlongrightarrow{E_{1,2}\left(-3\right)}
		\begin{rowequmat}{ccc|c}
			1 & -1 & 1 & 0 \\
			0 &  0 & 0 & 0
		\end{rowequmat}
	\end{equation*}
	Anche nella matrice aumentata il rango è 1, quindi per il teorema di Rouché Capelli:
	\begin{equation*}
		\rk\left(A\right) = \rk\left(A|b\right) < n \longrightarrow 1 = 1 < 3
	\end{equation*}
	Dove $n$ indica il numero di incognite, allora il sistema ammette $\infty^{n-\rk\left(A\right)}$ soluzioni. In questo caso $\infty^{3-1} = \infty^{2}$. Per cui la dimensione del nucleo dell'applicazione lineare $f$ è pari a $2$:
	\begin{equation*}
		N\left(f\right) = \dim\left(\ker\left(f\right)\right) = 2
	\end{equation*}
	Infine, il teorema delle dimensioni o teorema della nullità più rango (\underline{forse non è} \underline{stato fatto a lezione!}) conferma il calcolo poiché:
	\begin{equation*}
		\dim\left(f\right) = \dim\left(\ker\left(f\right)\right) + \dim\left(\mathrm{Im}\left(f\right)\right) \longrightarrow \dim\left(\mathbb{C}^{3}\right) = 2 + 1
	\end{equation*}
	
	
	\subsection{Soluzione \emph{d}}
	
	Si verifichi che l'insieme $\mathscr{C} = \left\{v_{1}, v_{2}, v_{3}\right\}$ con $v_{1} = \begin{pmatrix}
		2 \\ 1 \\ 0
	\end{pmatrix}, v_{2} = \begin{pmatrix}
		0 \\ 1 \\ 0
	\end{pmatrix}, v_{3} = \begin{pmatrix}
		1 \\ 0 \\ 1
	\end{pmatrix}$ è una base di $\mathbb{C}^{3}$.\newline
	
	\noindent
	L'insieme $\mathscr{C}$ è una base di $\mathbb{C}^{3}$ se i vettori che lo compongono sono un sistema di generatori e vettori linearmente indipendenti.
	
	\begin{proof}[\textbf{Dimostrazione che i vettori sono un sistema di generatori}]
		Per dimostrare che $\mathscr{C} = \left\{v_{1}, v_{2}, v_{3}\right\}$ è un sistema di generatori, è necessario stabilire se per ogni $w = \left(w_{1}, w_{2}, w_{3}\right)$ esistono tre scalari $a_{1}, a_{2}, a_{3} \in \mathbb{C}$ tali che:
		\begin{equation*}
			a_{1}v_{1} + a_{2}v_{2} + a_{3}v_{3} = w
		\end{equation*}
		Si sostituiscono i dati:
		\begin{equation*}
			a_{1} \begin{pmatrix}
				2 \\ 1 \\ 0
			\end{pmatrix} + a_{2} \begin{pmatrix}
				0 \\ 1 \\ 0
			\end{pmatrix} + a_{3} \begin{pmatrix}
				1 \\ 0 \\ 1
			\end{pmatrix} = \begin{pmatrix}
				w_{1} \\ w_{2} \\ w_{3}
			\end{pmatrix} \longrightarrow
			\begin{pmatrix}
				2a_{1} + a_{3} \\
				a_{1}  + a_{2} \\
				a_{3}
			\end{pmatrix} = \begin{pmatrix}
				w_{1} \\ w_{2} \\ w_{3}
			\end{pmatrix}
		\end{equation*}
		Si costruisce il sistema relativo:
		\begin{equation*}
			\begin{cases}
				2a_{1} + a_{3} = w_{1} \\
				a_{1}  + a_{2} = w_{2}\\
				a_{3} = w_{3}
			\end{cases}
		\end{equation*}
		Se il sistema ammette soluzione, allora i vettori $v_{1}, v_{2}, v_{3}$ sono un sistema di generatori. Quindi, viene calcolato il rango della matrice incompleta $A$ e completa $\left(A|b\right)$, così da applicare il teorema di Rouché Capelli:
		\begin{equation*}
			\begin{array}{rll}
				A &=& \begin{rowequmat}{ccc}
					2 & 0 & 1 \\
					1 & 1 & 0 \\
					0 & 0 & 1
				\end{rowequmat} \xlongrightarrow{E_{1,2}\left(-\frac{1}{2}\right)}
				\begin{rowequmat}{ccc}
					2 & 0 & 1 \\ [0.3em]
					0 & 1 & -\frac{1}{2} \\ [0.3em]
					0 & 0 & 1
				\end{rowequmat} \rightarrow \rk\left(A\right) = 3 \\ [1.8em]
				\left(A|b\right) &=& \begin{rowequmat}{ccc|c}
					2 & 0 & 1 & w_{1} \\
					1 & 1 & 0 & w_{2} \\
					0 & 0 & 1 & w_{3}
				\end{rowequmat} \xlongrightarrow{E_{1,2}\left(-\frac{1}{2}\right)}
				\begin{rowequmat}{ccc|c}
					2 & 0 & 1 & w_{1} \\ [0.3em]
					0 & 1 & -\frac{1}{2} & \frac{2w_{2} - w_{1}}{2} \\ [0.3em]
					0 & 0 & 1 & w_{3}
				\end{rowequmat} \rightarrow \rk\left(A|b\right) = 3
			\end{array}
		\end{equation*}
		Applicando il teorema di Rouché Capelli, il sistema ammette una e una sola soluzione poiché:
		\begin{equation*}
			\rk\left(A\right) = \rk\left(A|b\right) = n \xlongrightarrow{\text{sostituzione}} 3 = 3 = 3
		\end{equation*}
		Con $n$ che corrisponde al numero di incognite. Quindi, si può concludere la dimostrazione affermando che i tre vettori sono un insieme di generatori.
	\end{proof}\newpage
	
	\begin{proof}[\textbf{Dimostrazione che i vettori sono linearmente indipendenti}]
		Se i vettori sono linearmente indipendenti, allora l'insieme è una base. Si procede con la dimostrazione. Siano $b_{1}, b_{2}, b_{3} \in \mathbb{C}$ e si verifichi che:
		\begin{equation*}
			b_{1}v_{1} + b_{2}v_{2} + b_{3}v_{3} = 0 \Rightarrow b_{1} = b_{2} = b_{3} = 0
		\end{equation*}
		Quindi, si procede con la sostituzione:
		\begin{equation*}
			b_{1} \begin{pmatrix}
				2 \\ 1 \\ 0
			\end{pmatrix} + b_{2} \begin{pmatrix}
				0 \\ 1 \\ 0
			\end{pmatrix} + b_{3} \begin{pmatrix}
				1 \\ 0 \\ 1
			\end{pmatrix} = \begin{pmatrix}
				0 \\ 0 \\ 0
			\end{pmatrix} \longrightarrow
			\begin{pmatrix}
				2b_{1} + b_{3} \\
				b_{1}  + b_{2} \\
				b_{3}
			\end{pmatrix} = \begin{pmatrix}
				0 \\ 0 \\ 0
			\end{pmatrix}
		\end{equation*}
		Si costruisce il sistema relativo:
		\begin{equation*}
			\begin{cases}
				2b_{1} + b_{3} = 0 \\
				b_{1}  + b_{2} = 0 \\
				b_{3} = 0
			\end{cases}
		\end{equation*}
		Applicando banalmente il metodo di sostituzione, si arriva alla conclusione che l'unica soluzione del sistema è $b_{1} = b_{2} = 0$. Per cui, i vettori $v_{1}, v_{2}, v_{3}$ sono linearmente indipendenti.
	\end{proof} \:\newline
	
	\noindent
	L'esercizio si conclude affermando che l'insieme $\mathscr{C}$ è una base di $\mathbb{C}^{3}$ grazie alla dimostrazione delle due condizioni: sistema di generatori e linearmente indipendenti.\newpage
	
	\subsection{Soluzione \emph{e}}
	
	Si verifichi che l'insieme $\mathscr{B} = \left\{w_{1}, w_{2}\right\}$ con $w_{1} = \begin{pmatrix}
		1 \\ 1
	\end{pmatrix}, w_{2} = \begin{pmatrix}
		1 \\ 0
	\end{pmatrix}$ è una base di $\mathbb{C}^{2}$.\newline
	
	\noindent
	L'insieme $\mathscr{B}$ è una base di $\mathbb{C}^{2}$ se i vettori che lo compongono sono un sistema di generatori e vettori linearmente indipendenti.
	
	\begin{proof}[\textbf{Dimostrazione che i vettori sono un sistema di generatori}]
		Per dimostrare che $\mathscr{B} = \left\{w_{1}, w_{2}\right\}$ è un sistema di generatori, è necessario stabilire se per ogni $v = \left(v_{1}, v_{2}\right)$ esistono due scalari $a_{1}, a_{2} \in \mathbb{C}$ tali che:
		\begin{equation*}
			a_{1} w_{1} + a_{2} w_{2} = v
		\end{equation*}
		Si sostituiscono i dati:
		\begin{equation*}
			a_{1} \begin{pmatrix}
				1 \\ 1
			\end{pmatrix} +
			a_{2} \begin{pmatrix}
				1 \\ 0
			\end{pmatrix} =
			\begin{pmatrix}
				v_{1} \\ v_{2}
			\end{pmatrix} \longrightarrow
			\begin{pmatrix}
				a_{1} + a_{2} \\
				a_{1}
			\end{pmatrix} = 
			\begin{pmatrix}
				v_{1} \\ v_{2}
			\end{pmatrix}
		\end{equation*}
		Si costruisce il sistema relativo:
		\begin{equation*}
			\begin{cases}
				a_{1} + a_{2} = v_{1} \\
				a_{1} = v_{2}
			\end{cases}
		\end{equation*}
		Se il sistema ammette soluzione, allora i vettori $w_{1}, w_{2}$ sono un sistema di generatori. Quindi, viene calcolato il rango della matrice incompleta $A$ e completa $\left(A|b\right)$, così da applicare il teorema di Rouché Capelli:
		\begin{equation*}
			\begin{array}{rll}
				A &=& \begin{rowequmat}{cc}
					1 & 1 \\ 0 & 1
				\end{rowequmat} \longrightarrow \rk\left(A\right) = 2 \\ [1.8em]
				\left(A|b\right) &=& \begin{rowequmat}{cc|c}
					1 & 1 & v_{1} \\ 0 & 1 & v_{2}
				\end{rowequmat} \longrightarrow \rk\left(A|b\right) = 2
			\end{array}
		\end{equation*}
		Per fortuna le matrici erano già in forma ridotta. Adesso, applicando il teorema di Rouché Capelli, si può osservare che il sistema ammette una e una sola soluzione:
		\begin{equation*}
			\rk\left(A\right) = \rk\left(A|b\right) = n \xlongrightarrow{\text{sostituzione}} 2 = 2 = 2
		\end{equation*}
		Con $n$ che corrisponde al numero di incognite. Quindi, si conclude la dimostrazione affermando che i due vettori sono un insieme di generatori.
	\end{proof}\newpage
	
	\begin{proof}[\textbf{Dimostrazione che i vettori sono linearmente indipendenti}]
		Se i vettori sono linearmente indipendenti, allora l'insieme è una base. Si procede con la dimostrazione. Siano $b_{1}, b_{2} \in \mathbb{C}$ e si verifichi che:
		\begin{equation*}
			b_{1} w_{1} + b_{2} w_{2} = 0 \Rightarrow b_{1} = b_{2} = 0
		\end{equation*}
		Si procede con la sostituzione:
		\begin{equation*}
			b_{1} \begin{pmatrix}
				1 \\ 1
			\end{pmatrix} +
			b_{2} \begin{pmatrix}
				1 \\ 0
			\end{pmatrix} =
			\begin{pmatrix}
				0 \\ 0
			\end{pmatrix} \longrightarrow
			\begin{pmatrix}
				b_{1} + b_{2} \\
				b_{1}
			\end{pmatrix} = \begin{pmatrix}
				0 \\ 0
			\end{pmatrix}
		\end{equation*}
		Si costruisce il sistema relativo:
		\begin{equation*}
			\begin{cases}
				b_{1} + b_{2} = 0 \\
				b_{1} = 0
			\end{cases}
		\end{equation*}
		Applicando banalmente il metodo di sostituzione, si arriva alla conclusione che l'unica soluzione del sistema è $b_{1} = b_{2} = 0$. Per cui, i vettori $w_{1}, w_{2}$ sono linearmente indipendenti.
	\end{proof}\:\newline
	
	\noindent
	L'esercizio si conclude affermando che l'insieme $\mathscr{B}$ è una base di $\mathbb{C}^{2}$ grazie alla dimostrazione delle due condizioni: sistema di generatori e linearmente indipendenti.\newpage
	
	\subsection{Soluzione \emph{f}}
	
	Si determini la matrice associata a $f$ rispetto alla base $\mathscr{C}$ di $\mathbb{C}^{3}$ e alla base $\mathscr{B}$ di $\mathbb{C}^{2}$.\newline
	
	\noindent
	Ricordando che $f$ è l'applicazione lineare così definita:
	\begin{equation*}
		f: \mathbb{C}^{3} \longrightarrow \mathbb{C}^{2}, \hspace{2em}
		f \left(
		\begin{pmatrix}
			x \\ y \\ z
		\end{pmatrix}
		\right) = \begin{pmatrix}
			x-y+z \\
			3x-3y+3z
		\end{pmatrix}, \hspace{2em}
		\forall \begin{pmatrix}
			x \\ y \\ z
		\end{pmatrix} \in \mathbb{C}^{3}
	\end{equation*}
	E che le basi fornite sono:
	\begin{equation*}
		\begin{array}{lllll}
			\mathscr{C}_{\mathbb{C}^{3}} &=& \left\{v_{1}, v_{2}, v_{3}\right\} &=& \left\{\begin{pmatrix}
				2 \\ 1 \\ 0
			\end{pmatrix}, \begin{pmatrix}
				0 \\ 1 \\ 0
			\end{pmatrix}, \begin{pmatrix}
				1 \\ 0 \\ 1
			\end{pmatrix}\right\} \\ [1.8em]
			%
			\mathscr{B}_{\mathbb{C}^{2}} &=& \left\{w_{1}, w_{2}\right\} &=& \left\{\begin{pmatrix}
				1 \\ 1
			\end{pmatrix}, \begin{pmatrix}
				1 \\ 0
			\end{pmatrix}\right\}
		\end{array}
	\end{equation*}
	Si calcola inizialmente le immagini mediante l'applicazione lineare $f$ dei vettori della base $\mathscr{C}_{\mathbb{C}^{3}}$:
	\begin{equation*}
		\begin{array}{lllll}
			f\begin{pmatrix}
				2 \\ 1 \\ 0
			\end{pmatrix} &=& \begin{pmatrix}
				2 - 1 + 0 \\
				3 \cdot 2 - 3 \cdot 1 + 3 \cdot 0
			\end{pmatrix} &=& \begin{pmatrix}
				1 \\
				3
			\end{pmatrix} \\ [1.8em]
			%
			f\begin{pmatrix}
				0 \\ 1 \\ 0
			\end{pmatrix} &=& \begin{pmatrix}
				0 - 1 + 0 \\
				3 \cdot 0 - 3 \cdot 1 + 3 \cdot 0
			\end{pmatrix} &=& \begin{pmatrix}
				-1 \\
				-3
			\end{pmatrix} \\ [1.8em]
			%
			f\begin{pmatrix}
				1 \\ 0 \\ 1
			\end{pmatrix} &=& \begin{pmatrix}
				1 - 0 + 1 \\
				3 \cdot 1 - 3 \cdot 0 + 3 \cdot 1
			\end{pmatrix} &=& \begin{pmatrix}
				2 \\
				6
			\end{pmatrix}
		\end{array}
	\end{equation*}
	Adesso si cercano le coordinate rispetto alla base $\mathscr{B}$ dei vettori $f\left(v_{1}\right), f\left(v_{2}\right), f\left(v_{3}\right)$. Quindi, si cercano due scalari $a_{1}, a_{2} \in \mathbb{C}$ tali che:
	\begin{itemize}
		\item $f\left(v_{1}\right)$ sia combinazione lineare di $w_{1}, w_{2}$
		
		\item $f\left(v_{2}\right)$ sia combinazione lineare di $w_{1}, w_{2}$
		
		\item $f\left(v_{3}\right)$ sia combinazione lineare di $w_{1}, w_{2}$
	\end{itemize}\newpage
	
	\noindent
	Da notare che per tutti e tre il sistema di incognite è lo stesso, cambia ovviamente il valore dopo l'uguale a seconda dell'immagine applicata:
	\begin{equation*}
		\begin{array}{rllll}
			a_{1}w_{1} + a_{2}w_{2} = f\left(v_{1}\right) &\rightarrow& a_{1}\begin{pmatrix}
				1 \\ 1
			\end{pmatrix} + a_{2}\begin{pmatrix}
				1 \\ 0
			\end{pmatrix} = \begin{pmatrix}
				1 \\ 3
			\end{pmatrix} &\rightarrow& \begin{cases}
				a_{1} + a_{2} = 1 \\
				a_{1} = 3
			\end{cases} \\
			&& &\parallel& \\
			&& &\rightarrow& \begin{cases}
				a_{2} = -2 \\
				a_{1} = 3
			\end{cases} \\
			&& 3\begin{pmatrix}
				1 \\ 1
			\end{pmatrix} - 2\begin{pmatrix}
				1 \\ 0
			\end{pmatrix} = \begin{pmatrix}
				1 \\ 3
			\end{pmatrix} &\hookleftarrow& \text{Sostituisco } a_{1},a_{2} \\ [1.8em]
			%
			%
			a_{1}w_{1} + a_{2}w_{2} = f\left(v_{2}\right) &\rightarrow& a_{1}\begin{pmatrix}
				1 \\ 1
			\end{pmatrix} + a_{2}\begin{pmatrix}
				1 \\ 0
			\end{pmatrix} = \begin{pmatrix}
				-1 \\ -3
			\end{pmatrix} &\rightarrow& \begin{cases}
				a_{1} + a_{2} = -1 \\
				a_{1} = -3
			\end{cases} \\
			&& &\parallel& \\
			&& &\rightarrow& \begin{cases}
				a_{2} = 2 \\
				a_{1} = -3
			\end{cases} \\
			&& -3\begin{pmatrix}
				1 \\ 1
			\end{pmatrix} + 2\begin{pmatrix}
				1 \\ 0
			\end{pmatrix} = \begin{pmatrix}
				-1 \\ -3
			\end{pmatrix} &\hookleftarrow& \text{Sostituisco } a_{1},a_{2} \\ [1.8em]
			%
			%
			a_{1}w_{1} + a_{2}w_{2} = f\left(v_{3}\right) &\rightarrow& a_{1}\begin{pmatrix}
				1 \\ 1
			\end{pmatrix} + a_{2}\begin{pmatrix}
				1 \\ 0
			\end{pmatrix} = \begin{pmatrix}
				2 \\ 6
			\end{pmatrix} &\rightarrow& \begin{cases}
				a_{1} + a_{2} = 2 \\
				a_{1} = 6
			\end{cases}  \\
			&& &\parallel& \\
			&& &\rightarrow& \begin{cases}
				a_{2} = -4 \\
				a_{1} = 6
			\end{cases} \\
			&& 6\begin{pmatrix}
				1 \\ 1
			\end{pmatrix} - 4\begin{pmatrix}
				1 \\ 0
			\end{pmatrix} = \begin{pmatrix}
				2 \\ 6
			\end{pmatrix} &\hookleftarrow& \text{Sostituisco } a_{1},a_{2}
		\end{array}
	\end{equation*}
	La matrice associata $A$ alla trasformazione $f$ rispetto alle basi $\mathscr{C}_{\mathbb{C}^{3}}$ e $\mathscr{B}_{\mathbb{C}^{2}}$ è la matrice avente per colonne i coefficienti trovati ($a_{1}, a_{2}$):
	\begin{equation*}
		A = \begin{pmatrix}
			 3 & -3 &  6 \\
			-2 &  2 &  -4
		\end{pmatrix}
	\end{equation*}\newpage
	
	\section{Soluzione esercizio 4}
	
	Sia $\mathscr{C}$ la base di $\mathbb{C}^{3}$ dell'esercizio 3(d) e sia $\mathscr{D} = \left\{u_{1}, u_{2}, u_{3}\right\}$ dove $u_{1} = \begin{pmatrix}
		1 \\ 0 \\ 1
	\end{pmatrix}, u_{2} = \begin{pmatrix}
		6 \\ -1 \\ 8
	\end{pmatrix}, u_{3} = \begin{pmatrix}
		-8 \\ -8 \\ 1
	\end{pmatrix}$. Si verifichi che $\mathscr{D}$ è una base di $\mathbb{C}^{3}$ e si calcoli la matrice del cambio di base $\mathscr{C} \rightarrow \mathscr{D}$.\newline
	
	\noindent
	Innanzitutto si esegue la dimostrazione che $\mathscr{D}$ è una base di $\mathbb{C}^{3}$. Si esegue prima le verifica della condizione che i vettori siano un sistema di generatori e, successivamente, che siano linearmente indipendenti.
	
	\begin{proof}[\textbf{Dimostrazione che i vettori sono un sistema di generatori}]
		Per dimostrare che $\mathscr{D} = \left\{u_{1}, u_{2}, u_{3}\right\}$ è un sistema di generatori, è necessario stabilire se per ogni $w = \left\{w_{1}, w_{2}, w_{3}\right\}$ esistono tre scalari $a_{1}, a_{2}, a_{3} \in \mathbb{C}$ tali che:
		\begin{equation*}
			a_{1}u_{1} + a_{2}u_{2} + a_{3}u_{3} = w
		\end{equation*}
		Si sostituiscono i dati:
		\begin{equation*}
			a_{1}\begin{pmatrix}
				1 \\ 0 \\ 1
			\end{pmatrix} +
			a_{2}\begin{pmatrix}
				6 \\ -1 \\ 8
			\end{pmatrix} +
			a_{3}\begin{pmatrix}
				-8 \\ -8 \\ 1
			\end{pmatrix} =
			\begin{pmatrix}
				w_{1} \\ w_{2} \\ w_{3}
			\end{pmatrix} \longrightarrow
			\begin{pmatrix}
				a_{1} + 6a_{2} - 8a_{3} \\
				-a_{2} - 8a_{3} \\
				a_{1} + 8a_{2} + a_{3}
			\end{pmatrix} =
			\begin{pmatrix}
				w_{1} \\ w_{2} \\ w_{3}
			\end{pmatrix}
		\end{equation*}
		Si costruisce il sistema relativo:
		\begin{equation*}
			\begin{cases}
				a_{1} + 6a_{2} - 8a_{3} = w_{1} \\
				-a_{2} - 8a_{3} = w_{2} \\
				a_{1} + 8a_{2} + a_{3} = w_{3}
			\end{cases}
		\end{equation*}
		Se il sistema ammette soluzione, allora i vettori $u_{1}, u_{2}, u_{3}$ sono un sistema di generatori. Quindi, viene calcolato il rango della matrice incompleta $A$ e completa $\left(A|b\right)$, così da applicare il teorema di Rouché Capelli:
		\begin{equation*}
				\begin{array}{rll}
					A &=& \begin{rowequmat}{ccc}
						1 &  6 & -8 \\
						0 & -1 & -8 \\
						1 &  8 &  1
					\end{rowequmat} \xlongrightarrow[E_{2,3}\left(2\right)]{E_{1,3}\left(-1\right)}
					\begin{rowequmat}{ccc}
						1 &  6 & -8 \\
						0 & -1 & -8 \\
						0 &  0 &  7
					\end{rowequmat} \rightarrow \rk\left(A\right) = 3 \\ [1.8em]
					\left(A|b\right) &=& \begin{rowequmat}{ccc|c}
						1 &  6 & -8 & w_{1} \\
						0 & -1 & -8 & w_{2} \\
						1 &  8 &  1 & w_{3}
					\end{rowequmat} \xlongrightarrow[E_{2,3}\left(2\right)]{E_{1,3}\left(-1\right)}
					\begin{rowequmat}{ccc|c}
						1 &  6 & -8 & w_{1} \\
						0 & -1 & -8 & w_{2} \\
						0 &  0 &  7 & w_{3} - w_{1} + 2w_{2}
					\end{rowequmat} \rightarrow \rk\left(A|b\right) = 3
				\end{array}
		\end{equation*}
		Applicando il teorema di Rouché Capelli, il sistema ammette una e una sola soluzione poiché:
		\begin{equation*}
			\rk\left(A\right) = \rk\left(A|b\right) = n \xlongrightarrow{\text{sostituzione}} 3 = 3 = 3
		\end{equation*}
		Con $n$ che corrisponde al numero di incognite. Quindi, si può concludere la dimostrazione affermando che i tre vettori sono un insieme di generatori.
	\end{proof}\newpage
	
	\begin{proof}[\textbf{Dimostrazione che i vettori sono linearmente indipendenti}]
		Se i vettori sono linearmente indipendenti, allora l'insieme è una base. Si procede con la dimostrazione. Siano $b_{1}, b_{2}, b_{3} \in \mathbb{C}$ e si verifichi che:
		\begin{equation*}
			b_{1}u_{1} + b_{2}u_{2} + b_{3}u_{3} = 0 \Rightarrow b_{1} = b_{2} = b_{3} = 0
		\end{equation*}
		Quindi, si procede con la sostituzione:
		\begin{equation*}
			b_{1}\begin{pmatrix}
				1 \\ 0 \\ 1
			\end{pmatrix} +
			b_{2}\begin{pmatrix}
				6 \\ -1 \\ 8
			\end{pmatrix} +
			b_{3}\begin{pmatrix}
				-8 \\ -8 \\ 1
			\end{pmatrix} =
			\begin{pmatrix}
				0 \\ 0 \\ 0
			\end{pmatrix} \longrightarrow
			\begin{pmatrix}
				b_{1} + 6b_{2} - 8b_{3} \\
				-b_{2} - 8b_{3} \\
				b_{1} + 8b_{2} + b_{3}
			\end{pmatrix} =
			\begin{pmatrix}
				0 \\ 0 \\ 0
			\end{pmatrix}
		\end{equation*}
		Si costruisce il sistema relativo:
		\begin{equation*}
			\begin{cases}
				b_{1} + 6b_{2} - 8b_{3} = 0 \\
				-b_{2} - 8b_{3} = 0 \\
				b_{1} + 8b_{2} + b_{3} = 0
			\end{cases}
		\end{equation*}
		Si risolve sfruttando l'Eliminazione di Gauss:
		\begin{equation*}
			\begin{rowequmat}{ccc|c}
				1 &  6 & -8 & 0 \\
				0 & -1 & -8 & 0 \\
				1 &  8 &  1 & 0
			\end{rowequmat} \xlongrightarrow[E_{2,3}\left(2\right)]{E_{1,3}\left(-1\right)}
			\begin{rowequmat}{ccc|c}
				1 &  6 & -8 & 0 \\
				0 & -1 & -8 & 0 \\
				0 &  0 &  7 & 0
			\end{rowequmat}
		\end{equation*}
		Si riscrive il sistema e si procede con il classico metodo di sostituzione:
		\begin{equation*}
			\begin{array}{lllll}
				&\begin{cases}
					b_{1} + 6b_{2} - 8b_{3} = 0 \\
					-b_{2} - 8b_{3} = 0 \\
					7b_{3} = 0
				\end{cases} &\rightarrow&
				\begin{cases}
					b_{1} + 6b_{2} - 8 \cdot 0 = 0 \\
					-b_{2} - 8 \cdot 0 = 0 \\
					b_{3} = 0
				\end{cases} \rightarrow \\ [1.8em]
				\rightarrow & \begin{cases}
					b_{1} + 6 \cdot 0 = 0 \\
					b_{2} = 0 \\
					b_{3} = 0
				\end{cases} &\rightarrow&
				\begin{cases}
					b_{1} = 0 \\
					b_{2} = 0 \\
					b_{3} = 0
				\end{cases}
			\end{array}
		\end{equation*}
		La condizione $b_{1} = b_{2} = b_{3} = 0$ è stata rispettata poiché essa è l'unica soluzione del sistema (si potrebbe verificare anche con il teorema Rouché Capelli). Per cui, i vettori $u_{1}, u_{2}, u_{3}$ sono linearmente indipendenti.
	\end{proof}\newpage
	
	\noindent
	Dopo aver dimostrato che l'insieme $\mathscr{D}$ è una base di $\mathbb{C}^{3}$, grazie alla dimostrazione delle due condizioni (sistema di generatori e linearmente indipendenti), si prosegue calcolando la matrice del cambio di base $\mathscr{C} \rightarrow \mathscr{D}$. Per comodità si riportano qua di seguito le due basi:
	\begin{equation*}
		\begin{array}{lll}
			\mathscr{C} &=& \left\{v_{1} = \begin{pmatrix}
				2 \\ 1 \\ 0
			\end{pmatrix}, v_{2} = \begin{pmatrix}
				0 \\ 1 \\ 0
			\end{pmatrix}, v_{3} = \begin{pmatrix}
				1 \\ 0 \\ 1
			\end{pmatrix}\right\} \\ [1.8em]
			\mathscr{D} &=& \left\{u_{1} = \begin{pmatrix}
				1 \\ 0 \\ 1
			\end{pmatrix}, u_{2} = \begin{pmatrix}
				6 \\ -1 \\ 8
			\end{pmatrix}, u_{3} = \begin{pmatrix}
				-8 \\ -8 \\ 1
			\end{pmatrix}\right\}
		\end{array}
	\end{equation*}
	Per trovare la matrice del cambiamento di base che esprime il passaggio da $\mathscr{C} \rightarrow \mathscr{D}$, è necessario esprimere di vettori di $\mathscr{C}$ come combinazioni lineari dei vettori di $\mathscr{D}$:
	\begin{itemize}
		\item La prima:
		\begin{equation*}
			v_{1} = a_{1} u_{1} + b_{1} u_{2} + c_{1} u_{3}
		\end{equation*}
		Si esegue la sostituzione dei valori:
		\begin{equation*}
			\begin{pmatrix}
				2 \\ 1 \\ 0
			\end{pmatrix} =
			a_{1} \begin{pmatrix}
				1 \\ 0 \\ 1
			\end{pmatrix} +
			b_{1} \begin{pmatrix}
				6 \\ -1 \\ 8
			\end{pmatrix} +
			c_{1} \begin{pmatrix}
				-8 \\ -8 \\ 1
			\end{pmatrix}
		\end{equation*}
		Si costruisce il sistema e si applica l'Eliminazione di Gauss (EG) per semplificare il sistema:
		\begin{equation*}
			\begin{array}{l}
				\begin{cases}
					a_{1} + 6b_{1} - 8c_{1} = 2 \\
					-b_{1} - 8c_{1} = 1 \\
					a_{1} + 8c_{1} + c_{1} = 0
				\end{cases} \xlongrightarrow{EG}
				\begin{cases}
					a_{1} + 6b_{1} - 8c_{1} = 2 \\
					-b_{1} - 8c_{1} = 1 \\
					7c_{1} = 0
				\end{cases} \rightarrow
				\begin{cases}
					a_{1} + 6b_{1} - 8\cdot0 = 2 \\
					-b_{1} - 8\cdot0 = 1 \\
					c_{1} = 0
				\end{cases} \\ [2em]
				\rightarrow \begin{cases}
					a_{1} + 6 \cdot \left(-1\right) = 2 \\
					b_{1} = -1 \\
					c_{1} = 0
				\end{cases} \rightarrow
				\begin{cases}
					a_{1} =  8 \\
					b_{1} = -1 \\
					c_{1} =  0
				\end{cases}
			\end{array}
		\end{equation*}
		Quindi la prima combinazione lineare è:
		\begin{equation*}
			\begin{pmatrix}
				2 \\ 1 \\ 0
			\end{pmatrix} =
			8 \begin{pmatrix}
				1 \\ 0 \\ 1
			\end{pmatrix} +
			-1 \begin{pmatrix}
				6 \\ -1 \\ 8
			\end{pmatrix} +
			0 \begin{pmatrix}
				-8 \\ -8 \\ 1
			\end{pmatrix}
		\end{equation*}
		
		\item La seconda:
		\begin{equation*}
			v_{2} = a_{2} u_{1} + b_{2} u_{2} + c_{2} u_{3}
		\end{equation*}
		Si esegue la sostituzione dei valori:
		\begin{equation*}
			\begin{pmatrix}
				0 \\ 1 \\ 0
			\end{pmatrix} =
			a_{2} \begin{pmatrix}
				1 \\ 0 \\ 1
			\end{pmatrix} +
			b_{2} \begin{pmatrix}
				6 \\ -1 \\ 8
			\end{pmatrix} +
			c_{2} \begin{pmatrix}
				-8 \\ -8 \\ 1
			\end{pmatrix}
		\end{equation*}
		Si costruisce il sistema e si applica l'Eliminazione di Gauss (EG) per semplificare il sistema:
		\begin{equation*}
			\begin{array}{l}
				\begin{cases}
					a_{2} + 6b_{2} - 8c_{2} = 0 \\
					-b_{2} - 8c_{2} = 1 \\
					a_{2} + 8c_{2} + c_{2} = 0
				\end{cases} \xlongrightarrow{EG}
				\begin{cases}
					a_{2} + 6b_{2} - 8c_{2} = 0 \\
					-b_{2} - 8c_{2} = 1 \\
					7c_{2} = 2
				\end{cases} \rightarrow
				\begin{cases}
					a_{2} + 6b_{2} - 8 \cdot \frac{2}{7} = 0 \\
					-b_{2} - 8 \cdot \frac{2}{7} = 1 \\
					c_{2} = \frac{2}{7}
				\end{cases} \\ [2em]
				\rightarrow \begin{cases}
					a_{2} + 6 \cdot \frac{23}{7} - \frac{16}{7} = 0 \\
					b_{2} = -\frac{23}{7} \\
					c_{2} = \frac{2}{7}
				\end{cases} \rightarrow
				\begin{cases}
					a_{2} = -\frac{122}{7} \\
					b_{2} = -\frac{23}{7} \\
					c_{2} = \frac{2}{7}
				\end{cases}
			\end{array}
		\end{equation*}
		Quindi la seconda combinazione lineare è:
		\begin{equation*}
			\begin{pmatrix}
				0 \\ 1 \\ 0
			\end{pmatrix} =
			-\dfrac{122}{7} \begin{pmatrix}
				1 \\ 0 \\ 1
			\end{pmatrix} +
			-\dfrac{23}{7} \begin{pmatrix}
				6 \\ -1 \\ 8
			\end{pmatrix} +
			\dfrac{2}{7} \begin{pmatrix}
				-8 \\ -8 \\ 1
			\end{pmatrix}
		\end{equation*}
		
		\item La terza e ultima:
		\begin{equation*}
			v_{3} = a_{3} u_{1} + b_{3} u_{2} + c_{3} u_{3}
		\end{equation*}
		Si esegue la sostituzione dei valori:
		\begin{equation*}
			\begin{pmatrix}
				1 \\ 0 \\ 1
			\end{pmatrix} =
			a_{3} \begin{pmatrix}
				1 \\ 0 \\ 1
			\end{pmatrix} +
			b_{3} \begin{pmatrix}
				6 \\ -1 \\ 8
			\end{pmatrix} +
			c_{3} \begin{pmatrix}
				-8 \\ -8 \\ 1
			\end{pmatrix}
		\end{equation*}
		Si costruisce il sistema e si applica l'Eliminazione di Gauss (EG) per semplificare il sistema:
		\begin{equation*}
			\begin{array}{l}
				\begin{cases}
					a_{3} + 6b_{3} - 8c_{3} = 1 \\
					-b_{3} - 8c_{3} = 0 \\
					a_{3} + 8c_{3} + c_{3} = 1
				\end{cases} \xlongrightarrow{EG}
				\begin{cases}
					a_{3} + 6b_{3} - 8c_{3} = 1 \\
					-b_{3} - 8c_{3} = 0 \\
					7c_{3} = 0
				\end{cases} \rightarrow
				\begin{cases}
					a_{3} + 6b_{3} - 8 \cdot 0 = 1 \\
					-b_{3} - 8 \cdot 0 = 0 \\
					c_{3} = 0
				\end{cases} \\ [2em]
				\rightarrow \begin{cases}
					a_{3} + 6 \cdot 0 = 1 \\
					b_{3} = 0 \\
					c_{3} = 0
				\end{cases} \rightarrow
				\begin{cases}
					a_{3} = 1 \\
					b_{3} = 0 \\
					c_{3} = 0
				\end{cases}
			\end{array}
		\end{equation*}
		Quindi la terza combinazione lineare è:
		\begin{equation*}
			\begin{pmatrix}
				1 \\ 0 \\ 1
			\end{pmatrix} =
			1 \begin{pmatrix}
				1 \\ 0 \\ 1
			\end{pmatrix} +
			0 \begin{pmatrix}
				6 \\ -1 \\ 8
			\end{pmatrix} +
			0 \begin{pmatrix}
				-8 \\ -8 \\ 1
			\end{pmatrix}
		\end{equation*}
	\end{itemize}
	L'eliminazione di Gauss non viene mostrata perché identica alla pagina precedente con l'aggiunta dei valori diversi nell'ultima colonna. Si scrive il risultato nella prossima pagina così da avere più spazio.\newpage
	
	\noindent
	Riscrivendo le tre combinazioni qua di seguito:
	\begin{equation*}
		\begin{array}{lll}
			v_{1} &:& \begin{pmatrix}
				2 \\ 1 \\ 0
			\end{pmatrix} =
			8 \begin{pmatrix}
				1 \\ 0 \\ 1
			\end{pmatrix} +
			-1 \begin{pmatrix}
				6 \\ -1 \\ 8
			\end{pmatrix} +
			0 \begin{pmatrix}
				-8 \\ -8 \\ 1
			\end{pmatrix} \\ [1.8em]
			%
			v_{2} &:& \begin{pmatrix}
				0 \\ 1 \\ 0
			\end{pmatrix} =
			-\dfrac{122}{7} \begin{pmatrix}
				1 \\ 0 \\ 1
			\end{pmatrix} +
			-\dfrac{23}{7} \begin{pmatrix}
				6 \\ -1 \\ 8
			\end{pmatrix} +
			\dfrac{2}{7} \begin{pmatrix}
				-8 \\ -8 \\ 1
			\end{pmatrix} \\ [1.8em]
			%
			v_{3} &:& \begin{pmatrix}
				1 \\ 0 \\ 1
			\end{pmatrix} =
			1 \begin{pmatrix}
				1 \\ 0 \\ 1
			\end{pmatrix} +
			0 \begin{pmatrix}
				6 \\ -1 \\ 8
			\end{pmatrix} +
			0 \begin{pmatrix}
				-8 \\ -8 \\ 1
			\end{pmatrix}
		\end{array}
	\end{equation*}
	La matrice del cambio di base $\mathscr{C} \rightarrow \mathscr{D}$ è formata:
	\begin{equation*}
		M_{\mathscr{C} \rightarrow \mathscr{D}} = \begin{rowequmat}{ccc}
			8  & -\frac{122}{7} & 1 \\ [0.3em]
			-1 & -\frac{23}{7}  & 0 \\ [0.3em]
			0  &  \frac{2}{7}   & 0
		\end{rowequmat}
	\end{equation*}
\end{document}
\documentclass[a4paper]{article}
\usepackage[T1]{fontenc}			% pacchetto per \chapter
\usepackage[italian]{babel}
\usepackage[italian]{isodate}  		% formato delle date in italiano
\usepackage{graphicx}				% gestione delle immagini
\usepackage{amsfonts}
\usepackage{booktabs}				% tabelle di qualità superiore
\usepackage{amsmath}				% pacchetto matematica
\usepackage{enumitem}				% gestione delle liste
\usepackage{pifont}					% pacchetto con elenchi carini
\usepackage{listings}				% pacchetto per i codici
\usepackage[x11names]{xcolor}		% pacchetto colori RGB
% Link ipertestuali per l'indice
\usepackage{xcolor}
\usepackage[linkcolor=black, citecolor=blue, urlcolor=cyan]{hyperref}
\hypersetup{
	colorlinks=true
}

\newcommand{\longline}{\noindent\rule{\textwidth}{0.4pt}}

\definecolor{codegreen}{rgb}{0,0.6,0}
\definecolor{codegray}{rgb}{0.5,0.5,0.5}
\definecolor{codepurple}{rgb}{0.58,0,0.82}
\definecolor{backcolour}{rgb}{0.95,0.95,0.92}
\lstdefinestyle{mystyle}{
	backgroundcolor=\color{backcolour},   
	commentstyle=\color{codegreen},
	keywordstyle=\color{magenta},
	numberstyle=\tiny\color{codegray},
	stringstyle=\color{codepurple},
	basicstyle=\ttfamily\footnotesize,
	breakatwhitespace=false,         
	breaklines=true,                 
	captionpos=b,                    
	keepspaces=true,                 
	numbers=left,                    
	numbersep=5pt,                  
	showspaces=false,                
	showstringspaces=false,
	showtabs=false,                  
	tabsize=2
}

\lstdefinelanguage{JavaScript}{
	keywords={typeof, new, true, false, catch, function, return, null, catch, switch, var, if, in, while, do, else, case, break},
	keywordstyle=\color{blue}\bfseries,
	ndkeywords={class, export, boolean, throw, implements, import, this},
	ndkeywordstyle=\color{darkgray}\bfseries,
	identifierstyle=\color{black},
	sensitive=false,
	comment=[l]{//},
	morecomment=[s]{/*}{*/},
	commentstyle=\color{codegreen}\ttfamily,
	stringstyle=\color{red}\ttfamily,
	morestring=[b]',
	morestring=[b]"
}

\lstset{
	language=JavaScript,
	backgroundcolor=\color{lightgray},
	extendedchars=true,
	basicstyle=\footnotesize\ttfamily,
	showstringspaces=false,
	showspaces=false,
	numbers=left,
	numberstyle=\footnotesize,
	numbersep=9pt,
	tabsize=2,
	breaklines=true,
	showtabs=false,
	captionpos=b
}

\lstset{style=mystyle}

%\usepackage{showframe}				% visualizzazione bordi
%\usepackage{showkeys}				% visualizzazione etichetta

\newcommand{\dquotes}[1]{``#1''}

\begin{document}
	\author{VR443470}
	\title{Programmazione e sicurezza delle reti}
	\date{\printdayoff\today}
	\maketitle
	
	\newpage
	
	% indice
	\tableofcontents
	
	\newpage
	
	\section{Scrittura di applicazioni di rete mediante interfaccia socket}
	
	\subsection{Host, processo e applicazione}
	
	L'\textcolor{Red3}{\textbf{host}} (colui che ospita) è una \textbf{macchina} sempre identificata da un indirizzo IP a cui, opzionalmente, può essere associato un nome Internet.\newline
	
	\noindent
	Il \textcolor{Red3}{\textbf{processo}} è un \textbf{programma in esecuzione} sull'host, il quale trasmette/riceve pacchetti verso/da altri processi su altri host attraverso la rete. Viene identificato tramite un numero di porta nell'intervallo $0$ - $65535$.\newline
	
	\noindent
	Un \textcolor{Red3}{\textbf{applicazione}} è una collaborazione tra un \textbf{insieme di processi} sparsi sulla rete per fare qualcosa di utile per l'utente, per esempio chat, e-mail, ecc.\newline
	
	\noindent
	Alcuni \textcolor{Green4}{\textbf{esempi}}:
	\begin{itemize}
		\item Eseguendo una ricerca su internet:
		\begin{itemize}
			\item Il \emph{web} è l'applicazione;
			\item Mentre i browser (Chrome, Firefox, Edge, Safari) sono il processo di esecuzione;
			\item L'host è il PC, tablet o smartphone su cui viene aperto il browser;
			\item Apache o NGINX è il processo di esecuzione sulla macchina remota, anch'essa identificata come host.
		\end{itemize}
		
		\item Aprendo un'applicazione come Telegram:
		\begin{itemize}
			\item Telegram è l'applicazione;
			\item Il processo di esecuzione è sempre l'app Telegram che è in esecuzione sul dispositivo attualmente in uso (PC, tablet, ecc.) che funge da host;
			\item Il server di Telegram è il processo di esecuzione sulla macchina remota, anch'essa identificata come host.
		\end{itemize}
	\end{itemize}\newpage

	\subsection{Modalità di trasmissione in Internet}
	
	Su Internet la modalità di trasmissione è una sequenza di byte chiamata: pacchetto, Protocol Data Unit (PDU), Datagram. A seconda del livello del protocollo, ci sono nomi diversi.
	
	\longline
	
	\subsubsection{Applicazioni orientate al datagramma (UDP)}
	
	Alcune \textbf{applicazioni sono orientate al datagramma}, quindi \textbf{ogni pacchetto scambiato tra gli host è indipendente dai precedenti e successivi}. Le \textbf{perdite} di pacchetti non vengono tenute in considerazione ed un \textcolor{Green4}{\textbf{esempio}} può essere la \textbf{trasmissione di temperature}: il ricevitore può non tener conto di alcune perdite poiché le informazioni ricevute non sono necessarie per il futuro.
	
	\longline
	
	\subsubsection{Applicazioni orientate alla connessione (TCP)}
	
	Invece, alcune \textbf{applicazioni sono orientate alla connessione}. A differenza delle applicazioni orientate al datagramma, quelle orientate alla connessione devono tener conto delle perdite poiché solitamente le informazioni scambiate sono di dimensioni rilevanti (e.g. un'immagine). Di conseguenza, una perdita provocherebbe una lettura parziale o impossibile da parte del ricevitore.\newline
	
	\noindent
	Il \textbf{socket} si preoccupa di \textbf{numerare i pacchetti appartenenti alla stessa connessione} per rilevare eventuali pacchetti persi e poterli ritrasmettere.\newline
	
	\noindent
	Il sistema operativo introduce all'interno dei pacchetti un numero di sequenza così che possa rilevare eventuali pacchetti persi e ritrasmetterli.
	\begin{itemize}
		\item \textcolor{Green4}{\textbf{Vantaggi:}}
		\begin{itemize}
			\item L'utente scrive/legge su un archivio remoto con la stessa naturalezza di quando scrive/legge su un archivio locale come se la rete in mezzo non ci fosse.
		\end{itemize}
		
		\item \textcolor{Red3}{\textbf{Svantaggi:}}
		\begin{itemize}
			\item Gli host mittente e destinatario eseguono un lavoro più complesso con il sistema operativo;
			\item Ritardo di ritrasmissione nel caso in cui i pacchetti vengano persi.
		\end{itemize}
	\end{itemize}\newpage

	\subsection{Schemi di applicazioni che utilizzano la rete}
	
	Le applicazioni di rete sono insiemi di processi su host diversi che si scambiano messaggi attraverso la rete. \textbf{Esistono degli schemi base che regolano lo scambio di messaggi}:
	\begin{itemize}
		\item Modello client/server;
		\item Modello Publisher/Subscriber (Pub/Sub)
	\end{itemize}

	\longline
	
	\subsubsection{Modello client/server}
	
	Il \textcolor{Red3}{\textbf{modello client/server}} è quello più utilizzato e funziona nel seguente modo (attenzione all'ordine!):
	\begin{enumerate}
		\item Il \emph{client} esegue una \textbf{richiesta} inviando dei dati al \emph{server};
		
		\item Il \emph{server} riceve i dati del \emph{client}, processa i dati e invia la \textbf{risposta} al \emph{client}. Infine, si mette in attesa di altre richieste.
	\end{enumerate}
	I dati inviati dal \emph{client} possono essere delle trasmissioni di dati o delle richieste di dati. In ogni caso, \textbf{il ruolo è determinato dall'ordine dei messaggi e non dal contenuto}. Si noti che il \emph{client} e \emph{server} \textbf{sono processi e \underline{non host}}. Infatti, l'insieme di un client e un server costituisce l'applicazione di rete.\newline
	
	\noindent
	Un \textcolor{Green4}{\textbf{esempio}} di applicazione client/server è il \textbf{sensore di temperatura corporea} che funge da client e invia al server una temperatura. Le risposte del server sono \dquotes{OK} per confermare l'avvenuta ricezione.\newline
	
	\noindent
	Un altro \textcolor{Green4}{\textbf{esempio}} di applicazione client/server è il display che funge da cliente e chiede al server una temperatura. Le risposte del server in questo caso saranno i dati richiesti.
	
	\longline
	
	\subsection{Creazione dell'interfaccia Socket}
	
	Il programma, prima di utilizzare la rete, deve essere in grado di creare un oggetto di tipo \textbf{socket}. Esso è identificato principalmente da tre parametri:
	\begin{itemize}
		\item Indirizzo IP locale;
		
		\item Porta locale, la quale è un intero senza segno di 16 bit (quindi da 0 a 65'535). Nel modello client/server:
		\begin{itemize}
			\item Il \textbf{server} deve decidere esplicitamente il numero di porta affinché i client possano saperlo (da 0 a 1023 le porte sono chiamate \dquotes{porte note} perché sono utilizzate per protocolli famosi come HTTP);
			
			\item Il \textbf{client} può decidere se scegliere il numero di porta esplicitamente oppure delegare la scelta al sistema operativo.
		\end{itemize}

		\item Modalità di trasmissione: UDP o TCP.
	\end{itemize}\newpage
	
	\subsection[\textcolor{Green4}{\textbf{Esempi}} di codice]{Esempi di codice}
	
	\subsubsection{Esecuzione degli esempi}
	
	\begin{enumerate}
		\item Aprire due terminali
		
		\item Compilare il server e successivamente eseguirlo con il comando:
		\begin{lstlisting}[language=C]
			-gcc network.c serverUDP.c -o serverUDP -lpthread\end{lstlisting}
		
		\item Compilare il client e successivamente eseguirlo con il comando:
		\begin{lstlisting}[language=C]
			-gcc network.c clientUDP.c -o clientUDP -lpthread\end{lstlisting}
	\end{enumerate}

	\longline
	
	\subsubsection{Client UDP}\label{client UDP}
	
	Il client UDP ha il seguente codice:
	\lstinputlisting[language=C]{code/clientUDP.c}
	\begin{itemize}
		\item (4-8) dichiarazione delle variabili tra cui la variabile socket;
	
		\item (10) inizializzazione dell'interfaccia socket sulla porta 20'000;
	
		\item (13) invio, tramite UDP, il messaggio \dquotes{Ciao sono il client!} al destinatario avente indirizzo IP \dquotes{127.0.0.1} (\emph{localhost}) e porta 35'000;
	
		\item (14) attesta dell'arrivo della risposta del server;
		
		\item (15-16) scrittura sul terminale della porta, dell'indirizzo del mittente e del messaggio.
	\end{itemize}\newpage
	
	\subsubsection{Server UDP}\label{server UDP}
	
	Il server UDP ha il seguente codice:
	\lstinputlisting[language=C]{code/serverUDP.c}
	\begin{itemize}
		\item (4-8) dichiarazione delle variabili tra cui la variabile socket;
		
		\item (10) inizializzazione dell'interfaccia socket sulla porta 35'000;
		
		\item (12) attesta dell'arrivo di qualche messaggio da parte di qualche client;
		
		\item (13-14) ricezione di un messaggio da parte di un client e stampa sul terminale dell'indirizzo, della porta e del messaggio del mittente;
		
		\item (15) invio della risposta del server al client.
	\end{itemize}\newpage

	\subsubsection{Client\_inc UDP}\label{client_inc UDP}
	
	Il client UDP (paragrafo~\ref{client UDP}) può essere riscritto nel seguente modo:
	\lstinputlisting[language=C]{code/clientUDP_inc.c}
	\begin{itemize}
		\item (4-8) dichiarazione delle variabili tra cui la variabile socket;
		
		\item (10) inizializzazione dell'interfaccia socket sulla porta 20'000;
		
		\item (11-12) inserimento di un numero intero da parte dell'utente;
		
		\item (13) invio, tramite UDP, del numero inserito dall'utente al destinatario avente indirizzo IP \dquotes{127.0.0.1} (\emph{localhost}) e porta 35'000;
		
		\item (14) attesta dell'arrivo della risposta del server;
		
		\item (15-16) scrittura sul terminale della porta, dell'indirizzo del mittente e del messaggio.
	\end{itemize}\newpage

	\subsubsection{Server\_inc UDP}\label{server_inc UDP}
	
	Il server UDP (paragrafo~\ref{server UDP}) può essere riscritto nel seguente modo:
	\lstinputlisting[language=C]{code/serverUDP_inc.c}
	\begin{itemize}
		\item (4-8) dichiarazione delle variabili tra cui la variabile socket;
		
		\item (10) inizializzazione dell'interfaccia socket sulla porta 35'000;
		
		\item (12) attesta dell'arrivo di qualche messaggio da parte di qualche client;
		
		\item (13-14) ricezione di un messaggio da parte di un client e stampa sul terminale dell'indirizzo, della porta e del messaggio del mittente;
		
		\item (15) incremento di uno del valore intero ottenuto;
		
		\item (15) invio della risposta del server al client con il valore incrementato.
	\end{itemize}\newpage

	\subsubsection{Client TCP}\label{client TCP}
	
	Il client TCP ha il seguente codice:
	\lstinputlisting[language=C]{code/clientTCP.c}
	\begin{itemize}
		\item (4-5) dichiarazione delle variabili tra cui la variabile \textsf{connection} per gestire la connessione;
		
		\item (7-8) creazione di una connessione TCP con il server utilizzando \emph{localhost} e la porta 35'000. 
		
		\item (9-10) controllo del valore della connessione per verificare se c'è stato un errore. In tal caso, la connessione termina con la stampa dell'errore su terminale;
		
		\item (12-15) in caso di connessione riuscita, il client richiede l'inserimento di un valore intero all'utente;
		
		\item (16-17) invio del valore intero al server;
		
		\item (18) attesa di una risposta da parte del server;
		
		\item (19-20) al momento della ricezione della risposta, il client stampa la risposta del server e chiude la connessione.
	\end{itemize}\newpage
	
	\subsubsection{Server TCP}\label{server TCP}
	
	Il server TCP ha il seguente codice:
	\lstinputlisting[language=C]{code/serverTCP.c}
	\begin{itemize}
		\item (4-6) dichiarazione delle variabili tra cui la variabile \textsf{connection} per gestire la connessione e \textsf{socket} per gestire i dati;
		
		\item (8) inizializzazione di un socket TCP utilizzando la porta 35'000.
		
		\item (9-10) controllo del valore del socket per verificare se c'è stato un errore. In tal caso, la creazione termina con la stampa dell'errore su terminale;
		
		\item (12-14) in caso di creazione del socket riuscita, il server attende la connessione da parte di qualche client;
		
		\item (15-17) attesa di una richiesta da parte di qualche client. Nel momento in cui viene ricevuta una richiesta, il server la accetta e instaura la connessione e attende la ricezione dei dati;
		
		\item (18-19) all'arrivo dei dati da parte del client, il server esegue un incremento di uno del valore ricevuto dal client;
		
		\item (20-22) il server invia il valore al client e infine chiude la connessione.
	\end{itemize}\newpage

	\subsubsection{Codice per la copia di un file}
	
	Il codice per la copia di un file è strutturato nel seguente modo:
	\lstinputlisting[language=C]{code/fileCopy.c}
	\begin{itemize}
		\item (6-29) dichiarazione dei puntatori ai file e tentativi di apertura dei due file richiesti dall'utente;
		
		\item (32-37) viene eseguita la lettura dal primo file e salvata nella variabile \textsf{c}. A questo punto, finché viene letto un carattere valido, ovvero che non sia la fine del file (\emph{End Of File}, EOF), il contenuto della variabile \textsf{c} viene inserito nel secondo file;

		\item (39-43) al termine del processo di copia, viene stampato il file nel quale sono stati copiati i valori e chiusi i rispettivi file descriptors.
	\end{itemize}\newpage
	
	\subsection[\textcolor{Red3}{\textbf{Esercizi}}]{Esercizi}
	
	\subsubsection{Esercizio 1 - UDP}
	
	Lanciare prima il server e poi il client. Cosa si osserva? Invertire la sequenza di lancio. Cosa si osserva?\newline
	
	\noindent
	\textcolor{Green4}{\textbf{\emph{\underline{Soluzione}}}}\newline
	
	\noindent
	Lanciando il server e successivamente il client, il primo attende la connessione da parte di qualcuno. Quindi, una volta avviato il client, le due parti inizieranno a comunicare.
	
	Invece, avviando prima il client e successivamente il server, le due parti non riescono a comunicare. Questo perché il client tenta di raggiungere un host non esistente.
	
	\longline
	
	\subsubsection{Esercizio 2 - UDP}
	
	Modificare i sorgenti per consentire al server di ricevere sulla porta 10000 e il client di trasmettere sulla propria porta 30000 (ogni modifica dei sorgenti richiede una loro ricompilazione).\newline
	
	\noindent
	\textcolor{Green4}{\textbf{\emph{\underline{Soluzione}}}}\newline
	
	\noindent
	Le modifiche da effettuare sono banali, ecco qua di seguito i codici del client (spiegazione al paragrafo~\ref{client UDP}):
	\lstinputlisting[language=C]{code/clientUDP_mod2.c}\newpage
	
	\noindent
	E del server (spiegazione al paragrafo~\ref{server UDP}):
	\lstinputlisting[language=C]{code/serverUDP_mod2.c}
	
	\longline
	
	\subsubsection{Esercizio 3 - UDP}
	
	Mettere il server in ascolto sulla porta 100 e osservare cosa succede:
	\begin{itemize}
		\item Bisogna modificare anche il client? Se sì, dove?
		\item Per chi usa il proprio PC con Linux o una virtual machine Linux, lanciare il server con il comando \dquotes{\textsf{sudo ./serverUDP}} e osservare cosa cambia.
	\end{itemize}

	\noindent
	\textcolor{Green4}{\textbf{\emph{\underline{Soluzione}}}}\newline
	
	\noindent
	Modificando il codice e inserendo il numero di porta 100, il server non riesce ad essere eseguito per un problema della porta. Infatti, la porta 100 fa parte delle \emph{well-know port} e non può essere utilizzata per altri scopi.\newline
	
	\noindent
	Modificando anche il client e inviando il messaggio al localhost con porta 100, il codice viene compilato ed eseguito correttamente. Ovviamente il client rimane in attesa del server.\newline
	
	\noindent
	Compilando il server con la modalità \textsf{sudo} è possibile forzare la modifica della porta 100 e mettersi in ascolto su tale porta. Di conseguenza, il server si metterà in ascolto sulla porta 100 e il client che eseguirà l'invio di un messaggio in tale porta, riuscirà a trasmettere il messaggio.\newpage
	
	\subsubsection{Esercizio 4 - UDP}
	
	Sostituire \dquotes{127.0.0.1} (o la stringa \dquotes{localhost}) con localhost (o al contrario con 127.0.0.1) e poi con \dquotes{pippo} e osservare cosa succede.\newline
	
	\noindent
	\textcolor{Green4}{\textbf{\emph{\underline{Soluzione}}}}\newline
	
	\noindent
	Modificando il parametro della chiamata a funzione \textsf{UDPSend} nel client viene modificato l'indirizzo del destinatario:
	\begin{itemize}
		\item Inserendo \dquotes{127.0.0.1} si sta inserendo l'indirizzo privato creato per identificare l'host stesso;

		\item Inserendo \textsf{localhost} si sta utilizzando un \emph{alias}, dunque è come scrivere l'indirizzo \dquotes{127.0.0.1};

		\item Inserendo \dquotes{Pippo} si sta provando a identificare l'alias Pippo a qualche indirizzo. Purtroppo non esiste nessun alias all'interno del sistema operativo con tale valore, tuttavia su Linux (forse anche su Windows) è possibile creare alias di rete con il relativo indirizzo IP.
	\end{itemize}

	\longline
	
	\subsubsection{Esercizio 5 - UDP}
	
	(Da fare solo se in Lab Delta) Accordarsi per lavorare su coppie di macchine in modo che server e client siano su macchine diverse. Come bisogna modificare i sorgenti?\newline
	
	\noindent
	\textcolor{Green4}{\textbf{\emph{\underline{Soluzione}}}}\newline
	
	\noindent
	La modifica è alquanto semplice. Supponendo che i due host siano connessi sulla stessa rete, quindi non per forza la rete universitaria ma va bene anche un hotspot tramite telefono, è necessario modificare nel seguente modo i codici:
	\begin{itemize}
		\item Server: non apportare nessuna modifica in quanto il server (destinatario) deve solo aprire una porta, nel caso d'esempio la 35000;
		
		\item Client: indipendentemente dalla porta aperta nel client, di default nell'esempio la 20000, il client necessita di una modifica nei parametri della chiamata a funzione \textsf{UDPSend}. In particolare, al posto di \textsf{localhost} o dell'indirizzo \dquotes{127.0.0.1}, basterà inserire l'indirizzo IP del server che ha all'interno della rete. È doveroso cambiare anche il numero di porta nel caso in cui sia stata cambiata nel server.
		
		Attenzione, nelle virtual machine non è così semplice la faccenda. Infatti esse utilizzano un bridge per collegarsi alla scheda di rete e di conseguenza l'IP che viene visualizzato non è \dquotes{reale}. Si consiglia dunque un sistema operativo linux (o windows) non virtualizzato.
	\end{itemize}\newpage
	
	\subsubsection{Esercizio 6 - UDP}
	
	Modificare il server in maniera che soddisfi 5 richieste prima di terminare.
	\begin{itemize}
		\item E se volessi che non terminasse mai?
	\end{itemize}

	\noindent
	\textcolor{Green4}{\textbf{\emph{\underline{Soluzione}}}}\newline
	
	\noindent
	Per soddisfare almeno 5 richieste, è necessario modificare il codice del server di modo che esegua la parte di codice della ricezione almeno 5 volte. Quindi il relativo codice del server sarà:
	\lstinputlisting[language=C]{code/serverUDP_mod3.c}
	Mentre quello del client non viene modificato.
	
	\noindent
	La simulazione dell'esecuzione sarà la seguente (prossima pagina):
	\begin{figure}[!htp]
		\centering
		\includegraphics[width=\textwidth]{img/soluzioni_TCP-UDP/TCP-UDP_1.png}
		\caption{Terminale client.}
		\:\newline
		\includegraphics[width=\textwidth]{img/soluzioni_TCP-UDP/TCP-UDP_2.png}
		\caption{Terminale server.}
	\end{figure}\newpage

	\subsubsection{Esercizio 7 - UDP}
	
	Compilare ed eseguire il secondo esempio.\newline
	
	\noindent
	\textcolor{Green4}{\textbf{\emph{\underline{Soluzione}}}}\newline
	
	\noindent
	Si esegue il secondo esempio che riguarda il clientUDP e serverUDP in versione \_inc, rispettivamente paragrafo~\ref{client_inc UDP} e \ref{server_inc UDP}.
	
	\longline
	
	\subsubsection{Esercizio 8 - Sommatrice UDP}\label{esercizio 8 - Sommatrice UDP}
	
	Modificare il codice in modo tale da costruire una semplice sommatrice:
	\begin{itemize}
		\item Il client acquisisce ripetutamente da tastiera un numero intero e lo invia al server finché l'utente digita zero;
		
		\item Il server accumula in una variabile \dquotes{somma} i valori mandati dal client finché il client manda zero;
		
		\item Quando il client manda zero il server risponde al client con la somma ottenuta.
	\end{itemize}

	\noindent
	\textcolor{Green4}{\textbf{\emph{\underline{Soluzione}}}}\newline
	
	\noindent
	Il codice del client:
	\lstinputlisting[language=C]{code/clientUDP_inc_mod2.c}
	\begin{itemize}
		\item (11-16) Viene richiesto l'inserimento di un numero intero all'utente finché tale numero non è diverso da zero. Ogni numero viene inviato al server (15).
		
		\item (17-19) Nel momento in cui il numero inserito è zero, il programma termina aspettando il risultato che verrà inviato dal server. Alla ricezione, verrà stampato.
	\end{itemize}\newpage
	
	\noindent
	\begin{figure}[!htp]
		\centering
		\includegraphics[width=\textwidth]{img/soluzioni_TCP-UDP/TCP-UDP_3.png}
		\caption{Esempio di esecuzione del client.}
	\end{figure}

	\noindent
	Il codice del server:
	\lstinputlisting[language=C]{code/serverUDP_inc_mod2.c}
	\begin{itemize}
		\item (12-19) Viene dichiarata la variabile somma, come richiesto dall'esercizio. Successivamente, il server attende che qualche client gli invii un valore intero. Una volta arrivato tale informazione, il server somma il valore ad una sua variabile locale. Il ciclo continua finché non riceve un valore pari a zero.
		
		\item (20-22) Quando viene ricevuto un valore pari a zero, il server invia la somma effettuata al client e stampa chi è il client che ha richiesto la terminazione.
	\end{itemize}\newpage

	\noindent
	\begin{figure}[!htp]
		\centering
		\includegraphics[width=\textwidth]{img/soluzioni_TCP-UDP/TCP-UDP_4.png}
		\caption{Esempio di esecuzione del server.}
	\end{figure}\newpage

	\subsubsection{Esercizio 9 - Sommatrice UDP e perdita di pacchetti}\label{esercizio 9 - Sommatrice UDP e perdita di pacchetti}
	
	Usare la sommatrice su due macchine distinte provando, sulla macchina del client, a staccare il cavo di rete prima di un invio di un dato, ad esempio:
	\begin{itemize}
		\item Digitare \dquotes{2345} + INVIO
		\item Digitare \dquotes{5187} + INVIO
		\item \textbf{Staccare il cavo}
		\item Digitare \dquotes{2} + INVIO
		\item \textbf{Riattaccare il cavo e aspettare 30 sec che il sistema operativo si riassesti}
		\item Digitare \dquotes{1} + INVIO
		\item \dquotes{0}
	\end{itemize}
	Che somma leggo? È corretta?\newline
	
	\noindent
	\textcolor{Green4}{\textbf{\emph{\underline{Soluzione}}}}\newline
	
	\noindent
	Si parte con il rispondere prima alla domanda e poi a fornire la motivazione. La somma letta \underline{non} è corretta. La motivazione è la seguente:
	\begin{itemize}
		\item Il client invia il valore 2345 al server. Quest'ultimo riceve correttamente il valore. Somma progressiva: $2345$;
		
		\item Il client invia il valore 5187 al server. Quest'ultimo riceve correttamente il valore. Somma progressiva: $2345 + 5187 = 7532$;
		
		\item Viene staccato il cavo;
		
		\item Inserimento del valore 2 all'interno del client. Quest'ultimo invia il pacchetto al localhost, il quale non è collegato alla rete, di conseguenza il pacchetto viene perso. Dato che il protocollo è UDP, il client non sa se il pacchetto è stato ricevuto dal destinatario oppure no, di conseguenza ricomincia il ciclo e richiede un numero intero. La somma nel server rimane $7532$, mentre la somma corretta dovrebbe essere $7532+2=7534$;
		
		\item Viene riattaccato il cavo;
		
		\item Il client invia il valore 1 al server. Quest'ultimo riceve correttamente il valore. Somma progressiva: $7532+1 = 7533$;
		
		\item Valore zero, il client e il server si fermano.
	\end{itemize}\newpage

	\subsubsection{Esercizio 10 - Sommatrice UDP e influenze reciproche}\label{esercizio 10 - Sommatrice UDP e influenze reciproche}
	
	Invocare il server della sommatrice con due client diversi (tutti e tre possono anche essere sulla stessa macchina ovviamente su finestre terminali diverse), ad esempio:
	\begin{table}[!htbp]
		\centering
		\begin{tabular}{@{} l | l @{}}
			\toprule
			\textbf{Client A} & \textbf{Client B} \\
			\midrule
			Digitare \dquotes{2345} + INVIO	& Digitare \dquotes{2} + INVIO 	\\
			Digitare \dquotes{5187} + INVIO	& Digitare \dquotes{8} + INVIO 	\\
			Digitare \dquotes{2} + INVIO	& \dquotes{0} 					\\
			Digitare \dquotes{1} + INVIO	& 								\\
			\dquotes{0}						&								\\
			\bottomrule
		\end{tabular}
	\end{table}
	
	\noindent
	Che somma leggo da ciascun client? È la somma che ciascun client si aspetterebbe?\newline
	
	\noindent
	\textcolor{Green4}{\textbf{\emph{\underline{Soluzione}}}}\newline
	
	\noindent
	Mantenendo lo schema dell'esercizio 8 (paragrafo~\ref{esercizio 8 - Sommatrice UDP}), l'esecuzione dei due client (A e B) come nello schema dell'esercizio:
	\begin{figure}[!htp]
		\centering
		\includegraphics[width=\textwidth]{img/soluzioni_TCP-UDP/TCP-UDP_5.png}
	\end{figure}
	
	\noindent
	Ricordando di eseguire prima un'operazione del client A, poi un'operazione del client B, poi A, poi B, e così via. Il risultato ottenuto sul server è il seguente:\newpage
	\begin{figure}[!htp]
		\centering
		\includegraphics[width=\textwidth]{img/soluzioni_TCP-UDP/TCP-UDP_6.png}
	\end{figure}

	\noindent
	Quindi, nel client B viene letta la somma corretta, ovvero quella eseguita prima che il client B inviasse il valore zero e richiedesse la terminazione del server. Al contrario, nel client A il valore inserito 1 non viene ricevuto dal server, ma dato che è un protocollo UDP, il client continua ad eseguire il codice. Il problema nel client A sorge nel momento in cui esce dal ciclo \textsf{do...while}, poiché deve attendere una risposta (il risultato) dal server. Tuttavia, dato che il client B ha inviato il valore di terminazione \dquotes{0} e di conseguenza ha cessato la sua esecuzione, il client A rimarrà in un stato di attesa infinito.\newline
	
	\noindent
	Quale potrebbe essere una \textbf{possibile modifica per evitare questa attesa infinita}? Ci sono molteplici soluzioni, una tra queste è quella di inserire un ciclo \textsf{do...while} al termine del codice del server e inserendo al suo interno un'attesa di ricezione con il conseguente invio del valore corretto. Il codice muterebbe in questo modo:
	\lstinputlisting[language=C]{code/serverUDP_inc_mod3.c}
	E l'esecuzione del client diventerebbe:
	\begin{figure}[!htp]
		\centering
		\includegraphics[width=\textwidth]{img/soluzioni_TCP-UDP/TCP-UDP_7.png}
	\end{figure}

	\noindent
	Con il server che rimarrebbe in attesa di essere terminato dal programmatore:
	\begin{figure}[!htp]
		\centering
		\includegraphics[width=\textwidth]{img/soluzioni_TCP-UDP/TCP-UDP_8.png}
	\end{figure}\newpage

	\subsubsection{Esercizio 11 - Sommatrice TCP}
	
	Scrivere la sommatrice (quella dell'esercizio 8 al paragrafo~\ref{esercizio 8 - Sommatrice UDP}) usando TCP, compilare ed eseguire.\newline
	
	\noindent
	\textcolor{Green4}{\textbf{\emph{\underline{Soluzione}}}}\newline
	
	\noindent
	Il codice del client:
	\lstinputlisting[language=C]{code/clientTCP.c}
	Un esempio di esecuzione del client:
	\begin{figure}[!htp]
		\centering
		\includegraphics[width=\textwidth]{img/soluzioni_TCP-UDP/TCP-UDP_9.png}
	\end{figure}\newpage

	\noindent
	Il codice del server:
	\lstinputlisting[language=C]{code/serverTCP.c}
	Un esempio di esecuzione del server:
	\begin{figure}[!htp]
		\centering
		\includegraphics[width=\textwidth]{img/soluzioni_TCP-UDP/TCP-UDP_10.png}
	\end{figure}\newpage

	\subsubsection{Esercizio 12 - Sommatrice TCP e influenze reciproche}
	
	Provare a rifare l'esercizio 10 (paragrafo~\ref{esercizio 10 - Sommatrice UDP e influenze reciproche}) ma con questa nuova versione della sommatrice.\newline
	Cosa si può osservare? Che soluzioni si può trovare? C'è influenza reciproca tra i due client?\newline
	
	\noindent
	\textcolor{Green4}{\textbf{\emph{\underline{Soluzione}}}}\newline
	
	\noindent
	In questo caso, non vi è influenza reciproca poiché il protocollo TCP è più restrittivo:
	\begin{figure}[!htp]
		\centering
		\includegraphics[width=\textwidth]{img/soluzioni_TCP-UDP/TCP-UDP_11.png}
	\end{figure}
	
	\noindent
	Il server ha la seguente esecuzione:
	\begin{figure}[!htp]
		\centering
		\includegraphics[width=\textwidth]{img/soluzioni_TCP-UDP/TCP-UDP_10.png}
	\end{figure}\newpage

	\subsubsection{Esercizio 13 - Sommatrice TCP e perdita di pacchetti}
	
	Riprovare l'esercizio 9 (paragrafo~\ref{esercizio 9 - Sommatrice UDP e perdita di pacchetti}) utilizzando questa volta la sommatrice TCP. Si analizzi il risultato.\newline
	
	\noindent
	\textcolor{Green4}{\textbf{\emph{\underline{Soluzione}}}}\newline
	
	\noindent
	Nonostante non sia possibile provare questo esercizio in Delta, è possibile formulare la risposta grazie alle conoscenze teoriche sul protocollo TCP.\newline
	
	\noindent
	Nel momento in cui vengono inviati i primi due valori (2345 e 5187), il server riceve correttamente il valore. Successivamente, avviene un down di rete, ovvero viene staccato il cavo. Il client tenta di inviare un pacchetto contenente il valore 2 ma fallisce. Per definizione del protocollo, entra in gioco l'RTO (\emph{Retransmission TimeOut}) che inizia ad aumentare e a riprovare l'invio finché non riesce. Alla fine del down di rete, il server riceve il pacchetto con il valore 2, successivamente riceve il pacchetto inviato dal client con il valore 1 e termina con zero.










	
	\newpage

	\section{Dal Web ai Webservices}
	
	\subsection{Protocollo HTTP/HTTPS}
	
	Il \textcolor{Red3}{\textbf{protocollo HTTP}} venne inventato per fruire dei contenuti in rete (\emph{World Wide Web}). Tuttavia, al giorno d'oggi viene usato per l'invocazione di funzionalità remote, tecnica chiamata \textbf{Webservice}.\newline
	
	\noindent
	Il protocollo si divide in più \textbf{fasi}:
	\begin{enumerate}
		\item Apertura di una connessione TCP;
		
		\item Nel caso del protocollo HTTPS, avviene l'autenticazione del server e negoziazione di una chiave di cifratura;
		
		\item Invio di messaggi di \emph{request} e \emph{response};
		
		\item Chiusura della connessione TCP.
	\end{enumerate}
	\begin{figure}[!htp]
		\centering
		\includegraphics[width=\textwidth]{img/protocollo_http-https.png}
		\caption{Esempio di scambio di messaggi nel protocollo HTTP.}
	\end{figure}\newpage
	
	\noindent
	Nel \textcolor{Red3}{\textbf{protocollo HTTPS}} i messaggi che passano nella connessione TCP, sono gli stessi del protocollo HTTP con l'aggiunta di una \textbf{cifratura dei dati in transito} e di una \textbf{autenticazione del server mediante certificato digitale}. Inoltre, il server lavora sulla porta 443 e non sulla porta classica 80.
	\begin{figure}[!htp]
		\centering
		\includegraphics[width=.4\textwidth]{img/protocollo_https.pdf}
		\caption{Protocollo HTTPS.}
	\end{figure}

	\noindent
	Un \textcolor{Green4}{\textbf{esempio}} di richiesta:
	\begin{figure}[!htp]
		\centering
		\includegraphics[width=\textwidth]{img/richiesta_http.pdf}
		\caption{Messaggio di richiesta.}
	\end{figure}

	\noindent
	Un \textcolor{Green4}{\textbf{esempio}} di risposta:
	\begin{figure}[!htp]
		\centering
		\includegraphics[width=\textwidth]{img/risposta_http.pdf}
		\caption{Messaggio di risposta.}
	\end{figure}\newpage

	\subsection{Hyper Text Markup Language (HTML) e Cascading Style Sheets (CSS)}
	
	\textcolor{Red3}{\textbf{HTML}} è un linguaggio testuale di descrizione di una pagina, in particolare è la specializzazione del generico XML (\emph{eXtensible Markup Language}). HTML si basa sui \dquotes{tag} annidati, i quali eventualmente contengono attributi.\newline
	
	\noindent
	Questo linguaggio, spesso viene utilizzato con un altro linguaggio chiamato \textbf{CSS}.\newline
	
	\noindent
	\textcolor{Green4}{\textbf{Esempi}} di codice HTML:
	\lstinputlisting[language=HTML]{code/css.html}
	E CSS:
	\lstinputlisting{code/styles.css}
	
	\longline
	
	\subsubsection{HTML: tag per richiamare immagini}
	
	Viene utilizzato \dquotes{\textsf{img src}} per richiamare le immagini:
	\lstinputlisting[language=HTML]{code/immagini.html}\newpage
	
	\subsubsection{HTML: tag per il collegamento ipertestuale}
	
	Viene utilizzato \dquotes{\textsf{href}} per il collegamento ipertestuale:
	\lstinputlisting[language=HTML]{code/link.html}
	
	\longline
	
	\subsubsection{Document Object Model (DOM)}
	
	Il \textcolor{Red3}{\textbf{Document Object Model (DOM)}} è una forma di rappresentazione dei documenti (pagina) strutturati come modello orientato agli oggetti.
	
	\longline
	
	\subsection{Javascript}
	
	\textcolor{Red3}{\textbf{Javascript}} è un linguaggio di programmazione multi paradigma orientato agli eventi, utilizzato sia nella programmazione lato client web che lato server. È facile trovarlo all'interno di codice HTML anche grazie al suo tag riconoscibile: \textsf{<script>}.
	
	Tuttavia, è difficile trovare del codice Javascript pure scritto nelle pagine HTML. Solitamente vengono create vere e proprie librerie così da rendere il codice più leggibile e mantenibile.\newline
	
	\noindent
	Un \textcolor{Green4}{\textbf{esempio}} di codice Javascript:
	\lstinputlisting[language=HTML]{code/javascript.html}\newpage
	
	\noindent
	Javascript trova il suo grande utilizzo con gli \textbf{eventi causati dall'utente}, per esempio con la pressione di un bottone all'interno di una pagina web:
	\lstinputlisting[language=HTML]{code/javascript-evento.html}
	
	\longline
	
	\subsubsection{Javascript e Document Object Model (DOM)}
	
	Il \textbf{\emph{Document Object Model} consente di trasformare una pagina web da documento statico a \emph{Graphical User Interface} (GUI)}, cioè interattivo.
	
	Infatti, grazie al codice Javascript contenuto nella pagina HTML ed eseguito dal browser, l'utente può modificare lo stato della pagina web a seconda di determinate azioni. Quindi, la pagina web si automodifica e assume le sembianze di una applicazione web, chiamata in gergo \emph{web application}.\newline
	
	\noindent
	Alcuni \textcolor{Green4}{\textbf{esempi}} di codice interattivo:
	\begin{itemize}
		\item Per avere un riquadro con la pagina web ANSA:
		\lstinputlisting[language=HTML]{code/javascript-load.html}\newpage
		
		\item Per avere un timer che alla fine del tempo stampa \dquotes{Hello}:
		\lstinputlisting[language=HTML]{code/javascript-timer.html}
		
		\item Per avere lo stesso effetto del punto precedente ma utilizzando una \textbf{funzione}:
		\lstinputlisting[language=HTML]{code/javascript-timer2.html}
	\end{itemize}\newpage
	
	\subsection{Uniform Resource Locator (URL)}
	
	Chiamato anche Universal Resource Locator, l'\textcolor{Red3}{\textbf{URL}} consente di identificare in maniera univoca una risorsa HTTP in qualsiasi parte della rete mondiale.\newline
	
	\noindent
	È \textbf{strutturato} in tre parti:
	\begin{itemize}
		\item Il protocollo utilizzato a livello di applicazione, di trasporto e la porta utilizzata, per esempio HTTP con protocollo TCP e porta 80;
		
		\item Nome/IP dell'host che eroga tale risorsa;
		
		\item Nome della risorsa con il percorso logico completo.
	\end{itemize}\newpage
	
	\subsection{Passare dei dati al server web col metodo GET}
	
	Il codice HTML per utilizzare il metodo GET è il seguente:
	\lstinputlisting[language=HTML]{code/form-get.html}
	La pagina web visualizzata è la seguente:
	\begin{figure}[!htp]
		\centering
		\includegraphics[width=\textwidth]{img/form-get.pdf}
	\end{figure}
	
	\noindent
	Una volta inserito nome e cognome, alla pressione del tasto, i dati saranno inviati al \emph{localhost} aggiungendo come parametri nell'URL \textsf{fname=name} e \textsf{lname=name}, dove al posto di name vengono inseriti nome e cognome. Il link risulta: \url{http://127.0.0.1/action?fname=John&lname=Doe}.\newpage
	
	\begin{figure}[!htp]
		\centering
		\includegraphics[width=\textwidth]{img/richiesta_get.pdf}
		\caption{La richiesta GET HTTP.}
	\end{figure}\newpage

	\subsection{Passare dei dati al server web col metodo POST}
	
	Il codice HTML per utilizzare il metodo POST è il seguente:
	\lstinputlisting[language=HTML]{code/form-post.html}
	La pagina web visualizzata è la seguente:
	\begin{figure}[!htp]
		\centering
		\includegraphics[width=\textwidth]{img/form-post.pdf}
	\end{figure}
	
	\noindent
	Una volta inserito nome e cognome, alla pressione del tasto, i dati saranno inviati al \emph{localhost}. \textbf{A differenza del metodo GET}, i parametri non vengono specificati nell'URL, di conseguenza la sicurezza aumenta. Il link dunque risulta: \url{http://127.0.0.1/action}.\newline
	
	\noindent
	I valori vengono inseriti all'interno della richiesta HTTP.\newpage
	
	\begin{figure}[!htp]
		\centering
		\includegraphics[width=\textwidth]{img/richiesta_post.pdf}
		\caption{La richiesta POST HTTP.}
	\end{figure}\newpage

	\subsection{Common Gateway Interface (CGI)}
	
	Il \textcolor{Red3}{\textbf{Common Gateway Interface (CGI)}} è una tecnologia utilizzata dai \emph{web server} per interfacciarsi con applicazioni esterne generando contenuti web dinamici.\newline
	
	\noindent
	Ogni qualvolta che un \emph{client} richiede al web server un URL corrispondente a un documento HTML, gi viene restituito un documento statico. Al contrario, se l'URL corrisponde a un programma CGI, il server lo esegue in tempo reale, generando dinamicamente informazioni per l'utente. Sostanzialmente è l'\textbf{esecuzione di un determinato programma sul server}.
	
	Di conseguenza, il browser diventa il \emph{client} di molte applicazioni di rete, per esempio la posta elettronica.
	
	\begin{figure}[!htp]
		\centering
		\includegraphics[width=\textwidth]{img/CGI.pdf}
		\caption{Fasi del CGI.}
	\end{figure}

	\noindent
	Degli \textcolor{Green4}{\textbf{esempi}} di esecuzione lato server di un programma sono un eseguibile come il client della posta elettronica, oppure un codice PHP, Java, NodeJS, ecc.\newpage
	
	\begin{figure}[!htp]
		\centering
		\includegraphics[width=\textwidth]{img/CGI_esempio.pdf}
		\caption{Esempio di CGI, la classica posta elettronica.}
	\end{figure}\newpage

	\subsection{Web socket}
	
	La \textcolor{Red3}{\textbf{web socket}} è una tecnologia web che fornisce canali di comunicazione chiamati \emph{full-duplex}, cioè bidirezionali, attraverso una singola connessione TCP. Viene utilizzato principalmente per realizzare applicazioni che forniscono contenuti e giochi in tempo reale. Questo perché \textbf{il protocollo consente maggiore interazione tra browser e server} grazie al alcune caratteristiche.\newline
	
	\noindent
	Innanzitutto è un \textbf{protocollo a livello di applicazione}, per cui è un \textbf{metodo alternativo a HTTP e HTTPS}. Ha la caratteristica fondamentale di \textbf{comunicazione simmetrica} tra \emph{browser} e \emph{web server}, ovverosia che \textbf{i processi possono \dquotes{prendere l'iniziativa} e inviare dei dati alla controparte}. Inoltre, nasce da una sessione HTTP/HTTPS attraverso un'operazione chiamata \textbf{Protocol Upgrade}.
	\begin{figure}[!htp]
		\centering
		\includegraphics[width=.5\textwidth]{img/web_socket.pdf}
		\caption{Esempio di web socket e Protocol Upgrade.}
	\end{figure}
	
	\noindent
	Nel messaggio HTTP, ci sono due campi che vengono modificati per indicare il Protocol Upgrade: \textsf{Upgrade} e \textsf{Connection}. Entrambi i campi vengono modificati con i rispettivi valori \textsf{websocket} e \textsf{Upgrade}.\newpage
	
	\noindent
	\begin{figure}[!htp]
		\centering
		\includegraphics[width=\textwidth]{img/protocol_upgrade.pdf}
		\caption{Il Protocol Upgrade nel dettaglio.}
	\end{figure}\newpage

	\subsubsection{Approfondimento WebSocket}
	
	Il \textbf{WebSocket} è un protocollo di comunicazione web che fornisce un canale di \textbf{comunicazione bidirezionale} attraverso una singola connessione TCP inizialmente utilizzata per il protocollo HTTP.
	
	Il protocollo consente \textbf{maggiore interazione tra browser e server}, facilitando inoltre la realizzazione di applicazioni web che devono fornire contenuti in tempo reale. Tutto questo è possibile poiché i \textbf{WebSocket concedono al server di \dquotes{prendere l'iniziativa} ed effettuare dei \emph{push} autonomi di dati verso il browser per aggiornarlo}. Ovviamente questo non è possibile con il classico protocollo HTTP. Per \textbf{\emph{comunicazione bidirezionale}} si intende una comunicazione in entrambe le direzioni \underline{simultaneamente}.\newline
	
	\noindent
	I WebSocket sono basati sul protocollo TCP e nascono da una connessione HTTP grazie ad un \textbf{Upgrade Request} richiesto dal client al server. Il browser comunica questa richiesta speciale al server tramite alcune voci nell'intestazione del messaggio. Inoltre, il WebSocket consente connessioni per un lungo periodo.\newline
	
	\noindent
	In sintesi, le \textbf{\underline{caratteristiche}} di questo protocollo:
	\begin{itemize}
		\item Si \textbf{basa sul protocollo TCP}, la quale inizialmente utilizza il protocollo HTTP, ma grazie alla richiesta speciale Upgrade Request, essa muta nel protocollo WebSocket;
		
		\item \textbf{Connessione bidirezionale} (\emph{full-duplex}), quindi aumento della facilità nella realizzazione di applicazioni web che forniscono contenuti in tempo reale;
		
		\item \textbf{Utilizzo di porte note} (\emph{Well-know ports}), in particolare quelle dedicate al protocollo HTTP, quindi la 80 e la 443;
		
		\item \textbf{Connessioni per un lungo periodo}.
	\end{itemize}

	\longline
	
	\subsubsection{Limitazioni}
	
	Nonostante la grande utilità che apporta il protocollo WebSocket, esso non è la soluzione a tutto. Infatti, \textbf{HTTP ha ancora un ruolo chiave nella comunicazione client-server} per vari motivi:
	\begin{itemize}
		\item \textbf{Invio e chiusura delle connessioni per trasferimenti di dati di tipo one-time}, come i caricamenti iniziali. Il protocollo HTTP è più efficiente del WebSocket;
		
		\item \textbf{Utilizzo più intelligente} da parte di HTTP delle \textbf{risorse} grazie alla chiusura delle connessioni una volta terminate le operazioni. Al contrario, il WebSocket mantiene una connessione attiva più a lungo rischiando di sprecare risorse inutilmente;
		
		\item \textbf{WebSocket riservato solo agli utenti con JavaScript abilitato} e quindi a coloro che posseggono browser moderni a discapito, per esempio, dei sistemi \emph{embedded}.
	\end{itemize}\newpage

	\subsection{WebSocket-Chat}
	
	WebSocket-Chat è un progetto \dquotes{giocattolo} \textbf{realizzato per comprendere al meglio la tecnologia offerta dal protocollo WebSocket}. Esso utilizza vari linguaggi di programmazione: JavaScript, Node.js (\href{https://it.wikipedia.org/wiki/Node.js}{framework per realizzare applicazioni Web in JavaScript}), HTML, CSS e una parte aggiuntiva chiamata Console con Ispezione Network (monitoraggio da parte del browser con scambio tra i pacchetti).
	
	\longline
	
	\subsubsection{Node.js e l'approccio asincrono}
	
	Il framework \textcolor{Red3}{\textbf{Node.js}} (\href{https://it.wikipedia.org/wiki/Node.js}{Wikipedia}, \href{https://nodejs.org/it}{sito ufficiale Node.js}) è nato per realizzare applicazioni Web in JavaScript. Solitamente viene utilizzato lato client (\emph{client-side}) per realizzare applicazioni tipicamente lato server (\emph{server-side}).\newline
	
	\noindent
	La \textbf{caratteristica principale} di Node.js è la possibilità di accedere alle risorse del sistema operativo in modalità \emph{event-driven} (\href{https://it.wikipedia.org/wiki/Programmazione_a_eventi}{programmazione orientata agli eventi}) e \underline{non} sfruttando il classico modello basato su processi/thread concorrenti, utilizzato dai classici web server.\newline
	
	\noindent
	Il modello \textbf{\emph{event-driven}}, tradotto in programmazione orientata agli eventi, si basa sulla \textbf{mutazione di stato nel momento in cui si manifesta un evento}.
	
	A differenza della programmazione procedurale (\emph{C-style}) in cui ogni azione viene eseguita una dopo l'altra con un determinato ordine, nella programmazione ad eventi le azioni sono asincrone e seguono un ordine dettato dalla manifestazione degli eventi.\newline
	
	\noindent
	L'\textbf{approccio asincrono} comporta una \textcolor{Green4}{\textbf{grande efficienza}} soprattutto in ambito di \emph{networking} poiché capita spesso di effettuare richieste e di rimanere in attesa di un'eventuale risposta. Grazie all'approccio asincrono, \textbf{durante l'attesa possibile effettuare altre operazioni che non dipendono dalla richiesta effettuata}.
	
	\longline
	
	\subsubsection{Descrizione dell'applicazione}
	
	Il progetto mira ad avere una chat multiutente a cui collegarsi tramite browser. Le \emph{features} implementate sono le seguenti:
	\begin{itemize}
		\item Registrazione del nome di contatto che si vuole avere quando si accede alla chat.
		
		\item Invio dei messaggi in broadcast a tutti gli utenti attualmente collegati alla chat.
	
		\item Visualizzazione dei messaggi inviati col nome della persona che lo ha inviato.
	\end{itemize}\newpage

	\subsubsection{Codice Back-end server}\label{codice Back-end server}
	
	\lstinputlisting[language=JavaScript]{code/server.js}
	
	\begin{itemize}
		\item (3) \textcolor{Red3}{\textbf{Express.js}} è una libreria di Node.js che consente di costruire applicazioni web molto facilmente (\href{https://en.wikipedia.org/wiki/Express.js}{Wikipedia}, \href{https://expressjs.com/}{sito ufficiale}). L'unica cosa importante da sapere è che la prima linea di codice crea una variabile \textsf{express}, che necessita della relativa libreria Express.js, e viene \textbf{utilizzata per creare il server web in ascolto sulla porta 4000}.
		
		\item (4) \textcolor{Red3}{\textbf{Socket.IO}} è una libreria JavaScript \textbf{utilizzata per implementare il protocollo WebSocket} e racchiude molte \textbf{funzioni} tra le quali:
		\begin{itemize}
			\item \textbf{Broadcasting} a tutti i socket collegati;
			
			\item \textbf{Salvataggio} dei dati riguardanti ciascun utente;
			
			\item Approccio \textbf{asincrono di I/O}.
		\end{itemize}
	
		\item (6-13) Alla riga 7 viene effettuato il vero e proprio \emph{import} della libreria e viene creato il socket;
		
		Alla riga 11 il server si mette in ascolto, sulla porta 4000, e (riga 12) stampa sul terminale la stringa \dquotes{waiting for HTTP requests on port 4000,}.
		
		
		\item (21) Una volta ricevuta una connessione da parte di un client, il server ricerca all'interno della cartella chiamata \textsf{public}, il relativo file \dquotes{.html} da inviare al mittente.
		
		\item (29) Alla conferma di connessione instaurata e di file ricevuto, il server prende l'iniziativa ed esegue un \textbf{Upgrade Request} trasformando la connessione in una WebSocket.
		
		\item (30-38) Il server rimane in attesa, in particolare questo accade alla linea 35. Nel momento in cui un client scatena un evento, ovvero invia un messaggio con il tag \textsf{message}, il server lo inoltrerà a tutti i client connessi mediante l'oggetto \textsf{sockets}.
	\end{itemize}

	\longline
	
	\subsubsection{Codice Front-end HTML}
	
	\lstinputlisting[language=HTML]{code/index.html}
	Il codice HTML consente di far visualizzare l'interfaccia utente creata da JavaScript. In particolare, nella pagina è possibile leggere e inviare messaggi.\newline
	
	\noindent
	L'unica osservazione da fare è il tag \textsf{<script>}, il quale contiene il link del file .js da eseguire all'apertura del file HTML (paragrafo~\ref{codice Front-end JavaScript}). Ovviamente si intende il codice a riga 21, ovvero quello necessario per la chat.\newpage
	
	\subsubsection{Codice Front-end JavaScript}\label{codice Front-end JavaScript}
	
	\lstinputlisting[language=JavaScript]{code/chat.js}
	\begin{itemize}
		\item (3-6) Finché non viene inserito un nome, il codice continua a chiederlo.
		
		\item (9-13) Vengono inizializzate le variabili che acquisiscono i tag presenti nella pagina HTML.
		
		\item (15-16) Viene scritto il nome dell'utente nella pagina web e impostato il valore.
		
		\item (19) Viene creato il socket che deve connettersi al server.
		
		\item (22-30) Sul bottone di invio del messaggio, viene aggiunto un nuovo evento JavaScript. Quest'ultimo si attiverà nel momento in cui l'utente cliccherà sul bottone. Una volta premuto, se il messaggio non sarà vuoto, verrà inviato al server (con tag \textsf{message}!) il messaggio (riga 25) e il nome dell'utente (riga 26). Una volta inviato, viene svuotato il valore del messaggio.
		
		\item (32-35) Sono la stampa del messaggio ricevuto. Si noti il segno \dquotes{+=} che indica che i vari messaggi ricevuti vengono concatenati.
	\end{itemize}\newpage

	\subsubsection{Esecuzione del Front-end e del Back-end}
	
	Il server web utilizzato è Node.js che consente di eseguire codice JavaScript \emph{server-side} per creare il Back-end. Invece, il Front-end è realizzato mediante codice JavaScript eseguito dentro il browser.\newline
	
	\noindent
	Passaggi da eseguire su Windows:
	\begin{enumerate}
		\item Download del file di installazione dal sito ufficiale: \url{https://nodejs.org/it/download}
		
		\item Eseguire il Back-end:
		\begin{enumerate}
			\item Scaricare lo zip fornito su Moodle con nome \dquotes{WebSocket\_Chat};
			\item Aprire un terminale e posizionarsi nella cartella contente il file \textsf{server.js};
			\item Eseguire il comando: \textsf{node.exe server.js}.
		\end{enumerate}
		
		\item Eseguire il Front-end:
		\begin{enumerate}
			\item Aprire il browser alla pagina: \url{http://localhost:4000};
			\item (Opzionale) Aprire più finestre del browser così da simulare l'accesso da parte di più utenti.
		\end{enumerate}
	\end{enumerate}\newpage
	
	\subsubsection[\textcolor{Red3}{\textbf{Esercizi}}]{Esercizi}
	
	\noindent	
	\textcolor{Red3}{\textbf{\emph{\underline{Esercizio 1}}}}\newline
	
	\noindent
	Lanciare l'applicazione dopo aver fatto partire l'ispezione del Network tramite la console di sviluppo del browser. Ogni quanto tempo il client fa sapere al server che è ancora connesso? È un'azione dovuta all'implementazione della chat o insita nel WebSocket? A cosa sere tale procedura?\newline
	Lanciare Wireshark e vedere cosa passa in rete sulla connessione TCP interessata.\newline
	
	\noindent	
	\textcolor{Green4}{\textbf{\emph{\underline{Soluzione esercizio 1}}}}\newline
	
	\noindent
	Per analizzare la rete si utilizza il software Wireshark che consente di analizzare il flusso di pacchetti in entrata e in uscita. All'apertura del software, andando nella sezione \dquotes{\emph{Adapter for loopback traffic capture}} sarà possibile seguire tutti i pacchetti che riguardano il localhost. Per filtrare il risultato dei pacchetti, si inserisce la stringa \dquotes{websocket} nella barra in alto, così da mostrare solamente quei pacchetti con protocollo WebSocket:
	\begin{figure}[!htp]
		\centering
		\includegraphics[width=\textwidth]{img/soluzioni_websocket-chat/wireshark-1.png}
	\end{figure}

	\noindent
	A questo punto, si aprono tre, quattro client. Quindi, si scrive l'URL localhost:4000 nel browser. Dato che il server non è in esecuzione, il browser non riesce a collegarsi al localhost:4000 poiché vede tale porta inutilizzata. Di conseguenza, il traffico catturato da Wireshark è inesistente.\newline

	\noindent
	Avviando il server, in automatico vedrà il collegamento dei 3/4 client avviati precedentemente. Di conseguenza, su Wireshark appariranno dei pacchetti corrispondenti al collegamento dei client al server:
	\begin{figure}[!htp]
		\centering
		\includegraphics[width=\textwidth]{img/soluzioni_websocket-chat/wireshark-2.png}
	\end{figure}

	\noindent
	Cliccando su uno dei pacchetti con flag [MASKED] è possibile notare una cosa interessante riguardo il protocollo TCP. Ovvero, il numero di porta d'origine e destinazione. Per esempio, nell'immagine è possibile vedere come un client con porta 51021 (\emph{Source Port}) stia comunicando con il server sulla sua porta 4000 (\emph{Destination Port}):
	\begin{figure}[!htp]
		\centering
		\includegraphics[width=\textwidth]{img/soluzioni_websocket-chat/wireshark-3.png}
	\end{figure}\newpage
	
	\noindent
	Adesso che è chiaro quale siano i client (quelli \dquotes{marchiati} con MASKED e il \href{https://en.wikipedia.org/wiki/WebSocket#Client_to_Server_Masking}{motivo per cui i messaggi sono mascherati è dovuto ad una questione di sicurezza}) e quale il server, è possibile vedere sulla colonna (la terza) di sinistra qual'è il tempo in cui ogni client comunica al server che è ancora vivo:
	\begin{figure}[!htp]
		\centering
		\includegraphics[width=\textwidth]{img/soluzioni_websocket-chat/wireshark-4.png}
	\end{figure}
	
	\noindent
	In questo caso, al tempo 259 il client con porta 51021 ha comunicato al server che è ancora vivo. Ovviamente il server ha risposto con un ACK e successivamente gli altri 3 client hanno comunicato al server la loro presenza. Al tempo 284, nuovamente il client con porta 51021 ricomunica al server che è ancora vivo (idem per gli altri). Si deduce che il client fa sapere al server che è ancora connesso ogni 25 secondi circa ($284-259=25$).\newline
	Documentazione ufficiale riguardo al Keep-Alive nel protocollo WebSocket: \url{https://websockets.readthedocs.io/en/stable/topics/timeouts.html}\newline
	RFC documentation: \url{https://www.rfc-editor.org/rfc/rfc6455#page-36}\newpage
	
	\noindent	
	\textcolor{Red3}{\textbf{\emph{\underline{Esercizio 2}}}}\newline
	
	\noindent
	Modificare il sorgente del codice per fare in modo che ad ogni utente connesso alla chat arrivi nella console il messaggio \dquotes{l'utente sta scrivendo...}.\newline
	
	\noindent
	NOTA: Lato client, bisogna spedire al server un evento apposito (ad es. \dquotes{typing}) quando l'utente scrive sulla tastiera (catturando l'evento di sistema \dquotes{keypress}). Lato server, la chiamata \textsf{webSocket.broadcast.emit('typing', data)} rilancia l'evento \dquotes{typing} a tutti i client connessi tranne che a quello dalla quale si è ricevuto il messaggio. Lato client infine gestire la ricezione de messaggio \dquotes{typing} che arriva dal server (si veda la gestione del messaggio \dquotes{UploadChat}).\newline
	
	\noindent	
	\textcolor{Green4}{\textbf{\emph{\underline{Soluzione esercizio 2}}}}\newline
	
	\noindent
	\lstinputlisting[language=JavaScript]{code/server_ex2.js}
	Il codice è rimasto lo stesso (paragrafo~\ref{codice Back-end server}), l'unica modifica effettuata è stata dalla riga 38 alla riga 40 in cui si impone al server di inoltrare il messaggio ricevuto, con tag \textsf{typing}, a tutti gli altri client.\newpage
	
	\noindent
	\lstinputlisting[language=JavaScript]{code/chat_ex2.js}
	Il codice del front-end è rimasto pressoché identico (paragrafo~\ref{codice Front-end JavaScript}) tranne a due modifiche importanti:
	\begin{itemize}
		\item (20-26) Sull'elemento \textsf{message} della pagina HTML, si crea un evento. Nel momento in cui una lettera viene premuta, il client invierà il messaggio con tag \textsf{typing} al server, inserendo nel payload il nome del mittente (\textsf{sender.value}).
		
		\item (50-60) Alla ricezione dell'evento \textsf{typing} da parte del server, il client stamperà la scritta \textsf{typing...} se e solo se supera il controllo alla riga 52, ovvero se non è lui il mittente (come da specifica dell'esercizio).
	\end{itemize}
	
	\longline\newline
	
	\noindent	
	\textcolor{Red3}{\textbf{\emph{\underline{Esercizio 3}}}}\newline
	
	\noindent
	Modificare a piacimento il contenuto del file \textsf{public/index.html} e valutare l'impatto grafico.\newline
	
	\noindent	
	\textcolor{Green4}{\textbf{\emph{\underline{Soluzione esercizio 3}}}}\newline
	
	\noindent
	Una modifica che è possibile fare è l'aggiunta di un altro titolo con tag h2. Ovviamente i colori saranno in tema con quelli specificati dal file CSS (styles.css).\newline
	
	\longline\newline
	
	\noindent	
	\textcolor{Red3}{\textbf{\emph{\underline{Esercizio 4}}}}\newline
	
	\noindent
	Provare a collegarsi allo stesso server da browser presenti su diversi PC collegati in rete.\newline
	
	\noindent	
	\textcolor{Green4}{\textbf{\emph{\underline{Soluzione esercizio 4}}}}\newline
	
	\noindent
	In teoria dovrebbe funzionare anche con localhost, ma non ci sono riuscito per cui non posso fornire più di tante informazioni a riguardo.\newpage
	
	\subsection{Architetture orientate ai servizi (Service-Oriented Architecture, SOA)}
	
	Solitamente le applicazioni sono monolitiche, ovvero hanno un'interfaccia utente, la quale richiama delle funzionalità fornite da una serie di librerie linkate in un unico programma che è eseguito sulla macchina dell'utente.\newline
	
	\noindent
	Al giorno d'oggi esiste un altro approccio di sviluppo delle applicazioni che riguarda SOA. Le \textcolor{Red3}{\textbf{architetture orientate ai servizi}} (\emph{Service-Oriented Architecture}) riguarda lo sviluppo di applicazioni complesse attraverso la \textbf{combinazione di diversi programmi attraverso la rete}:
	\begin{itemize}
		\item L'interfaccia utente e qualche funzionalità di base sono eseguito sull'host dell'utente.
		\item Le funzionalità principali dell'applicazione sono fornite da programmi che sono eseguiti su uno o più server.
	\end{itemize}

	\noindent
	I \textcolor{Green4}{\textbf{vantaggi}} sono molteplici:
	\begin{itemize}
		\item \textbf{Potenza di calcolo e memoria} sono delegate al server;
		
		\item \textbf{Protezione} della proprietà intellettuale su \textbf{algoritmi} strategici;
		
		\item \textbf{Annullamento} della necessità di distribuire \textbf{aggiornamenti} software quando le modifiche riguardano solo il codice dei server;
		
		\item Nuovo modello economico: \textbf{pay per use};
		
		\item \textbf{Eliminazione della pirateria}.
	\end{itemize}
	Nonostante i grandi vantaggi che propone, c'è un \textbf{requisito fondamentale} che deve essere rispettato: \textbf{la presenza e l'affidabilità della rete}.
	
	\longline
	
	\subsubsection{Funzioni remote e webservice}
	
	\noindent
	Queste architetture \textbf{si basano} completamente sui servizi offerti, ovvero sulle \textbf{funzioni remote}.
	
	Le \textcolor{Red3}{\textbf{funzioni remote}} hanno il seguente funzionamento. Il \textbf{server espone una API}\footnote{Application program interface (API): insieme delle funzioni/metodi esposte da una certa libreria.}, \textbf{la quale descrive una serie di funzioni che il client} (\underline{non} il web browser) \textbf{può invocare}. Quindi, l'implementazione effettiva del \textbf{codice si trova \emph{server-side}}, mentre il codice che richiama una funzione specifica dell'API si trova \emph{client-side}.\newline
	
	\noindent
	In passato le tecnologie utilizzate per eseguire una chiamata a funzione remota erano scritte completamente in C e venivano chiamate \emph{Remote Procedure Call} (RPC). Dopodiché, vennero semplificate e programmate in Java tramite il \emph{Java Remote Method Invocation} (JAVA RMI). Successivamente, ci fu l'invenzione del \emph{Common Object Request Broker Architecture} (CORBA) che permise di scrivere e di far comunicare più client/server con linguaggi differenti.\newline
	Al giorno d'oggi, grazie all'approdo delle \textcolor{Red3}{\textbf{webservice}}, al protocollo HTTP/HTTPS e alla metodologia REST, le cose sono molto più semplici.
	
	\longline
	
	\subsubsection{Webservice basati su REST}
	
	I webservice basati su REST hanno molteplici vantaggi. Questa metodologia consente di poter \textbf{utilizzare due linguaggi di programmazione differenti \emph{client-side} e \emph{server-side}}. Anche l'\textbf{architettura può essere differente}. Questo grazie a due componenti fondamentali:
	\begin{itemize}
		\item \textbf{Funzione STUB}, \emph{client-side}: codifica dei parametri trasmessi e la decodifica dei valori di ritorno;
		
		\item \textbf{Componente SKELETON}, \emph{server-side}: decodificare l'input, eseguire il codice della libreria, codificare l'eventuale risultato e inviarlo al client.
	\end{itemize}
	La metodologia REST utilizza il protocollo HTTP/HTTPS. Il suo \textbf{servizio principale sono le chiamate remote e per farlo utilizza l'URL}, il quale è stato mappato (\emph{mapping}).
	
	I \textbf{parametri possono essere passati tramite URL}, metodo GET, oppure dopo l'header \textbf{nel payload}, con il metodo POST o PUT.\newline
	
	\noindent
	Alcune dei metodi HTTP più famosi in base ai quali viene eseguita una funzione specifica:
	\begin{itemize}
		\item \textbf{POST}: usato da funzioni che \textbf{creano un nuovo oggetto} sul server;
		
		\item \textbf{PUT}: usato da funzioni che \textbf{aggiornano un oggetto esistente} sul server;
		
		\item \textbf{GET}: usato da funzioni che \textbf{recuperano informazioni di un oggetto} sul server;
		
		\item \textbf{DELETE}: usato da funzioni che \textbf{distruggono un oggetto esistente} sul server;
	\end{itemize}
	Solitamente i \textbf{valori di ritorno} sono sotto forma di file JSON.\newline
	
	\noindent
	I \textcolor{Green4}{\textbf{vantaggi}} sono:
	\begin{itemize}
		\item A differenza di CORBA, il \textbf{webservice utilizza il protocollo HTTP/HTTPS}. Di conseguenza, elementi come \textbf{firewall o NAT sono già predisposti all'utilizzo} senza troppe complicazioni;
		
		\item Il \textbf{debugging è facilitato} (e quindi anche lo sviluppo di applicazioni) dall'utilizzo di contenuti testuali nelle transazioni, per esempio i file JSON.
	\end{itemize}\newpage
	
	\subsubsection{Breve introduzione ai file JSON}
	
	I file \textcolor{Red3}{\textbf{JSON}} sono dei file testuali nati con JavaScript, ma ad oggi supportati da quasi tutti i linguaggi di programmazione.\newline
	
	\noindent
	La sua è una \textbf{struttura gerarchica} facilmente leggibile da un umano e parserizzabile da un programma. La sua sintassi è semplice ed è formato da coppie \dquotes{\textsf{attributo:valore}}.\newline
	
	\noindent
	Possono essere presenti anche array, rappresentati con le parentesi quadre, e strutture dati, rappresentate con le parentesi graffe. Un \textcolor{Green4}{\textbf{esempio}} di formattazione JSON:
	\begin{lstlisting}
{"impiegati":[
	{
		"nome":"Giovanni",
		"cognome":"Rossi"
	},
	{
		"nome":"Anna",
		"cognome":"Bianchi"
	},
	{
		"nome":"Pietro",
		"cognome":"Verdi"
	}
]}\end{lstlisting}\newpage

	\section{Introduzione alla sicurezza}
	
	Per comprendere al meglio la sicurezza informatica, è necessario procedere a piccoli passi. Prima di tutto, ci sono delle domande da porsi:
	\begin{itemize}
		\item Quali risorse si vogliono proteggere? Risposta al paragrafo \ref{quali risorse si vogliono proteggere}
		\item Come vengono minacciate tali risorse? Risposta al paragrafo \ref{come vengono minacciate le risorse}
		\item Cosa è necessario fare per contrastare tali minacce? Risposta al paragrafo \ref{come contrastare le minacce}
	\end{itemize}

	\longline
	
	\subsection{Quali risorse si vogliono proteggere}\label{quali risorse si vogliono proteggere}
	
	Le risorse che solitamente si vogliono proteggere sono:
	\begin{itemize}
		\item Risorse \textbf{hardware}, come i sistemi, i componenti, dischi. In questo caso si parla di \textbf{sicurezza \dquotes{fisica}};
		
		\item Risorse \textbf{software}, come i sistemi operativi e gli applicativi;
		
		\item Risorse di \textbf{dati}, come file e database;
		
		\item Risorse di \textbf{rete}, come i collegamenti e gli apparati.
	\end{itemize}
	Per \textcolor{Red3}{\textbf{proteggere}} queste risorse è necessario \textbf{garantire le proprietà di}:
	\begin{itemize}
		\item Confidenzialità
		\item Integrità
		\item Disponibilità
		\item Autenticità
		\item Tracciabilità
	\end{itemize}
	
	\longline
	
	\subsubsection{Confidenzialità}
	
	Per \textcolor{Red3}{\textbf{confidenzialità}} si intende che \textbf{nessun utente deve poter ottenere o dedurre dal sistema informazioni che non è autorizzato a conoscere}. In questo caso ci sono due caratteristiche fondamentali:
	\begin{itemize}
		\item La \textbf{riservatezza dei dati}, ovvero le informazioni confidenziali non devono essere rilevate o rilevabili da utenti non autorizzati;
		
		\item La \textbf{privacy}, ovvero dare l'opportunità all'utente di controllare i dati che il sistema con cui sta interagendo può collezionare e memorizzare.
	\end{itemize}\newpage
		
	\subsubsection{Integrità}
	
	Per \textcolor{Red3}{\textbf{integrità}} si intende di \textbf{impedire l'alterazione diretta o indiretta delle informazioni, sia da parte di utente e processi non autorizzati, che a seguito di eventi accidentali}. Nel caso in cui i dati venissero alterati, sarebbe necessario fornire strumenti per poter verificare facilmente tale alterazione.\newline
	
	\noindent
	Anche qui ci sono due caratteristiche fondamentali:
	\begin{itemize}
		\item \textbf{Integrità dei dati}, ovvero le informazioni e i programmi possono essere modificati solo se autorizzati;
		
		\item \textbf{Integrità del sistema}, ovvero il sistema funzione e non è compromesso.
	\end{itemize}

	\longline
	
	\subsubsection{Disponibilità}
	
	Per \textcolor{Red3}{\textbf{disponibilità}} si intende la possibilità di \textbf{rendere disponibili a ciascun utente abilitato, le informazioni alle quali ha diritto di accedere, nei tempi e modi previsti}. Quindi in determinate condizioni o in un preciso istante o in un intervallo di tempo. Nei sistemi informatici, i requisiti di disponibilità includono caratteristiche di prestazioni e robustezza.
	
	\longline
	
	\subsubsection{Autenticità}
	
	Per \textcolor{Red3}{\textbf{autenticità}} si intende il \textbf{dovere da parte di ciascun utente di verificare l'autenticità delle informazioni}. Per esempio messaggi, mittenti e destinatari. Per garantire l'autenticità, si richiede di poter verificare se un'informazione è stata manipolata.
	
	\longline
	
	\subsubsection{Tracciabilità}
	
	Per \textcolor{Red3}{\textbf{tracciabilità}} si intende che le \textbf{azioni di un'entità devono essere tracciate in modo univoco così da supportare la non-ripudiabilità e l'isolamento della responsabilità}. Ad esempio, nessun utente deve poter ripudiare o negare in tempi successivi messaggi da lui spediti o firmati.\newpage
	
	\subsection{Come vengono minacciate le risorse} \label{come vengono minacciate le risorse}
	
	Le minacce compromettono le proprietà di confidenzialità, integrità e disponibilità. Per \textcolor{Green4}{\textbf{esempio}}:
	
	\begin{table}[!htp]
		\centering
		\begin{tabular}{@{} l p{9em} p{9em} p{9em} @{}}
			\toprule
			& Confidenzialità & Integrità & Disponibilità \\
			\midrule
			HW & & & Calcolatore rubato \\
			&&&\\
			SW & Copia non autorizzata & Eseguibile modificato & Eseguibili cancellati \\
			&&&\\
			Dati & Lettura non autorizzata & File modificati & File cancellati \\
			&&&\\
			Rete & Lettura messaggi inviati & Messaggi modificati/ritardati/duplicati & Messaggi distrutti, rete fuori uso \\
			\bottomrule
		\end{tabular}
	\end{table}
	
	\noindent
	Una \textcolor{Red3}{\textbf{minaccia}} è una \textbf{possibile violazione della sicurezza}. Mentre un \textcolor{Red3}{\textbf{attacco}} è una \textbf{violazione effettiva della sicurezza}. Gli \textbf{attacchi} possono essere:
	\begin{itemize}
		\item \textbf{Attivi}, tentativi di \textbf{alterare risorse o modificare il funzionamento} dei sistemi;
		
		\item \textbf{Passivi}, tentativi di \textbf{ottenere informazioni e utilizzarle} senza intaccare le risorse;
		
		\item \textbf{Interni}, iniziati da un'\textbf{entità interna} al sistema;
		
		\item \textbf{Esterni}, iniziati da un'\textbf{entità esterna}, tipicamente attraverso la rete.
	\end{itemize}\newpage
	
	\begin{figure}[!htp]
		\centering
		\includegraphics[width=\textwidth]{img/sicurezza/sicurezza.png}
		\caption{Esempi di attacchi.}
	\end{figure}

	\noindent
	Infine, esistono alcune classi di attacchi o minacce:
	\begin{itemize}
		\item \textbf{\emph{Disclosure}}, accesso non autorizzato alle informazioni
		
		\item \textbf{\emph{Deception}}, accettazione di dati falsi
		
		\item \textbf{\emph{Disruption}}, interruzione o prevenzione di operazioni corrette
		
		\item \textbf{\emph{Usurpation}} controllo non autorizzato di alcune parti del sistema
	\end{itemize}\newpage

	\subsection{Come contrastare le minacce}\label{come contrastare le minacce}
	
	Il contrasto delle minacce è la parte più difficile della sicurezza. Innanzitutto non esiste una risposta unica poiché dipende da una serie di fattori. Inoltre, le risorse cambiano con il tempo e quello che oggi è considerato come \dquotes{sistema sicuro}, un domani non potrebbe più esserlo.\newline
	
	\noindent
	\textbf{Durante la progettazione di sistemi} è necessario considerare la possibilità di \textbf{attacchi potenziali}. Quindi, l'obbiettivo è di considerare quali possibili attacchi un attaccante potrebbe mettere in atto. Nello sviluppo di meccanismi di difesa, è intelligente utilizzare delle \textbf{soluzioni contro-intuitive} così da complicare la vita ad eventuali malintenzionati.\newline
	
	\noindent
	I \textbf{meccanismi di sicurezza devono essere inseriti sia a livello fisico che a livello logico}, quindi protocollare. Tuttavia, nonostante il grande impegno che può esserci per sviluppare sistemi sicuri, la \textbf{sicurezza dipende anche dagli utenti}. Per esempio le informazioni possedute, come password, e la creazione/distribuzione/protezione di tali informazioni dovrebbero avere un occhio di riguardo.\newline
	
	\noindent
	Purtroppo sempre più spesso la \textbf{sicurezza viene vista come un impedimento e/o rallentamento} del normale funzionamento dei sistemi per una serie di motivi:
	\begin{itemize}
		\item Gli \textbf{amministratori conducono una battaglia no-stop contro gli attaccanti}, ai quali, a differenza dei difensori che devono eliminare \underline{qualsiasi} minaccia possibile, è sufficiente sfruttare una singola vulnerabilità;
		
		\item Dispendio di energie e forze senza un reale risultato tangibile. Questo si traduce in uno svantaggio e non in un beneficio. Ovviamente finché non accade un incidente di sicurezza...
		
		\item Sempre più elementi aggiuntivi al sistema.
	\end{itemize}\newpage
	
	\subsubsection{Principi fondamentali di progettazione della sicurezza}
	
	Nonostante anni di ricerca, è impossibile progettare sistema senza falle di sicurezza. Tuttavia, grazie ad un insieme di pratiche e regole è possibile creare un sistema molto difficile da attaccare con successo. Queste \textbf{regole} vengono chiamate \textcolor{Red3}{\textbf{principi fondamentali di progettazione della sicurezza}}:
	\begin{itemize}
		\item \textbf{Aspetti economici dei meccanismi}, la progettazione delle misure di sicurezza deve essere il più semplice possibile, sia da implementare che da verificare;
		
		\item \textbf{\emph{Fail-safe default}}, i comportamenti non specificati devono prevedere un default sicuro, ad esempio i permessi di accesso;
		
		\item \textbf{Progettazione aperta} è preferibile rispetto al codice segreto;
		
		\item \textbf{Tracciabilità delle operazioni}, così che qualsiasi operazione può essere ricostruita e il sistema ripristinato;
		
		\item \textbf{Separazione dei privilegi} dalle risorse create da ciascun utente e da quelle critiche;
		
		\item \textbf{Separazione delle funzionalità}, ovvero la distinzione dei ruoli nei diversi punti del sistema fisico e logico;
		
		\item \textbf{Isolamento dei sottosistemi}, ovvero un sistema compromesso non dovrebbe compromettere gli altri;
		
		\item \textbf{Modularità}, quindi meccanismi di sicurezza indipendenti, sostituibili e riusabili.
	\end{itemize}\newpage
	
	\subsubsection{Politiche di sicurezza e meccanismi}
	
	Una \textcolor{Red3}{\textbf{politica di sicurezza}} è un'indicazione di \textbf{cosa è permesso e di cosa non è concesso}. Le regole riguardano i dati, le operazioni possibili, gli utenti singoli e i profili.\newline
	
	\noindent
	Un \textcolor{Red3}{\textbf{meccanismo di sicurezza}} è un metodo (strumento/procedura) per garantire una politica di sicurezza.\newline
	
	\noindent
	Quindi, data l'unione di queste due definizione, si dice: data una politica che distingue le azioni sicure da quelle non sicure, i meccanismi di sicurezza \textbf{prevengono}, \textbf{scoprono} o \textbf{recupero} da un attacco.
	\begin{itemize}
		\item \textbf{Prevenzione} di un attacco, ovvero il meccanismo di sicurezza deve rende impossibile l'attacco. Spesso l'adozione di questa politica interferisce con il sistema al punto da renderlo scomodo da usare. Per \textcolor{Green4}{\textbf{esempio}}, la richiesta di password come modo di autenticazione.
		
		\item \textbf{Scoperta} di un attacco, quindi il meccanismo di sicurezza è in grado di scoprire che un attacco è in corso. È \textbf{utile} quando:
		\begin{itemize}
			\item Non è possibile prevenire l'attacco;
			\item È necessario valutare le misure preventive.
		\end{itemize}
		Solitamente vengono monitorate le risorse del sistema, cercando eventuali tracce di attacchi.
		
		\item \textbf{Recupero} da un attacco. Questa politica è possibile applicarla in due modi diversi:
		\begin{enumerate}
			\item \textbf{Fermare l'attacco} e recuperare o ricostruire la situazione prima dell'attacco, per esempio con un backup;
			
			\item \textbf{Continuare a far funzionare il sistema correttamente durante l'attacco}, in gergo viene chiamata \emph{fault-tolerant}.
		\end{enumerate}
	\end{itemize}
	Degli \textcolor{Green4}{\textbf{esempi}} di meccanismi (specifici) di sicurezza sono: la \textbf{crittografia}, cioè la trasformazione dei dati in un formato non facilmente comprensibile da un essere umano, la \textbf{firma digitale}, usata per provare la sorgente, l'\textbf{autenticazione e controllo degli accessi}, come la gestione dei diritti degli utenti rispetto le risorse. I meccanismi generali possono essere il rilevamento degli eventi, la gestione degli Audit (\href{https://it.wikipedia.org/wiki/Audit}{Wikipedia}) o le Recovery (\href{https://www.cybersecurity360.it/soluzioni-aziendali/cyber-event-recovery-strutturare-un-piano-dazione-per-ripristinare-dati-sistemi-e-servizi/}{CyberSecurity360}).\newpage
	
	\subsubsection{Ottenere un sistema sicuro}
	
	Per ottenere un \textcolor{Red3}{\textbf{sistema sicuro}}, i primi passi sono uno sviluppo consapevole. È dunque necessario seguire le seguenti \textbf{fasi}:
	\begin{enumerate}
		\item \textbf{Specifica}, descrive il funzionamento del sistema desiderato e avere una visione d'insieme;
		\item \textbf{Progetto}, tradurre le specifiche in componenti che le implementano ed elencare le loro caratteristiche;
		\item \textbf{Implementazione}, creare il sistema vero e proprio che soddisfi le specifiche.
	\end{enumerate}
	Durante queste fasi è \underline{fondamentale} continuare a verificare la correttezza dell'implementazione.\newline
	
	\noindent
	Inoltre, è importante \textbf{considerare alcune scelte implementative} inevitabili durante la progettazione di un sistema:
	\begin{itemize}
		\item \textbf{Analizzare} i \textbf{costi-benefici} dell'implementazione della \textbf{sicurezza} e dei suoi meccanismi;
		
		\item \textbf{Analizzare i rischi} in caso di attacchi e gli eventuali danni/perdite che possono causare;
		
		\item \textbf{Aspetti legali} riguardanti la sicurezza e la privacy ed eventuali \textbf{aspetti morali};
		
		\item \textbf{Analizzare i problemi organizzativi}, infatti l'implementazione di meccanismi di sicurezza, come detto nei paragrafi precedenti, potrebbe creare situazioni di difficoltà per gli utenti che utilizzano il sistema. Questo si potrebbe tradurre nell'aumento di perdite invece di utili;
		
		\item \textbf{Aspetti comportamentali} delle persone coinvolte.
	\end{itemize}
	
	\newpage


%%%%%%%%%%%%%%%%%%%%%%%%%%%%%%%%%%%%%%%%%%%%%%%%%%%%%%%%%%%%%%%%%%%%%%%%%%%%%%%%%%%%%%%%%%%%%%%%%%%%%%%%%%%%%%%%%%%%%%%%%%%%%%%%%%%%%%%%%%%%%%%%%%%%%%%%%%%%%%%%%%%%%%%%%%%%%%%%%%%%%%%%%%%%%%%%%%%%%%%%%%%%%%%%%%%%%%%%%%%%%%%%%%%%%%%%%%%%%%%%%%%%%%%%
%%%%%%%%%%%%%%%%%%%%%%%%%%%%%%%%%%%%%%%%%%%%%%%%%%%%%%%%%%%%%%%%%%%%%%%%%%%%%%%%%%%%%%%%%%%%%%%%%%%%%%%%%%%%%%%%%%%%%%%%%%%%%%%%%%%%%%%%%%%%%%%%%%%%%%%%%%%%%%%%%%%%%%%%%%%%%%%%%%%%%%%%%%%%%%%%%%%%%%%%%%%%%%%%%%%%%%%%%%%%%%%%%%%%%%%%%%%%%%%%%%%%%%%%%%%%%%%%%%%%%%%%%%%%%%%%%%%%%%%%%%%%%%%%%%%%%%%%%%%%%%%%%%%%%%%%%%%%%%%%%%%%%%%%%%%%%%%%%%%%%%%%%%%%%%%%%%%%%%%%%%%%%%%%%%%%
%%%%%%%%%%%%%%%%%%%%%%%%%%%%%%%%%%%%%%%%%%%%%%%%%%%%%%%%%%%%%%%%%%%%%%%%%%%%%%%%%%%%%%%%%%%%%%%%%%%%%%%%%%%%%%%%%%%%%%%%%%%%%%%%%%%%%%%%%%%%%%%%%%%%%%%%%%%%%%%%%%%%%%%%%%%%%%%%%%%%%%%%%%%%%%%%%%%%%%%%%%%%%%%%%%%%%%%%%%%%%%%%%%%%%%%%%%%%%%%%%%%%%%%%%%%%%%%%%%%%%%%%%%%%%%%%%%%%%%%%%%%%%%%%%%%%%%%%%%%%%%%%%%%%%%%%%%%%%%%%%%%%%%%%%%%%%%%%%%%%%%%%%%%%%%%%%%%%%%%%%%%%%%%%%%%%%%%%%%%%%%%%%%%%%%%%%%%%%%%%%%%%%%%%%%%%%%%%%%%%%%%%%%%%%%%%%%%%%%%%%%%%%%%%%%%%%%%%%%%%%%%%%%%%%%%%%%%%%%%%%%%%%%%%%%%%%%%%
%%%%%%%%%%%%%%%%%%%%%%%%%%%%%%%%%%%%%%%%%%%%%%%%%%%%%%%%%%%%%%%%%%%%%%%%%%%%%%%%%%%%%%%%%%%%%%%%%%%%%%%%%%%%%%%%%%%%%%%%%%%%%%%%%%%%%%%%%%%%%%%%%%%%%%%%%%%%%%%%%%%%%%%%%%%%%%%%%%%%%%%%%%%%%%%%%%%%%%%%%%%%%%%%%%%%%%%%%%%%%%%%%%%%%%%%%%%%%%%%%%%%%%%%%%%%%%%%%%%%%%%%%%%%%%%%%%%%%%%%%%%%%%%%%%%%%%%%%%%%%%%%%%%%%%%%%%%%%%%%%%%%%%%%%%%%%%%%%%%%%%%%%%%%%%%%%%%%%%%%%%%%%%%%%%%%
	\section{Analisi di rete con Wireshark e da linea di comando}
	
	\subsection{Introduzione agli analizzatori di rete}
	
	Esistono diversi \textbf{strumenti software che consentono di analizzare i pacchetti} che arrivano alla propria interfaccia di rete:
	\begin{itemize}
		\item \textbf{TCDUMP} tool da linea di comando per sistemi operativi Linux;
		
		\item \textbf{WinDump} tool da linea di comando per sistemi operativi Windows;
		
		\item \textbf{Wireshark} tool sia da linea di comando che da GUI per sistemi operativi Linux, Windows e MacOS.
	\end{itemize}
	Tutti gli \textbf{strumenti di rete si basano sulla libreria} del linguaggio programmazione C, chiamata \textcolor{Red3}{\textbf{\textsf{libpcap}}}. Le sue funzionalità sono tre:
	\begin{enumerate}
		\item Cercare e trovare interfacce di rete;
		
		\item Gestione avanzata di filtri di cattura;
		
		\item Gestione degli errori e statistiche di cattura.
	\end{enumerate}
	
	\longline
	
	\subsubsection{Sniffing e la motivazione del \textsf{sudo}}\label{sniffing e la motivazione del sudo}
	
	Con il termine \textcolor{Red3}{\textbf{\emph{sniffing}}} viene intesa l'abilità del software, in questo caso Wireshark, di \textbf{catturare i pacchetti in arrivo e in partenza} dalla propria interfaccia di rete. Esistono due tipi di \emph{sniffing}:
	\begin{itemize}
		\item \emph{Sniffing} all'interno di \textcolor{Red3}{\textbf{reti non-switched}}. Come accade, per esempio, nelle \textbf{reti WiFi}, tutte le schede di rete dei PC collegati al router ricevono tutti i pacchetti, sia i propri sia quelli destinati ad altri.
		
		In questo caso, lo \emph{sniffing} consente di \textbf{catturare sia i pacchetti destinati al dispositivo che sta eseguendo questa operazione, sia i pacchetti di cui non è il destinatario}. Per eseguire questa operazione è necessario avviare Wireshark in modalità amministratore (comando \textsf{sudo} su Linux). Questo perché a priori l'\textbf{interfaccia di rete scarta i pacchetti a cui non è interessato}, mentre con Wireshark viene attivata la \textcolor{Red3}{\textbf{modalità promiscua}} che \textbf{consente di bypassare il sistema operativo e interrogare direttamente la CPU per salvare tutti i pacchetti in entrata e in uscita}. La modalità promiscua viene disattivata dal sistema operativo per motivi di \emph{performance}. Infatti, se fosse sempre attiva, il sistema operativo avrebbe un carico di lavoro talmente alto sulla CPU che calerebbero le prestazioni generali del sistema.
		
		\item \emph{Sniffing} all'interno di \textcolor{Red3}{\textbf{reti Ethernet switched}}. In questo caso, l'apparato centrale della rete (\emph{switch}), si preoccupa di \textbf{inoltrare su ciascuna porta solo il traffico destinato ai dispositivi collegati a quella porta}. Di conseguenza, ciascuna interfaccia di rete riceve solo i pacchetti destinati a lei (o multicast/broadcast).
		
		In altre parole, la \textbf{modalità promiscua \underline{non} consente la cattura di pacchetti di altre schede di rete}.
	\end{itemize}\newpage

	\subsection{Interfaccia grafica di Wireshark}
	
	Come specificato nel paragrafo~\ref{sniffing e la motivazione del sudo}, Wireshark necessita di essere avviato con la modalità amministratore (o \textsf{sudo} su Linux).
	
	Una volta avviato, la schermata iniziale sarà la seguente. Al centro sono presenti una serie di voci in cui vengono indicati i vari dispositivi di rete su cui è possibile ascoltare il traffico. Oltre all'interfaccia Wi-Fi (se presente sulla macchina), è possibile trovare interfacce del tipo bluetooth, ethernet (LAN), USB e interfacce di rete delle macchine virtuali.
	\begin{figure}[!htp]
		\centering
		\includegraphics[width=\textwidth]{img/wireshark/grafica-wireshark-1.png}
		\caption{Schermata iniziale di Wireshark.}
	\end{figure}\newpage

	\subsubsection{Sniffing della rete}

	Cliccando su una delle interfacce, il software inizierà ad ascoltare il traffico di pacchetti nella rete. La grafica si presenta nel seguente modo e con i seguenti colori:
	\begin{figure}[!htp]
		\centering
		\includegraphics[width=\textwidth]{img/wireshark/grafica-wireshark-2.png}
		\caption{Schermata di \emph{sniffing} nella rete.}
	\end{figure}

	\longline
	
	\subsubsection{Applicazione dei filtri}
	
	Data la grande mole di pacchetti che possono presentarsi durante lo \emph{sniffing}, Wireshark dà la possibilità di applicare una serie di filtri (la serie di comandi che è possibile utilizzare: \href{https://www.wireshark.org/docs/man-pages/wireshark-filter.html}{documentazione ufficiale}). Essi possono essere applicati scrivendo il relativo comando nella barra in alto in cui è scritto \dquotes{\textsf{Apply a display filter ...}}. Per \textcolor{Green4}{\textbf{esempio}} è possibile ricercare un protocollo scrivendolo nella barra (se quest'ultima viene evidenziata di verde, il comando è corretto, altrimenti viene evidenziata di rosso per segnalare un'inesattezza nel comando).
	\begin{figure}[!htp]
		\centering
		\includegraphics[width=\textwidth]{img/wireshark/grafica-wireshark-3.png}
		\caption{Applicazione corretta di un filtro.}
		\vspace{2em}
		\includegraphics[width=\textwidth]{img/wireshark/grafica-wireshark-4.png}
		\caption{Applicazione errata di un filtro.}
	\end{figure}\newpage

	\subsubsection{Seguire il flusso di una conversazione}
	
	È possibile seguire il flusso dati della \dquotes{conversazione} di un pacchetto. Per farlo è necessario cliccare sul pacchetto interessato, andare nella barra degli strumenti e cliccare \emph{Analyze} $\rightarrow$ \emph{Follow} e selezionare il protocollo interessato. Per \textcolor{Green4}{\textbf{esempio}}, nella seguente conversazione viene seguito il protocollo UDP.
	
	\begin{figure}[!htp]
		\centering
		\includegraphics[width=\textwidth]{img/wireshark/grafica-wireshark-5.png}
		\caption{Analisi del flusso di un pacchetto 1 di 2.}
		\vspace{2em}
		\includegraphics[width=\textwidth]{img/wireshark/grafica-wireshark-6.png}
		\caption{Analisi del flusso di un pacchetto 2 di 2.}
	\end{figure}\newpage

	\subsection{Comando \textsf{ping}}
	
	Il comando \textcolor{Red3}{\textsf{ping}} consente di \textbf{verificare la raggiungibilità di un computer connesso alla rete e il relativo Round Trip Time} (RTT)\footnote{Tempo che intercorre dalla partenza del pacchetto inviato fino al ritorno della risposta.}. Per questa operazione viene utilizzato il protocollo ICMP (Internet Control Message Protocol) che è un servizio per trasmettere informazioni riguardanti malfunzionamenti, informazioni di controllo o messaggi tra vari componenti di una rete di calcolatori.
	
	\begin{figure}[!htp]
		\centering
		\includegraphics[width=\textwidth]{img/altri-strumenti/ping.png}
		\caption{Esempio di esecuzione del comando \textsf{ping}.}
	\end{figure}\newpage
	
	\subsection{Comando \textsf{traceroute} (\textsf{tracert} Windows)}
	
	Il comando \textcolor{Red3}{\textsf{traceroute}} (\textsf{tracert} in Windows) è un semplice strumento per \textbf{tracciare il precorso che un pacchetto segue dalla sorgente alla destinazione}. Il comando mostra un elenco di tutte le interfacce dei router che il pacchetto attraversa finché non raggiunge la destinazione. Per questioni di sicurezza, alcuni nodi possono non essere visibili tramite il comando \textsf{traceroute}, questo per evitare che venga resa nota la struttura della rete.
	
	\begin{figure}[!htp]
		\centering
		\includegraphics[width=\textwidth]{img/altri-strumenti/traceroute.png}
		\caption{Esempio di esecuzione del comando \textsf{tracert} su Windows (\textsf{traceroute} su Linux).}
	\end{figure}\newpage

	\subsection{Comando \textsf{nslookup}}
	
	Il comando \textcolor{Red3}{\textsf{nslookup}} consente di \textbf{effettuare un'interrogazione ai server DNS (Domain Name System)\footnote{Sistema di server organizzato gerarchicamente per la gestione del namespace.} per poter ottenere da un hostname il relativo indirizzo IP, o viceversa}. Esso può essere utilizzato in modalità: \textbf{interattiva} o \textbf{non interattiva}.\newline
	
	\noindent
	La \textbf{modalità interattiva} consente di \textbf{effettuare più interrogazioni} e visualizza i singoli risultati. Viene attivata eseguendo il comando senza parametri.
	
	\begin{figure}[!htp]
		\centering
		\includegraphics[width=\textwidth]{img/altri-strumenti/nslookup1.png}
		\caption{Esempio di esecuzione del comando \textsf{nslookup} in modalità interattiva.}
	\end{figure}

	\noindent
	La \textbf{modalità \underline{non} interattiva} consente di \textbf{effettuare una sola interrogazione} visualizzandone il risultato. Viene attivata se viene inserito un parametro che corrisponde ad un \emph{host-to-find}.
	
	\begin{figure}[!htp]
		\centering
		\includegraphics[width=\textwidth]{img/altri-strumenti/nslookup2.png}
		\caption{Esempio di esecuzione del comando \textsf{nslookup} in modalità \underline{non} interattiva.}
	\end{figure}\newpage
	
	\subsection{Comando \textsf{ifconfig} (\textsf{ipconfig} Windows)}
	
	Il comando \textcolor{Red3}{\textsf{ifconfig}} (\textsf{ipconfig} in Windows) è utilizzato per \textbf{configurare e controllare un'interfaccia di rete TCP/IP} da riga di comando. La sua esecuzione con il parametro \dquotes{\textsf{-a}} stampa su terminale le informazioni di tutte le interfacce di rete.\newline
	
	\noindent
	Vengono stampate molteplici interfacce di rete, ma tra queste le più famose sono:
	\begin{itemize}
		\item \textsf{eth0} è la prima interfaccia Ethernet;
		
		\item \textsf{lo} è l'interfaccia \emph{loopback}, sempre presente. È \dquotes{speciale} poiché il sistema la utilizza per comunicare con sé stesso;
		
		\item \textsf{wlan0} è il nome della prima interfaccia di rete Wireless del sistema.
	\end{itemize}
	
	\begin{figure}[!htp]
		\centering
		\includegraphics[width=\textwidth]{img/altri-strumenti/ifconfig.png}
		\caption{Esempio di esecuzione del comando \textsf{ifconfig} su Linux.}
	\end{figure}\newpage

	\subsection{Comando \textsf{route} (\textsf{route PRINT} Windows)}
	
	Il comando \textcolor{Red3}{\textsf{route}} (\textsf{route PRINT} su Windows) è utilizzato per \textbf{visualizzare e modificare le tabelle di routing}. L'esecuzione consente di visualizzare la tabella di routing dell'host.

	\begin{figure}[!htp]
		\centering
		\includegraphics[width=\textwidth]{img/altri-strumenti/route.png}
		\caption{Esempio di esecuzione del comando \textsf{ifconfig} su Linux.}
	\end{figure}

	\longline	

	\subsection{Comando \textsf{whois}}
	
	Il comando \textsf{whois} consente, mediante l'interrogazione di database server da parte di un client, di \textbf{stabilire il nome privato, o dell'azienda, o dell'ente, al quale è intestato un determinato indirizzo IP o uno specifico DNS}. Solitamente vengono mostrate anche informazioni riguardanti l'intestatario, data di registrazione e data di scadenza.
	
	\begin{figure}
		\centering
		\includegraphics[width=.78\textwidth]{img/altri-strumenti/whois.png}
		\caption{Esempio di esecuzione del comando \textsf{whois} su Linux.}
	\end{figure}\newpage
	
	\subsection[\textcolor{Red3}{\textbf{Esercizi}}]{Esercizi}
	
	\subsubsection{Esercizio 1 - File \textsf{capture.cap}}
	
	\subsubsection{Esercizio 2 - File \textsf{simpleHTTP.cap}}
	
	\subsubsection{Esercizio 3 - File \textsf{busyNetwork.cap}}
	
	\subsubsection{Esercizio 4 - File \textsf{pingCapture.cap}}
	
	\subsubsection{Esercizio 5 - Comando \textsf{traceroute}}
	
	\subsubsection{Esercizio 6 - Interfacce di rete}
\end{document}
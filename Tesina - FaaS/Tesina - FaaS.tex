\documentclass[a4paper]{article}
\usepackage[T1]{fontenc}			% pacchetto per \chapter
\usepackage[italian]{babel}
\usepackage[italian]{isodate}  		% formato delle date in italiano
\usepackage{graphicx}				% gestione delle immagini
\usepackage{amsfonts}
\usepackage{booktabs}				% tabelle di qualità superiore
\usepackage{amsmath}				% pacchetto matematica
\usepackage{enumitem}				% gestione delle liste
\usepackage{pifont}					% pacchetto con elenchi carini
\usepackage{listings}				% pacchetto per i codici
\usepackage[x11names]{xcolor}		% pacchetto colori RGB
% Link ipertestuali per l'indice
\usepackage{xcolor}
\usepackage[linkcolor=black, citecolor=blue, urlcolor=cyan]{hyperref}
\hypersetup{
	colorlinks=true
}

\newcommand{\longline}{\noindent\rule{\textwidth}{0.4pt}}
\newcommand{\dquotes}[1]{``#1''}

\definecolor{codegreen}{rgb}{0,0.6,0}
\definecolor{codegray}{rgb}{0.5,0.5,0.5}
\definecolor{codepurple}{rgb}{0.58,0,0.82}
\definecolor{backcolour}{rgb}{0.95,0.95,0.92}
\lstdefinestyle{mystyle}{
	backgroundcolor=\color{backcolour},   
	commentstyle=\color{codegreen},
	keywordstyle=\color{magenta},
	numberstyle=\tiny\color{codegray},
	stringstyle=\color{codepurple},
	basicstyle=\ttfamily\footnotesize,
	breakatwhitespace=false,         
	breaklines=true,                 
	captionpos=b,                    
	keepspaces=true,                 
	numbers=left,                    
	numbersep=5pt,                  
	showspaces=false,                
	showstringspaces=false,
	showtabs=false,                  
	tabsize=2
}

\lstdefinelanguage{JavaScript}{
	keywords={typeof, new, true, false, catch, function, return, null, catch, switch, var, if, in, while, do, else, case, break},
	keywordstyle=\color{blue}\bfseries,
	ndkeywords={class, export, boolean, throw, implements, import, this},
	ndkeywordstyle=\color{darkgray}\bfseries,
	identifierstyle=\color{black},
	sensitive=false,
	comment=[l]{//},
	morecomment=[s]{/*}{*/},
	commentstyle=\color{codegreen}\ttfamily,
	stringstyle=\color{red}\ttfamily,
	morestring=[b]',
	morestring=[b]"
}

\lstset{
	language=JavaScript,
	backgroundcolor=\color{lightgray},
	extendedchars=true,
	basicstyle=\footnotesize\ttfamily,
	showstringspaces=false,
	showspaces=false,
	numbers=left,
	numberstyle=\footnotesize,
	numbersep=9pt,
	tabsize=2,
	breaklines=true,
	showtabs=false,
	captionpos=b
}

\lstset{style=mystyle}

%\usepackage{showframe}				% visualizzazione bordi
%\usepackage{showkeys}				% visualizzazione etichetta

\begin{document}
	\author{VR443470 - Valentini Andrea}
	\title{Università degli studi di Verona \\
		\:\\
		Tesina su FaaS (Function-as-a-Service)}
	\date{\printdayoff\today}
	\maketitle
	
	\newpage
	
	% indice
	\tableofcontents
	
	\newpage
	
	\section{Introduzione}
	
	Con l'avanzare della tecnologia e del \emph{cloud computing}, è aumentata sempre di più la richiesta di servizi online che consentissero di utilizzare calcolatori già pronti e con grandi disponibilità di calcolo.
	
	La crescita del \emph{cloud computing} è stata esponenziale nell'ultimo decennio, soprattutto anche grazie, purtroppo, alla pandemia del COVID-19. Tant'è che il CEO di Microsoft, Satya Nadella, disse:
	\begin{center}
		\dquotes{\emph{We’ve seen two years of digital transformation in two months.}}
	\end{center}
	
	\longline
	
	% https://www.ibm.com/topics/faas#What+is+FaaS%3F
	% https://www.redhat.com/en/topics/cloud-native-apps/what-is-faas#overview
	\subsection{Che cos'è FaaS}
	
	Function-as-a-Service (FaaS) è una tipologia di servizio \emph{cloud computing} che consente ai programmatori di sviluppare, eseguire e gestire pacchetti di applicazioni come se fossero funzioni, senza preoccuparsi della manutenzione di una propria infrastruttura.
	
	Tipicamente, l'\emph{hosting} di un'applicazione software su Internet richiede: la gestione di un server virtuale o fisico e la gestione di un sistema operativo. Con FaaS, viene tutto gestito in automatico dal \emph{cloud service provider}.
	
	\begin{figure}[!htp]
		\centering
		\includegraphics[width=\textwidth]{img/faas-1.jpg}
	\end{figure}
	
	\noindent
	I vantaggi di questa tecnologia sono molteplici e verranno spiegati più avanti. Per esempio, i programmatori possono concentrarsi solamente sul codice delle loro applicazioni.
	
	\newpage
	
	% https://www.ibm.com/topics/faas#FaaS+vs.+serverless
	% https://www.ibm.com/topics/faas#FaaS+vs.+serverless
	\subsection{FaaS e serverless}
	
	% https://www.redhat.com/en/topics/cloud-native-apps/what-is-faas#how-does-faas-work
	\subsection{Panoramica generale su come funziona FaaS}
	
	% https://www.redhat.com/en/topics/cloud-native-apps/what-is-faas#dynamic-scaling
	\subsection{Scalabilità dinamica}
	
	\section{Architettura FaaS}
	
	\section{Aziende che offrono un servizio FaaS}
	
	\subsection{IBM: cloud functions}
	
	\subsubsection{Panoramica}
	
	\subsubsection{Caso studio}
	
	\subsubsection{Prezzo del servizio}
	
	\subsection{Amazon: AWS Lambda}
	
	\subsubsection{Panoramica}
	
	\subsubsection{Caso studio}
	
	\subsubsection{Prezzo del servizio}
	
	\subsection{Google: cloud functions}
	
	\subsubsection{Panoramica}
	
	\subsubsection{Caso studio}
	
	\subsubsection{Prezzo del servizio}
	
	\subsection{Microsoft: Azure functions}
	
	\subsubsection{Panoramica}
	
	\subsubsection{Caso studio}
	
	\subsubsection{Prezzo del servizio}
	
	\subsection{Oracle: OCI}
	
	\subsubsection{Panoramica}
	
	\subsubsection{Caso studio}
	
	\subsubsection{Prezzo del servizio}
	
	\section{Esempio di applicazione}	
\end{document}
\documentclass[a4paper]{article}
\usepackage[italian]{babel}
\usepackage[italian]{isodate}  		% formato delle date in italiano
\usepackage{graphicx}				% gestione delle immagini
\usepackage{amsfonts}
\usepackage{booktabs}				% tabelle di qualità superiore
\usepackage{amsmath}				% pacchetto matematica
\usepackage{cancel}					% cancellare per approssimare matematicamente
\usepackage{stmaryrd} 				% per '\llbracket' e '\rrbracket'
\usepackage{amsthm}					% teoremi migliorati
\usepackage{enumitem}				% gestione delle liste
\usepackage{pifont}					% pacchetto con elenchi carini

\usepackage[x11names]{xcolor}		% pacchetto colori RGB
% Link ipertestuali per l'indice
\usepackage{xcolor}
\usepackage[linkcolor=black, citecolor=blue, urlcolor=cyan]{hyperref}
\hypersetup{
	colorlinks=true
}

%\usepackage{showframe}				% visualizzazione bordi
%\usepackage{showkeys}				% visualizzazione etichetta

\newcommand{\dquotes}[1]{``#1''}

\begin{document}
	\author{VR443470}
	\title{Elaborazione di segnali e immagini}
	\date{\printdayoff\today}
	\maketitle
	
	\newpage
	% indice
	\tableofcontents
	
	\newpage
	
	\section{Fondamenti}
	
	\subsection{Matematica preliminare}
	
	\subsubsection{Numeri complessi}
	
	Un numero complesso $c$ appartiene all'insieme dei complessi $\mathbb{C}$ e la sua forma è del tipo:

	\begin{equation*}
		c = \Re + j \Im
	\end{equation*}

	\noindent
	con $\Re, \Im$ variabili $\in\mathbb{R}$ e $j$ chiamata \emph{unità immaginaria} rappresentata come $j = \sqrt{-1}$. Inoltre, $\Re$ rappresenta la \emph{parte reale} e $\Im$ la \emph{parte immaginaria}. Il coniugato di $c$ è
	
	\begin{equation*}
		\tilde{c} = \Re - j \Im
	\end{equation*}

	I numeri complessi, dal punto di vista geometrico, possono essere visti come punti su un piano (chiamato \emph{piano complesso}) e descritti da coordinate $(R, I)$. Nel piano complesso, le ascisse ($x$) sono rappresentate dalla parte reale, mentre le ordinate ($y$) dalla parte immaginaria.
	
	Spesso è utile rappresentare i numeri complessi in coordinate polari formate nel seguente modo $\left(modulo, angolo\right)$. Questa forma viene denominata \emph{forma polare} di un numero complesso:
	
	\begin{equation*}
		c = \Re + j \Im = |c| (\cos{\theta} + j \sin{\theta})
	\end{equation*}

	\noindent
	dove:
	
	\begin{equation*}
		|c| = \sqrt{\Re^2 + \Im^2} \longrightarrow \text{chiamato \emph{modulo} o \emph{magnitudo}}
	\end{equation*}

	\noindent
	invece, \emph{theta} rappresenta:
	
	\begin{equation*}
		\theta \cong \arctan{\left(\dfrac{\Im}{\Re}\right)} \longrightarrow \text{chiamato \emph{angolo}, \emph{fase} o \emph{argomento \underline{in radianti}}}
	\end{equation*}

	Grazie alla formula di Eulero:
	
	\begin{equation*}
		e^{j \theta} = \cos{\theta} + j \sin{\theta}
	\end{equation*}
	
	\noindent
	è possibile riscrivere la forma polare di un numero complesso in maniera alternativa, ossia:
	
	\begin{equation*}
		c = \Re + j \Im = |c|\left(\cos{\theta} + j \sin{\theta}\right) = |c| e^{j \theta}
	\end{equation*}

	La \textbf{somma} e la \textbf{moltiplcazione} di due numeri complessi diventa:
	
	\begin{gather*}
		c_1 = R_1 + j I_1 \hspace{2em} c_2 = R_2 + j I_2 \\
		\text{Somma: } c_1 + c_2 = \left(R_1 + R_2\right) + j \left(I_1 + I_2\right) \\
		\text{Moltiplicazione con Eulero: } c_1\cdot c_2 = \left(R_1 R_2 - I_1 I_2\right) + j \left(R_1 I_2 + I_1 R_2\right) \longrightarrow = |c_1| |c_2| e^{j\left(\theta_1 + \theta_2\right)}
	\end{gather*}

	\newpage

	\subsubsection{Funzioni complesse di variabile reale}

	Dato $t \in \mathbb{R}$, una funzione $f$ complessa di variabile reale è $f: D_1 \subseteq \mathbb{R} \rightarrow D_2 \subseteq \mathbb{C}$. Viene introdotto questo concetto poiché il \textbf{\emph{fasore}} è un \underline{esempio fondamentale}. Le \textbf{caratteristiche} di questa funzione:
	
	\begin{itemize}
		\item È una funzione complessa che modella la posizione di un punto che ruota attorno all'orgiine con raggio determinato $|c|$ e velocità angolare costante $\theta{(t)}$.
		
		\item Se la funzione fosse nei numeri reali, sarebbe più dispendioso in termini di numero di funzioni da utilizzare.
	\end{itemize}
	
	L'\textbf{obbiettivo} dei fasori è quello di \emph{passare dal dominio del \underline{tempo}} (o spazio) \emph{a quello dell'\underline{analisi frequenziale}}.\newline
	La particolarità è che nel tempo il fasore riesce a variare un numero complesso (in forma polare) mantenendo il modulo $|c|$ fisso:

	\begin{equation*}
		|c| e^{j\theta} \rightarrow |c| e^{j\theta{(t)}}
	\end{equation*}

	\noindent
	dove $\theta{(t)}$ indica la \textbf{\emph{velocità angolare}}. Quest'ultima può essere calcolata tramite:
	
	\begin{equation*}
		\theta{(t)} \longrightarrow \dfrac{2\pi}{T_0} t + \phi
	\end{equation*}

	\noindent
	dove $T_0$ indica il \emph{tempo} impiegato per eseguire $2\pi$ radianti.
	
	Solitamente si utilizza il fasore con le seguenti supposizioni:
	
	\begin{itemize}
		\item[\ding{45}] Coordinate rappresentate con $(R, I)$
		\item[\ding{45}] Impostata una distanza unitaria fissa dall'origine $|c| = 1$
		\item[\ding{45}] Velocità angolare \underline{costante} pari a $2\pi/sec.$, ossia $\theta{(t)} = 2\pi t, T_0 = 1\mathrm{sec.}$
		\item[\ding{45}] Con $t = 0$ si ha $\theta = 0$
		\item[\ding{45}] Viene mantenuto $\phi = 0$
	\end{itemize}

	\newpage
	
	\subsubsection{Funzioni pari e dispari}
	
	Una funzione $f:\mathbb{R}\rightarrow\mathbb{R}$ è \textbf{\emph{pari}} se e solo se:
	
	\begin{equation*}
		f(t) = f(-t)
	\end{equation*}

	\noindent
	Invece, una funzione $f:\mathbb{R}\rightarrow\mathbb{R}$ è \textbf{\emph{dispari}} se e solo se:
	
	\begin{equation*}
		f(t) = -f(-t)
	\end{equation*}

	\newpage
	
	\subsubsection{Segnali periodici}
	
	Un segnale $f$ è \textbf{\emph{periodico}} di periodo $T$ o $T$-periodico se:
	
	\begin{equation*}
		\exists\:T_0 \in R^+ : f \left(t + T_0\right) = f(t), \hspace{1em} \forall t \in D_1
	\end{equation*}

	\noindent
	e $T_0$ è il minor numero per cui la condizione di ripetizione si verifica.
	
	Dato un periodo $T_0$ con la lettera $\mu_0$ si indica la \textbf{\emph{frequenza fondamentale}}:
	
	\begin{equation*}
		\mu_0 = \dfrac{1}{T_0}
	\end{equation*}

	Fissato $T_0 > 0$ i \textbf{\emph{segnali trigonometrici}}  di \underline{minimo periodo} $T_0$ sono:
	
	\begin{equation*}
		f(t) = \cos{\left(2 \pi \mu_{0} t \right)} \hspace{2em} f(t) = \sin{\left(2 \pi \mu_0 t\right)}
	\end{equation*}

	\noindent
	dove $\mu$ è una frequenza generale, mentre $\mu_0 = \dfrac{1}{T_0}$ è la \textbf{frequenza fondamentale}. Invece, spesso la \textbf{velocità angolare} o \textbf{\emph{pulsazione}} viene rappresentata come:
	
	\begin{equation*}
		2 \pi \mu_0 = \dfrac{2\pi}{T_0} = \omega_0
	\end{equation*}

	Inoltre, fissato un $\theta\in\mathbb{R}$ chiamato \textbf{\emph{fase}} si osserva che anche le funzioni:
	
	\begin{equation*}
		f(t) = \cos{\left(2 \pi \mu_0 t + \theta\right)} \hspace{2em} f(t) = \sin{\left(2 \pi \mu_0 t + \theta\right)}
	\end{equation*}

	\noindent
	hanno il medesimo periodo $T$.
	
	\noindent
	Infine, la fase $\theta$ permette di eseguire operazione di \emph{shift}.
	
	\newpage
	
	\subsection{Operazioni fondamentali}
	
	\subsubsection{Somma}
	
	La \textbf{\emph{somma}} di due segnali è facile quando essi non interferiscono, ovvero quando \textbf{non} sono contemporaneamente $\ne 0$. Alcuni esempi qui di seguito.
	
	\begin{figure}[!htp]
		\centering
		\includegraphics[width=1\textwidth]{img/op_somma_1.pdf}
		\includegraphics[width=1\textwidth]{img/op_somma_2.pdf}
	\end{figure}

	\newpage
	
	\subsubsection{Shift (o traslazione)}
	
	Lo \textbf{\emph{shift}} (o traslazione) è il cambio di posizione di un segnale. Può essere effettuato:
	
	\begin{itemize}
		\item \textbf{Traslazione a destra} con la funzione $f(t-\tau)$
		\item \textbf{Traslazione a sinistra} con la funzione $f(t+\tau)$
	\end{itemize}
	
	\begin{figure}[!htp]
		\centering
		\includegraphics[width=0.5\textwidth]{img/op_shift_dx.pdf}\label{op_shift_dx}
		\caption{Shift a destra}
		\includegraphics[width=0.5\textwidth]{img/op_shift_sx.pdf}\label{op_shift_sx}
		\caption{Shift a sinistra}
	\end{figure}

	\newpage
	
	\subsubsection{Funzione box $\Pi$ e impulso di Dirac}\label{funzione box e impulso di Dirac}
	
	La funzione \textbf{\emph{box}} è definita nel seguente modo:
	
	\begin{equation*}
		A \Pi\left(\dfrac{x}{b}\right) \hspace{2em} x\in\left[-\dfrac{b}{2}, \dfrac{b}{2}\right]
	\end{equation*}

	\begin{figure}[!htp]
		\centering
		\includegraphics[width=0.4\textwidth]{img/box.pdf}\label{box}
		\caption{Box generica}
	\end{figure}

	La funzione $\delta(x)$ è chiamata \textbf{\emph{impulso unitario}} o \textbf{\emph{impulso di Dirac}} perché è definita nel seguente modo:
	
	\begin{equation*}
		\delta(x) = 
		\begin{cases}
			\infty  & \text{se } x=0 \\
			0		& \text{se } x\ne 0
		\end{cases}
		\hspace{2em} \int_{-\infty}^{\infty} \delta(x)\: dx = 1
	\end{equation*}

	\noindent
	Quindi è un impulso che tende all'infinito solamente quando la $x$ è nell'origine, ma il suo integrale è uguale a $1$. Alcune \textbf{proprietà} dell'impulso:
	
	\begin{enumerate}
		\item $\delta(x-x_0) = 0 \hspace{1em} \forall x\ne x_0$
		\item Data una funzione generica $f$ (\textbf{setacciamento}\label{setacciamento}): $\displaystyle \int_{-\infty}^{\infty} f(x)\delta(x-x_0)\: dt = f(x_0)$
		\item $\delta(x - x_0) = \delta(x_0 - x)$
		\item $\delta(ax) = \dfrac{1}{|a|} \delta(x) \hspace{1em} \forall x \in \mathbb{R} \text{, fissato } a \in \mathbb{R}-\{0\}$
	\end{enumerate}

	\newpage
	
	\subsubsection{Funzione sinc}\label{funzione sinc}
	
	La funzione \textbf{\emph{sinc}} è definita nel seguente modo:
	
	\begin{equation*}
		\mathrm{sinc}(t) = \dfrac{\sin{\left(\pi t\right)}}{\pi t}
	\end{equation*}

	\noindent
	Ha due \textbf{caratteristiche} importanti: (1) l'intersezione con l'asse delle $x$ avviene sempre nei numeri interi positivi e negativi (quindi $1$ e $-1$, $2$ e $-2$, ecc.); (2) il limite $\displaystyle \lim_{t\rightarrow \pm\infty}\mathrm{sinc}(t) = 0$.\newline
	Questa funzione è \textbf{importante per l'analisi nel dominio del tempo} (o \textbf{frequenza}).
	
	\subsubsection{Funzione triangolo $\Lambda$}
	
	La funzione \textbf{\emph{triangolo}} è definita nel seguente modo:
	
	\begin{equation*}
		\Lambda(x) =
		\begin{cases}
			1-|x|, 	& |x| < 1 \\
			0		& \text{altrimenti}
		\end{cases}
	\end{equation*}

	\noindent
	Questa funzione è \textbf{importante per l'analisi spettrale e} per le \textbf{operazioni di convoluzione}.
	
	\subsubsection{Funzione segno ($sgn$)}
	
	La funzione \textbf{\emph{segno}} è definita nel seguente modo:
	
	\begin{equation*}
		\mathrm{sgn}(x) =
		\begin{cases}
			-1, 	& x < 0 \\
			+1,		& x > 0 \\
			0		& x = 0
		\end{cases}
	\end{equation*}
	
	\noindent
	Questa funzione ribalta segnali sopra o sotto l'asse delle $x$.
	
	\subsubsection{Funzione gradino}
	
	La funzione \textbf{\emph{gradino}} è definita nel seguente modo:
	
	\begin{equation*}
		u(x) =
		\begin{cases}
			0 & x < 0 \\
			1 & x \ge 0
		\end{cases}
	\end{equation*}
	
	\noindent
	Questa funzione rappresenta un \textbf{segnale} che si attiva a partire dal tempo specificato e rimane attivo indefinitamente. Attenzione! Non si confonda questo segnale con il segno.
	
	\newpage
	
	\subsubsection{Treno di impulsi}
	
	Il \textbf{\emph{treno di impulsi}} $S_{\Delta T}(x)$ è la somma di un numero infinito di impulsi periodici discreti distanziati di una quantità $\Delta T$:
	
	\begin{equation*}
		S_{\Delta T}(x) = \sum_{n = -\infty}^{\infty} \delta (x - n \Delta T) \hspace{1em} n \in \mathbb{Z}
	\end{equation*}

	\begin{figure}[!htp]
		\centering
		\includegraphics[width=0.5\textwidth]{img/treno_di_impulsi.pdf}\label{treno_di_impulsi}
		\caption{Treno di impulsi}
	\end{figure}

	\subsubsection{Energia di un segnale}
	
	L'\textbf{\emph{energia di un segnale}} è definita nel seguente modo:
	
	\begin{equation*}
		E_f =
		\begin{cases}
			\displaystyle \int_{-\infty}^{+\infty} f^{2}(t)\: dt & \text{se } f \in \mathbb{R} \\
			\displaystyle \int_{-\infty}^{+\infty} \left| f(t) \right|^{2} dt \hspace{1em} \text{con } \left| f(t) \right|^{2} = \tilde{f}(t) f(t), & f \in \mathbb{C}
		\end{cases}
	\end{equation*}
	
	\noindent
	Un segnale si dice \textbf{ad energia finita} (o \textbf{di energia}) se l'integrale che rappresenta l'energia converge ed è diverso da $0$. Quindi:
	
	\begin{itemize}
		\item[\ding{43}] \textbf{Condizione \emph{sufficiente}} all'esistenza della sua trasformata di Fourier. Le funzioni trigonometriche non sono di energia ma hanno comunque la Trasformata di Fourier.
		\item[\ding{42}] \textbf{Condizione \emph{necessaria}} per essere un segnale ad energia finita, all'infinito ($+\infty$ e $-\infty$) l'\textbf{ampiezza} va a zero.
	\end{itemize}

	\noindent
	Alcuni esempi:

	\begin{itemize}
		\item[\ding{80}] \textbf{Segnali di energia.} Impulsi rettangolari, oscillazioni smorzate ($\mathrm{sinc}$);
		\item[\ding{80}] \textbf{Segnali \underline{non} di energia.} Funzioni trigonometriche $\sin$ e $\cos$.
	\end{itemize}

	\noindent
	L'\textbf{unità di misura} è il \emph{joule}.
	
	\subsubsection{Potenza media di un segnale}
	
	La \textbf{\emph{potenza media di un segnale}} è definita nel seguente modo:
	
	\begin{equation*}
		P_f =
		\begin{cases}
			\displaystyle \lim_{T \rightarrow +\infty} \dfrac{1}{T} \int_{-\dfrac{T}{2}}^{+\dfrac{T}{2}} f^{2}(t)\: dt & \text{se } f \in \mathbb{R} \\
			\displaystyle \lim_{T \rightarrow +\infty} \dfrac{1}{T} \int_{-\dfrac{T}{2}}^{+\dfrac{T}{2}} \left|f(t)\right|^{2}\: dt \hspace{1em} \text{con } \left| f(t) \right|^{2} = \tilde{f}(t) f(t), & f \in \mathbb{C}
		\end{cases}
	\end{equation*}
	
	\noindent
	Un segnale si dice \textbf{a potenza finita} (o \textbf{di potenza}) se l'integrale che rappresenta la potenza converge ed è diverso da $0$. L'\textbf{unità di misura} è il \emph{watt}.\newline
	Infine, un segnale ad energia finita ha la potenza che tende a zero (per cui un segnale non può appartenere ad entrambe le categorie). Invece, esistono segnali che non sono né di energia, n* di potenza finita.
	
	\newpage
	
	\subsection{Altre operazioni fondamentali}
	
	\subsubsection{Rescaling (o riscalatura)}
	
	La funzione di \textbf{\emph{rescaling}} è definita nel seguente modo:
	
	\begin{equation*}
		\forall f(t) : D_1 \in \mathbb{R}, \hspace{1em} \omega \ne 0
	\end{equation*}
	
	\noindent
	Simile allo \emph{shift}, il \emph{rescaling} ha una definizione generica e due varianti:
	
	\begin{itemize}
		\item \textbf{Definizione generica} con la funzione semplice $f(t)$ (immagine~\ref{rescaling}).
		\item \textbf{Ritardo \underline{lineare} del segnale di un fattore $\omega$} con la funzione $f(\omega t), 0 < \omega < 1$ (immagine~\ref{rescaling_ritardo}).
		\item \textbf{Accelero \underline{lineare} del segnale di un fattore $\omega$} con la funzione $f(\omega t), \omega > 1$ (immagine~\ref{rescaling_accelero}).
	\end{itemize}
	
	\begin{figure}[!htp]
		\centering
		\includegraphics[width=0.5\textwidth]{img/rescaling.pdf}
		\caption{Definizione generica}\label{rescaling}
		\includegraphics[width=0.5\textwidth]{img/rescaling_ritardo.pdf}
		\caption{Ritardo \underline{lineare} del segnale di un fattore $\omega$}\label{rescaling_ritardo}
		\includegraphics[width=0.5\textwidth]{img/rescaling_accelero.pdf}
		\caption{Accelero \underline{lineare} del segnale di un fattore $\omega$}\label{rescaling_accelero}
	\end{figure}

	\subsubsection{Cross-Correlazione}
	
	Dati $f_1(\tau), f_2(\tau)$ segnali continui, $\tau\in\mathbb{R}$ il segnale di \textbf{\emph{cross-correlazione}} viene definito come:
	
	\begin{equation*}
		\displaystyle f_1 \otimes f_2 (t) = \int_{-\infty}^{+\infty} \tilde{f_1}(\tau) f_2(\tau - t)\: d\tau
	\end{equation*}

	\noindent
	In cui $\tilde{f_1}(\tau)$ rappresenta un \emph{complesso coniugato}. Nel caso in cui $f_1$ è reale, allora $\tilde{f_1}(\tau) \rightarrow f_1(\tau)$. \newline
	Infine, con $t = 0$ si ha l'\textbf{\emph{integrale di cross-correlazione}}, il quale è definito se l'integrale converge (ovviamente se il segnale non è né di energia, né di potenza, la convergenza non esiste!).
	
	\newpage
	
	
	
	
	\subsubsection{Esercizi d'esame}
	
	\textcolor{Red3}{\textbf{\emph{Esercizio.}}}
	
	Il primo esercizio fornisce una funzione $f(t)$:
	
	\begin{equation*}
		f(t) = \Pi \left(\dfrac{t-2}{4}\right) e^{-2t}
	\end{equation*}

	\noindent
	Le \textbf{richieste} dell'esercizio sono le seguenti:
	
	\begin{enumerate}[label=\Roman*]
		\item Rappresentare graficamente il segnale;
		
		\item Calcolare sia l'energia che la potenza media. Inoltre, dire se $f(t)$ è una funzione di energia o di potenza fornendo una motivazione valida. Infine, calcolare l'energia o la potenza nel caso in cui $f(t)$ sia solo composta da $e^{-2t}$;
		
		\item Scrivere l'espressione analitica rispetto $z(t) = -f(-t)$ e $v(t) = f(t+4)$
	\end{enumerate}

	\noindent
	\textcolor{Green4}{\textbf{\emph{Risoluzione I.}}}
	
	\noindent
	Il \textbf{primo passo} è quello di scomporre la funzione così da avere una visione più chiara sulle operazioni da effettuare:
	
	\begin{equation*}
		f(t) = \Pi \left(\dfrac{t-2}{4}\right) e^{-2t} \longrightarrow f(t) = \Pi \left(\dfrac{1}{4} \cdot \left(t - 2\right)\right)
	\end{equation*}

	\noindent
	Come si può osservare, ci sono due operazioni da eseguire. Quindi, dopo l'esplicitazione si esegue la rappresentazione del segnale base $\Pi(t)$:
	
	\begin{figure}[!htp]
		\centering
		\includegraphics[width=0.4\textwidth]{img/ex_exam/Pi_func_1.pdf}
		\caption{Rappresentazione della funzione $f(t)$, ovvero un box.}
	\end{figure}

	\newpage
	Adesso si esegue l'operazione di moltiplicazione per un fattore che in questo caso è $\dfrac{1}{4}$. Quindi si rappresenta la box $\Pi\left(\dfrac{1}{4}\cdot t\right)$:
	
	\begin{figure}[!htp]
		\centering
		\includegraphics[width=0.4\textwidth]{img/ex_exam/Pi_func_1-Mod.pdf}
		\caption{Box $\Pi\left(\dfrac{1}{4}\cdot t\right)$ allargata.}
	\end{figure}

	\noindent
	L'operazione che è stata effettuata è stata semplicemente considerare la box del tipo $\Pi\left(\dfrac{t}{4}\right)$. Ricordandosi le nozioni del corso di Sistemi, per definizione quindi la box è definita nell'intervallo $-2, +2$.
	
	Infine, viene applicata l'ultima operazione, ovvero il $-2$ all'incognita $t$. Quindi, la funzione box diventerà $\Pi\left(\dfrac{1}{4}\left(t-2\right)\right)$ e la sua rappresentazione grafica sarà uno shift a destra (ritardo):
	
	\begin{figure}[!htp]
		\centering
		\includegraphics[width=0.4\textwidth]{img/ex_exam/Pi_func_1-Mod2.pdf}
		\caption{Box $\Pi\left(\dfrac{1}{4}\left(t-2\right)\right)$ dopo lo shift a destra.}
	\end{figure}

	\newpage

	Il \textbf{primo punto si conclude} con la rappresentazione del segnale $e^{-2t}$ e la sua combinazione con la box. Quindi:
	
	\begin{figure}[!htp]
		\centering
		\includegraphics[width=0.4\textwidth]{img/ex_exam/E_func_1.pdf}
		\caption{Rappresentazione della funzione $e^{-2t}$}
	\end{figure}

	\noindent
	E infine la sua concatenazione con la box, quindi una sorta di applicazione di un filtro:
	
	\begin{figure}[!htp]
		\centering
		\includegraphics[width=0.4\textwidth]{img/ex_exam/Sol_func_1.pdf}
		\caption{Rappresentazione finale della funzione $f(t) = \Pi \left(\dfrac{t-2}{4}\right) e^{-2t}$}\label{ex1_grafico}
	\end{figure}

	\newpage

	\noindent
	\textcolor{Green4}{\textbf{\emph{Risoluzione II.}}}

	Guardando la figura~\ref{ex1_grafico} si può già intuire che tipo di segnale sia. Infatti, dato che è limitato e non si estende all'infinito, per definizione è un \textbf{segnale finito}, quindi \textbf{di energia} e \underline{non} di potenza. Per dimostrare questa affermazione, si eseguono i calcoli:
	
	\begin{gather*}
		\textbf{Definizione di energia: } E_{f} = \int_{-\infty}^{\infty}{\left|f(t)\right|^2\:\mathrm{d}t} = \int_{-\infty}^{\infty}{f(t)^2\:\mathrm{d}t} \\
		\textbf{Definizione di potenza: } P_{f} = \lim_{T\rightarrow\infty}{\dfrac{1}{T} \int_{-\frac{T}{2}}^{\frac{T}{2}}{f^{2}(t)\:\mathrm{d}t}}
	\end{gather*}

	\noindent
	Dopo le definizioni, si esegue l'effettivo calcolo con i valori numerici:
	
	\begin{gather*}
		\textbf{Energia finita} \\
		E_{f} = \int_{0}^{4}{e^{-4t}\:\mathrm{d}t} = \left.\dfrac{e^{-4t}}{-4}\right\vert_{0}^{4} = \dfrac{-e^{-16}+1}{4} = \dfrac{1}{4} \ne 0 \\
		\textbf{Potenza finita} \\
		P_{f} = \lim_{T\rightarrow\infty}{\dfrac{1}{T} \int_{0}^{4}{e^{-4t}\:\mathrm{d}t}} = \lim_{T\rightarrow\infty}{\dfrac{1}{T}\cdot\dfrac{1}{4}} = 0
	\end{gather*}

	\noindent
	Come si osserva dai risultati, \underline{è} un segnale di energia finita poiché è un valore noto, invece \underline{non è} un segnale di potenza poiché il risultato è zero e non rispetta la definizione.
	
	Al contrario, se la funzione fosse composta solamente dall'esponenziale, il calcolo dell'energia e della potenza sarebbe:
	
	\begin{gather*}
		\textbf{Energia: } E_{f} = \int_{-\infty}^{\infty}{e^{-4t}\:\mathrm{d}t} = \left.\dfrac{e^-4t}{-4}\right\vert_{-\infty}^{\infty} = \lim_{T\rightarrow\infty}{\dfrac{e^{-4t} - e^{4t}}{-4}} = \infty \\
		\textbf{Potenza: } P_{f} = \lim_{T\rightarrow\infty}{\dfrac{1}{T} \int_{-\frac{T}{2}}^{\frac{T}{2}}{e^{-4t}\:\mathrm{d}t}} = \lim_{T\rightarrow\infty}\left.\dfrac{e^{-4t}}{-4}\cdot\dfrac{1}{T}\right\vert_{-\frac{T}{2}}^{\frac{T}{2}} = \lim_{T\rightarrow\infty}\dfrac{e^{-2T} - e^{2T}}{-4T} = \infty
	\end{gather*}

	\noindent
	Come si evince dai calcoli, il segnale non è né di energia né di potenza perché entrambi i risultati sono uguali a infinito.\newline
	
	\noindent
	\textcolor{Green4}{\textbf{\emph{Risoluzione III.}}}
	
	Considerando la funzione $z(t)$, si osserva che è la copia simmetrica rispetto all'origine di $f(t)$. Invece, la funzione $v(t)$ è identica alla funzione $f(t)$ ma ``shiftata'' a sinistra di $4$:
	
	\begin{equation*}
		f(t) = -f(-t) \hspace{2em} v(t) = f(t + 4)
	\end{equation*}

	\newpage
	
	\noindent
	\textcolor{Red3}{\textbf{\emph{Esercizio 2.}}}
	
	\noindent
	Il secondo esercizio fornisce una funzione $f(t)$:
	
	\begin{equation*}
		f(t) = \mathrm{sgn }\left(a\cdot\cos{\left(\dfrac{2\pi}{T_0} t\right)}\right)
	\end{equation*}
	
	\noindent
	Con $T_0 = 2$. Le \textbf{richieste} dell'esercizio sono le seguenti:
	
	\begin{enumerate}[label=\Roman*]
		\item Rappresentare graficamente il segnale;
		
		\item Calcolare sia l'energia che la potenza media. Inoltre, dire se $f(t)$ è una funzione di energia o di potenza fornendo una motivazione valida.
	\end{enumerate}
	
	\noindent
	\textcolor{Green4}{\textbf{\emph{Risoluzione I.}}}
	
	\noindent
	Viene rappresentato il segnale della funzione segno $\mathrm{sng}$:
	
	\begin{figure}[!htp]
		\centering
		\includegraphics[width=0.4\textwidth]{img/ex_exam/sng_func_2.pdf}
		\caption{Funzione segno $\mathrm{sng}$.}
	\end{figure}

	\noindent
	Si esplicitando le operazioni della funzione:
	
	\begin{equation*}
		f(t) = \mathrm{sgn }\left(a\cdot\cos{\left(\dfrac{2\pi}{T_0} t\right)}\right) = \cos\left(\dfrac{1}{T_0}\cdot 2\pi t\right)
	\end{equation*}

	\noindent
	E si rappresenta inizialmente la funzione $\cos\left(2\pi\right)$ con $T_0 = 1$:
	
	\begin{figure}[!htp]
		\centering
		\includegraphics[width=0.4\textwidth]{img/ex_exam/sng_func_2-Mod.pdf}
		\caption{Funzione coseno $\cos\left(2\pi\right)$.}
	\end{figure}

	\newpage

	\noindent
	Si conclude la rappresentazione grafica aumentando $T_0$ in maniera molto semplice:
	
	\begin{figure}[!htp]
		\centering
		\includegraphics[width=0.4\textwidth]{img/ex_exam/sng_func_2-Mod2.pdf}
		\caption{Funzione coseno $\cos\left(2\pi\right)$ moltiplicata per $\dfrac{1}{T_0}=\dfrac{1}{2}$.}\label{ex2_grafico}
	\end{figure}
	\noindent
	\textcolor{Green4}{\textbf{\emph{Risoluzione II.}}}
	
	\noindent
	Si conclude l'esercizio calcolando l'energia o la potenza del segnale. Per farlo, dato che non è definito in un intervallo ma continua all'infinito, si calcolano i rispettivi integrali in un intervallo arbitrario $n$ e poi lo si estende all'infinito:
	
	\begin{gather*}
		E_{f} = \int_{-\infty}^{\infty} f^{2}(t) \:\mathrm{d}t = \lim_{n\rightarrow\infty} \int_{-n\cdot\frac{T_0}{2}}^{n\cdot\frac{T_0}{2}} f^{2}(t) \:\mathrm{d}t = \lim_{n\rightarrow\infty} n\cdot\int_{-\frac{T_0}{2}}^{\frac{T_0}{2}} f^{2} (t) \:\mathrm{d}t = \infty \\
		P_{f} = \lim_{T\rightarrow\infty} \dfrac{1}{T} \int_{-\frac{T}{2}}^{\frac{T}{2}} f^{2}(t) \:\mathrm{d}t = \lim_{n\rightarrow\infty} \dfrac{1}{nT_{0}} \int_{-n\frac{T_{0}}{2}}^{n\frac{T_{0}}{2}} f^{2}(t)\:\mathrm{d}t = 
		\cancel{\lim_{n\rightarrow\infty}} \dfrac{1}{\cancel{n}T_{0}} \cdot \cancel{n} \cdot \int_{-\frac{T_{0}}{2}}^{\frac{T_{0}}{2}} f^{2}(t)\:\mathrm{d}t = \\
		= \dfrac{1}{T_{0}} \cdot T_{0} = \dfrac{1}{2} \cdot 2 = 1 \longrightarrow \ne 0
	\end{gather*}

	\noindent
	È evidente che il segnale è di potenza. Come si evince dalla figura~\ref{ex2_grafico}, i tratti di colore verde indicano il rettangolo formato dal segnale. Calcolando l'area del rettangolo, si ottiene esattamente il valore di $T_{0}$. Infatti, la base del rettangolo (verticale) è $2$, mentre l'altezza (orizzontale) è $1$.
	
	\newpage
	
	\subsubsection{Cross-Correlazione Normalizzata}
	
	Ha l'\textbf{obbiettivo} di trattare segnali con range di valori diversi e consente di eseguire \textbf{confronti uno-a-molti} (\emph{one-to-many}):
	
	\begin{equation*}
		f_{1} \bar{\otimes} f_{2}\left(t\right) = \dfrac{\displaystyle \int_{-\infty}^{+\infty} \tilde{f_{1}}\left(\tau\right) f_{2}\left(\tau - t\right) \: \mathrm{d}\tau}{\displaystyle \sqrt{E_{f_{1}} E_{f_{2}}}}
	\end{equation*}
	
	\noindent
	In cui $E_{f}$ indica l'\textbf{energia} del segnale $f$. Ci sono due caratteristiche importanti:
	
	\begin{itemize}
		\item $f_{1} \bar{\otimes} f_{2}\left(t\right) \in \left[-1, 1\right]$
		
		\item $\left|f_{1} \bar{\otimes} f_{2}\left(t\right)\right| = 1 \iff f_{1}\left(\tau\right) = \alpha f_{2} \left(\tau - t\right)$
	\end{itemize}
	
	\noindent
	Inoltre, si parla di \textbf{autocorrelazione} (normalizzata e non) quando $f_{1} = f_{2}$. Utile per i segnali stocastici.\newline
	
	\noindent
	Nel \textbf{\emph{caso di segnali discreti}}, dati $x_{1}\left(k\right), x_{2}\left(k\right)$:
	
	\begin{equation*}
		x_{1} \otimes x_{2}\left(n\right) = \sum_{k = -\infty}^{+\infty} \tilde{x_{1}}\left(k\right) x_{2} \left(k - n\right) \hspace{1em} k \in \mathbb{Z}
	\end{equation*}

	\noindent
	Sotto l'ipotesi di convergenza della serie, cioè la serie deve convergere.\newline
	
	\noindent
	Nel caso in cui $x_{1}\left(k\right)$ e $x_{2}\left(k\right)$ sono limitati di lunghezza M ed N rispettivamente, allora la \textbf{cross correlazione è di lunghezza} $M+N-1$.
	
	\newpage
	
	\noindent
	\textcolor{Green4}{\textbf{Cross-Correlazione 1D}}\newline
	
	\noindent
	Data la definizione:
	
	\begin{equation*}
		x_{1} \otimes x_{2}\left(n\right) = \sum_{k = -\infty}^{+\infty} x_{1}\left(k\right) x_{2}\left(k - n\right)
	\end{equation*}
	
	\noindent
	Esistono diverse casistiche:
	
	\begin{itemize}
		\item $n = 0$ si confronta tra $x_{1}$ e $x_{2}$ nei loro domini temporali originali.
		
		\item $n > 0$ sposto $x_{2}$ a destra poiché c'è l'anticipo di $x_{2}$
		
		\item $n < 0$ sposto $x_{2}$ a sinistra poiché c'è ritardo di $x_{2}$
	\end{itemize}

	\begin{figure}[!htp]
		\centering
		\includegraphics[width=0.9\textwidth]{img/ex_exam/eg_cross-correlazione-1D.pdf}
		\caption{Esempio di cross-correlazione normalizzata 1D.}
	\end{figure}

	\noindent
	Il triangolo $x_{2}$ va verso sinistra e il lasso di tempo che $x_{2}$ non combacia con $x_{1}$, viene rappresentato come una linea orizzontale sull'asse delle $n$ nel piano cartesiano di destra.
	
	\newpage
	
	\noindent
	\textcolor{Green4}{\textbf{Cross-Correlazione 2D}}\newline
	
	\noindent
	Data la definizione:
	
	\begin{equation*}
		x_{1} \otimes x_{2}\left(m, n\right) = \sum_{u = -\infty}^{+\infty} \sum_{v = -\infty}^{+\infty} x_{1}\left(u, v\right) x_{2}\left(u - m, v - n\right) \hspace{1em} u,v,m,n \in \mathbb{Z}
	\end{equation*}
	
	\noindent
	Nel 2D $x_{1}$ e $x_{2}$ possono essere pensate come \textbf{immagini infinite}.\newline
	
	\noindent
	Di solito $x_{1}$ e $x_{2}$ sono \textbf{immagini finite} (segnali digitali ad intervallo limitato), e gli estremi di sommatoria sono quindi finiti.\newline
	
	\noindent
	Il primo segnale $x_{1}$ viene chiamato \textbf{\emph{template}}, o \textbf{\emph{matrice kernel}}, mentre $x_{2}$ genericamente \textbf{immagine} (di solito, la matrice kernel $x_{1}$ ha una dimensionalità minore di quella dell'immagine).\newline
	
	\noindent
	Nel caso $x_{1} = x_{2}$ si ha \textbf{autocorrelazione 2D}.\newline
	
	\noindent
	\textcolor{Green4}{\textbf{Cross-Correlazione normalizzata 2D}}\newline
	
	\noindent
	Si definisce come:
	
	\begin{equation*}
		x_{1} \otimes x_{2}\left(m,n\right) = \dfrac
		{\sum_{u = -\infty}^{+\infty} \sum_{v = -\infty}^{+\infty} \left[x_{1}\left(u, v\right)\right]\left[x_{2}\left(u-m, v-n\right)\right]}
		{\sqrt{\sum_{u = -\infty}^{+\infty} \sum_{v = -\infty}^{+\infty} \left[x_{1}\left(u,v\right)\right]^{2} {\sum_{u = -\infty}^{+\infty} \sum_{v = -\infty}^{+\infty} \left[x_{2}\left(u,v\right)\right]^{2}}}}
	\end{equation*}

	\noindent
	In altre parole, fissato il punto di applicazione $n, m$, si sottrae la media ad ogni punto nell'interno di applicazione dalla matrice kernel. Successivamente, si divide per il prodotto della varianza dei due segnali, estraendo a radice alla fine.
	
	\begin{figure}[!htp]
		\centering
		\includegraphics[width=1\textwidth]{img/ex_exam/eg_cross-correlazione-2D.pdf}
		\caption{Esempio di Cross-Correlazione normalizzata 2D.}
	\end{figure}

	\newpage
	
	\noindent
	\textcolor{Red3}{\textbf{Esercizio Cross-Correlazione 2D}}\newline
	
	\noindent
	Dati le due immagini $x_{1}$ di dimensione $5 \times 5$ e $x_{2}$ di dimensione $3 \times 3$, si calcola la cross-correlazione 2D. Quindi, si effettua la rappresentazione grafica.
	
	\begin{figure}[!htp]
		\centering
		\includegraphics[width=0.5\textwidth]{img/cross-correlazione-2D_ex1.pdf}
		\caption{Piano cartesiano di $x_{2}$ di dimensione $5 \times 5$.}
		\includegraphics[width=0.5\textwidth]{img/cross-correlazione-2D_ex1_1.pdf}
		\caption{Piano cartesiano di $x_{1}$ di dimensione $3 \times 3$.}
	\end{figure}

	\noindent
	E vengono fornite dall'esercizio le due matrici:
	
	\begin{equation*}
		x_{2} =
		\begin{bmatrix}
			1 & 0 & 0 & 0 & 0 \\
			0 & 1 & 0 & 0 & 0 \\
			0 & 0 & 1 & 0 & 0 \\
			0 & 1 & 1 & 0 & 1 \\
			0 & 0 & 0 & 1 & 0 \\
		\end{bmatrix}
		\hspace{2em}
		x_{1} =
		\begin{bmatrix}
			1 & 0 & 0 \\
			0 & 1 & 0 \\
			0 & 0 & 1 \\
		\end{bmatrix}		
	\end{equation*}

	\noindent
	Esse indicano i valori nei punti corrispondenti. L'\textbf{obbiettivo dell'esercizio} è trovare:
	
	\begin{itemize}
		\item L'argomento massimo della cross-correlazione ($\arg \max x_{1} \otimes x_{2}\left(m,n\right)$);
		
		\item Il massimo della cross-correlazione ($\max x_{1} \otimes x_{2} \left(m,n\right)$).
	\end{itemize}

	\noindent
	L'argomento massimo è con i valori $m = 1$ e $n = -1$ poiché così facendo la diagonale incontra tutti i valori positivi e che formano il massimo. Infatti, prendendo in considerazione la matrice $x_{2}\:5 \times 5$ e osservando l'operazione di cross-correlazione 2D:
	
	\begin{gather*}
		\sum_{u} \sum_{v} x_{1} \left(u,v\right) \cdot x_{2}\left(u - m, v - n\right) \\
		\xrightarrow{\text{sostituzione termini noti } (m, n)} \sum_{u} \sum_{v} x_{1} \left(u,v\right) \cdot x_{2}\left(u - 1, v - \left(-1\right)\right)
	\end{gather*}

	\noindent
	Risulta evidente come si debba spostare a destra, rispetto l'origine, la matrice $x_{2}$ di un solo valore\footnote{Shift a destra poiché $u-1$ nell'equazione rappresenta un ritardo.} e sotto, rispetto sempre l'origine, di un valore negativo\footnote{Spostamento sotto l'asse delle ascisse poiché è un valore positivo $v+1$.}. Così facendo, la diagonale della matrice $x_{2}$ corrisponderà esattamente a tutti i valori $1$ della matrice $x_{1}$.

	\newpage
	
	\subsubsection{Convoluzione}
	
	La \textbf{convoluzione} è un parente stretto della cross-correlazione, ma è leggermente diverso. È definito nel seguente modo:
	
	\begin{equation*}
		f_{1} * f_{2}\left(t\right) = \int_{-\infty}^{+\infty} f_{1}\left(\tau\right) f_{2}\left(t - \tau\right) \mathrm{d}\tau
	\end{equation*}

	\noindent
	Con $t in \mathbb{R}$. Si ricordi che se i \textbf{segnali non sono né di energia né di potenza, l'integrale \underline{converge}}.\newline
	
	\noindent
	Nel caso in cui i \textbf{segnali} siano \textbf{discreti}, dati $x_{1}\left(n\right)$, $x_{1}\left(n\right)$:
	
	\begin{equation*}
		x_{1} * x_{2} \left(n\right) = \sum_{k = -\infty}^{+\infty} x_{1}\left(k\right) x_{2}\left(n - k\right)
	\end{equation*}

	\noindent
	Con $k \in \mathbb{Z}$.
	
	\noindent
	Nel caso in cui $x_{1} \left(n\right)$ e $x_{2} \left(n\right)$ sono limitati di lunghezza $M$ ed $N$ rispettivamente, allora la \textbf{convoluzione è di lunghezza} $M+N-1$.\newline
	
	\noindent
	\textcolor{Green4}{\textbf{Convoluzione 2D}}\newline
	
	\noindent
	Nel caso delle immagini, quindi del 2D, $x_{1}$ ed $x_{2}$ sono solitamente \textbf{segnali digitali ad intervallo limitato}, e la convoluzione diventa dunque:
	
	\begin{equation*}
		x_{1} * x_{2}\left(m,n\right) = \sum_{u = -\infty}^{+\infty} \sum_{v = -\infty}^{+\infty} x_{1} \left(u,v\right) x_{2}\left(m - u, n - v\right) \hspace{2em} u,v,m,n\in\mathbb{Z}
	\end{equation*}

	\noindent
	Solitamente il primo segnale $x_{1}$ viene chiamato \textbf{\underline{filtro}}, o \textbf{\underline{matrice kernel}}, mentre $x_{2}$ genericamente \textbf{\underline{immagine}} (solitamente la matrice kernel ha una dimensione inferiore di quella dell'immagine).
	
	\newpage
	
	\section{Analisi di Fourier}
	
	\subsection{Serie di Fourier}\label{serie di fourier}
	
	Una funzione, chiamata \textbf{\underline{funzione di sintesi}}, $f: \mathbb{R} \rightarrow \mathbb{R}$ di variabile continua $t$, periodica di periodo $T$, si esprime come:
	
	\begin{equation*}
		f(t) = \sum_{n = -\infty}^{+\infty} c_{n} \underbrace{e^{j \frac{2\pi n}{T}t}}_{\text{fasore}} \hspace{2em} n\in\mathbb{Z}
	\end{equation*}

	\noindent
	Dove $c_{n}$ è un numero complesso. Invece, una \textbf{\underline{funzione di analisi}} è espressa come:
	
	\begin{equation*}
		c_{n} \in \mathbb{C} = \dfrac{1}{T} \int_{-\frac{T}{2}}^{+\frac{T}{2}} f\left(t\right) \underbrace{e^{-j \frac{2\pi n}{T}t}}_{\text{fasore}} \mathrm{d}t \hspace{2em} n \in \mathbb{Z}
	\end{equation*}

	\noindent
	\textbf{N.B.} si ricorda che $e^{j \frac{2\pi n}{T}t}$ è un \textbf{fasore rotante} di velocità angolare $\dfrac{2\pi n}{T} t$.\newline
	La \textbf{\underline{funzione di sintesi}} quindi non è altro che una somma di infiniti termini. Ciascuno è composto dalla moltiplicazione tra un numero complesso ed un fasore, il quale \emph{produce un altro fasore}. Esprimendo $c_{n}$ come numero complesso in forma polare:
	
	\begin{equation*}
		c_{n} e^{j \frac{2 \pi n}{T} t} = |c_{n}| e^{j \theta_{n}} e^{j \frac{2 \pi n}{T} t} = |c_{n}| e^{j \left(\frac{2 \pi n}{T} t + \theta_{n}\right)}
	\end{equation*}

	\noindent
	Si può notare come questa conversione corrisponda ad \textbf{estendere} il fasore $e^{j\frac{2 \pi n}{T} t}$ ad una lunghezza $|c_{n}|$ facendolo partire con un \textbf{angolo di partenza} uguale a $\theta_{n}$ (chiamato \textbf{\underline{angolo di fase}}).\newline
	
	\noindent
	\textbf{\underline{Altra osservazione:}} se $c_{n}$ appartiene all'insieme $\mathbb{R}$, significa che $\theta_{n}$ non compare. Questo comporta un cambiamento nella lunghezza dell'$n$-esimo fasore pari a $|c_{n}|$:
	
	\begin{equation*}
		c_{n} = |c_{n}| \cancel{e^{j \theta_{n}}}
	\end{equation*}

	\newpage
	
	\noindent
	\textcolor{Green4}{\textbf{\underline{Esempio 1}}}\newline
	
	\noindent
	Il primo esempio di serie di Fourier si applica per il segnale trigonometrico:
	
	\begin{equation*}
		f\left(t\right) = \cos\left(2 \pi t\right) \hspace{2em} \text{con } T = 1
	\end{equation*}

	\noindent
	Applicando la \textbf{funzione di analisi} e saltando i passaggi perché complessi, si ottengono i seguenti valori:
	
	\begin{equation*}
		c_{-1} = \dfrac{1}{2} \hspace{3em}
		c_{0} = 0 \hspace{3em}
		c_{1} = \dfrac{1}{2} \hspace{3em}
		c_{i \le -2, i \ge 2} = 0
	\end{equation*}

	\noindent
	E sostituendo nella \textbf{funzione di sintesi}:
	
	\begin{equation*}
		\cos\left(2 \pi t\right) = \dfrac{1}{2} e^{-j 2 \pi t} + \dfrac{1}{2} e^{j 2 \pi t} = \dfrac{e^{j 2 \pi t} + e^{-j 2 \pi t}}{2}
	\end{equation*}

	\noindent
	Ci sono \textbf{tre osservazioni} da fare:
	
	\begin{enumerate}[label=\Roman*.]
		\item $\dfrac{2\pi}{T} = f_{0}$;
		
		\item $c_{n} = |c_{n}| e^{j\theta_{n}}$;
		
		\item In questo caso, $c_{n} \in \mathbb{R}$ quindi l'angolo di fase non è presente.
	\end{enumerate}

	\noindent
	\textbf{Le parti} dell'equazione sono le seguenti:
	
	\begin{equation*}
		\cos{\left(2 \pi t\right)} = \dfrac{1}{2} e^{-j 2 \pi t} + \dfrac{1}{2} e^{j 2 \pi t}
	\end{equation*}

	\begin{itemize}[label=\ding{43}]
		\item $\cos{\left(2 \pi t\right)} \rightarrow$ La funzione trigonometrica da studiare
		
		\item $\dfrac{1}{2} e^{-j 2 \pi t} \rightarrow$ Fasore di modulo $0.5$ e velocità angolare $-2\pi t$
		
		\item $\dfrac{1}{2} e^{j 2 \pi t} \rightarrow$ Fasore di modulo $0.5$ e velocità angolare $2\pi t$
	\end{itemize}

	\begin{figure}[!htp]
		\centering
		\includegraphics[width=0.5\textwidth]{img/fourier_eg1.pdf}
		\caption{Grafico rappresentante i due fasori. La freccia verde rappresenta il valore assunto da $\cos{\left(2\pi t\right)}$ per $t = \dfrac{1}{8}$.}
	\end{figure}

	\noindent
	I coefficienti $c_{n = -1}$ e $c_{n = 1}$ sono relativi ai \textbf{\underline{moduli o ampiezze}} \textbf{dei fasori} complessi di frequenza $f_{0} \cdot n$ con $n = -1,1$ e ricordando che:
	
	\begin{equation*}
		\exp\left(j \left(\dfrac{2 \pi n}{T} t\right)\right) = \exp\left(j \left(f_{0} n t\right)\right)
	\end{equation*}

	\noindent
	Che si possono annotare con le variabili $f_{-1}$ e $f_{1}$ per $f_{0} \cdot n$ con $n = -1, 1$ e analogamente per gli altri $n \in \mathbb{Z}$.\newline
	
	\noindent
	Inoltre, è possibile disegnare lo \textbf{\underline{spettro di ampiezza}} che \textbf{mostra i moduli dei fasori costruiti con la trasformata di Fourier}, in particolare la funzione di sintesi.
	
	\begin{figure}[!htp]
		\centering
		\includegraphics[width=0.9\textwidth]{img/fourier_spettro_di_ampiezza.pdf}
		\caption{Grafico che rappresenta lo spettro di ampiezza.}
	\end{figure}

	\newpage
	
	\noindent
	\textcolor{Green4}{\textbf{\underline{Esempio 2}}}\newline
	
	\noindent
	Il secondo esempio di serie di Fourier è il segnale trigonometrico:
	
	\begin{equation*}
		f\left(t\right) = \sin\left(2 \pi t\right) \hspace{2em} \text{con } T = 1
	\end{equation*}

	\noindent
	Applicando la \textbf{funzione di analisi} e saltando i passaggi perché complessi, si ottengono i seguenti valori:
	
	\begin{equation*}
		c_{-1} = -\dfrac{1}{2j} \hspace{3em}
		c_{0} = 0 \hspace{3em}
		c_{1} = \dfrac{1}{2j} \hspace{3em}
		c_{i \le -2, i \ge 2} = 0
	\end{equation*}

	\noindent
	Dove questa volta $c_{n} \in \mathbb{C}$ ed in particolare:
	
	\begin{equation*}
		\pm \dfrac{1}{2j} = \pm \dfrac{1}{2j} \cdot \dfrac{j}{j} = \pm \dfrac{1}{2} \cdot \dfrac{j}{j^{2}} = j \cdot \mp \dfrac{1}{2}
	\end{equation*}

	\noindent
	Si passa alla forma di esponenziale complesso:
	
	\begin{gather*}
		j \cdot \dfrac{1}{2} = 0 + j \cdot \dfrac{1}{2} \\
		|c| = \sqrt{0^{2} + \left(\dfrac{1}{2}\right)^{2}} = \dfrac{1}{2} \\
		\theta = \arctan\left(\dfrac{0.5}{0}\right) \rightarrow \dfrac{\pi}{2} \\
		\dfrac{1}{2} e^{j \cdot \frac{\pi}{2}} = c_{-1}
	\end{gather*}

	\begin{figure}[!htp]
		\centering
		\includegraphics[width=0.3\textwidth]{img/fourier_eg2.pdf}
		\caption{Grafico di $c_{-1}$.}
	\end{figure}

	\newpage
	\noindent
	Analogamente:
	
	\begin{gather*}
		j \cdot -\dfrac{1}{2} = 0 + j \cdot \left(-\dfrac{1}{2}\right) \\
		|c| = \sqrt{0^{2} + \left(-\dfrac{1}{2}\right)^{2}} = \dfrac{1}{2} \\
		\theta = \arctan\left(-\dfrac{0.5}{0}\right) \rightarrow -\dfrac{\pi}{2} \\
		\dfrac{1}{2} e^{j \cdot \left(-\frac{\pi}{2}\right)} = c_{1}
	\end{gather*}

	\begin{figure}[!htp]
		\centering
		\includegraphics[width=0.3\textwidth]{img/fourier_eg2_2.pdf}
		\caption{Grafico di $c_{1}$.}
	\end{figure}

	\newpage
	\noindent
	Applicando l'\textbf{equazione di sintesi} e sostituendo i termini noti:
	
	\begin{gather*}
		\sin{\left(2 \pi t\right)} = \sum_{n = -\infty}^{+\infty} c_{n} e^{j \frac{2 \pi n}{T} t} = c_{-1} e^{j-2\pi t} + c_{1} e^{j2\pi t} \\
		\xrightarrow{\text{sostituzione dei termini noti }c_{-1}, c_{1}} = \dfrac{1}{2} e^{j \frac{\pi}{2}} e^{j \cdot \left(-2 \pi t\right)} + \dfrac{1}{2}  e^{j \frac{-\pi}{2}} e^{j 2 \pi t} \\
		\xrightarrow{\text{forma finale }} = \dfrac{1}{2} \exp{\left(j\left(-2 \pi t + \dfrac{\pi}{2}\right)\right)} + \exp{\left(j\left(2 \pi t - \dfrac{\pi}{2}\right)\right)}
	\end{gather*}

	\begin{figure}[!htp]
		\centering
		\includegraphics[width=0.5\textwidth]{img/fourier_eg2_3.pdf}
		\caption{Grafico finale.}
	\end{figure}

	\noindent
	Infine, si disegna lo \textbf{\underline{spettro di ampiezza}} e lo \textbf{\underline{spettro di fase}}, quest'ultimo è un \textbf{grafico in cui si riportano gli angoli di fase della funzione}.
	
	\begin{figure}[!htp]
		\centering
		\includegraphics[width=0.9\textwidth]{img/fourier_spettro_di_ampiezza-e-fase.pdf}
		\caption{Spettro di ampiezza e di fase della funzione $\sin\left(2 \pi t\right)$.}
	\end{figure}

	\newpage
	
	\subsubsection{Proprietà della serie di Fourier}
	
	\noindent
	Lo \textbf{spettro di ampiezza e di fase} sono funzioni nel dominio delle frequenze che formano lo {\underline{\textcolor{Red3}{\textbf{spettro di Fourier}}}}. Lo spettro di Fourier per i segnali periodici gode delle \textbf{seguenti \underline{proprietà}}:
	
	\begin{itemize}
		\item Lo \textbf{spettro di ampiezza è \emph{simmetrico}} rispetto all'asse $y$;
		
		\item Lo \textbf{spettro di fase è \emph{antisimmetrico}} rispetto all'asse $y$;
		
		\item Se i coefficienti $c_{n}$ sono reali, \textbf{non esiste lo spettro di fase};
		
		\item Entrambe gli spettri sono funzioni a pettine\label{funzioni a pettine}, definite su frequenze multiple rispetto a quella fondamentale:
		
		\begin{equation*}
			\left\{\dfrac{2 \pi n}{T}\right\}_{n \in \mathbb{Z}} = \left\{f_{0} \cdot n\right\}_{n \in \mathbb{Z}} \equiv \left\{f_{n}\right\}_{n \in \mathbb{Z}}
		\end{equation*}
	\end{itemize}

	\newpage
	
	\subsection{Trasformata di Fourier continua} \label{trasformata di fouerier continua}
	
	\subsubsection{Trasformata di Fourier}
	
	Sia $f\left(t\right)$ un \textbf{segnale reale continuo} $f:\mathbb{R} \rightarrow \mathbb{R}$ periodico o non, si definisce la \textcolor{Red3}{\textbf{Trasformata di Fourier}} (TdF) $\mathcal{F}\left(f\left(t\right)\right) = F\left(\mu\right)$ il segnale $\mathcal{F}: \mathbb{R} \rightarrow \mathbb{C}$:
	
	\begin{equation*}
		\mathcal{F}\left(f\left(t\right)\right) = F\left(\mu\right) = \int_{-\infty}^{+\infty} f\left(t\right) e^{-j 2 \pi \mu t} \: \mathrm{d}t
	\end{equation*}

	\noindent
	L'\textbf{unità frequenziale} $\mu$ è l'angolo di $\frac{n}{T}$ della serie di Fourier (per esempio, con $n = 1$, $T = 1$ sec. $\rightarrow \mu = 1 \text{ sec.}^{-1} = 1$ Hz).\newline
	
	\noindent
	La Trasformata di Fourier esiste se $f\left(t\right)$ è un \textbf{segnale di energia}. Condizione sufficiente e non necessaria perché altri segnali ammettono la TdF.
	
	\subsubsection{Trasformata di Fourier inversa}
	
	Sia $F\left(\mu\right)$ la trasformata di Fourier di un segnale $f: \mathbb{R} \rightarrow \mathbb{R}$. Si definisce la \textcolor{Red3}{\textbf{trasformata di Fourier inversa}} il segnale $\mathcal{F}^{-1} \left(F\left(\mu\right)\right) = f\left(t\right)$:
	
	\begin{equation*}
		\mathcal{F}^{-1} = \left(F\left(\mu\right)\right) = f\left(t\right) = \int_{-\infty}^{+\infty} F\left(\mu\right) e^{j 2 \pi \mu t} \: \mathrm{d}\mu
	\end{equation*}

	\noindent
	La trasformata di Fourier restituisce, per una data frequenza $\mu$, un coefficiente di \dquotes{presenza} $F\left(\mu\right)$. Infatti, la sua inversa \textbf{\underline{permette di ricostruire $f$ a partire da $F$}}.
	
	\begin{figure}[!htp]
		\centering
		\includegraphics[width=1\textwidth]{img/trasformata_fourier1.pdf}
		\caption{(Anti)Trasformata di Fourier su un segnale $f\left(t\right)$.}
	\end{figure}

	\newpage

	\noindent
	Nel caso in cui il segnale $f\left(t\right)$ non è reale, la trasformata è complessa:
	
	\begin{itemize}
		\item $t$ rappresenta il \textbf{\underline{tempo}} (in secondi), allora $\mu$ rappresenta gli \textbf{Hertz}, cioè $\frac{\text{numero cicli}}{\text{secondi}}$;
		
		\item $t$ rappresenta lo \textbf{\underline{spazio}} (in metri), allora $\mu$ rappresenta la \textbf{frequenza spaziale}, cioè $\frac{\text{numero cicli}}{\text{metri}}$
	\end{itemize}

	\noindent
	Mentre nella serie di Fourier le funzioni rappresentate negli spettri di ampiezza e di fase erano a \dquotes{pettine} (paragrafo~\ref{funzioni a pettine}), in questo caso le funzioni sono solitamente continue, nello spettro di ampiezza, o continue a tratti:
	
	\begin{figure}[!htp]
		\centering
		\includegraphics[width=0.9\textwidth]{img/trasformata_fourier2.pdf}
		\caption{Esempio di spettro di ampiezza.}
	\end{figure}
	
	\newpage
	
	\subsubsection{Proprietà della trasformata di Fourier}
	
	\begin{itemize}[label=\ding{42}]
		\item \textcolor{Red3}{\textbf{\underline{Linearità}}}
			\begin{equation*}
				a_{1} f_{1}\left(t\right) + a_{2}f_{2}\left(t\right) \hspace{1em} \xrightarrow{\mathcal{F}} \hspace{1em} a_{1}F_{1}\left(\mu\right) + a_{2}F_{2}\left(\mu\right)
			\end{equation*}
		
		\item \textcolor{Red3}{\textbf{\underline{Scalatura temporale}}}
			\begin{equation*}
				z\left(t\right) = f\left(at\right) \hspace{1em} \xrightarrow{\mathcal{F}} \hspace{1em} Z\left(\mu\right) = \dfrac{1}{a} \: F\left(\dfrac{\mu}{a}\right)
			\end{equation*}
		
		\item \textcolor{Red3}{\textbf{\underline{Dualità}}}
			\begin{gather*}
				f\left(t\right) \hspace{1em} \xrightarrow{\mathcal{F}} \hspace{1em} F\left(\mu\right) \\
				F\left(t\right) \hspace{1em} \xrightarrow{\mathcal{F^{-1}}} \hspace{1em} f\left(-\mu\right)
			\end{gather*}
			N.B. derivando la forma analitica per una trasformata, la sua antitrasformata ne produce un'altra con segno opposto.
			
		\item \textcolor{Red3}{\textbf{\underline{Time shift}}}
			\begin{equation*}
				\begin{matrix}
					\mathcal{F}\left(f\left(t-t_{0}\right)\right) & = & \displaystyle \int_{-\infty}^{+\infty} f\left(t-t_{0}\right) e^{-j 2 \pi \mu t} \: \mathrm{d}t \\
					\\
					& = & \displaystyle \int_{-\infty}^{+\infty} f\left(u\right) e^{-j 2 \pi \mu \left(u + t_{0}\right)} \: \mathrm{d}u \\
					\\
					& = & \displaystyle \int_{-\infty}^{+\infty} f\left(u\right) e^{-j 2 \pi \mu u} e^{-j 2 \pi \mu t_{0}} \: \mathrm{d}u \\
					\\
					& = & \displaystyle e^{-j 2 \pi \mu t_{0}} \: \int_{-\infty}^{+\infty} f\left(u\right) e^{-j 2 \pi \mu u} \: \mathrm{d}u \\
					\\
					& = & F\left(\mu\right) e^{\overbrace{- j 2 \pi \mu t_{0}}^{fase}}
				\end{matrix}
			\end{equation*}
	\end{itemize}

	\newpage
	
	\subsubsection{Trasformata di Fourier di una box}
	
	La trasformata di Fourier di una box (paragrafo~\ref{funzione box e impulso di Dirac}) è la seguente:
	
	\begin{equation*}
		\mathcal{F}\left(f\left(t\right)\right) = \int_{-\infty}^{+\infty} A \Pi \left(\dfrac{t}{w}\right) e^{-j 2 \pi \mu t} \: \mathrm{d}t = F\left(\mu\right)
	\end{equation*}

	\noindent
	Il \textbf{\underline{risultato corrisponde alla funzione}} $\mathrm{sinc}$:
	
	\begin{equation*}
		f\left(\mu\right) = Aw \cdot \mathrm{sinc }\left(\mu w\right)
	\end{equation*}
	
	\noindent
	Dove la funzione $\mathrm{sinc}$ è uguale a:
	
	\begin{equation*}
		\mathrm{sinc} = \dfrac{\sin\left(\pi \mu w\right)}{\pi \mu w}
	\end{equation*}

	\noindent
	Per ripassare la funzione $\mathrm{sinc}$, si rimanda al paragrafo~\ref{funzione sinc}. Tuttavia, si ricorda che la sua forma generale è del tipo:
	
	\begin{equation*}
		\mathrm{sinc} \left(m\right) = \dfrac{\sin \left(\pi m\right)}{\pi m}
	\end{equation*}

	\noindent
	E risultata uguale a:
	
	\begin{itemize}
		\item $\mathrm{sinc}\left(0\right) = 1$
		\item $\mathrm{sinc}\left(m\right) = 0 \hspace{1em} \forall m \in \mathbb{Z}$
	\end{itemize}

	\noindent
	Prima di concludere, si ricorda che:
	
	\begin{itemize}[label=\ding{43}]
		\item All'\textbf{aumentare} della \textbf{larghezza} della box, la funzione $\mathrm{sinc}$ tenderà a \textbf{stringersi};
		
		\item La box è \textbf{limita}, invece la $\mathrm{sinc}$ è \textbf{infinita} a destra e sinistra, anche se il termine al denominatore attenua il valore della funzione comportando un limite a $0$.
			
		\item In sintesi, la TdF di una box è:
			\begin{equation*}
				\Pi\left(\dfrac{t}{w}\right) \hspace{1em} \xrightarrow{\mathcal{F}} \hspace{1em} w \cdot \mathrm{sinc} \left(\mu w\right)
			\end{equation*}
	\end{itemize}

	\begin{figure}[!htp]
		\centering
		\includegraphics[width=0.8\textwidth]{img/sinc.pdf}
		\caption{Grafico della funzione sinc.}\label{grafico sinc}
	\end{figure}

	\newpage

	\subsubsection[Trasformata di Fourier di un $\mathrm{sinc}$]{Trasformata di Fourier di un $\boldsymbol{\mathrm{sinc}}$}
	
	La trasformata di Fourier di un segnale $\mathrm{sinc}$ (segnale rappresentato in figura~\ref{grafico sinc}) è la seguente:
	
	\begin{equation*}
		\mathcal{F}\left(f\left(t\right)\right) = \int_{-\infty}^{+\infty} \mathrm{sinc}\left(tw\right) e^{-j 2 \pi \mu t} \: \mathrm{d}t = F\left(\mu\right)
	\end{equation*}

	\noindent
	Dato che la TdF di una box è:
	
	\begin{equation*}
		\Pi\left(\dfrac{t}{w}\right) \hspace{1em} \xrightarrow{\mathcal{F}} \hspace{1em} w \cdot \mathrm{sinc} \left(\mu w\right)
	\end{equation*}

	\noindent
	Al contrario, si ottiene la \textbf{\underline{trasformata di Fourier di un $\mathrm{sinc}$}}:
	
	\begin{equation*}
		\mathrm{sinc}\left(tw\right) \hspace{1em} \xrightarrow{\mathcal{F}} \hspace{1em} \dfrac{1}{w} \Pi \left(-\dfrac{\mu}{w}\right) = \dfrac{1}{w} \cdot \Pi \left(\dfrac{\mu}{w}\right)
	\end{equation*}

	\newpage

	\subsubsection{Trasformata di Fourier di un impulso}
	
	La trasformata di Fourier di un impulso\footnote{Definizione di impulso al paragrafo~\ref{funzione box e impulso di Dirac}.} è la seguente:
	
	\begin{equation*}
		\mathcal{F}\left(f\left(t\right)\right) = F\left(\mu\right) = \int_{-\infty}^{+\infty} \delta\left(t\right) e^{-j 2 \pi \mu t} \: \mathrm{d}t
	\end{equation*}

	\noindent
	Il risultato della trasformata di Fourier di un impulso è molto semplice grazie alle sue proprietà. Infatti, il risultato è uguale a:
	
	\begin{equation*}
		\int_{-\infty}^{+\infty} \delta\left(t\right) e^{-j 2 \pi \mu t} \: \mathrm{d}t = \int_{-\infty}^{+\infty} \delta\left(0\right) e^{-j 2 \pi \mu 0} \: \mathrm{d}t = 1
	\end{equation*}

	\noindent
	La proprietà che consente di ottenere il risultato uguale a $1$ è la seguente:
	
	\begin{equation*}
		\delta\left(t\right) =
		\begin{cases}
			\infty  & \text{se } t = 0 \\
			0		& \text{se } t \ne 0
		\end{cases}
		\hspace{2em} \longrightarrow \hspace{2em}
		\int_{-\infty}^{+\infty}\delta\left(t\right)\mathrm{d}t = 1
	\end{equation*}

	\noindent
	\textbf{\underline{N.B. In questo caso è rappresentabile solo lo spettro di ampiezza!}}\newline
	
	\noindent
	Analogamente, con un impulso centrato in $t_{0}$, quindi non nell'origine:
	
	\begin{equation*}
		\mathcal{F}\left(f\left(t\right)\right) = F\left(\mu\right) = \int_{-\infty}^{+\infty} \delta\left(t-t_{0}\right) e^{-j 2 \pi \mu t} \: \mathrm{d}t = e^{-j 2 \pi \mu t_{0}}
	\end{equation*}

	\noindent
	Il risultato è stato ottenuto grazie alla proprietà di setacciamento (definita a pagina~\pageref{setacciamento}). Tuttavia, in questo caso i valori non sono più reali ma complessi.
	
	\newpage
	
	\subsubsection{Trasformata di Fourier di un treno di impulsi}
	
	Data la definizione di treno di impulsi (funzione definita nel paragrafo~\ref{treno_di_impulsi}):
	
	\begin{equation*}
		S_{\Delta T}\left(t\right) = \sum_{n = -\infty}^{+\infty} \delta\left(t - n\Delta T\right) \hspace{1em} n \in \mathbb{Z}
	\end{equation*}

	\noindent
	Si ottiene la sua relativa trasformata di Fourier:
	
	\begin{equation*}
		\mathcal{F}\left(S_{\Delta T}\left(t\right)\right) = \int_{-\infty}^{+\infty} S_{\Delta T}\left(t\right) e^{-j 2 \pi \mu t}\: \mathrm{d}t = F\left(\mu\right)
	\end{equation*}

	\noindent
	Tralasciando i vari calcoli numerici per arrivare al risultato, si può scrivere la trasformata di Fourier in maniera più semplice:
	
	\begin{equation*}
		S_{\Delta T}\left(t\right) \hspace{2em} \xrightarrow{\mathcal{F}} \hspace{2em} \sum_{n = -\infty}^{+\infty} \dfrac{1}{\Delta T} \delta\left(\mu - \dfrac{n}{\Delta T}\right)
	\end{equation*}

	\newpage
	
	\subsubsection{Sintesi}
	
	Qui di seguito si lascia un riassunto rapido delle trasformate di Fourier continue dei segnali più importanti.
	
	\begin{table}[!htbp]
		\centering
		\begin{tabular}{@{} l l c l @{}}
			\toprule
			Segnale & & & Trasformata di Fourier \\
			\midrule
			Box:				& $A\Pi\left(\dfrac{t}{w}\right)$	& $\xrightarrow{\mathcal{F}}$ & $Aw \cdot \mathrm{sinc}\left(\mu w\right)$ \\
			&&&\\
			Sinc:				& $\mathrm{sinc}\left(tw\right)$	& $\xrightarrow{\mathcal{F}}$ & $\dfrac{1}{w}\cdot\Pi \left(-\dfrac{\mu}{w}\right) = \dfrac{1}{w}\cdot \Pi\left(\dfrac{\mu}{w}\right)$ \\
			&&&\\
			Impulso:			& $\delta\left(t\right)$			& $\xrightarrow{\mathcal{F}}$ &
			$\begin{cases}
				1 						& \text{se valori reali}\\
				e^{-j 2 \pi \mu t_{0}}	& \text{se valori complessi}
			\end{cases}$ \\
			&&&\\
			Treno di impulsi:	& $S_{\Delta T}\left(t\right)$ 		& $\xrightarrow{\mathcal{F}}$ & $\displaystyle\sum_{n = -\infty}^{+\infty} \dfrac{1}{\Delta T} \cdot \delta \left(\mu - \dfrac{n}{\Delta T}\right)$ \\
			\bottomrule
		\end{tabular}
		\caption{Trasformate di Fourier continue.}
	\end{table}

	\newpage
	
	\subsection{Trasformata di Fourier a tempo discreto}\label{trasformata di fourier a tempo discreto}
	
	\subsubsection{Campionamento}
	
	Sia $f\left(t\right)$ un segnale reale continuo definito $f : \left] -\infty, + \infty \right[ \in \mathbb{R} \rightarrow \mathbb{R}$ (attenzione al dominion non limitato), anche non periodico:
	
	\begin{figure}[!htp]
		\centering
		\includegraphics[width=0.65\textwidth]{img/campionamento_1.png}
	\end{figure}

	\noindent
	Questo tipo di segnale, per essere elaborato al computer deve essere \textbf{campionato} ad intervalli discreti. Per farlo, si prenda in considerazione il treno di impulsi:
	
	\begin{equation*}
		S_{\Delta T} \left(t\right) = \sum_{n = -\infty}^{+\infty} \delta\left(t - n\Delta T\right)
	\end{equation*}

	\noindent
	Con periodo $\Delta T$, ossia con \textbf{frequenza di campionamento} pari a: $\mu_{S} = \dfrac{1}{\Delta T}$

	\begin{figure}[!htp]
		\centering
		\includegraphics[width=0.65\textwidth]{img/campionamento_2.png}
	\end{figure}

	\noindent
	Si assuma che il treno di impulsi sia un \textbf{segnale discreto}. Matematicamente parlando, \textbf{campionare un segnale \underline{significa} moltiplicarlo per un treno di impulsi}:
	
	\begin{equation*}
		\tilde{f}\left(t\right) = f\left(t\right) \cdot S_{\Delta T}\left(t\right) = f\left(t\right) \cdot \sum_{n = -\infty}^{+\infty} \delta\left(t - n \Delta T\right)
	\end{equation*}

	\begin{figure}[!htp]
		\centering
		\includegraphics[width=0.65\textwidth]{img/campionamento_3.png}
	\end{figure}
	
	\newpage
	
	\subsubsection{Trasformata di Fourier a tempo discreto}
	
	Sia $F\left(\mu\right)$ la trasformata di Fourier di un segnale $f\left(t\right): \mathbb{R} \rightarrow \mathbb{R}$. Si considera $\tilde{f}\left(t\right)$ e si calcola la trasformata di Fourier $\tilde{F}\left(\mu\right)$ (entrambi sono a tempo discreto). Grazie alla convoluzione si ottiene:
	
	\begin{equation*}
		\tilde{F}\left(\mu\right) = \mathcal{F}\left\{\tilde{f}\left(t\right)\right\} = F\left(\mu\right) * S_{\Delta T}\left(\mu\right)
	\end{equation*}

	\noindent
	Si ricorda che:
	
	\begin{equation*}
		S_{\Delta T} \left(\mu\right) = \dfrac{1}{\Delta T} \sum_{n = -\infty}^{+\infty}\delta\left(\mu - \dfrac{n}{\Delta T}\right)
	\end{equation*}	

	\noindent
	E risolvendo la convoluzione, si ottiene che la \textcolor{Red3}{\textbf{\underline{trasformata di Fourier a tempo}}} \textcolor{Red3}{\textbf{\underline{discreto}}} corrisponde a:
	
	\begin{equation*}
		F\left(\mu\right) * S_{\Delta T}\left(\mu\right) = \int_{-\infty}^{+\infty} F\left(\tau\right) \cdot S_{\Delta T} \left(\mu - \tau\right) \: \mathrm{d}\tau = \dfrac{1}{\Delta T} \sum_{n = -\infty}^{+\infty} F \left(\mu - \dfrac{n}{\Delta T}\right)
	\end{equation*}

	\noindent
	Analizzando la formula si evidenziano alcuni termini:
	
	\begin{itemize}
		\item $F\left(\mu\right)$ è la trasformata di Fourier della funzione originale $f\left(t\right)$;
		
		\item $F\left(\mu - \dfrac{n}{\Delta T}\right)$ è la trasformata di Fourier della funzione originale $f\left(t\right)$ shiftato a destra di una quantità pari a $\dfrac{n}{\Delta T}$;
		
		\item $\dfrac{1}{\Delta T} \sum_{n = -\infty}^{+\infty} F \left(\mu - \dfrac{n}{\Delta T}\right)$ sono infinite copie dello spettro $F\left(\mu\right)$, ripetute ovviamente ogni $\dfrac{1}{\Delta T}$. \newline
		Inoltre, è un \textbf{segnale periodico} (nelle frequenze) di periodo $\dfrac{1}{\Delta T}$, ovvero si ripete ogni $\dfrac{1}{\Delta T}$ Hz.\newline
		La sua scalatura nell'ampiezza è pari a $\dfrac{1}{\Delta T}$ e rappresenta la T.d.F. a tempo discreto.
	\end{itemize}

	\newpage
	
	\subsubsection{Teorema del campionamento}
	
	Un \textbf{\underline{segnale reale continuo}} $f\left(t\right)$, limitato in banda, può essere \textbf{\underline{ricostruito}} \textbf{\underline{senza errori completamente dai suoi campioni}} se essi sono acquisiti con un tempo di campionamento $\Delta T$ tale per cui:
	
	\begin{equation*}
		\dfrac{1}{\Delta T} = \mu_{S} \ge 2 \mu_{\text{max}}
	\end{equation*}

	\noindent
	Ovvero se nel tempo si adotta una frequenza di campionamento $\dfrac{1}{\Delta T}$ almeno doppia rispetto alla frequenza massima del segnale $\mu_{\text{max}}$.\newline
	In altre parole, il teorema del campionamento afferma che tutte le proprietà di un segnale possono essere espresse usando dei campioni.\newline
	
	\noindent
	\textbf{\underline{Attenzione!}} L'espressione $\dfrac{1}{\Delta T}$ viene chiamata \textbf{\emph{frequenza di Nyquist}} e per frequenze minori si crea aliasing, fenomeno che impedisce la ricostruzione senza errori.
	
	\subsubsection{Considerazioni}
	
	Dal \textbf{\underline{punto di vista teorico}} la trasformata di Fourier a tempo discreto consente di ricostruire il segnale. Tuttavia, è impossibile implementarla in un computer poiché tende, come limiti, all'infinito e ci vorrebbe un numero infinito di campioni e di segnali di tipo $\mathrm{sinc}$.
	
	\newpage
	
	\subsection{Trasformata di Fourier discreta}\label{trasformata di fourier discreta}
	
	La trasformata di Fourier di un segnale reale continuo $f\left(t\right)$ di dominio illimitato e non periodico, campionato nel tempo con periodo di campionamento $\Delta T$, è una funzione continua, periodica (di periodo $\dfrac{1}{\Delta T}$) anch'essa di dominio illimitato:
	
	\begin{equation*}
		\tilde{f}\left(t\right) = f\left(t\right) \cdot \sum_{n = -\infty}^{+\infty}\delta\left(t - n\Delta T\right) \: \textcolor{Red3}{\overset{\mathcal{F}}{\longrightarrow}} \: \tilde{F}\left(\mu\right) = \dfrac{1}{\Delta T} \sum_{n = -\infty}^{+\infty} F\left(\mu - \dfrac{n}{\Delta T}\right)
	\end{equation*}

	\noindent
	Il \textbf{problema} di questa formulazione è data l'espressione analitica dello spettro, la quale suppone che si è a \textbf{conoscenza della T.d.F.} teorica $F$ \textbf{del segnale di partenza}. Questo, spesso, è molto difficile.\newline
	
	\noindent
	Si ricava dunque una \emph{forma più semplice} da manipolare. Essa consente di \textbf{costruire una rappresentazione spettrale a partire dai campioni della funzione originale} $f\left(t\right)$:
	
	\begin{equation*}
		\tilde{F}\left(\mu\right) = \sum_{n = -\infty}^{+\infty} f_{n} \: e^{-j 2 \pi \mu n \Delta T}
	\end{equation*}
	
	\noindent
	Al contrario, la prima formulazione era più chiara per comprendere il fatto che la T.d.F. di una funzione campionata genera delle repliche dello spettro originale $F\left(\mu\right)$.\newline
	
	\noindent
	L'equazione alternativa deve essere modificata, eseguendo un campionamento per il dominio spettrale, per poterla implementare su un computer. Per farlo, si prende in considerazione solo l'intervallo frequenziale da $0$ a $\dfrac{1}{\Delta T} = \mu_{S}$.
	
	\begin{figure}[!htp]
		\centering
		\includegraphics[width=0.7\textwidth]{img/trasformata_fourier_discreta1.pdf}
	\end{figure}

	\noindent
	Inoltre, si prendono in considerazione $M$ campioni tramite l'operazione di campionamento.
	
	\begin{figure}[!htp]
		\centering
		\includegraphics[width=0.7\textwidth]{img/trasformata_fourier_discreta2.pdf}
	\end{figure}

	\newpage

	\noindent
	In cui:
	
	\begin{equation*}
		\tilde{\mu} = \dfrac{m}{M} \cdot \dfrac{1}{\Delta T} \hspace{2em} \text{con } m = 0, ..., M-1 \text{ e dove } \dfrac{m}{M} \in \left[0,1 - \dfrac{1}{M}\right]
	\end{equation*}

	\noindent
	La $m$ indica il \textbf{range di variazione} degli indici dei campioni frequenziali. Calcolando la trasformata di Fourier a tempo discreto sui campioni $M$, si giunge alla \textcolor{Red3}{\textbf{\underline{forma finale della trasformata di Fourier discreta}}}:
	
	\begin{equation*}
		\tilde{F}\left(\tilde{\mu}\right) = \tilde{F}\left(\dfrac{m}{M} \cdot \dfrac{1}{\Delta T}\right) = \sum_{n = 0}^{M - 1} f_{n} e^{-j 2 \pi \frac{m}{M} n} \hspace{2em} \text{con } m = 0, ..., M - 1
	\end{equation*}

	\noindent
	La \textcolor{Red3}{\textbf{\underline{trasformata di Fourier discreta inversa}}}, ovvero l'antitrasformata:
	
	\begin{equation*}
		\tilde{f}\left(n \Delta T\right) = f\left(n \Delta T\right) = f_{n} = \dfrac{1}{M} \sum_{m = 0}^{M - 1} F_{m} e^{j 2 \pi \frac{m}{M} n}
	\end{equation*}

	\newpage
	
	\subsection{Riassunto Trasformate}
	
	Qui di seguito vengono rappresentate le trasformate più importanti.
	
	\begin{table}[!htbp]
		\centering
		\begin{tabular}{@{} l l c l @{}}
			\toprule
			%Segnale & & & Trasformata di Fourier \\
			Funzione & & & Serie di Fourier \\
			\midrule
			&&& \\
			Funzione di sintesi: & $f(t)$ && $\sum_{n = -\infty}^{+\infty} c_{n} \underbrace{e^{j \frac{2\pi n}{T}t}}_{\text{fasore}} \hspace{2em} n\in\mathbb{Z}$ \\
			&&& \\
			Funzione di analisi: & $c_{n} \in \mathbb{C}$ && $\dfrac{1}{T} \int_{-\frac{T}{2}}^{+\frac{T}{2}} f\left(t\right) \underbrace{e^{-j \frac{2\pi n}{T}t}}_{\text{fasore}} \mathrm{d}t \hspace{2em} n \in \mathbb{Z}$ \\
			&&& \\
			\toprule
			Segnale  & & & Trasformata di Fourier continua \\
			\midrule
			Box:				& $A\Pi\left(\dfrac{t}{w}\right)$	& $\xrightarrow{\mathcal{F}}$ & $Aw \cdot \mathrm{sinc}\left(\mu w\right)$ \\
			&&&\\
			Sinc:				& $\mathrm{sinc}\left(tw\right)$	& $\xrightarrow{\mathcal{F}}$ & $\dfrac{1}{w}\cdot\Pi \left(-\dfrac{\mu}{w}\right) = \dfrac{1}{w}\cdot \Pi\left(\dfrac{\mu}{w}\right)$ \\
			&&&\\
			Impulso:			& $\delta\left(t\right)$			& $\xrightarrow{\mathcal{F}}$ &
			$\begin{cases}
				1 						& \text{se valori reali}\\
				e^{-j 2 \pi \mu t_{0}}	& \text{se valori complessi}
			\end{cases}$ \\
			&&& \\
			Treno di impulsi:	& $S_{\Delta T}\left(t\right)$ 		& $\xrightarrow{\mathcal{F}}$ & $\displaystyle\sum_{n = -\infty}^{+\infty} \dfrac{1}{\Delta T} \cdot \delta \left(\mu - \dfrac{n}{\Delta T}\right)$ \\
			&&& \\
			\toprule
			Funzione & & & Trasformata di Fourier a tempo discreto \\
			\midrule
			&&& \\
			$F\left(\mu\right) * S_{\Delta T}\left(\mu\right)$ & & & $\dfrac{1}{\Delta T} \sum_{n = -\infty}^{+\infty} F \left(\mu - \dfrac{n}{\Delta T}\right)$ \\
			&&& \\
			\toprule
			Funzione & & & Trasformata di Fourier discreta \\
			\midrule
			&&& \\
			$\tilde{F}\left(\tilde{\mu}\right)$ &&& $\tilde{F}\left(\dfrac{m}{M} \cdot \dfrac{1}{\Delta T}\right) = \sum_{n = 0}^{M - 1} f_{n} e^{-j 2 \pi \frac{m}{M} n}$ \\
			&&& \\
			&&& $m = 0, ..., M - 1$ \\
			&&& \\
			\toprule
			Funzione & & & Trasformata di Fourier discreta inversa \\
			\midrule
			&&& \\
			$\tilde{f}\left(n \Delta T\right) = f\left(n \Delta T\right) = f_{n}$ &&& $\dfrac{1}{M} \sum_{m = 0}^{M - 1} F_{m} e^{j 2 \pi \frac{m}{M} n}$ \\
			&&& \\
			\bottomrule
		\end{tabular}
		\caption{Trasformate di Fourier.}
	\end{table}
	
	\newpage
	
	\subsection{Domanda da esame}
	
	All'esame è possibile che sia richiesto come domanda: \dquotes{quali sono le trasformate di Fourier studiate durante il corso?}
	
	La risposta, anche se banale, è la seguente: le trasformate di Fourier studiate durante il corso sono $4$:
	
	\begin{enumerate}[label=\Roman*.]
		\item Serie di Fourier (paragrafo~\ref{serie di fourier})
		\item Trasformata di Fourier continua (paragrafo~\ref{trasformata di fouerier continua})
		\item Trasformata di Fourier a tempo discreto (paragrafo~\ref{trasformata di fourier a tempo discreto})
		\item Trasformata di Fourier discreta (paragrafo~\ref{trasformata di fourier discreta})
	\end{enumerate}

	\newpage
	
	\section{Elaborazione di immagini - Dominio spaziale}
	
	L'\textcolor{Red3}{\textbf{\underline{elaborazione delle immagini}}} consiste nel prendere come input un'immagine (segnale) e restituirne un'altra (sempre segnale) come output.\newline
	
	\noindent
	Il \textcolor{Red3}{\textbf{\underline{rinforzo (\emph{enhancement}) di immagini}}} è un tipo di elaborazione delle immagini. Il suo \textbf{obbiettivo} è elaborare un'immagine in modo che il risultato sia più adatto alle esigenze soggettive dell'utente. La definizione è \emph{problem-oriented} poiché, per esempio, per visualizzare lo spettro di ampiezza di un'immagine, è necessario eseguire un'operazione di rinforzo (\emph{log-transformation}); oppure, per migliorare la visibilità dei dettagli di un'immagine, si effettua un'altra operazione di rinforzo (\emph{sharpening}).\newline
	
	\noindent
	La \textcolor{Red3}{\textbf{\underline{qualità di un'immagine}}} è la combinazione pesata di tutti gli attributi significativi di un'immagine. Infatti, \emph{non} esiste una ricetta univoca per determinare quando un'immagine sia di qualità poiché è un'opinione soggettiva. Tuttavia, è più facile dire quando un'immagine \emph{non} è di qualità. In genere, un'\textcolor{Red3}{\textbf{\underline{immagine \emph{non} è di qualità}}} quando non viene interpretata facilmente da un operatore umano.\newline
	
	\noindent
	A differenza del rinforzo, il \textcolor{Red3}{\textbf{\underline{restauro (\emph{restoration})}}} è un processo di ricostruzione dell'immagine a partire da un modello di degradazione noto.
	
	\newpage
	
	\subsection{Strumento per l'elaborazione: istogramma}
	
	I pixel di un'immagine sono una \dquotes{popolazione} sulla quale è possibile calcolare tutte le quantità statistiche descrittive che vengono usate normalmente come media, mediana, varianza, deviazione standard, quartili, percentili, etc. Uno \textbf{strumento fondamentale} per l'elaborazione delle immagini è l'\textcolor{Red3}{\textbf{\underline{istogramma}}}, il quale può essere visto come una funzione continua o discreta.\newline
	
	\noindent
	Infatti, per ogni livello di grigio (in un'immagine solo a livelli di grigi) vengono riportati il numero di pixel. Per un'immagine $I\left[M,N\right]$ si identifica con $M,N$ il \textbf{numero di pixel righe per colonne} e con la funzione $H\left(r\right)$ il \textbf{numero di pixel di valore} $r$, quest'ultimo è definito nell'intervallo $0 \le r \le L-1$ con $r,L \in \mathbb{N}$ dove $L$ indica i livelli di grigio. Inoltre:
	
	\begin{equation*}
		\sum_{r = 0}^{L - 1} H\left(r\right) = M \cdot N
	\end{equation*}

	\noindent
	Grazie all'istogramma, è possibile comprendere immediatamente le caratteristiche dell'immagine (come in figura).
	
	\begin{figure}[!htp]
		\centering
		\includegraphics[width=0.8\textwidth]{img/istogramma.pdf}
		\caption{Esempio di istogramma.}
	\end{figure}

	\noindent
	Un istogramma può essere anche visto come una distribuzione di probabilità:
	
	\begin{equation*}
		p_{h}\left(r\right) = \dfrac{H\left(r\right)}{M \cdot N} \hspace{2.5em} \sum_{r} p_{h}\left(r\right) = 1
	\end{equation*}

	\noindent
	Uno \textcolor{Red3}{\textbf{\underline{svantaggio}}} di questo strumento è che immagini diverse potrebbero avere istogrammi simili, questo perché l'istogramma non tiene conto della distribuzione spaziale dei pixel. Dunque, \textbf{utilizzando solo questo metodo è impossibile ricostruire un'immagine}.\newline
	
	\noindent
	Al contrario, un \textcolor{Green4}{\textbf{\underline{vantaggio}}} dell'istogramma è la possibilità di identificare facilmente il \textcolor{Red3}{\textbf{contrasto}}: rapporto o differenza tra il valore più alto (punto più luminoso) e il valore più basso (punto più scuro) della luminosità (che corrisponde al livello di grigio per le immagini a livello di grigio).\newline
	Un'immagine viene definita:
	
	\begin{itemize}
		\item Valori più alti sulla destra:
		\begin{itemize}
			\item \textbf{chiara}, caratteristica dell'immagine;
			\item \textbf{sovraesposta}, caratteristica di come è stata acquisita l'immagine.
		\end{itemize}
	
		\item Valori più alti sulla sinistra:
		\begin{itemize}
			\item \textbf{scura}, caratteristica dell'immagine;
			\item \textbf{sottoesposta}, caratteristica di come è stata acquisita l'immagine.
		\end{itemize}
	\end{itemize}

	\newpage
	
	\subsection{Domini}
	
	L'elaborazione delle immagini può avvenire nel \textbf{dominio spaziale} o nel \textbf{dominio frequenziale} (dopo aver applicato la T.d.F. discreta 2D). Nel \textcolor{Red3}{\textbf{\underline{dominio spaziale}}}, l'elaborazione delle immagine può essere espressa come:
	
	\begin{equation*}
		g\left(x,y\right) = T\left[f\left(x,y\right)\right]
	\end{equation*}

	\noindent
	In cui:
	
	\begin{itemize}[label=-]
		\item $f$ è l'immagine di ingresso (input) da elaborare;
		\item $g$ è l'immagine d'uscita (output) elaborata;
		\item $T$ è un operatore su $f$ definito in un intorno di $\left(x,y\right)$.
	\end{itemize}

	\noindent
	L'\textcolor{Red3}{\textbf{\underline{operatore}}} definito in un intorno di $\left(x,y\right)$ può essere di tre tipi:
	
	\begin{itemize}
		\item \textbf{Puntuale:} $\left[f\left(x,y\right)\right] = f\left(x,y\right)$, l'intorno coincide con il pixel stesso;
		
		\item \textbf{Locale:} $\left[f\left(x,y\right)\right]$ rappresenta una regione, per esempio quadrata, attorno al pixel di locazione $\left(x,y\right)$;
		
		\item \textbf{Globale:} $\left[f\left(x,y\right)\right]$ rappresenta l'intera immagine $f$.
	\end{itemize}

	\newpage
	
	\subsection{Operazioni puntuali}
	
	Si dice \textcolor{Red3}{\textbf{\underline{operatore puntuale}}}, un operatore che ha preso in input il valore di un pixel e ne restituisce uno cambiato, il quale dipende esclusivamente dal valore del pixel in ingresso.\newline
	
	\noindent
	L'\textbf{\underline{obbiettivo}} è quello di variare il contrasto. Infatti, eseguendo questa operazione, si evidenziano le differenze strutturali dell'oggetto rappresentato. Per farlo, basta cambiare il valore di un pixel per renderlo più scuro o più chiaro.\newline
	
	\noindent
	Un operatore puntuale può essere rappresentato tramite una \textbf{\underline{funzione}} che prende in input un valore $r$ e lo modifica in un valore $s = T\left(r\right)$ con $s,r$ appartenenti allo stesso campo di definizione (esempio tra 0 e 255). Più in generale viene definita come:
	
	\begin{equation*}
		T: \left[0, L - 1\right] \subset \mathbb{R} \longrightarrow \left[0, L - 1\right] \subset \mathbb{R}
	\end{equation*}

	\noindent
	Dato che un operatore puntuale dipende solo dal singolo valore del pixel, esso è dunque descritto da una tabella di questo tipo:
	
	\begin{center}
		\begin{tabular}{|c|c|c|c|c|c|c|c|c|}
			\hline
			&&&&&&&& \\
			r & 0 & 1 & 2 & 3 & 4 & 5 & 6 & $\cdots$ \\
			&&&&&&&& \\
			\hline
			&&&&&&&& \\
			s & $T\left(0\right)$ & $T\left(1\right)$ & $T\left(2\right)$ & $T\left(3\right)$ & $T\left(4\right)$ & $T\left(5\right)$ & $T\left(6\right)$ & $\cdots$ \\
			&&&&&&&& \\
			\hline
		\end{tabular}
	\end{center}

	\newpage
	
	\subsubsection{Identità}
	
	È l'operazione più semplice e non fa nulla:
	
	\begin{equation*}
		s = r
	\end{equation*}

	\subsubsection{Negativo}
	
	Rende l'immagine più scura:
	
	\begin{equation*}
		s = L - 1 - r
	\end{equation*}

	\noindent
	Nel caso dei livelli di grigio:
	
	\begin{equation*}
		s = 255 - r
	\end{equation*}

	\noindent
	Viene \textbf{utilizzata} quando si hanno dettagli grigi immersi in zone nere che si vogliono evidenziare.

	\subsubsection{Clamping}
	
	Limita l'intensità ad un range definito $\left[a,b\right]$:
	
	\begin{equation*}
		T\left(r\right) = \begin{cases}
			a & \text{se } r < a \\
			r & \text{se } a \le r \le b \\
			b & \text{se } r > b \\
		\end{cases}
	\end{equation*}

	\noindent
	Viene \textbf{utilizzata} nel caso in cui ci siano dei pixel di rumore molto chiari o molto scuri che \emph{mascherano} l'immagine. Quindi, si pensi per esempio ad un'immagine con dei puntini bianchi al quale si applica il \emph{clamping} per rimuoverli.
	
	\subsubsection{Stretching/Shrinking}
	
	Stira/comprime le intensità di un range $\left[r_{\min}, r_{\max}\right]$ ad un range definito $\left[a,b\right]$:
	
	\begin{equation*}
		s = \left[\dfrac{r - r_{\min}}{r_{\max} - r_{\min}}\right] \left[b - a\right] + a
	\end{equation*}

	\noindent
	In cui:
	
	\begin{itemize}
		\item $r_{\min / \max}$ sono il più piccolo/grande livello di grigio del range che si vuole trattare;
		\item $a,b$ sono il minimo e il massimo \dquotes{stretchati}.
	\end{itemize}

	\noindent
	\textbf{Nota bene:} l'operazione è seguita da \emph{rounging} (arrotondamento) nel caso di dominio di valori nei naturali (come in 0-255). Inoltre, lo \textbf{stretching \underline{non} risolve il problema del rumore impulsivo (puntini neri o bianchi)}, neanche se mascherato con il clamping.
	
	\newpage
	
	\subsubsection{Trasformazione logaritmica}\label{trasformazione logaritmica}
	
	La forma generale è:
	
	\begin{equation*}
		s = c \cdot \log\left(1 + r\right) \hspace{2em} r \in \left[0, L - 1\right]
	\end{equation*}

	\noindent
	Con $c$ che rappresenta la \textbf{costante di normalizzazione}:
	
	\begin{equation*}
		c = \dfrac{L - 1}{\log\left(L\right)}
	\end{equation*}
	
	\noindent
	La quale assicura la mappatura in $\left[0, L - 1\right]$. Inoltre, l'aggiunta di $1$ permette di evitare il calcolo di quantità $\in\left[0,1\right[$ che generano valori minori di zero ed in particolare il calcolo di $\log\left(0\right) = -\infty$.\newline
	
	\noindent
	Viene \textbf{utilizzata} quando si vuole mappare fasce strette di valori dell'immagine originale in fasce più ampie, aumentandone così il range del contrasto, rendendo inoltre l'interpretazione umana più informativa.
	
	\subsubsection{Trasformazione esponenziale}
	
	Al contrario della trasformazione logaritmica, la trasformazione esponenziale consente di aumentare il range di una fascia determinata di livelli di grigi chiari:
	
	\begin{equation*}
		s = \left(e^{r}\right)^{\frac{1}{c}} - 1 \hspace{2em} r \in \left[0, L - 1\right]
	\end{equation*}

	\noindent
	Con la \textbf{costante di normalizzazione}:
	
	\begin{equation*}
		c = \dfrac{L - 1}{\log\left(L\right)}
	\end{equation*}

	\newpage

	\subsubsection{Trasformazione di potenza}
	
	La trasformazione di potenza può essere espressa come:
	
	\begin{equation*}
		s = cr^{\gamma} \hspace{2em} c,\gamma > 0 \in \mathbb{R}
	\end{equation*}

	\noindent
	La costante $c$ è scelta \textbf{in dipendenza da} $\gamma$ in modo da normalizzare i valori di $s$ nell'intervallo $\left[0,255\right]$.
	
	\begin{itemize}[label=-]
		\item $\gamma < 1$, la trasformazione ha effetti analoghi alla trasformazione logaritmica (\ref{trasformazione logaritmica}), cioè espansione della dinamica per bassi valori di $r$, mentre compressione della dinamica per alti valori di $r$;
		
		\item $\gamma > 1$, la trasformazione ha effetti opposti ai valori negativi di gamma.
	\end{itemize}

	\noindent
	Nella pratica, il termine di normalizzazione $c$ è complicato da definire analiticamente, quindi si preferiscono due versioni di $s$:
	
	\begin{itemize}
		\item \textbf{Non normalizzata:} $\tilde{s} = r^{\gamma}$;
		
		\item \textbf{Normalizzata} $s = cr^{\gamma}$
	\end{itemize}

	\noindent
	Per passare dalla versione \textbf{non normalizzata alla versione normalizzata} si esegue lo \emph{stretching}:
	
	\begin{equation*}
		s = \left[\dfrac{\tilde{s} - \tilde{s}_{\min}}{\tilde{s}_{\max}} - \tilde{s}_{\min}\right] \left[\max-\min\right]
	\end{equation*}

	\noindent
	Dove $\tilde{s}_{\min / \max}$ sono il più piccolo/grande livello di grigio e $\max$ e $\min$ sono il massimo e il minimo livello di grigio possibile (255, 0).
	
	\subsubsection{Binarizzazione}
	
	Produce un'immagine che ha solo due livelli: nero e bianco. Si \textbf{ottiene} scegliendo una soglia $T$, si imposta a colore nero tutti i pixel il cui valore è minore a $T$ e si imposta a colore bianco tutti gli altri.\newline
	
	\noindent
	Si \textbf{utilizza} la binarizzazione per discriminare un oggetto dalla scena.
	
	\begin{figure}[!htp]
		\centering
		\includegraphics[width=0.2\textwidth]{img/binarizzazione1.png}
		\includegraphics[width=0.2\textwidth]{img/binarizzazione2.png}
	\end{figure}

	\noindent
	Solitamente il suo utilizzo è prevalente nell'ambito delle immagini biomedicali e di videosorveglianza. La difficoltà maggiore di questa tecnica è il saper scegliere la soglia $T$ più ragionevole.
	
	\subsubsection{Binarizzazione attraverso il metodo di Otsu}
	
	Questo metodo assume che ci siano due regioni da scegliere, come nella seguente figura:
	
	\begin{figure}[!htp]
		\centering
		\includegraphics[width=0.5\textwidth]{img/binarizzazione_otsu.pdf}
	\end{figure}

	\noindent
	Se l'immagine ha un \textbf{istogramma bimodale la binarizzazione è efficace}, altrimenti no. Le due regioni sono definite come:
	
	\begin{equation*}
		\begin{array}{lll}
			p_{0 \rightarrow T}\left(r\right) & = & \dfrac{H\left(r\right)}{\displaystyle\sum_{r = 0}^{T} H\left(r\right)} \\
			&& \\
			\sigma_{0 \rightarrow T}^{2} & = & \displaystyle\sum_{r = 0}^{T} p_{0 \rightarrow T}\left(r\right)\left(r - \mu_{0 \rightarrow T}\right)^{2} \\
			&& \\
			\mu_{0 \rightarrow T} & = & \displaystyle\sum_{r = 0}^{T}r \cdot p_{0 \rightarrow T}\left(r\right)
		\end{array}
	\end{equation*}

	\noindent
	La formula \textbf{da minimizzare su $T$} è la seguente:
	
	\begin{equation*}
		\sigma_{w}^{2}\left(T\right) = W_{0}\left(T\right) \sigma_{0}^{2}\left(T\right) + W_{1}\left(T\right) \sigma_{1}^{2}\left(T\right)
	\end{equation*}

	\noindent
	Dove si considera la versione probabilistica dell'istogramma, ovvero la sua versione normalizzata, e si ha:
	
	\begin{equation*}
		\begin{array}{lll}
			W_{0}\left(T\right) & = & \displaystyle\sum_{r=0}^{T-1} p\left(r\right) \\
			&& \\
			W_{1}\left(T\right) & = & \displaystyle\sum_{r=T}^{L-1} p\left(r\right) \\
			&& \\
			p\left(r\right) & = & \dfrac{1}{M \cdot N} \displaystyle\sum_{r=0}^{L-1} H\left(r\right)
		\end{array}
	\end{equation*}

	\noindent
	Dove $W_{0}, W_{1}$ sono le probabilità che le due classi siano separate da $T$ e $\sigma_{0}^{2}$ e $\sigma_{1}^{2}$ sono le varianze sui valori di istogramma assunti dalle due classi. \textbf{L'approccio \dquotes{cicla} su tutti i possibili valori di $T$ e restituisce:}
	
	\begin{equation*}
		T_{best} = \arg_{T} \min \left(\sigma_{w}^{2}\left(T\right)\right)
	\end{equation*}

	\subsubsection{Equalizzazione}
	
	Un'immagine si dice \textbf{equalizzata} quando il contributo di ogni differente tonalità di grigio è simile. L'istogramma tende ad essere uniforme o appiattito.\newline
	
	\noindent
	L'\textbf{obbiettivo} è vedere l'istogramma come una distribuzione e di renderla il più simile a quella uniforme. Una \textbf{distribuzione uniforme} ha un'\textbf{entropia massima}\footnote{Secondo l'entropia, un sistema isolato si trasforma ed evolve nel tempo fino a raggiungere uno stato di equilibrio finale macroscopico in cui le differenze locali sono minime.}. Nelle immagini, ogni valore della distribuzione è un valore di grigio, per cui ogni valore di grigio appartiene all'entropia massima.\newline
	
	\noindent
	Si \textbf{utilizza} questo operatore poiché un istogramma piatto assicura a livello percettivo una risposta del cervello migliore (in termini di numero di dettagli che si riesce a riconoscere), per cui l'immagine diventa più \dquotes{informativa} da osservare.\newline
	
	\noindent
	Se $r_{k}$ è il $k$-esimo livello di grigio $k = 0, ..., L - 1$ e $H\left(r_{k}\right)$ è il conteggio dato dall'istogramma dell'immagine di dimensione $M \times N$, allora si può definire:
	
	\begin{equation*}
		p_{r}\left(r_{k}\right) = \dfrac{H\left(r_{k}\right)}{M \cdot N}
	\end{equation*}

	\noindent
	L'\textcolor{Red3}{\textbf{\underline{equalizzazione dell'istogramma}}} si basa sulla seguente funzione $T$, con $s_{k}$ che rappresenta il $k$-esimo valore di grigio in cui viene \dquotes{mappato} $r_{k}$:
	
	\begin{equation*}
		s_{k} = T\left(r_{k}\right) = \left(L-1\right) \displaystyle\sum_{j=0}^{k} p_{r}\left(r_{k}\right) = \dfrac{\displaystyle\sum_{j=0}^{k} H\left(r_{j}\right)}{\dfrac{M \cdot N}{\left(L-1\right)}}
	\end{equation*}

	\noindent
	\textcolor{Red3}{\textbf{\emph{\underline{Algoritmo}}}}
	
	\begin{enumerate}[label=\Roman*.]
		\item Calcolare le $L$ somme cumulative $\displaystyle\sum_{j=0}^{k} p_{r}\left(r_{j}\right)$ dei valori dell'istogramma visto come distribuzione con $k = 0, ..., L-1$;
		
		\item Moltiplicare i valori del passo precedente per il massimo di livelli di grigio $L - 1$;
		
		\item Normalizzazione dei valori calcolati al primo passo, dividendo per il numero totale di pixel $M \cdot N$ e arrotondamento;
		
		\item Applicare il mapping $T$ ottenuto.
	\end{enumerate}

	\newpage
	
	\noindent
	\textcolor{Green4}{\textbf{\underline{\emph{Esempio}}}}\newline
	
	\noindent
	Sia data un'immagine con $L = 8, 64\times64$ pixel $\left(M \cdot N = 4096\right)$, con la seguente distribuzione d'intensità:
	
	\begin{table}[!htbp]
		\centering
		\begin{tabular}{@{} l c c @{}}
			\toprule
			$r_{k}$ & $H\left(r_{k}\right)$ & $p_{r}\left(r_{k}\right) = \dfrac{H\left(r_{k}\right)}{M \cdot N}$ \\
			\midrule
			$r_{0} = 0$	& $790$	& $0.19$ \\
			&& \\
			$r_{1} = 1$	& $1023$& $0.25$ \\
			&& \\
			$r_{2} = 2$	& $850$	& $0.21$ \\
			&& \\
			$r_{3} = 3$	& $656$	& $0.16$ \\
			&& \\
			$r_{4} = 4$	& $329$	& $0.08$ \\
			&& \\
			$r_{5} = 5$	& $245$	& $0.06$ \\
			&& \\
			$r_{6} = 6$	& $122$	& $0.03$ \\
			&& \\
			$r_{7} = 7$	& $81$	& $0.02$ \\
			\bottomrule
		\end{tabular}
	\end{table}

	\begin{figure}[!htp]
		\centering
		\includegraphics[width=0.5\textwidth]{img/eg_equalizzazione.png}
		\caption{Rappresentazione della distribuzione di intensità.}
	\end{figure}
	
	\noindent
	Si applica la formula di equalizzazione:
	
	\begin{equation*}
		\begin{array}{lllllllll}
			s_{0} & = & T\left(r_{0}\right) & = & 7\displaystyle\sum_{j=0}^{0} p_{r}\left(r_{j}\right) & = & 7p_{r}\left(r_{0}\right) & = & 1.33 \\
			&&&&&&&&\\
			s_{1} & = & T\left(r_{1}\right) & = & 7\displaystyle\sum_{j=0}^{1} p_{r}\left(r_{j}\right) & = & 7p_{r}\left(r_{0}\right) + 7p_{r}\left(r_{1}\right) & = & 3.08
		\end{array}
	\end{equation*}
	
	\noindent
	Analogamente anche per gli altri valori si applica la formula e si trovano i seguenti valori:
	
	\begin{equation*}
		\begin{array}{lll}
			s_{2} & = & 4.55 \\
			s_{3} & = & 5.67 \\
			s_{4} & = & 6.23 \\
			s_{5} & = & 6.65 \\
			s_{6} & = & 6.86 \\
			s_{7} & = & 7.00 \\
		\end{array}
	\end{equation*}

	\begin{figure}[!htp]
		\centering
		\includegraphics[width=0.7\textwidth]{img/eg_equalizzazione.pdf}
		\caption{LUT (\emph{Lookup Table})}
	\end{figure}

	\noindent
	\textbf{LUT} (\emph{Lookup Table}) è un termine utilizzato per descrivere una predeterminata lista di numeri che offre una \dquotes{scorciatoia} per una specifica computazione. Nel contesto dei colori, una LUT trasforma i colori, ricevuti come input (camera), in un output desiderato (final footage).\newline
	
	\noindent
	L'immagine è quantizzata, quindi si effettua l'arrotondamento dei valori ottenendo l'intero più vicino:
	
	\begin{equation*}
		\begin{array}{lllll}
			s_{0} & = & 1.33 & \longrightarrow & 1 \\
			s_{1} & = & 3.08 & \longrightarrow & 3 \\
			s_{2} & = & 4.55 & \longrightarrow & 5 \\
			s_{3} & = & 5.67 & \longrightarrow & 6 \\
			s_{4} & = & 6.23 & \longrightarrow & 6 \\
			s_{5} & = & 6.65 & \longrightarrow & 7 \\
			s_{6} & = & 6.86 & \longrightarrow & 7 \\
			s_{7} & = & 7.00 & \longrightarrow & 7 \\
		\end{array}
	\end{equation*}

	\newpage

	\noindent
	Dopo l'arrotondamento, si ottiene una nuova immagine e il suo relativo istogramma.
	
	\begin{figure}[!htp]
		\centering
		\includegraphics[width=0.5\textwidth]{img/eg_equalizzazione2.png}
	\end{figure}

	\newpage
	
	\subsection{Operazioni locali}
	
	Un'\textcolor{Red3}{\textbf{\underline{operazione locale}}} restituisce un pixel che dipende da un limitato intorno del corrispondente punto in input. Tali operazioni vengono \textbf{utilizzati} per migliorare la qualità delle immagini o per estrarre delle informazioni dall'immagine.\newline
	
	\noindent
	Le operazioni locali sono come dei filtraggi spaziali dell'immagine. Il \textbf{\underline{filtraggio spaziale}} è un'elaborazione $T$ dell'immagine $f$ dove un pixel di locazione $\left(n,m\right)$, di intensità $f\left(n,m\right)$, viene cambiato in $g\left(n,m\right)$ da una funzione dei pixel in un intorno di $\left(n,m\right)$, ossia:
	
	\begin{equation*}
		g\left(n,m\right) = T\left(\left[f\left(n,m\right)\right]\right)
	\end{equation*}

	\noindent
	Dove le parentesi quadrate indicano che viene preso in considerazione un intorno di $n,m$. Ovviamente, il risultato dell'operazione, se applicato a tutti i pixel dell'immagine $f$, è una nuova immagine $g$.\newline
	
	\noindent
	Gli intorni presi maggiormente in considerazione sono di grandezza $K \times K$, con $K$ dispari (per fare in modo di considerare uniformemente i pixel attorno al punto $\left(n,m\right)$ di applicazione), di solito $3 \times 3, 5 \times 5, 7 \times 7$. I pixel al di fuori dell'intorno non prendono parte alla funzione.\newline
	
	\noindent
	\textcolor{Red3}{\textbf{\underline{\emph{Pseudocodice}}}}\newline
	
	\begin{itemize}
		\item \textbf{Input:}
		\begin{itemize}
			\item Immagine $f$ definita con un suo valore di pixel generico $f\left(n,m\right) \in \left[0 ... L - 1\right] \subset \mathbb{N}$ con $\left(n,m\right) \in \left[1 ... \mathbb{N}\right] \times \left[1 ... M\right] \subset \mathbb{N} \times \mathbb{N}$;
			
			\item Intorno di valori di pixel $\left[f\left(n,m\right)\right]$ ossia $f\left(n-u, m-v\right)$ definita come:
			\begin{equation*}
				\left(u,v\right) \in \left[-\dfrac{K-1}{2} \cdots \dfrac{K-1}{2}\right] \times \left[-\dfrac{K-1}{2} \cdots \dfrac{K-1}{2}\right] \subset \mathbb{N} \times \mathbb{N} \hspace{2em} K \in \left\{3,5,7,...\right\}
			\end{equation*}
		\end{itemize}
		
		\item \textbf{Output:}
		\begin{itemize}
			\item Nuova immagine $g\left(n,m\right) \in \mathbb{R}$ che attraverso operazioni puntuali può essere riportata in $g\left(n,m\right) \in \left[0 ... L-1\right] \subset \mathbb{N}$
		\end{itemize}
	
		\item \textbf{Procedimento:}
		\begin{equation*}
			\begin{array}{l}
				\mathrm{for} \: n = 1 ... N \\
				\hspace{2em} \mathrm{for} \: m = 1 ... M \\
				\hspace{4em} g\left(n,m\right) = T\left(\left[f\left(n,m\right)\right]\right)
			\end{array}
		\end{equation*}
	\end{itemize}

	\newpage
	
	\subsubsection{Filtraggi spaziali: lineari e non lineari}
	
	Le due principali categorie di operazioni locali sono lineari e non lineari:
	
	\begin{itemize}
		\item \textcolor{Red3}{\textbf{\underline{Filtraggio lineare}}} se $T$ è una combinazione lineare dei valori di pixel nel vicinato. Quindi, la convoluzione di un'immagine con un kernel è una somma di fattori ognuno dei quali è una moltiplicazione di un valore dell'immagine per un coefficiente del filtro:
		\begin{equation*}
			g\left(n,m\right) = T\left(\left[f\left(n,m\right)\right]\right) = h * f\left(n,m\right) = \sum_{u=-k}^{+k} \sum_{v=-k}^{+k} h\left(u,v\right) f\left(n-u, m-v\right)
		\end{equation*}
		\begin{equation*}
			k = \dfrac{K-1}{2}
		\end{equation*}
		Un esempio di operazione lineare è la convoluzione;
		
		\item \textcolor{Red3}{\textbf{\underline{Filtraggio non lineare}}} se $T$ contiene operazioni non lineari sulle variabili indipendenti. Un esempio di operazioni non lineari sono la mediana dei pixel nell'intorno e il valore massimo dei pixel nel vicinato.
	\end{itemize}

	\noindent
	I filtraggi lineari \textbf{\underline{non}} presentano un \textbf{\underline{problema ai bordi}} nel momento in cui l'intorno è definito all'interno dell'immagine (cioè non cade fuori dall'immagine). Alcuni filtri lineari:
	
	\begin{itemize}[label=-]
		\item \textbf{Cropping} è un filtro dove solo l'intorno cade all'interno dell'immagine, quindi il filtro viene applicato ad un'area ristretta e non a tutta l'immagine. Un esempio:
		\begin{equation*}
			\text{Input } = 
			\begin{bmatrix}
				1 & 2 & 2 & 3 & 1 \\
				3 & 2 & 2 & 1 & 4 \\
				2 & 5 & 2 & 7 & 1 \\
				9 & 0 & 1 & 1 & 2 \\
				3 & 1 & 2 & 4 & 1
			\end{bmatrix}
			\longrightarrow
			\text{ Output } = 
			\begin{bmatrix}
				* & * & * & * & * \\
				* & 30 & 45 & 30 & * \\
				* & 46 & 27 & 37 & * \\
				* & 34 & 41 & 28 & * \\
				* & * & * & * & *
			\end{bmatrix}
		\end{equation*}
		Le aree con un * non verranno calcolate;
		
		\item \textbf{Zero Padding} utilizzato per inserire degli zero che creano degli artefatti così da consentire il filtraggio. Un esempio:
		\begin{equation*}
			\text{Input } = 
			\begin{bmatrix}
				0 & 0 & 0 & 0 & 0 & 0 & 0 \\
				0 & 1 & 2 & 2 & 3 & 1 & 0 \\
				0 & 3 & 2 & 2 & 1 & 4 & 0 \\
				0 & 2 & 5 & 2 & 7 & 1 & 0 \\
				0 & 9 & 0 & 1 & 1 & 2 & 0 \\
				0 & 3 & 1 & 2 & 4 & 1 & 0 \\
				0 & 0 & 0 & 0 & 0 & 0 & 0
			\end{bmatrix}
			\longrightarrow
			\text{ Output } = 
			\begin{bmatrix}
				\colorbox{gray}{11} & \colorbox{gray}{19} & \colorbox{gray}{17} & \colorbox{gray}{22} & \colorbox{gray}{11} \\
				\colorbox{gray}{25} & 30 & 45 & 30 & \colorbox{gray}{31} \\
				\colorbox{gray}{25} & 46 & 27 & 37 & \colorbox{gray}{19} \\
				\colorbox{gray}{35} & 34 & 41 & 28 & \colorbox{gray}{29} \\
				\colorbox{gray}{16} & \colorbox{gray}{27} & \colorbox{gray}{12} & \colorbox{gray}{18} & \colorbox{gray}{10}
			\end{bmatrix}
		\end{equation*}
		Le aree in grigio non vengono calcolate.
		
		\newpage
		
		\item \textbf{Replicazione} utilizzata per creare artefatti, infatti l'immagine risultante non è realistica. Un esempio:
		\begin{equation*}
			\text{Input } = 
			\begin{bmatrix}
				1 & 1 & 2 & 2 & 3 & 1 & 1 \\
				1 & 1 & 2 & 2 & 3 & 1 & 1 \\
				3 & 3 & 2 & 2 & 1 & 4 & 4 \\
				2 & 2 & 5 & 2 & 7 & 1 & 1 \\
				9 & 9 & 0 & 1 & 1 & 2 & 2 \\
				3 & 3 & 1 & 2 & 4 & 1 & 1 \\
				3 & 3 & 1 & 2 & 4 & 1 & 1
			\end{bmatrix}
			\longrightarrow
			\text{ Output } = 
			\begin{bmatrix}
				\colorbox{gray}{25} & \colorbox{gray}{27} & \colorbox{gray}{29} & \colorbox{gray}{31} & \colorbox{gray}{33} \\
				\colorbox{gray}{34} & 30 & 45 & 30 & \colorbox{gray}{39} \\
				\colorbox{gray}{51} & 46 & 27 & 37 & \colorbox{gray}{32} \\
				\colorbox{gray}{54} & 34 & 41 & 28 & \colorbox{gray}{35} \\
				\colorbox{gray}{48} & \colorbox{gray}{32} & \colorbox{gray}{24} & \colorbox{gray}{34} & \colorbox{gray}{26}
			\end{bmatrix}
		\end{equation*}
		Le aree in grigio non vengono calcolate.
	\end{itemize}

	\newpage
	
	\subsection{Rumore nelle immagini}
	
	Il \textcolor{Red3}{\textbf{\underline{rumore nelle immagini}}} è un disturbo dell'immagine introdotto dal sistema di acquisizione (e.g. fotocamera) o dal mezzo di propagazione che ne degrada la qualità (e.g. Whatsapp). Il rumore è tipicamente \emph{modellato} come \textbf{additivo} e \textbf{casuale}:
	
	\begin{equation*}
		\tilde{f}\left(n,m\right) = f\left(n,m\right) + \varepsilon\left(n,m\right)
	\end{equation*}

	\noindent
	Dove $f$ è l'\textbf{immagine} priva di rumore e $\varepsilon$ è un processo aleatorio che genera delle quantità che seguono una distribuzione particolare, indipendentemente da dove il processo è collocato nell'immagine, ovvero indipendentemente da $n,m$.\newline
	
	\noindent
	Esistono due \textbf{tipi di rumore}: gaussiano additivo bianco (rumore generato da una distribuzione gaussiana) e impulsivo (rumore generato da una distribuzione bernoulliana).\newline
	
	\noindent
	La \textbf{quantità di rumore} può essere stimata attraverso la misura di $SNR$\label{SNR} (\emph{signal to noise ratio}), di cui esistono varie versioni. La più utilizzata è la \emph{\textbf{mean square, $SNR_{ms}$}}:
	
	\begin{equation*}
		SNR_{ms} = \dfrac{
		\displaystyle\sum_{n=1}^{N}\sum_{m=1}^{M} \tilde{f}\left(n,m\right)^{2}
		}{
		\displaystyle\sum_{n=1}^{N}\sum_{m=1}^{M} \left[\tilde{f}\left(n,m\right) - f\left(n,m\right)\right]^{2}
		}
	\end{equation*}

	\noindent
	Una forma alternativa della $SNR$ può essere stimata grazie alla varianza $\sigma_{n}^{2}$, o alla deviazione standard $\sigma_{n}$:
	
	\begin{equation*}
		SNR = \dfrac{\sigma_{s}}{\sigma_{n}}
	\end{equation*}

	\noindent
	Dove $\sigma_{s}$ è la deviazione standard del segnale e $\sigma_{n}$ è la deviazione standard dell'immagine affetta da rumore. Per questo motivo si utilizzano immagini ad alto contrasto, poiché $\sigma_{s}$ risulta maggiore!
	
	\newpage
	
	\subsubsection{Rumore gaussiano additivo bianco}
	
	Il \textcolor{Red3}{\textbf{\underline{rumore gaussiano additivo bianco}}} è un processo stocastico, ovvero una variabile aleatoria che emette valori casuali nel tempo $\varepsilon\left(t\right)$ o nello spazio $\varepsilon\left(n,m\right)$ con le seguenti caratteristiche:
	
	\begin{itemize}
		\item Si somma al segnale pulito;
		
		\item Non è periodico nel tempo o nello spazio;
		
		\item I valori vengono prodotti con la seguente probabilità:
		\begin{equation*}
			P\left(\varepsilon\left(n,m\right) = l\right) = \dfrac{1}{\sqrt{2\pi\sigma^{2}}} \cdot \exp\left(-\dfrac{\left(l - 0\right)^{2}}{2\sigma^{2}}\right)
		\end{equation*}
		In cui $0$ è uguale a $\mu$, ovvero indica la media del rumore. Inoltre, data una distribuzione gaussiana $\mu$, $\sigma^{2}$ il $98\%$ di valori $l \in \left[\mu - 2.5\sigma, \mu + 2.5\sigma\right]$.
		
		\item I valori seguono una distribuzione gaussiana di media pari a zero, ed una particolare varianza $\sigma^{2}$ (o deviazione standard $\sigma$) dove più è alta la varianza, più distanti da zero saranno i numeri prodotti e sommati all'immagine pulita, più rumorosa l'immagine finale.
	\end{itemize}

	\begin{figure}[!htp]
		\centering
		\includegraphics[width=\textwidth]{img/rumore_gaussiano_additivo_bianco.pdf}
		\caption{Esempio di rumore gaussiano a diversi valori di $\sigma$.}
	\end{figure}

	\newpage
	
	\subsubsection{Rumore impulsivo}
	
	Il \textcolor{Red3}{\textbf{\underline{rumore impulsivo}}} è causato da alterazione brusche nel segnale, viene parametrizzato da un fattore $D$ (una percentuale) che è la densità con cui esso si localizza su pixel dell'immagine: maggiore il valore di intensità $D$, maggiore sarà il numero di pixel affetti.\newline
	
	\noindent
	Per esempio, il disturbo sale e pepe (\href{https://it.wikipedia.org/wiki/Rumore_sale_e_pepe}{\emph{salt-and-pepper noise}}) può essere utilizzato selezionando una percentuale $D$ di pixel, ovvero $D\%$, in maniera uniforme nell'immagine e per ogni pixel si assegna un valore minimo o massimo con probabilità uniforme pari a $p = 0.5$.
	
	\begin{figure}[!htp]
		\centering
		\includegraphics[width=\textwidth]{img/rumore_impulsivo.png}
		\caption{Esempio di applicazione dell'effetto \emph{salt-and-pepper noise}.}
	\end{figure}

	\newpage
	
	\subsection{Altre operazioni locali: tipologie di filtraggio}
	
	Esistono altre $3$ tipologie principali di filtraggio:
	
	\begin{itemize}
		\item \textcolor{Red3}{\textbf{\emph{Smoothing}}}, utilizzato per aumentare il $SNR$ (pagina~\pageref{SNR}), ovvero per rimuovere il rumore.\newline
		Per esempio, il filtro di media, mediano, gaussiano;

		\item \textcolor{Red3}{\textbf{\emph{Sharpening}}}, utilizzato per aumentare il grado di dettaglio delle immagini.\newline
		Per esempio, il filtro laplaciano;

		\item \textcolor{Red3}{\textbf{\emph{Estrazione di caratteristiche}}}, utilizzato per estrarre rappresentazioni alternative alle immagini di partenza, che ne evidenzino aspetti particolari (edge, microstrutture, oggetti).\newline
		Per esempio, il filtro prewitt, sobel, cenny.
	\end{itemize}

	\newpage

	\subsubsection{Smoothing - Filtro media}
	
	Il \textcolor{Red3}{\textbf{\underline{filtro media}}} è utilizzato per \textbf{rimuovere il rumore gaussiano}. È un filtraggio $T$ lineare, si attua attraverso la convoluzione dell'immagine con la maschera media la quale ha le seguenti caratteristiche:
	
	\begin{itemize}[label=-]
		\item Dimensioni $K \times K$ con $K$ dispari;
		\item I suoi coefficienti sono tutti uguali e pari a $\dfrac{1}{K^{2}}$;
		\item Il suo funzionamento è il seguente: dato un intorno di applicazione $\left[\left(n,m\right)\right]$, esso calcola la media dei valori vii compresi $\left[\tilde{f}\left(n,m\right)\right]$, e la sostituisce al posto del valore $\tilde{f}\left(n,m\right)$:
		\begin{equation*}
			g\left(n,m\right) = T\left(\left[\tilde{f}\left(n,m\right)\right]\right) = E\left(\left[f\left(n,m\right)\right]\right)
		\end{equation*}
		Dove $E$ è l'operatore di media, perché $T$ essenzialmente è l'operatore di media
	\end{itemize}

	\noindent
	Si \textbf{osservi} che la somma dei valori del kernel è $1$:
	
	\begin{equation*}
		\begin{bmatrix}
			\frac{1}{9} & \frac{1}{9} & \frac{1}{9} \\
			&& \\
			\frac{1}{9} & \frac{1}{9} & \frac{1}{9} \\
			&& \\
			\frac{1}{9} & \frac{1}{9} & \frac{1}{9}
		\end{bmatrix}
	\end{equation*}

	\noindent
	Questo significa che il filtraggio in una locazione $\left(x,y\right)$ è una \textbf{combinazione lineare convessa}. In altre parole, la somma dei livelli di grigio dell'immagine originale $f$ e di quella processata $g$ sono uguali (a meno di padding!).\newline
	\textbf{Maggiore è l'ampiezza} $K$ della maschera, \textbf{più severo è l'effetto della media} sulla struttura dell'immagine.
	
	\begin{figure}[!htp]
		\centering
		\includegraphics[width=\textwidth]{img/filtro_media.png}
		\caption{Esempio di filtro media all'aumentare di $K$, ovvero della grandezza.}
	\end{figure}

	\subsubsection{Smoothing - Filtro mediano}
	
	Il \textcolor{Red3}{\textbf{\underline{filtro mediano}}} è utilizzato per \textbf{rimuovere il rumore impulsivo}. È un filtraggio $T$ non lineare, che si realizza attraverso un algoritmo. Data la matrice dell'immagine:
	
	\begin{equation*}
		\begin{bmatrix}
			240 & 245 & 0   \\
			247 & 0   & 244 \\
			251 & 246 & 250
		\end{bmatrix}
	\end{equation*}

	\begin{enumerate}
		\item Si calcola la media di tutti i valori della matrice:
		\begin{equation*}
			\dfrac{240+245+0+247+0+244+251+246+250}{9} \approx 191
		\end{equation*}
	
		\item Si inseriscono in un vettore riga i valori in ordine crescente:
		\begin{equation*}
			\begin{bmatrix}
				0 & 0 & 240 & 244 & 245 & 246 & 247 & 250 & 251
			\end{bmatrix}
		\end{equation*}
	
		\item Si calcola la mediana del vettore:
		\begin{equation*}
			9 \div 2 = 4.5 \longrightarrow \begin{bmatrix}
				0 & 0 & 240 & 244 & \underline{245} & 246 & 247 & 250 & 251
			\end{bmatrix}
		\end{equation*}
	\end{enumerate}

	\begin{figure}[!htp]
		\centering
		\includegraphics[width=\textwidth]{img/filtro_mediano.png}
		\caption{Esempio di applicazione di filtro mediano.}
	\end{figure}

	\newpage
	
	\subsubsection{Smoothing - Filtro gaussiano}
	
	Il \textcolor{Red3}{\textbf{\underline{filtro gaussiano}}} è quello di rendere più \dquotes{lisica (\emph{smooth})} l'immagine, in modo simile al filtraggio di media, e come parametro l'operatore prende il valore $\sigma$ che rappresenta la \textbf{forza} (maggiore è il valore, più è forte lo \emph{smoothing}). La \textbf{differenza} sostanziale tra il filtraggio di media e il filtraggio gaussiano è che quest'ultimo è una media pesata, dove i pesi più vicini al centro della maschera hanno valori più alti. Così facendo si ha:
	
	\begin{itemize}
		\item \textcolor{Green4}{\textbf{\emph{Vantaggio}}}
		\begin{itemize}
			\item Il filtraggio gaussiano effettua uno \emph{smoothing} più lieve, \textbf{preservando i contorni meglio} di quanto faccia il filtraggio media. Quindi, la struttura viene preservata meglio.
		\end{itemize}
		
		\item \textcolor{Red3}{\textbf{\emph{Svantaggio}}}
		\begin{itemize}
			\item Il rumore viene rimosso in maniera inferiore e questo provoca l'\textbf{impossibilità di applicare la formula di annullamento del rumore} visto per il filtro media.
		\end{itemize}
	\end{itemize}

	\noindent
	Questo filtro può essere \textbf{implementato in maniera efficiente} in quanto la maschera è separabile, ovvero è possibile eseguirlo facendo un filtraggio prima su tutte le $N$ righe dell'immagine come se fossero funzioni $1D$, e poi su tutte le $M$ colonne:
	
	\begin{equation*}
		\begin{array}{lll}
			I_{G}\left(i,j\right) & = & \displaystyle\sum_{h = -\frac{m}{2}}^{\frac{m}{2}} \displaystyle\sum_{k = -\frac{m}{2}}^{\frac{m}{2}} G\left(h,k\right) I\left(i+h, j+k\right) \\
			&& \\
			& = & \displaystyle\sum_{h = -\frac{m}{2}}^{\frac{m}{2}} \displaystyle\sum_{k = -\frac{m}{2}}^{\frac{m}{2}} \exp\left(-\dfrac{h^{2} + k^{2}}{2\sigma^{2}}\right) I\left(i+h, j+k\right) \\
			&& \\
			& = & \displaystyle\sum_{h = -\frac{m}{2}}^{\frac{m}{2}} \exp\left(-\dfrac{h^{2}}{2\sigma^{2}}\right) \displaystyle\sum_{k = - \frac{m}{2}}^{\frac{m}{2}} \exp\left(-\dfrac{k^{2}}{2\sigma^{2}}\right) I\left(i+h, j+k\right)
		\end{array}
	\end{equation*}

	\noindent
	Dove le variabili $i,j,h,k,m$ sono indici per comprendere la \textbf{separabilità}. Quest'ultima consente di progettare manualmente un filtro gaussiano come segue nella prossima pagina.
	
	\newpage
	
	\noindent
	Si definiscono i parametri $\sigma$ e $W$. Quindi, si fissi $\sigma$ e si trovi la dimensione della maschera $W$ sapendo che $W$ deve essere tale da contenere un'elevata percentuale di probabilità (uguale all'area della densità gaussiana, come in figura). In particolare, la statistica dice che con $W = 5\sigma$ si copre il $98.75\%$ dell'area della densità gaussiana. In altre parole, se si vuole $\sigma = 1$ allora $W = 5 \cdot 1 = 5$; se si vuole $\sigma = 0.6$ allora $W = 5 \cdot 0.6 = 3$.
	
	\begin{figure}[!htp]
		\centering
		\includegraphics[width=\textwidth]{img/filtro_gaussiano.pdf}
	\end{figure}

	\subsubsection{Domanda da \textcolor{Red3}{esame}}
	
	\textcolor{Red3}{\textbf{\emph{Domanda}}}\newline
	
	\noindent
	Il livello di noise gaussiano presente in un'immagine, se esso noto (e.g. $\sigma_{1} = 0.01$) può guidare la scelta del parametro $\sigma_{2}$ della maschera di filtro gaussiano?\newline
	
	\noindent
	\textcolor{Green4}{\textbf{\emph{Risposta}}}\newline
	
	\noindent
	No, perché il rumore gaussiano lavora su tutti i valori di grigio, mentre il filtro gaussiano sulle coordinate e non c'è correlazione. Invece, il filtro media è quello ideale.
	
	\newpage
	
	\subsubsection{Filtraggi di sharpening}
	
	I \textcolor{Red3}{\textbf{\underline{filtraggi di sharpening}}} servono per evidenziare i dettagli o come post processing dopo filtraggi di \emph{smoothing} (questo perché i filtraggi di \emph{smoothing} eliminano i dettagli). Per lo stesso motivo, i filtraggi di sharpening possono incrementare il rumore (un'immagine di rumore è un'immagine ad alta frequenza). Esistono due categorie: \textbf{\emph{basic highpass spatial filtering}} e \textbf{\emph{high boost filtering}}.\newline
	
	\noindent
	I filtri di sharpening sono detti anche \textbf{filtri di derivata}, poiché calcolano numericamente nell'intorno in cui sono definiti la derivata locale (prima o seconda) dell'immagine.
	
	\begin{center}
		\textcolor{Red3}{\textbf{\underline{Rispetto a $\boldsymbol{x}$}}}
	\end{center}

	\noindent
	\textcolor{Green4}{\textbf{Derivata asimmetrica}}
	
	\begin{equation*}
		\begin{array}{lll}
			I_{x}\left(x,y\right) & = & \dfrac{\partial I}{\partial x}\left(x,y\right) = \lim_{h \rightarrow 0} \dfrac{I\left(x+h, y\right) - I\left(x,y\right)}{h} \\
			&& \\
			I_{x}\left[x,y\right] & = & \dfrac{\partial I}{\partial x}\left[x,y\right] = I\left[x+1, y\right] - I\left[x,y\right]
		\end{array}
	\end{equation*}

	\noindent
	\textcolor{Green4}{\textbf{Derivata simmetrica}}
	
	\begin{equation*}
		\begin{array}{lll}
			I_{x}\left(x,y\right) & = & \dfrac{\partial I}{\partial x}\left(x,y\right) = \lim_{h \rightarrow 0} \dfrac{I\left(x+h, y\right) - I\left(x-h,y\right)}{2h} \\
			&& \\
			I_{x}\left[x,y\right] & = & \dfrac{\partial I}{\partial x}\left[x,y\right] = \dfrac{1}{2} \left(I\left[x+1, y\right] - I\left[x-1,y\right]\right)
		\end{array}
	\end{equation*}

	\noindent
	\textcolor{Green4}{\textbf{Filtro differenziale asimmetrico}}
	
	\begin{equation*}
		\partial_{x} = \begin{bmatrix}
			-1 & 1
		\end{bmatrix}
	\end{equation*}

	\noindent
	\textcolor{Green4}{\textbf{Filtro differenziale simmetrico}}
	
	\begin{equation*}
		\partial_{x} = \dfrac{1}{2} \begin{bmatrix}
			-1 & 0 & 1
		\end{bmatrix}
	\end{equation*}

	\newpage

	\begin{center}
		\textcolor{Red3}{\textbf{\underline{Rispetto a $\boldsymbol{y}$}}}
	\end{center}

	\noindent
	\textcolor{Green4}{\textbf{Derivata asimmetrica}}
	
	\begin{equation*}
		\begin{array}{lll}
			I_{y}\left(x,y\right) & = & \dfrac{\partial I}{\partial y}\left(x,y\right) = \lim_{h \rightarrow 0} \dfrac{I\left(x, y+h\right) - I\left(x,y\right)}{h} \\
			&& \\
			I_{y}\left[x,y\right] & = & \dfrac{\partial I}{\partial y}\left[x,y\right] = I\left[x, y+1\right] - I\left[x,y\right]
		\end{array}
	\end{equation*}

	\noindent
	\textcolor{Green4}{\textbf{Derivata simmetrica}}
	
	\begin{equation*}
		\begin{array}{lll}
			I_{y}\left(x,y\right) & = & \dfrac{\partial I}{\partial y}\left(x,y\right) = \lim_{h \rightarrow 0} \dfrac{I\left(x, y+h\right) - I\left(x,y-h\right)}{2h} \\
			&& \\
			I_{y}\left[x,y\right] & = & \dfrac{\partial I}{\partial y}\left[x,y\right] = \dfrac{1}{2} \left(I\left[x, y+1\right] - I\left[x,y-1\right]\right)
		\end{array}
	\end{equation*}
	
	\noindent
	\textcolor{Green4}{\textbf{Filtro differenziale asimmetrico}}
	
	\begin{equation*}
		\partial_{y} = \begin{bmatrix}
			-1 \\
			1
		\end{bmatrix}
	\end{equation*}

	\noindent
	\textcolor{Green4}{\textbf{Filtro differenziale simmetrico}}
	
	\begin{equation*}
		\partial_{y} = \dfrac{1}{2} \begin{bmatrix}
			-1 \\
			0 \\
			1
		\end{bmatrix}
	\end{equation*}
	
	\noindent
	\textbf{Proprietà} della \textbf{derivata prima}:
	
	\begin{itemize}[label=-]
		\item Nulla in regioni di intensità costante;
		\item Non nulla in presenza di variazioni di intensità.
	\end{itemize}
	
	\noindent
	\textbf{Proprietà} della \textbf{derivata seconda}:
	
	\begin{itemize}[label=-]
		\item Nulla in regioni di intensità costante;
		\item Nulla in presenza di variazioni costanti di intensità (rampe);
		\item Non nulla in presenza di variazioni non costanti (all'inizio e alla fine di rampe).
	\end{itemize}

	\newpage
	
	\subsubsection{Sharpening - Basic Highpass Spatial Filtering}
	
	Il filtraggio di sharpening chiamato \textcolor{Red3}{\textbf{\underline{basic highpass spatial filtering}}} è un \textbf{filtraggio lineare} con il laplaciano, con la maschera $H$ caratterizzata da coefficienti di un segno (e.g. positivo) vicino al centro e di segno opposto (e.g. negativo) nella periferia esterna.\newline
	
	\noindent
	Una tipica maschera di filtraggio, chiamata laplaciana:
	
	\begin{equation*}
		\dfrac{1}{9} \times \begin{bmatrix}
			-1 & -1 & -1 \\
			-1 & 8  & -1 \\
			-1 & -1 & -1
		\end{bmatrix}
	\end{equation*}

	\noindent
	Altre maschere laplaciane:
	
	\begin{equation*}
		\begin{bmatrix}
			0 & 1  & 0 \\
			1 & -4 & 1 \\
			0 & 1  & 0
		\end{bmatrix}
		\hspace{2em}
		\begin{bmatrix}
			1 & 1  & 1 \\
			1 & -8 & 1 \\
			1 & 1  & 1
		\end{bmatrix}
	\end{equation*}
	\begin{equation*}
		\begin{bmatrix}
			0  & -1 & 0 \\
			-1 & 4  & -1 \\
			0  & -1 & 0
		\end{bmatrix}
		\hspace{2em}
		\begin{bmatrix}
			-1 & -1 & -1 \\
			-1 &  8 & -1 \\
			-1 & -1 & -1
		\end{bmatrix}
	\end{equation*}

	\noindent
	Alcune caratteristiche:
	
	\begin{itemize}
		\item La \textbf{somma dei coefficienti è zero}. Questo indica che quando il filtro passa su regioni con livelli di grigio quasi stabili, l'output della maschera è zero o molto piccolo;
		
		\item L'\textbf{uscita è alta} quando il valore centrale differisce dai valori periferici;
		
		\item L'\textbf{immagine di output non assomiglierà} a quella originale;
		
		\item L'\textbf{immagine di output mostra tutti i dettagli};
		
		\item Sono inclusi alcuni \textbf{ridimensionamenti} e/o \emph{\textbf{clipping}}, necessari per compensare eventuali livelli di grigio negativi dopo il filtraggio.
	\end{itemize}

	\noindent
	Il filtraggio lineare utilizza il \emph{basic highpass spatial filtering} per \textbf{creare un'immagine realistica}, simile a quella di partenza, con gli edge amplificati:
	
	\begin{equation*}
		g\left(n,m\right) = T\left(\left[f\left(n,m\right)\right]\right) = f\left(n,m\right) + c \cdot h * f\left(n,m\right)
	\end{equation*}

	\noindent
	In cui $h$ indica la \textbf{maschera laplaciana}, $c$ è una \textbf{costante} pari a uno nel caso in cui il pixel centrale della maschera laplaciana sia positivo, altrimenti $-1$.
	
	\newpage
	
	\subsubsection{Sharpening - Filtro Laplaciano}
	
	La funzione \textcolor{Red3}{\textbf{\underline{laplaciana}}} prende in ingresso un parametro $\alpha$ il cui significato è legato all'importanza che si vuole dare agli edge verticali e orizzontali $\left(\alpha = 0\right)$, diagonali $\left(\alpha = 1\right)$, tutti gli edge $\left(\alpha = 0.5\right)$, attraverso questa formula:
	
	\begin{equation*}
		h = \dfrac{1}{\alpha + 1} \begin{bmatrix}
			-\alpha  & \alpha-1 & -\alpha  \\
			\alpha-1 & \alpha+5 & \alpha-1 \\
			-\alpha	 & \alpha-1 & -\alpha
		\end{bmatrix}
	\end{equation*}
	
	\subsubsection{Sharpening - High Boost Filtering}
	
	L'immagine filtrata dallo \emph{sharpening} si ottiene sottraendo l'immagine filtrata con \emph{smoothing} dall'immagine originale:
	
	\begin{equation*}
		\textbf{Immagine filtrata dallo sharpening} = \text{Originale } - \text{ Im. filtrata con smoothing}
	\end{equation*}

	\noindent
	Se la costante $A$ rappresenta un \textbf{fattore di amplificazione degli edge}, allora il filtro \textcolor{Red3}{\textbf{\underline{high boost filtering}}} è definito come:
	
	\begin{equation*}
		\textbf{High-boost} = A \cdot \text{ Originale } + \text{ Im. filt. dallo sharpening}
	\end{equation*}

	\noindent
	A differenza degli altri filtri, questo dà maggiore libertà al progettista. Infatti, il blur può avvenire attraverso una maschera di supporto arbitrariamente grande.
\end{document}
\documentclass[a4paper]{article}
\usepackage[T1]{fontenc}			% pacchetto per \chapter
\usepackage[italian]{babel}
\usepackage[italian]{isodate}  		% formato delle date in italiano
\usepackage{lipsum}
\usepackage{graphicx}				% gestione delle immagini
\usepackage{amsfonts}
\usepackage{booktabs}				% tabelle di qualità superiore
\usepackage{amsmath}				% pacchetto matematica
\usepackage{amsthm}					% teoremi migliorati
\usepackage{enumitem}				% gestione delle liste
\usepackage{pifont}					% pacchetto con elenchi carini
\usepackage{listings}				% pacchetto per i codici
\usepackage{caption}
\usepackage{cancel}
\usepackage[x11names]{xcolor}		% pacchetto colori RGB
% Link ipertestuali per l'indice
\usepackage{xcolor}
\usepackage[linkcolor=black, citecolor=blue, urlcolor=cyan]{hyperref}
\hypersetup{
	colorlinks=true
}

\definecolor{codegreen}{rgb}{0,0.6,0}
\definecolor{codegray}{rgb}{0.5,0.5,0.5}
\definecolor{codepurple}{rgb}{0.58,0,0.82}
\definecolor{backcolour}{rgb}{0.95,0.95,0.92}
\lstdefinestyle{mystyle}{
	backgroundcolor=\color{backcolour},   
	commentstyle=\color{codegreen},
	keywordstyle=\color{magenta},
	numberstyle=\tiny\color{codegray},
	stringstyle=\color{codepurple},
	basicstyle=\ttfamily\footnotesize,
	breakatwhitespace=false,         
	breaklines=true,                 
	captionpos=b,                    
	keepspaces=true,                 
	numbers=left,                    
	numbersep=5pt,                  
	showspaces=false,                
	showstringspaces=false,
	showtabs=false,                  
	tabsize=2
}
\lstset{style=mystyle}

\definecolor{maroon}{rgb}{0.5,0,0}
\definecolor{darkgreen}{rgb}{0,0.5,0}
\lstdefinelanguage{XML}
{
	basicstyle=\ttfamily\footnotesize,
	morestring=[s]{"}{"},
	morecomment=[s]{?}{?},
	morecomment=[s]{!--}{--},
	commentstyle=\color{darkgreen},
	moredelim=[s][\color{black}]{>}{<},
	moredelim=[s][\color{red}]{\ }{=},
	stringstyle=\color{blue},
	identifierstyle=\color{maroon}
}

%\usepackage{showframe}				% visualizzazione bordi
%\usepackage{showkeys}				% visualizzazione etichetta

\newcommand{\longline}{\noindent\rule{\textwidth}{0.4pt}}
\newcommand{\dquotes}[1]{``#1''}
\renewcommand{\qedsymbol}{QED}

\begin{document}
	\author{VR443470}
	\title{Esami - Basi di dati}
	\date{\printdayoff\today}
	\maketitle

	\newpage
	
	% indice
	\tableofcontents
	
	\newpage
		
	\section{Domande di teoria - Primo parziale}
	
	Le domande più frequenti che si possono incontrare nel primo parziale di Basi di dati sono:
	\begin{enumerate}
		\item Si illustri il concetto/costrutto di \textbf{entità} nel modello Entità-Relazione.
		
		\item Si illustri il concetto/costrutto di \textbf{relazione} nel modello Entità-Relazione.
		
		\item Si illustri il concetto/costrutto di \textbf{generalizzazione} nel modello Entità-Relazione.
		
		\item Si illustri il concetto/costrutto di \textbf{identificatore} nel modello Entità-Relazione.
		
		\item Si illustri il concetto/costrutto di \textbf{superchiave} nel modello Entità-Relazione.
		
		\item Si illustri il concetto/costrutto di \textbf{attributo multivalore} nel modello Entità-Relazione.
	\end{enumerate}
	La risposta, per essere considerata perfetta, deve includere le seguenti caratteristiche:
	\begin{itemize}
		\item Semantica
		\item Sintassi grafica con esempio
		\item Istanza
		\item Eventuali proprietà
	\end{itemize}
	Qui di seguito vengono date le possibili risposte alle domande di teoria:
	\begin{enumerate}
		\item \textcolor{Green4}{\emph{Si illustri il concetto/costrutto di \textbf{entità} nel modello Entità-Relazione.}}
		
		\textbf{Semantica}. L'entità rappresenta una classe di oggetti che hanno proprietà comuni ed esistenza \dquotes{autonoma} ai fini dell'applicazione di interesse. Il nome dato ad ogni entità è identificativo di quella determinata classe di oggetti e deve essere univoco all'interno dello schema.\newline
		\textbf{Sintassi grafica}. Per esempio, l'entità studenti rappresenta la classe di oggetti degli studenti di un'università e gli attributi possibili possono essere: matricola, nome, cognome, data di nascita, ecc. La sintassi grafica è la seguente:
		\begin{figure}[!htp]
			\centering
			\includegraphics[width=.3\textwidth]{img/entita_def.pdf}
		\end{figure}
		
		\noindent
		\textbf{Istanza}. L'istanza di un'entità è un oggetto della classe che lo rappresenta e non un unico valore. Per esempio, nell'entità studenti, lo studente Mario Rossi (in carne ed ossa) rappresenta un'istanza dell'entità.
		\newpage
		
		
		\item \textcolor{Green4}{\emph{Si illustri il concetto/costrutto di \textbf{relazione} nel modello Entità-Relazione.}}
		
		\textbf{Semantica}. La relazione rappresenta legami logici tra una o più entità. Ogni relazione deve avere un nome univoco all'interno dello schema e non può avere identificatori. Esistono due tipi di relazioni: \emph{ricorsive}, cioè in cui è coinvolta una sola entità, \emph{n-arie}, in cui sono coinvolte \emph{n} entità. Esse nascono solo quando le entità coinvolte contengono almeno una tupla.\newline
		\textbf{Sintassi grafica}. Un esempio di relazione è la \dquotes{Residenza} tra le entità \dquotes{Città} e \dquotes{Impiegato}. La sua sintassi grafica è un rombo:
		\begin{figure}[!htp]
			\centering
			\includegraphics[width=.5\textwidth]{img/relazione_def.pdf}
		\end{figure}
		
		\noindent
		\textbf{Istanza}. Data una relazione $R$ tra $n$ entità $E_{1}, E_{2}, ..., E_{n}$, un'istanza è composta da una ennupla del tipo:
		\begin{equation*}
			\left(e_{1}, e_{2}, ..., e_{i}\right) \: \text{dove} \: e_{i} \in I\left(E_{i}\right), 1 \le i \le n
		\end{equation*}
		Inoltre, esiste un'importante proprietà che afferma:
		\begin{equation*}
			I\left(R\right) \subseteq I\left(E_{1}\right) \times I\left(E_{2}\right) \times ... \times I\left(E_{n}\right)
		\end{equation*}
		\newpage
		
		
		\item \textcolor{Green4}{\emph{Si illustri il concetto/costrutto di \textbf{generalizzazione} nel modello Entità-Relazione.}}
		
		\textbf{Semantica}. Le generalizzazioni rappresentano legami logici tra un'entità $E$, chiamata genitore, e più entità $E_{1}, ..., E_{n}$, chiamate figlie. Quindi, si dice che l'entità $E$ (genitore) è la generalizzazione delle entità $E_{1},...,E_{n}$ (figlie) e quest'ultime vengono chiamate specializzazioni. Inoltre, ogni occorrenza dell'entità figlia è anche un'occorrenza dell'entità padre, e ogni proprietà dell'entità padre è anche una proprietà dell'entità figlia.\newline
		La classificazioni sono coppie di valori che hanno diverso significato:
		\begin{itemize}
			\item (totale, esclusiva)
			
			\item (totale, sovrapposta)
			
			\item (parziale, esclusiva)
			
			\item (parziale, sovrapposta)
		\end{itemize}
		Con totale, il genitore ha ogni occorrenza posseduta da almeno un'entità figlia. In caso contrario è parziale.\newline
		Con esclusiva, il genitore ha ogni occorrenza che si ripete solamente in una delle entità figlie. In caso un'occorrenza del genitore sia di più entità figlie, si dice sovrapposta.\newline
		\textbf{Sintassi grafica}. Un esempio è la generalizzazione \dquotes{Persona} con le specializzazioni \dquotes{Uomo} e \dquotes{Donna}. La sintassi grafica:
		\begin{figure}[!htp]
			\centering
			\includegraphics[width=.5\textwidth]{img/generalizzazione_sintassi.pdf}
		\end{figure}
		\newpage
		
		
		\item \textcolor{Green4}{\emph{Si illustri il concetto/costrutto di \textbf{identificatore} nel modello Entità-Relazione.}}
		
		\textbf{Semantica}. Gli identificatori descrivono i concetti (attributi/entità) dello schema che consentono di identificare in maniera univoca le occorrenze delle entità. Devono essere specificati per ogni entità e non possono apparire all'interno di relazioni. Un identificatore può essere:
		\begin{itemize}
			\item Interno, ovvero viene scelto un attributo dell'entità;
			\item Esterno, viene scelto un identificatore di un'altra identità;
		\end{itemize}
		È possibile utilizzare sia identificatori interni ed esterni insieme.\newline
		\textbf{Sintassi grafica}. Un esempio è l'entità \dquotes{Studente} che possiede come identificatore la \dquotes{Matricola} poiché unica. La sintassi grafica:
		\begin{figure}[!htp]
			\centering
			\includegraphics[width=.6\textwidth]{img/identificatore_def.pdf}
		\end{figure}
		
		
		\item \textcolor{Green4}{\emph{Si illustri il concetto/costrutto di \textbf{superchiave} nel modello Entità-Relazione.}}
		
		\textbf{Semantica}. Una superchiave è un'insieme di attributi che non contiene tuple duplicate al suo interno. Una superchiave è una chiave prima se e solo se è una superchiave minimale. Invece, una chiave primaria è sempre superchiave (non viceversa!).\newline
		\textbf{Sintassi grafica}. Non esiste una sintassi grafica poiché è un concetto, ma un esempio:
		\begin{figure}[!htp]
			\centering
			\includegraphics[width=\textwidth]{img/superchiave.png}
		\end{figure}
		
		\noindent
		Nessuna tupla si ripete, quindi \dquotes{Matricola, Cognome, Nome} è una superchiave valida. Non è minimale poiché esiste \dquotes{Matricola} che è chiave primaria e superchiave minimale.
		\newpage
		
		
		\item \textcolor{Green4}{\emph{Si illustri il concetto/costrutto di \textbf{attributo} nel modello Entità-Relazione.}}
		
		\textbf{Semantica}. Gli attributi descrivono le proprietà elementari di entità o relazioni che sono di interesse ai fini dell'applicazione. Ogni attributo ha un suo dominio e quindi può essere visto come una funzione che ha come dominio le istanze dell'entità/relazione e come codominio l'insieme dei valori ammissibili:
		\begin{equation*}
			f_{a} : I\left(E\right) \rightarrow D
		\end{equation*}
		Dove $a$ è un attributo dell'entità $E$, $I\left(E\right)$ è l'insieme delle istanze di $E$ e $D$ è l'insieme dei valori ammissibili.\newline
		\textbf{Sintassi grafica}. Un esempio di attributo è \dquotes{Cognome}, \dquotes{Stipendio} ed \dquotes{Età} dell'entità \dquotes{Impiegato}. La sintassi grafica è la seguente:
		\begin{figure}[!htp]
			\centering
			\includegraphics[width=.5\textwidth]{img/attributo_def.pdf}
		\end{figure}
		
		\noindent
		\textbf{Istanza}. L'istanza si ottiene tramite una funzione che data un'istanza dell'entità $E$ (o relazione $R$), restituisce l'attributo $a$:
		\begin{equation*}
			\text{valore di} \: a \: \text{su} \: e = f_{a}\left(e\right)
		\end{equation*}
	\end{enumerate}\newpage
	
	\section{Esercizi terzo parziale}
	
	\subsection[$\mathrm{B^{+}\text{-}tree}$]{$\boldsymbol{\mathrm{B^{+}\text{-}tree}}$}
	
	\subsubsection{Esercizio 1}
	
	Costruire un $\mathrm{B}^{+}$-tree di $\text{fan-out} = 5$ con i seguenti nodi foglia: $\left(A,B,C,D\right)$, $\left(F,G,M,N\right)$, $\left(O,P\right)$, $\left(S,T\right)$, $\left(W,Z\right)$. I vincoli di riempimento sono:
	\begin{itemize}
		\item $2 \le \text{\#chiavi} \le 4$
		
		\item $3 \le \text{\#puntatori} \le 5$
	\end{itemize}
	Dopodiché, inserire il valore chiave H nel $\mathrm{B}^{+}$-tree ottenuto precedentemente. Infine, l'esercizio si conclude eseguendo una rimozione del valore chiave Z ottenuto precedentemente.\newline
	
	\noindent
	\textcolor{Green4}{\textbf{\emph{\underline{Soluzione}}}}\newline
	
	\noindent
	Il primo passo è costruire i vari livelli dei nodi foglia:
	\begin{figure}[!htp]
		\centering
		\includegraphics[width=\textwidth]{img/b+-tree.pdf}
	\end{figure}
	
	\noindent
	Adesso è necessario costruire tutti i puntatori richiesti. Fan-out è uguale a 5 quindi viene costruito un nodo intermedio con 5 puntatori e si collegano tutti i nodi:
	\begin{figure}[!htp]
		\centering
		\includegraphics[width=\textwidth]{img/b+-tree-1.pdf}
	\end{figure}
	
	\noindent
	Adesso si aggiungono le lettere che devono essere raggiunte dopo aver visitato ogni nodo:
	\begin{figure}[!htp]
		\centering
		\includegraphics[width=\textwidth]{img/b+-tree-2.pdf}
	\end{figure}\newpage
	
	\noindent
	Per inserire il valore chiave è necessario avere a disposizione una posizione libera. Tuttavia, questo non è possibile, per cui viene applicato uno split. Viene divisa la radice contenente $\left(F,G,M,N\right)$ così da inserire la chiave H tra la G e la M:
	\begin{figure}[!htp]
		\centering
		\includegraphics[width=\textwidth]{img/b+-tree-3.pdf}
	\end{figure}
	
	\noindent
	A questo punto è necessario riadattare il nodo radice che attualmente punta ad un nodo errato (attenzione c'è un errore, il nodo V in realtà è il nodo W):
	\begin{figure}[!htp]
		\centering
		\includegraphics[width=\textwidth]{img/b+-tree-4.pdf}
	\end{figure}
	
	\noindent
	Per farlo, è necessario eseguire una divisione anche nel nodo radice aggiustando i valori delle chiavi (attenzione c'è un errore, il nodo V in realtà è il nodo W):
	\begin{figure}[!htp]
		\centering
		\includegraphics[width=\textwidth]{img/b+-tree-5.pdf}
	\end{figure}\newpage
	
	\noindent
	E infine, collegare i due nodi divisi con un nodo di congiunzione. Inoltre, quest'ultimo viene riempito con un valore chiave (attenzione c'è un errore, il nodo V in realtà è il nodo W):
	\begin{figure}[!htp]
		\centering
		\includegraphics[width=\textwidth]{img/b+-tree-6.pdf}
	\end{figure}
	
	\noindent
	La rimozione della chiave Z comporta che l'ultimo nodo abbia come chiave solo il valore W. Questo comporta un'irregolarità poiché il numero minimo di ogni chiave in ogni nodo deve essere minimo di due e massimo di quattro. Per cui è necessario effettuare un merge:
	\begin{figure}[!htp]
		\centering
		\includegraphics[width=\textwidth]{img/b+-tree-7.pdf}
		\caption*{Eliminazione della chiave Z.}
	\end{figure}
	
	\begin{figure}[!htp]
		\centering
		\includegraphics[width=\textwidth]{img/b+-tree-8.pdf}
		\caption*{Merge degli ultimi due nodi.}
	\end{figure}\newpage
	
	\subsubsection{Esercizio 2}
	
	Data la seguente lista di possibili valori chiave:
	\begin{equation*}
		\begin{array}{lll}
			L &=& (1,2,3,4,5,6,7,8,9,10,11,12,13,14,15,16,17,18,19,20,21,22,23,24,25, \\
			&& \phantom{(}26,27,28,29,30)
		\end{array}
	\end{equation*}
	Costruire un B+-tree (\emph{fan-out} = 5) che contenga i seguenti nodi foglia:
	\begin{figure}[!htp]
		\centering
		\includegraphics[width=\textwidth]{img/b+-tree-9.pdf}
	\end{figure}
	
	\noindent
	Mostrare l'albero dopo l'inserimento del valore chiave 5 e partendo da questo risultato, mostrare l'albero dopo la cancellazione del valore chiave 28.\newline
	
	\noindent
	\textcolor{Green4}{\textbf{\emph{\underline{Soluzione}}}}\newline
	
	\noindent
	Si costruisce la lista:
	\begin{figure}[!htp]
		\centering
		\includegraphics[width=\textwidth]{img/b+-tree-10.pdf}
	\end{figure}
	
	\noindent
	E poi l'albero:
	\begin{figure}[!htp]
		\centering
		\includegraphics[width=\textwidth]{img/b+-tree-11.pdf}
	\end{figure}\newpage
	
	\noindent
	Si inserisce il valore 5:
	\begin{figure}[!htp]
		\centering
		\includegraphics[width=\textwidth]{img/b+-tree-12.pdf}
	\end{figure}
	
	\noindent
	E infine si rimuove il valore 28:
	\begin{figure}[!htp]
		\centering
		\includegraphics[width=\textwidth]{img/b+-tree-13.pdf}
	\end{figure}\newpage
	
	\subsection{Verificare che uno schedule sia VSR (View-serializzabile)}
	
	\subsubsection{Esercizio 1 - Perdita di aggiornamento}
	
	Date due transazioni $T_{1}$ e $T_{2}$ di seguito descritte:
	\begin{equation*}
		\begin{array}{lll}
			T_{1} &:& r_{1}\left(x\right) \: w_{1}\left(x\right) \\
			T_{2} &:& r_{2}\left(x\right) \: w_{2}\left(x\right)
		\end{array}
	\end{equation*}
	Lo schedule che rappresenta l'anomalia è il seguente:
	\begin{equation*}
		S_{PA} = r_{1}\left(x\right) \: r_{2}\left(x\right) \: w_{2}\left(x\right) \: w_{1}\left(x\right)
	\end{equation*}
	Per verificare che uno schedule sia VSR o meno, è necessario caratterizzare $S_{PA}$ calcolando l'insieme delle relazioni LeggeDa e l'insieme delle ScrittureFinali.\newline
	
	\noindent
	Quindi, per l'insieme LeggeDa viene cercato per ogni operazione di lettura, una precedente scrittura sulla stessa risorsa fatta da un'altra transazione. In questo caso, l'insieme è vuoto poiché nessuna risorsa scrive prime di una lettura.\newline
	
	\noindent
	Invece, per l'insieme ScrittureFinali, per ogni risorsa indicata nello schedule si specifica l'ultima scrittura eseguita. In questo caso, l'unica risorsa è $x$ e l'ultima scrittura è $w_{1}\left(x\right)$.\newline
	
	\noindent
	Quindi, gli insiemi sono composti da:
	\begin{equation*}
		\begin{array}{rll}
			\text{LeggeDa}\left(S_{PA}\right) &=& \emptyset \\
			\text{ScrittureFinali}\left(S_{PA}\right) &=& \left\{w_{1}\left(x\right)\right\}
		\end{array}
	\end{equation*}
	Adesso si generano tutti i possibili schedule seriali che eseguono le due transazioni. Essi si ottengono generando le possibili permutazioni dell'insieme di transazioni che partecipano allo schedule. In questo caso, date solo due transazioni $T_{1}$ e $T_{2}$, le possibili combinazioni sono:
	\begin{equation*}
		\begin{array}{lllll}
			S_{1} &=& T_{1} \: T_{2} &=& r_{1}\left(x\right) \: w_{1}\left(x\right) \: r_{2}\left(x\right) \: w_{2}\left(x\right) \\
			S_{2} &=& T_{2} \: T_{1} &=& r_{2}\left(x\right) \: w_{2}\left(x\right) \: r_{1}\left(x\right) \: w_{1}\left(x\right) \\
		\end{array}
	\end{equation*}
	Adesso, si verifica che almeno uno dei due schedule seriali è view-equivalente a $S_{PA}$.\newpage
	
	\noindent
	Verifica partendo dallo schedule $S_{1}$:
	\begin{enumerate}
		\item Creazione dell'insieme LeggeDa$\left(S_{1}\right)$. Data la sequenza:
		\begin{equation*}
			S_{1} = r_{1}\left(x\right) \: w_{1}\left(x\right) \: r_{2}\left(x\right) \: w_{2}\left(x\right)
		\end{equation*}
		L'unica scrittura che precede una lettura è $w_{1}\left(x\right)$. Quindi, l'insieme è composto dalla scrittura che avviene prima della lettura e da quest'ultima:
		\begin{equation*}
			\text{LeggeDa}\left(S_{1}\right) = \left\{\left(r_{2}\left(x\right), w_{1}\left(x\right)\right)\right\}
		\end{equation*}
		
		\item Creazione dell'insieme ScrittureFinali$\left(S_{1}\right)$. Data la sequenza:
		\begin{equation*}
			S_{1} = r_{1}\left(x\right) \: w_{1}\left(x\right) \: r_{2}\left(x\right) \: w_{2}\left(x\right)
		\end{equation*}
		L'unica risorsa $x$ ha come ultima scrittura $w_{2}\left(x\right)$, quindi l'insieme è composto da:
		\begin{equation*}
			\text{ScrittureFinali}\left(S_{1}\right) = \left\{w_{2}\left(x\right)\right\}
		\end{equation*}
		
		\item Si esegue il confronto degli insiemi ottenuti da $S_{1}$ e dagli insiemi ottenuti da $S_{PA}$:
		\begin{equation*}
			\begin{array}{lll}
				\text{LeggeDa}\left(S_{PA}\right)	&=& \emptyset \\
				\text{LeggeDa}\left(S_{1}\right)	&=& \left\{\left(r_{2}\left(x\right), w_{1}\left(x\right)\right)\right\} \\
				\text{ScrittureFinali}\left(S_{PA}\right)	&=& \left\{w_{1}\left(x\right)\right\} \\
				\text{ScrittureFinali}\left(S_{1}\right)	&=& \left\{w_{2}\left(x\right)\right\}
			\end{array}
		\end{equation*}
		Come è evidente, nessuno dei due insiemi è equivalente:
		\begin{equation*}
			\begin{array}{lll}
				\text{LeggeDa}\left(S_{PA}\right)	&\cancel{\equiv}& \text{LeggeDa}\left(S_{1}\right) \\
				\text{ScrittureFinali}\left(S_{PA}\right)	&\cancel{\equiv}& \text{ScrittureFinali}\left(S_{1}\right)
			\end{array}
		\end{equation*}
		Quindi, è possibile concludere che $S_{PA}$ non è view-equivalente a $S_{1}$.
	\end{enumerate}\newpage
	
	\noindent
	Verifica partendo dallo schedule $S_{2}$:
	\begin{enumerate}
		\item Creazione dell'insieme LeggeDa$\left(S_{2}\right)$. Data la sequenza:
		\begin{equation*}
			S_{2} = r_{2}\left(x\right) \: w_{2}\left(x\right) \: r_{1}\left(x\right) \: w_{1}\left(x\right)
		\end{equation*}
		L'unica scrittura che precede una lettura è $w_{2}\left(x\right)$. Quindi, l'insieme è composto dalla scrittura che avviene prima della lettura e da quest'ultima:
		\begin{equation*}
			\text{LeggeDa}\left(S_{2}\right) = \left\{\left(r_{1}\left(x\right), w_{2}\left(x\right)\right)\right\}
		\end{equation*}
		
		\item Creazione dell'insieme ScrittureFinali$\left(S_{2}\right)$. Data la sequenza:
		\begin{equation*}
			S_{2} = r_{2}\left(x\right) \: w_{2}\left(x\right) \: r_{1}\left(x\right) \: w_{1}\left(x\right)
		\end{equation*}
		L'unica risorsa $x$ ha come ultima scrittura $w_{2}\left(x\right)$, quindi l'insieme è composto da:
		\begin{equation*}
			\text{ScrittureFinali}\left(S_{2}\right) = \left\{w_{1}\left(x\right)\right\}
		\end{equation*}
		
		\item Si esegue il confronto degli insiemi ottenuti da $S_{1}$ e dagli insiemi ottenuti da $S_{PA}$:
		\begin{equation*}
			\begin{array}{lll}
				\text{LeggeDa}\left(S_{PA}\right)	&=& \emptyset \\
				\text{LeggeDa}\left(S_{1}\right)	&=& \left\{\left(r_{1}\left(x\right), w_{2}\left(x\right)\right)\right\} \\
				\text{ScrittureFinali}\left(S_{PA}\right)	&=& \left\{w_{1}\left(x\right)\right\} \\
				\text{ScrittureFinali}\left(S_{1}\right)	&=& \left\{w_{1}\left(x\right)\right\}
			\end{array}
		\end{equation*}
		Come è evidente, soltanto uno dei due insiemi è equivalente:
		\begin{equation*}
			\begin{array}{lll}
				\text{LeggeDa}\left(S_{PA}\right)	&\cancel{\equiv}& \text{LeggeDa}\left(S_{1}\right) \\
				\text{ScrittureFinali}\left(S_{PA}\right)	&\equiv& \text{ScrittureFinali}\left(S_{1}\right)
			\end{array}
		\end{equation*}
		Quindi, è possibile concludere che $S_{PA}$ non è view-equivalente a $S_{1}$ poiché entrambi gli insiemi non sono equivalenti.
	\end{enumerate}
	L'esercizio si conclude qua. Nessuna combinazione è view-equivalente allo schedule di partenza $S_{PA}$. Quindi, si conclude affermando che $S_{PA}$ non è VSR.\newpage
	
	\subsubsection{Esercizio 2 - Lettura inconsistente}
	
	Date due transazioni $T_{1}$ e $T_{2}$ di seguito descritte:
	\begin{equation*}
		\begin{array}{lll}
			T_{1} &:& r_{1}\left(x\right) \: r_{1}'\left(x\right) \\
			T_{2} &:& r_{2}\left(x\right) \: w_{2}\left(x\right)
		\end{array}
	\end{equation*}
	Lo schedule che rappresenta l'anomalia è il seguente:
	\begin{equation*}
		S_{LI} = r_{1}\left(x\right) \: r_{2}\left(x\right) \: w_{2}\left(x\right) \: r_{1}'\left(x\right)
	\end{equation*}
	Per verificare che uno schedule sia VSR o meno, è necessario caratterizzare $S_{LI}$ calcolando l'insieme delle relazioni LeggeDa e l'insieme delle ScrittureFinali.\newline
	
	\noindent
	Quindi, per l'insieme LeggeDa viene cercato per ogni operazione di lettura, una precedente scrittura sulla stessa risorsa fatta da un'altra transazione. In questo caso, l'insieme è composto da $w_{2}\left(x\right)$ perché precede $r_{1}'\left(x\right)$.\newline
	
	\noindent
	Invece, per l'insieme ScrittureFinali, per ogni risorsa indicata nello schedule si specifica l'ultima scrittura eseguita. In questo caso, l'unica risorsa è $x$ e l'ultima scrittura è $w_{2}\left(x\right)$.\newline
	
	\noindent
	Quindi, gli insiemi sono composti da:
	\begin{equation*}
		\begin{array}{rll}
			\text{LeggeDa}\left(S_{LI}\right) &=& \left\{\left(r_{1}'\left(x\right), w_{2}\left(x\right)\right)\right\} \\
			\text{ScrittureFinali}\left(S_{LI}\right) &=& \left\{w_{2}\left(x\right)\right\}
		\end{array}
	\end{equation*}
	Adesso si generano tutti i possibili schedule seriali che eseguono le due transazioni. Essi si ottengono generando le possibili permutazioni dell'insieme di transazioni che partecipano allo schedule. In questo caso, date solo due transazioni $T_{1}$ e $T_{2}$, le possibili combinazioni sono:
	\begin{equation*}
		\begin{array}{lllll}
			S_{1} &=& T_{1} \: T_{2} &=& r_{1}\left(x\right) \: r_{1}'\left(x\right) \: r_{2}\left(x\right) \: w_{2}\left(x\right) \\
			S_{2} &=& T_{2} \: T_{1} &=& r_{2}\left(x\right) \: w_{2}\left(x\right) \: r_{1}\left(x\right) \: r_{1}'\left(x\right) \\
		\end{array}
	\end{equation*}
	Adesso, si verifica che almeno uno dei due schedule seriali è view-equivalente a $S_{LI}$.\newpage
	
	\noindent
	Verifica partendo dallo schedule $S_{1}$:
	\begin{enumerate}
		\item Creazione dell'insieme LeggeDa$\left(S_{1}\right)$. Data la sequenza:
		\begin{equation*}
			S_{1} = r_{1}\left(x\right) \: r_{1}'\left(x\right) \: r_{2}\left(x\right) \: w_{2}\left(x\right)
		\end{equation*}
		L'unica scrittura che precede una lettura è $w_{1}\left(x\right)$. Quindi, l'insieme è composto dalla scrittura che avviene prima della lettura e da quest'ultima:
		\begin{equation*}
			\text{LeggeDa}\left(S_{1}\right) = \emptyset
		\end{equation*}
		
		\item Creazione dell'insieme ScrittureFinali$\left(S_{1}\right)$. Data la sequenza:
		\begin{equation*}
			S_{1} = r_{1}\left(x\right) \: r_{1}'\left(x\right) \: r_{2}\left(x\right) \: w_{2}\left(x\right)
		\end{equation*}
		L'unica risorsa $x$ ha come ultima scrittura $w_{2}\left(x\right)$, quindi l'insieme è composto da:
		\begin{equation*}
			\text{ScrittureFinali}\left(S_{1}\right) = \left\{w_{2}\left(x\right)\right\}
		\end{equation*}
		
		\item Si esegue il confronto degli insiemi ottenuti da $S_{1}$ e dagli insiemi ottenuti da $S_{LI}$:
		\begin{equation*}
			\begin{array}{lll}
				\text{LeggeDa}\left(S_{LI}\right)	&=& \left\{\left(r_{1}'\left(x\right), w_{2}\left(x\right)\right)\right\} \\
				\text{LeggeDa}\left(S_{1}\right)	&=& \emptyset \\
				\text{ScrittureFinali}\left(S_{LI}\right)	&=& \left\{w_{2}\left(x\right)\right\} \\
				\text{ScrittureFinali}\left(S_{1}\right)	&=& \left\{w_{2}\left(x\right)\right\}
			\end{array}
		\end{equation*}
		Come è evidente, soltanto uno dei due insiemi è equivalente:
		\begin{equation*}
			\begin{array}{lll}
				\text{LeggeDa}\left(S_{LI}\right)	&\cancel{\equiv}& \text{LeggeDa}\left(S_{1}\right) \\
				\text{ScrittureFinali}\left(S_{LI}\right)	&\equiv& \text{ScrittureFinali}\left(S_{1}\right)
			\end{array}
		\end{equation*}
		Quindi, è possibile concludere che $S_{LI}$ non è view-equivalente a $S_{1}$.
	\end{enumerate}\newpage
	
	\noindent
	Verifica partendo dallo schedule $S_{2}$:
	\begin{enumerate}
		\item Creazione dell'insieme LeggeDa$\left(S_{2}\right)$. Data la sequenza:
		\begin{equation*}
			S_{2} = r_{2}\left(x\right) \: w_{2}\left(x\right) \: r_{1}\left(x\right) \: r_{1}'\left(x\right)
		\end{equation*}
		L'unica scrittura che precede due letture è $w_{2}\left(x\right)$. Quindi, l'insieme è composto dalla scrittura che avviene con due letture:
		\begin{equation*}
			\text{LeggeDa}\left(S_{2}\right) = \left\{\left(r_{1}'\left(x\right), w_{2}\left(x\right)\right), \left(r_{1}\left(x\right), w_{2}\left(x\right)\right)\right\}
		\end{equation*}
		
		\item Creazione dell'insieme ScrittureFinali$\left(S_{2}\right)$. Data la sequenza:
		\begin{equation*}
			S_{2} = r_{2}\left(x\right) \: w_{2}\left(x\right) \: r_{1}\left(x\right) \: r_{1}'\left(x\right)
		\end{equation*}
		L'unica risorsa $x$ ha come ultima scrittura $w_{2}\left(x\right)$, quindi l'insieme è composto da:
		\begin{equation*}
			\text{ScrittureFinali}\left(S_{2}\right) = \left\{w_{2}\left(x\right)\right\}
		\end{equation*}
		
		\item Si esegue il confronto degli insiemi ottenuti da $S_{1}$ e dagli insiemi ottenuti da $S_{PA}$:
		\begin{equation*}
			\begin{array}{lll}
				\text{LeggeDa}\left(S_{LI}\right)	&=& \left\{\left(r_{1}'\left(x\right), w_{2}\left(x\right)\right)\right\} \\
				\text{LeggeDa}\left(S_{2}\right)	&=& \left\{\left(r_{1}'\left(x\right), w_{2}\left(x\right)\right), \left(r_{1}\left(x\right), w_{2}\left(x\right)\right)\right\} \\
				\text{ScrittureFinali}\left(S_{LI}\right)	&=& \left\{w_{2}\left(x\right)\right\} \\
				\text{ScrittureFinali}\left(S_{2}\right)	&=& \left\{w_{2}\left(x\right)\right\}
			\end{array}
		\end{equation*}
		Come è evidente, soltanto uno dei due insiemi è equivalente:
		\begin{equation*}
			\begin{array}{lll}
				\text{LeggeDa}\left(S_{PA}\right)	&\cancel{\equiv}& \text{LeggeDa}\left(S_{1}\right) \\
				\text{ScrittureFinali}\left(S_{PA}\right)	&\equiv& \text{ScrittureFinali}\left(S_{1}\right)
			\end{array}
		\end{equation*}
		Quindi, è possibile concludere che $S_{LI}$ non è view-equivalente a $S_{2}$ poiché entrambi gli insiemi non sono equivalenti.
	\end{enumerate}
	L'esercizio si conclude qua. Nessuna combinazione è view-equivalente allo schedule di partenza $S_{LI}$. Quindi, si conclude affermando che $S_{LI}$ non è VSR.\newpage
	
	\subsubsection{Sintesi dell'algoritmo}
	
	In sintesi l'algoritmo per capire se uno schedule è VSR:
	\begin{enumerate}
		\item Si tiene bene in considerazione lo schedule che rappresenta l'anomalia, ovvero quello che viene dato;
		
		\item Si compongono i due insiemi:
		\begin{enumerate}
			\item Creazione insieme LeggeDa cercando per ogni operazione di lettura ($r_{i}\left(\text{risorsa}\right)$) una precedente operazione di scrittura sulla stessa risorsa fatta da un'altra transazione. Nel caso in cui si trovi, si aggiunge all'insieme la scrittura incriminata e la relativa lettura;
			
			\item Creazione insieme ScrittureFinali cercando per ogni risorsa indicata nello schedule l'ultima scrittura eseguita.
		\end{enumerate}
		
		\item Date le varie transazioni, si creano tutti i possibili schedule creando così una lista;
		
		\item Si verifica che almeno uno schedule della lista sia view-equivalente allo schedule dato al punto 1. Per farlo si esegue questo piccolo algoritmo:
		\begin{enumerate}
			\item Creazione dell'insieme LeggeDa (vedi punto 2.a);
			
			\item Creazione dell'insieme ScrittureFinali (vedi punto 2.b);
			
			\item Confronto degli insiemi creati precedente con quelli creati per lo schedule dato al punto 1. Se non sono uguali tutti uguali, allora lo schedule creato tramite combinazione non è equivalente allo schedule dato al punto 1. Altrimenti, è possibile affermare di aver trovato una combinazione view-equivalente.
		\end{enumerate}
		
		\item Al termine della creazione degli insiemi e dei vari confronti, se esiste almeno una combinazione che è view-equivalente allo schedule del punto 1, allora è possibile affermare che lo schedule di partenza è VSR.
	\end{enumerate}\newpage
	
	\subsection{Verificare che uno schedule sia CSR (Conflict-serializzabile)}
	
	\subsubsection{Esercizio 1 - Perdita di aggiornamento}
	
	Date due transizioni $T_{1}$ e $T_{2}$:
	\begin{equation*}
		\begin{array}{lll}
			T_{1} &:& r_{1}\left(x\right) \: w_{1}\left(x\right) \\
			T_{2} &:& r_{2}\left(x\right) \: w_{2}\left(x\right)
		\end{array}
	\end{equation*}
	Lo schedule che rappresenta l'anomalia è il seguente:
	\begin{equation*}
		S_{PA} = r_{1}\left(x\right) \: r_{2}\left(x\right) \: w_{2}\left(x\right) \: w_{1}\left(x\right)
	\end{equation*}
	Per verificare CSR è necessario caratterizzare $S_{PA}$ calcolando l'insieme dei conflitti. Si ricorda che due azioni sono in conflitto se operano sullo stesso oggetto e se almeno una di esse è in scrittura (quindi le combinazioni: $rw, wr, ww$). Quindi, si calcola l'insieme dei conflitti di $S_{PA}$:
	\begin{enumerate}
		\item $\textcolor{Red3}{r_{1}\left(x\right)} \: r_{2}\left(x\right) \: \textcolor{Red3}{w_{2}\left(x\right)} \: w_{1}\left(x\right)$
		
		\item $r_{1}\left(x\right) \: \textcolor{Red3}{r_{2}\left(x\right)} \: w_{2}\left(x\right) \: \textcolor{Red3}{w_{1}\left(x\right)}$
		
		\item $r_{1}\left(x\right) \: r_{2}\left(x\right) \: \textcolor{Red3}{w_{2}\left(x\right)} \: \textcolor{Red3}{w_{1}\left(x\right)}$
	\end{enumerate}
	L'insieme è quindi così costituito:
	\begin{equation*}
		\text{Conflitti}\left(S_{PA}\right) = \left\{\left(r_{1}\left(x\right), w_{2}\left(x\right)\right), \left(r_{2}\left(x\right), w_{1}\left(x\right)\right), \left(w_{2}\left(x\right), w_{1}\left(x\right)\right)\right\}
	\end{equation*}
	Si costruisce il grafo nel seguente modo. Si rappresentano tanti nodi quanti sono le transazioni e ogni arco (orientato) viene tracciato da $t_{i}$ a $t_{j}$ se vengono rispettate due condizioni: se c'è almeno un conflitto fra un'azione $a_{i}$ e un'azione $a_{j}$ tale che $a_{i}$ precede $a_{j}$. Quindi:
	\begin{figure}[!htp]
		\centering
		\includegraphics[width=.55\textwidth]{img/CSR-1.pdf}
	\end{figure}
	
	\noindent
	Se il grafo è aciclico allora $S_{PA}$ è CSR. In questo caso, il grafo non è aciclico ma ciclico, per cui $S_{PA}$ non è CSR.\newpage
	
	\subsubsection{Esercizio 2 - Lettura inconsistente}
	
	Date due transizioni $T_{1}$ e $T_{2}$:
	\begin{equation*}
		\begin{array}{lll}
			T_{1} &:& r_{1}\left(x\right) \: r_{1}'\left(x\right) \\
			T_{2} &:& r_{2}\left(x\right) \: w_{2}\left(x\right)
		\end{array}
	\end{equation*}
	Lo schedule che rappresenta l'anomalia è il seguente:
	\begin{equation*}
		S_{LI} = r_{1}\left(x\right) \: r_{2}\left(x\right) \: w_{2}\left(x\right) \: r_{1}'\left(x\right)
	\end{equation*}
	Per verificare CSR è necessario caratterizzare $S_{LI}$ calcolando l'insieme dei conflitti. Si ricorda che due azioni sono in conflitto se operano sullo stesso oggetto e se almeno una di esse è in scrittura (quindi le combinazioni: $rw, wr, ww$). Quindi, si calcola l'insieme dei conflitti di $S_{LI}$:
	\begin{enumerate}
		\item $\textcolor{Red3}{r_{1}\left(x\right)} \: r_{2}\left(x\right) \: \textcolor{Red3}{w_{2}\left(x\right)} \: r_{1}'\left(x\right)$
		
		\item $r_{1}\left(x\right) \: r_{2}\left(x\right) \: \textcolor{Red3}{w_{2}\left(x\right)} \: \textcolor{Red3}{r_{1}'\left(x\right)}$
	\end{enumerate}
	L'insieme è quindi così costituito:
	\begin{equation*}
		\text{Conflitti}\left(S_{LI}\right) = \left\{\left(r_{1}\left(x\right), w_{2}\left(x\right)\right), \left(w_{2}\left(x\right), r_{1}'\left(x\right)\right)\right\}
	\end{equation*}
	Si costruisce il grafo nel seguente modo. Si rappresentano tanti nodi quanti sono le transazioni e ogni arco (orientato) viene tracciato da $t_{i}$ a $t_{j}$ se vengono rispettate due condizioni: se c'è almeno un conflitto fra un'azione $a_{i}$ e un'azione $a_{j}$ tale che $a_{i}$ precede $a_{j}$. Quindi:
	\begin{figure}[!htp]
		\centering
		\includegraphics[width=.55\textwidth]{img/CSR-1.pdf}
	\end{figure}
	
	\noindent
	Se il grafo è aciclico allora $S_{LI}$ è CSR. In questo caso, il grafo non è aciclico ma ciclico, per cui $S_{LI}$ non è CSR.
	
	\longline
	
	\subsubsection{Sintesi dell'algoritmo}
	
	In sintesi l'algoritmo per capire se uno schedule è CSR:
	\begin{enumerate}
		\item Si calcola l'insieme dei conflitti. Un conflitto si manifesta quando due azioni \textbf{differenti} operano sullo stesso oggetto e quando almeno una di esse è in scrittura. Quindi, le combinazioni che possono esserci sono: $rw, wr, ww$;
		
		\item Si costruire il grafo dall'insieme dei conflitti. I nodi rappresentano le transizioni e gli archi si disegnano solo se due azioni non riguardano la stessa transizione.
	\end{enumerate}\newpage
	
	\subsection{Verificare che uno schedule sia NonSR, VSR e/o CSR}
	
	\subsubsection{Testo esercizio}
	
	Classificare i seguenti schedule (come: NonSR, VSR, CSR); nel caso uno schedule sia VSR oppure CSR, indicare tutti gli schedule seriali a esso equivalenti.
	\begin{enumerate}
		\item $S_{1} = r_{1}\left(x\right) \: w_{1}\left(x\right) \: r_{2}\left(z\right) \: r_{1}\left(y\right) \: w_{1}\left(y\right) \: r_{2}\left(x\right) \: w_{2}\left(x\right) \: w_{2}\left(z\right)$
		
		\item $S_{2} = r_{1}\left(x\right) \: w_{1}\left(x\right) \: w_{3}\left(x\right) \: r_{2}\left(y\right) \: r_{3}\left(y\right) \: w_{3}\left(y\right) \: w_{1}\left(y\right) \: r_{2}\left(x\right)$
		
		\item $S_{3} = r_{1}\left(x\right) \: r_{2}\left(x\right) \: w_{2}\left(x\right) \: r_{3}\left(x\right) \: r_{4}\left(z\right) \: w_{1}\left(x\right) \: w_{3}\left(y\right) \: w_{3}\left(x\right) \: w_{1}\left(y\right) \: w_{5}\left(x\right) \: w_{1}\left(z\right) \: w_{5}\left(y\right) \: r_{5}\left(z\right)$
		
		\item $S_{4} = r_{1}\left(x\right) \: r_{3}\left(y\right) \: w_{1}\left(y\right) \: w_{4}\left(x\right) \: w_{1}\left(t\right) \: w_{5}\left(x\right) \: r_{2}\left(z\right) \: r_{3}\left(z\right) \: w_{2}\left(z\right) \: w_{5}\left(z\right) \: r_{4}\left(t\right) \: r_{5}\left(t\right)$
	\end{enumerate}
	
	\longline
	
	\subsubsection{Schedule 1}
	
	Dato il seguente schedule:
	\begin{equation*}
		S_{1} = r_{1}\left(x\right) \: w_{1}\left(x\right) \: r_{2}\left(z\right) \: r_{1}\left(y\right) \: w_{1}\left(y\right) \: r_{2}\left(x\right) \: w_{2}\left(x\right) \: w_{2}\left(z\right)
	\end{equation*}
	Le transizioni sono:
	\begin{equation*}
		\begin{array}{lll}
			T_{1} &:& r_{1}\left(x\right) \: w_{1}\left(x\right) \: r_{1}\left(y\right) \: w_{1}\left(y\right) \\
			T_{2} &:& r_{2}\left(z\right) \: r_{2}\left(x\right) \: w_{2}\left(x\right) \: w_{2}\left(z\right)
		\end{array}
	\end{equation*}
	Si verifica se è CSR. Quindi, si crea l'insieme dei conflitti:
	\begin{enumerate}
		\item $\textcolor{Red3}{r_{1}\left(x\right)} \: w_{1}\left(x\right) \: r_{2}\left(z\right) \: r_{1}\left(y\right) \: w_{1}\left(y\right) \: r_{2}\left(x\right) \: \textcolor{Red3}{w_{2}\left(x\right)} \: w_{2}\left(z\right)$
		
		\item $r_{1}\left(x\right) \: \textcolor{Red3}{w_{1}\left(x\right)} \: r_{2}\left(z\right) \: r_{1}\left(y\right) \: w_{1}\left(y\right) \: \textcolor{Red3}{r_{2}\left(x\right)} \: w_{2}\left(x\right) \: w_{2}\left(z\right)$
		
		\item $r_{1}\left(x\right) \: \textcolor{Red3}{w_{1}\left(x\right)} \: r_{2}\left(z\right) \: r_{1}\left(y\right) \: w_{1}\left(y\right) \: r_{2}\left(x\right) \: \textcolor{Red3}{w_{2}\left(x\right)} \: w_{2}\left(z\right)$
	\end{enumerate}
	Quindi l'insieme è:
	\begin{equation*}
		\text{Conflitti}\left(S_{1}\right) = \left\{\left(r_{1}\left(x\right) \: w_{2}\left(x\right)\right), \left(w_{1}\left(x\right) \: r_{2}\left(x\right)\right), \left(w_{1}\left(x\right) \: w_{2}\left(x\right)\right)\right\}
	\end{equation*}
	Si costruisce il grafo:
	\begin{figure}[!htp]
		\centering
		\includegraphics[width=.55\textwidth]{img/NoSR-VSR-CSR-1.pdf}
	\end{figure}
	
	\noindent
	Il ciclo è aciclico quindi è $S_{1}$ è CSR. Dato che CSR $\subset$ VSR, allora $S_{1}$ è anche VSR.\newpage
	
	\subsubsection{Schedule 2}
	
	Dato il seguente schedule:
	\begin{equation*}
		S_{2} = r_{1}\left(x\right) \: w_{1}\left(x\right) \: w_{3}\left(x\right) \: r_{2}\left(y\right) \: r_{3}\left(y\right) \: w_{3}\left(y\right) \: w_{1}\left(y\right) \: r_{2}\left(x\right)
	\end{equation*}
	Le transazioni sono:
	\begin{equation*}
		\begin{array}{lll}
			T_{1} &:& r_{1}\left(x\right) \: w_{1}\left(x\right) \: w_{1}\left(y\right) \\
			T_{2} &:& r_{2}\left(y\right) \: r_{2}\left(x\right) \\
			T_{3} &:& w_{3}\left(x\right) \: r_{3}\left(y\right) \: w_{3}\left(y\right)
		\end{array}
	\end{equation*}
	Si verifica se è VSR. Si inizia analizzando l'insieme $S_{2}$:
	\begin{equation*}
		\begin{array}{lll}
			\text{LeggeDa}\left(S_{2}\right) 			&=& \left\{\left(r_{2}\left(x\right), w_{3}\left(x\right)\right)\right\} \\
			\text{ScrittureFinali}\left(S_{2}\right) 	&=& \left\{w_{3}\left(x\right), w_{1}\left(y\right)\right\}
		\end{array}
	\end{equation*}
	Dato che è impossibile provare tutte le combinazioni (3 transazioni e quindi $3! = 6$), si fanno alcune considerazioni. Per esempio, dato che nelle LeggeDa si deve mantenere l'ordine $\left(r_{2}\left(x\right), w_{3}\left(x\right)\right)$, e sapendo che $r_{2}$ appartiene a $T_{2}$ e $w_{3}$ a $T_{3}$, si può concludere che $T_{3}$ deve per forza precedere $T_{2}$. Quindi, le combinazioni si riducono a:
	\begin{itemize}
		\item $T_{1} \: T_{3} \: T_{2}$
		
		\item $T_{3} \: T_{1} \: T_{2}$
		
		\item $T_{3} \: T_{2} \: T_{1}$
	\end{itemize}
	Tuttavia, se $T_{3}$ anticipa $T_{2}$, tutte le combinazioni avranno come insieme LeggeDa \underline{almeno} i due valori:
	\begin{equation*}
		\text{LeggeDa}\left(S_{2}\right) = \left\{\left(r_{2}\left(x\right), w_{3}\left(x\right)\right), \left(r_{2}\left(y\right) \: w_{3}\left(y\right)\right)\right\}
	\end{equation*}
	Quindi, è possibile concludere che nessuna combinazione ha un insieme LeggeDa equivalente al LeggeDa di $S_{2}$. È possibile concludere che $S_{2}$ non è VSR.\newpage
	
	\subsubsection{Schedule 3}
	
	Dato il seguente schedule:
	\begin{equation*}
		S_{3} = r_{1}\left(x\right) \: r_{2}\left(x\right) \: w_{2}\left(x\right) \: r_{3}\left(x\right) \: r_{4}\left(z\right) \: w_{1}\left(x\right) \: w_{3}\left(y\right) \: w_{3}\left(x\right) \: w_{1}\left(y\right) \: w_{5}\left(x\right) \: w_{1}\left(z\right) \: w_{5}\left(y\right) \: r_{5}\left(z\right)
	\end{equation*}
	Le transazioni sono:
	\begin{equation*}
		\begin{array}{lll}
			T_{1} &:& r_{1}\left(x\right) \: w_{1}\left(x\right) \: w_{1}\left(y\right) \: w_{1}\left(z\right) \\
			T_{2} &:& r_{2}\left(x\right) \: w_{2}\left(x\right) \\
			T_{3} &:& r_{3}\left(x\right) \: w_{3}\left(y\right) \: w_{3}\left(x\right) \\
			T_{4} &:& r_{4}\left(z\right) \\
			T_{5} &:& w_{5}\left(x\right) \: w_{5}\left(y\right) \: r_{5}\left(z\right)
		\end{array}
	\end{equation*}
	Si verifica se CSR. Quindi, si cerca l'insieme di conflitti:
	\begin{enumerate}
		\item $\textcolor{Red3}{\boldsymbol{r_{1}\left(x\right)} \: }r_{2}\left(x\right) \: \textcolor{Red3}{\boldsymbol{w_{2}\left(x\right)} \: }r_{3}\left(x\right) \: r_{4}\left(z\right) \: w_{1}\left(x\right) \: w_{3}\left(y\right) \: w_{3}\left(x\right) \: w_{1}\left(y\right) \: w_{5}\left(x\right) \: w_{1}\left(z\right) \: w_{5}\left(y\right) \: r_{5}\left(z\right)$
		
		\item $\textcolor{Red3}{\boldsymbol{r_{1}\left(x\right)} \: }r_{2}\left(x\right) \: w_{2}\left(x\right) \: r_{3}\left(x\right) \: r_{4}\left(z\right) \: w_{1}\left(x\right) \: w_{3}\left(y\right) \: \textcolor{Red3}{\boldsymbol{w_{3}\left(x\right)} \: }w_{1}\left(y\right) \: w_{5}\left(x\right) \: w_{1}\left(z\right) \: w_{5}\left(y\right) \: r_{5}\left(z\right)$
		
		\item $\textcolor{Red3}{\boldsymbol{r_{1}\left(x\right)} \: }r_{2}\left(x\right) \: w_{2}\left(x\right) \: r_{3}\left(x\right) \: r_{4}\left(z\right) \: w_{1}\left(x\right) \: w_{3}\left(y\right) \: w_{3}\left(x\right) \: w_{1}\left(y\right) \: \textcolor{Red3}{\boldsymbol{w_{5}\left(x\right)} \: }w_{1}\left(z\right) \: w_{5}\left(y\right) \: r_{5}\left(z\right)$
		
		\item $r_{1}\left(x\right) \: \textcolor{Red3}{\boldsymbol{r_{2}\left(x\right)} \: }w_{2}\left(x\right) \: r_{3}\left(x\right) \: r_{4}\left(z\right) \: \textcolor{Red3}{\boldsymbol{w_{1}\left(x\right)} \: }w_{3}\left(y\right) \: w_{3}\left(x\right) \: w_{1}\left(y\right) \: w_{5}\left(x\right) \: w_{1}\left(z\right) \: w_{5}\left(y\right) \: r_{5}\left(z\right)$
		
		\item $r_{1}\left(x\right) \: \textcolor{Red3}{\boldsymbol{r_{2}\left(x\right)} \: }w_{2}\left(x\right) \: r_{3}\left(x\right) \: r_{4}\left(z\right) \: w_{1}\left(x\right) \: w_{3}\left(y\right) \: \textcolor{Red3}{\boldsymbol{w_{3}\left(x\right)} \: }w_{1}\left(y\right) \: w_{5}\left(x\right) \: w_{1}\left(z\right) \: w_{5}\left(y\right) \: r_{5}\left(z\right)$
		
		\item $r_{1}\left(x\right) \: \textcolor{Red3}{\boldsymbol{r_{2}\left(x\right)} \: }w_{2}\left(x\right) \: r_{3}\left(x\right) \: r_{4}\left(z\right) \: w_{1}\left(x\right) \: w_{3}\left(y\right) \: w_{3}\left(x\right) \: w_{1}\left(y\right) \: \textcolor{Red3}{\boldsymbol{w_{5}\left(x\right)} \: }w_{1}\left(z\right) \: w_{5}\left(y\right) \: r_{5}\left(z\right)$
		
		\item $r_{1}\left(x\right) \: r_{2}\left(x\right) \: \textcolor{Red3}{\boldsymbol{w_{2}\left(x\right)} \: }\textcolor{Red3}{\boldsymbol{r_{3}\left(x\right)} \: }r_{4}\left(z\right) \: w_{1}\left(x\right) \: w_{3}\left(y\right) \: w_{3}\left(x\right) \: w_{1}\left(y\right) \: w_{5}\left(x\right) \: w_{1}\left(z\right) \: w_{5}\left(y\right) \: r_{5}\left(z\right)$
		
		\item $r_{1}\left(x\right) \: r_{2}\left(x\right) \: \textcolor{Red3}{\boldsymbol{w_{2}\left(x\right)} \: }r_{3}\left(x\right) \: r_{4}\left(z\right) \: \textcolor{Red3}{\boldsymbol{w_{1}\left(x\right)} \: }w_{3}\left(y\right) \: w_{3}\left(x\right) \: w_{1}\left(y\right) \: w_{5}\left(x\right) \: w_{1}\left(z\right) \: w_{5}\left(y\right) \: r_{5}\left(z\right)$
		
		\item $r_{1}\left(x\right) \: r_{2}\left(x\right) \: \textcolor{Red3}{\boldsymbol{w_{2}\left(x\right)} \: }r_{3}\left(x\right) \: r_{4}\left(z\right) \: w_{1}\left(x\right) \: w_{3}\left(y\right) \: \textcolor{Red3}{\boldsymbol{w_{3}\left(x\right)} \: }w_{1}\left(y\right) \: w_{5}\left(x\right) \: w_{1}\left(z\right) \: w_{5}\left(y\right) \: r_{5}\left(z\right)$
		
		\item $r_{1}\left(x\right) \: r_{2}\left(x\right) \: \textcolor{Red3}{\boldsymbol{w_{2}\left(x\right)} \: }r_{3}\left(x\right) \: r_{4}\left(z\right) \: w_{1}\left(x\right) \: w_{3}\left(y\right) \: w_{3}\left(x\right) \: w_{1}\left(y\right) \: \textcolor{Red3}{\boldsymbol{w_{5}\left(x\right)} \: }w_{1}\left(z\right) \: w_{5}\left(y\right) \: r_{5}\left(z\right)$
		
		\item $r_{1}\left(x\right) \: r_{2}\left(x\right) \: w_{2}\left(x\right) \: \textcolor{Red3}{\boldsymbol{r_{3}\left(x\right)} \: }r_{4}\left(z\right) \: \textcolor{Red3}{\boldsymbol{w_{1}\left(x\right)} \: }w_{3}\left(y\right) \: w_{3}\left(x\right) \: w_{1}\left(y\right) \: w_{5}\left(x\right) \: w_{1}\left(z\right) \: w_{5}\left(y\right) \: r_{5}\left(z\right)$
		
		\item $r_{1}\left(x\right) \: r_{2}\left(x\right) \: w_{2}\left(x\right) \: \textcolor{Red3}{\boldsymbol{r_{3}\left(x\right)} \: }r_{4}\left(z\right) \: w_{1}\left(x\right) \: w_{3}\left(y\right) \: w_{3}\left(x\right) \: w_{1}\left(y\right) \: \textcolor{Red3}{\boldsymbol{w_{5}\left(x\right)} \: }w_{1}\left(z\right) \: w_{5}\left(y\right) \: r_{5}\left(z\right)$
		
		\item $r_{1}\left(x\right) \: r_{2}\left(x\right) \: w_{2}\left(x\right) \: r_{3}\left(x\right) \: \textcolor{Red3}{\boldsymbol{r_{4}\left(z\right)} \: }w_{1}\left(x\right) \: w_{3}\left(y\right) \: w_{3}\left(x\right) \: w_{1}\left(y\right) \: w_{5}\left(x\right) \: \textcolor{Red3}{\boldsymbol{w_{1}\left(z\right)} \: }w_{5}\left(y\right) \: r_{5}\left(z\right)$
		
		\item $r_{1}\left(x\right) \: r_{2}\left(x\right) \: w_{2}\left(x\right) \: r_{3}\left(x\right) \: r_{4}\left(z\right) \: \textcolor{Red3}{\boldsymbol{w_{1}\left(x\right)} \: }w_{3}\left(y\right) \: \textcolor{Red3}{\boldsymbol{w_{3}\left(x\right)} \: }w_{1}\left(y\right) \: w_{5}\left(x\right) \: w_{1}\left(z\right) \: w_{5}\left(y\right) \: r_{5}\left(z\right)$
		
		\item $r_{1}\left(x\right) \: r_{2}\left(x\right) \: w_{2}\left(x\right) \: r_{3}\left(x\right) \: r_{4}\left(z\right) \: \textcolor{Red3}{\boldsymbol{w_{1}\left(x\right)} \: }w_{3}\left(y\right) \: w_{3}\left(x\right) \: w_{1}\left(y\right) \: \textcolor{Red3}{\boldsymbol{w_{5}\left(x\right)} \: }w_{1}\left(z\right) \: w_{5}\left(y\right) \: r_{5}\left(z\right)$
		
		\item $r_{1}\left(x\right) \: r_{2}\left(x\right) \: w_{2}\left(x\right) \: r_{3}\left(x\right) \: r_{4}\left(z\right) \: w_{1}\left(x\right) \: \textcolor{Red3}{\boldsymbol{w_{3}\left(y\right)} \: }w_{3}\left(x\right) \: \textcolor{Red3}{\boldsymbol{w_{1}\left(y\right)} \: }w_{5}\left(x\right) \: w_{1}\left(z\right) \: w_{5}\left(y\right) \: r_{5}\left(z\right)$
		
		\item $r_{1}\left(x\right) \: r_{2}\left(x\right) \: w_{2}\left(x\right) \: r_{3}\left(x\right) \: r_{4}\left(z\right) \: w_{1}\left(x\right) \: \textcolor{Red3}{\boldsymbol{w_{3}\left(y\right)} \: }w_{3}\left(x\right) \: w_{1}\left(y\right) \: w_{5}\left(x\right) \: w_{1}\left(z\right) \: \textcolor{Red3}{\boldsymbol{w_{5}\left(y\right)} \: }r_{5}\left(z\right)$
		
		\item $r_{1}\left(x\right) \: r_{2}\left(x\right) \: w_{2}\left(x\right) \: r_{3}\left(x\right) \: r_{4}\left(z\right) \: w_{1}\left(x\right) \: w_{3}\left(y\right) \: \textcolor{Red3}{\boldsymbol{w_{3}\left(x\right)} \: }w_{1}\left(y\right) \: \textcolor{Red3}{\boldsymbol{w_{5}\left(x\right)} \: }w_{1}\left(z\right) \: w_{5}\left(y\right) \: r_{5}\left(z\right)$
		
		\item $r_{1}\left(x\right) \: r_{2}\left(x\right) \: w_{2}\left(x\right) \: r_{3}\left(x\right) \: r_{4}\left(z\right) \: w_{1}\left(x\right) \: w_{3}\left(y\right) \: w_{3}\left(x\right) \: \textcolor{Red3}{\boldsymbol{w_{1}\left(y\right)} \: }w_{5}\left(x\right) \: w_{1}\left(z\right) \: \textcolor{Red3}{\boldsymbol{w_{5}\left(y\right)} \: }r_{5}\left(z\right)$
		
		\item $r_{1}\left(x\right) \: r_{2}\left(x\right) \: w_{2}\left(x\right) \: r_{3}\left(x\right) \: r_{4}\left(z\right) \: w_{1}\left(x\right) \: w_{3}\left(y\right) \: w_{3}\left(x\right) \: w_{1}\left(y\right) \: w_{5}\left(x\right) \: \textcolor{Red3}{\boldsymbol{w_{1}\left(z\right)} \: }w_{5}\left(y\right) \: \textcolor{Red3}{\boldsymbol{r_{5}\left(z\right)}}$
	\end{enumerate}\newpage

	\noindent
	L'insieme dei conflitti è quindi formato da:
	\begin{equation*}
		\begin{array}{lll}
			\text{Conflitti}\left(S_{3}\right) = \{ & \left(r_{1}\left(x\right) \: w_{2}\left(x\right)\right), \left(r_{1}\left(x\right) \: w_{3}\left(x\right)\right), \left(r_{1}\left(x\right) \: w_{5}\left(x\right)\right), & \\[0.5em]
													& \left(r_{2}\left(x\right) \: w_{1}\left(x\right)\right), \left(r_{2}\left(x\right) \: w_{3}\left(x\right)\right), \left(r_{2}\left(x\right) \: w_{5}\left(x\right)\right), & \\[0.5em]
													& \left(w_{2}\left(x\right) \: r_{3}\left(x\right)\right), \left(w_{2}\left(x\right) \: w_{1}\left(x\right)\right), \left(w_{2}\left(x\right) \: w_{3}\left(x\right)\right), \left(w_{2}\left(x\right) \: w_{5}\left(x\right)\right), & \\[0.5em]
													& \left(r_{3}\left(x\right) \: w_{1}\left(x\right)\right), \left(r_{3}\left(x\right) \: w_{5}\left(x\right)\right), & \\[0.5em]
													& \left(r_{4}\left(z\right) \: w_{1}\left(z\right)\right), & \\[0.5em]
													& \left(w_{1}\left(x\right) \: w_{3}\left(x\right)\right), \left(w_{1}\left(x\right) \: w_{5}\left(x\right)\right), & \\[0.5em]
													& \left(w_{3}\left(y\right) \: w_{1}\left(y\right)\right), \left(w_{3}\left(y\right) \: w_{5}\left(y\right)\right), & \\[0.5em]
													& \left(w_{3}\left(x\right) \: w_{5}\left(x\right)\right), & \\[0.5em]
													& \left(w_{1}\left(y\right) \: w_{5}\left(y\right)\right), & \\[0.5em]
													& \left(w_{1}\left(z\right) \: r_{5}\left(z\right)\right) & \}
		\end{array}
	\end{equation*}
	Adesso si crea il grafo dei conflitti. Le transazioni sono 5, quindi ci saranno 5 nodi:
	\begin{figure}[!htp]
		\centering
		\includegraphics[width=.7\textwidth]{img/CSR-2.pdf}
	\end{figure}
	
	\noindent
	Il grafo non è aciclico, quindi $S_{3}$ non è CSR, quindi si verifica se può essere VSR. Si inizia calcolando l'insieme LeggeDa e ScrittureFinali:
	\begin{equation*}
		\begin{array}{lll}
			\text{LeggeDa}\left(S_{3}\right) &=& \left\{ \left(r_{3}\left(x\right) \: w_{2}\left(x\right)\right), \left(r_{5}\left(z\right) \: w_{1}\left(z\right)\right)\right\} \\
			\text{ScrittureFinali}\left(S_{3}\right) &=& \left\{w_{5}\left(x\right), w_{1}\left(z\right), w_{5}\left(y\right)\right\}
		\end{array}
	\end{equation*}
	Adesso si cerca almeno una combinazione tra le 5 transizioni, tale per cui una di esse sia view-equivalente a $S_{3}$. Con una considerazione rapida è possibile già ottenere il risultato finale. Guardando l'insieme delle ScrittureFinali, è possibile vedere che la $5^{a}$ transizione dovrebbe avere tra la scrittura sulla risorsa $x$ e tra la scrittura sulla risorsa $y$, la scrittura sulla risorsa $z$. Questo non è possibile poiché quest'ultima appartiene alla transazione 1. Per cui, è già possibile affermare che nessuna combinazione tra le 5 transazioni, può dare una combinazione view-equivalente a $S_{3}$. Si conclude che $S_{3}$ non è VSR e quindi è NonSR.
	
	\subsubsection{Schedule 4}
	
	Dato il seguente schedule:
	\begin{equation*}
		S_{4} = r_{1}\left(x\right) \: r_{3}\left(y\right) \: w_{1}\left(y\right) \: w_{4}\left(x\right) \: w_{1}\left(t\right) \: w_{5}\left(x\right) \: r_{2}\left(z\right) \: r_{3}\left(z\right) \: w_{2}\left(z\right) \: w_{5}\left(z\right) \: r_{4}\left(t\right) \: r_{5}\left(t\right)
	\end{equation*}
	Le transazioni sono:
	\begin{equation*}
		\begin{array}{lll}
			T_{1} &:& r_{1}\left(x\right) \: w_{1}\left(y\right) \: w_{1}\left(t\right) \\
			T_{2} &:& r_{2}\left(z\right) \: w_{2}\left(z\right) \\
			T_{3} &:& r_{3}\left(y\right) \: r_{3}\left(z\right) \\
			T_{4} &:& w_{4}\left(x\right) \: r_{4}\left(t\right) \\
			T_{5} &:& w_{5}\left(x\right) \: w_{5}\left(z\right) \: r_{5}\left(t\right)
		\end{array}
	\end{equation*}
	Si verifica CSR prima di tutto. Quindi, si inizia costruendo l'insieme dei conflitti:
	\begin{enumerate}
		\item $\textcolor{Red3}{\boldsymbol{r_{1}\left(x\right)} \: }r_{3}\left(y\right) \: w_{1}\left(y\right) \: \textcolor{Red3}{\boldsymbol{w_{4}\left(x\right)} \: }w_{1}\left(t\right) \: w_{5}\left(x\right) \: r_{2}\left(z\right) \: r_{3}\left(z\right) \: w_{2}\left(z\right) \: w_{5}\left(z\right) \: r_{4}\left(t\right) \: r_{5}\left(t\right)$
		
		\item $\textcolor{Red3}{\boldsymbol{r_{1}\left(x\right)} \: }r_{3}\left(y\right) \: w_{1}\left(y\right) \: w_{4}\left(x\right) \: w_{1}\left(t\right) \: \textcolor{Red3}{\boldsymbol{w_{5}\left(x\right)} \: }r_{2}\left(z\right) \: r_{3}\left(z\right) \: w_{2}\left(z\right) \: w_{5}\left(z\right) \: r_{4}\left(t\right) \: r_{5}\left(t\right)$
		
		\item $r_{1}\left(x\right) \: \textcolor{Red3}{\boldsymbol{r_{3}\left(y\right)} \: }\textcolor{Red3}{\boldsymbol{w_{1}\left(y\right)} \: }w_{4}\left(x\right) \: w_{1}\left(t\right) \: w_{5}\left(x\right) \: r_{2}\left(z\right) \: r_{3}\left(z\right) \: w_{2}\left(z\right) \: w_{5}\left(z\right) \: r_{4}\left(t\right) \: r_{5}\left(t\right)$
		
		\item $r_{1}\left(x\right) \: r_{3}\left(y\right) \: w_{1}\left(y\right) \: \textcolor{Red3}{\boldsymbol{w_{4}\left(x\right)} \: }w_{1}\left(t\right) \: \textcolor{Red3}{\boldsymbol{w_{5}\left(x\right)} \: }r_{2}\left(z\right) \: r_{3}\left(z\right) \: w_{2}\left(z\right) \: w_{5}\left(z\right) \: r_{4}\left(t\right) \: r_{5}\left(t\right)$
		
		\item $r_{1}\left(x\right) \: r_{3}\left(y\right) \: w_{1}\left(y\right) \: w_{4}\left(x\right) \: \textcolor{Red3}{\boldsymbol{w_{1}\left(t\right)} \: }w_{5}\left(x\right) \: r_{2}\left(z\right) \: r_{3}\left(z\right) \: w_{2}\left(z\right) \: w_{5}\left(z\right) \: \textcolor{Red3}{\boldsymbol{r_{4}\left(t\right)} \: }r_{5}\left(t\right)$
		
		\item $r_{1}\left(x\right) \: r_{3}\left(y\right) \: w_{1}\left(y\right) \: w_{4}\left(x\right) \: \textcolor{Red3}{\boldsymbol{w_{1}\left(t\right)} \: }w_{5}\left(x\right) \: r_{2}\left(z\right) \: r_{3}\left(z\right) \: w_{2}\left(z\right) \: w_{5}\left(z\right) \: r_{4}\left(t\right) \: \textcolor{Red3}{\boldsymbol{r_{5}\left(t\right)}}$
		
		\item $r_{1}\left(x\right) \: r_{3}\left(y\right) \: w_{1}\left(y\right) \: w_{4}\left(x\right) \: w_{1}\left(t\right) \: w_{5}\left(x\right) \: \textcolor{Red3}{\boldsymbol{r_{2}\left(z\right)} \: }r_{3}\left(z\right) \: w_{2}\left(z\right) \: \textcolor{Red3}{\boldsymbol{w_{5}\left(z\right)} \: }r_{4}\left(t\right) \: r_{5}\left(t\right)$
		
		\item $r_{1}\left(x\right) \: r_{3}\left(y\right) \: w_{1}\left(y\right) \: w_{4}\left(x\right) \: w_{1}\left(t\right) \: w_{5}\left(x\right) \: r_{2}\left(z\right) \: \textcolor{Red3}{\boldsymbol{r_{3}\left(z\right)} \: }\textcolor{Red3}{\boldsymbol{w_{2}\left(z\right)} \: }w_{5}\left(z\right) \: r_{4}\left(t\right) \: r_{5}\left(t\right)$
		
		\item $r_{1}\left(x\right) \: r_{3}\left(y\right) \: w_{1}\left(y\right) \: w_{4}\left(x\right) \: w_{1}\left(t\right) \: w_{5}\left(x\right) \: r_{2}\left(z\right) \: \textcolor{Red3}{\boldsymbol{r_{3}\left(z\right)} \: }w_{2}\left(z\right) \: \textcolor{Red3}{\boldsymbol{w_{5}\left(z\right)} \: }r_{4}\left(t\right) \: r_{5}\left(t\right)$
		
		\item $r_{1}\left(x\right) \: r_{3}\left(y\right) \: w_{1}\left(y\right) \: w_{4}\left(x\right) \: w_{1}\left(t\right) \: w_{5}\left(x\right) \: r_{2}\left(z\right) \: r_{3}\left(z\right) \: \textcolor{Red3}{\boldsymbol{w_{2}\left(z\right)} \: } \textcolor{Red3}{\boldsymbol{w_{5}\left(z\right)} \: } r_{4}\left(t\right) \: r_{5}\left(t\right)$
	\end{enumerate}
	Quindi, l'insieme dei conflitti è:
	\begin{equation*}
		\begin{array}{lll}
			\text{Conflitti}\left(S_{4}\right) = \{ & \left(r_{1}\left(x\right) \: w_{4}\left(x\right)\right), \left(r_{1}\left(x\right) \: w_{5}\left(x\right)\right), & \\ [0.5em]
													& \left(r_{3}\left(y\right) \: w_{1}\left(y\right)\right), & \\ [0.5em]
													& \left(w_{4}\left(x\right) \: w_{5}\left(x\right)\right), & \\ [0.5em]
													& \left(w_{1}\left(t\right) \: r_{4}\left(t\right)\right), \left(w_{1}\left(t\right) \: r_{5}\left(t\right)\right) & \\ [0.5em]
													& \left(r_{2}\left(z\right) \: w_{5}\left(z\right)\right), & \\ [0.5em]
													& \left(r_{3}\left(z\right) \: w_{2}\left(z\right)\right), \left(r_{3}\left(z\right) \: w_{5}\left(z\right)\right), & \\ [0.5em]
													& \left(w_{2}\left(z\right) \: w_{5}\left(z\right)\right) & \}
		\end{array}
	\end{equation*}\newpage

	\noindent
	Adesso si costruisce il grafo. I nodi sono 5 poiché il numero di transazioni è 5:
	\begin{figure}[!htp]
		\centering
		\includegraphics[width=.7\textwidth]{img/CSR-3.pdf}
	\end{figure}

	\noindent
	Il grafo risulta aciclico, quindi è possibile affermare con certezza che $S_{4}$ è CSR e per definizione anche VSR ($CSR \subset VSR$). Per concludere l'esercizio, è necessario calcolare gli schedule seriali equivalenti. Per capire quali combinazioni sono accettate, si fanno alcune considerazioni guardando il grafo:
	\begin{itemize}
		\item Il nodo $T_{5}$ non ha archi orientati in uscita, ma solo in entrata. Quindi, è possibile affermare che la transazione $T_{5}$ deve essere l'ultima.
		
		\item Il nodo $T_{3}$ non ha archi orientati in entrata, ma solo in uscita. Quindi, è possibile affermare che la transazione $T_{3}$ deve essere la prima.
		
		\item I nodi $T_{2}$ e $T_{1}$ sono gli unici che vengono raggiunti da $T_{3}$, ovvero dal nodo d'inizio. Quindi, è possibile affermare che le uniche combinazioni potranno essere:
		\begin{enumerate}
			\item $T_{3}, T_{1}, T_{2}, T_{4}, T_{5}$
			\item $T_{3}, T_{1}, T_{4}, T_{2}, T_{5}$
			\item $T_{3}, T_{2}, T_{1}, T_{4}, T_{5}$
			\item $T_{3}, T_{2}, T_{4}, T_{1}, T_{5}$
		\end{enumerate}

		\item Dal nodo $T_{1}$ è possibile andare al nodo $T_{4}$ senza problemi ma non viceversa poiché altrimenti si creerebbe una ciclo. Quindi le uniche combinazioni ammesse sono:
		\begin{enumerate}
			\item $T_{3}, T_{1}, T_{2}, T_{4}, T_{5}$
			\item $T_{3}, T_{1}, T_{4}, T_{2}, T_{5}$
			\item $T_{3}, T_{2}, T_{1}, T_{4}, T_{5}$
			\item $T_{3}, T_{2}, T_{4}, T_{1}, T_{5}$
		\end{enumerate}
	\end{itemize}\newpage
	
	\subsection{Ottimizzazione e stima di costo}
	
	\subsubsection{Esercizio 1}
	
	Si consideri il seguente schema relazionale contenete le ricette di una catena di ristoranti:
	\begin{equation*}
		\begin{array}{rl}
			\textsf{INGREDIENTE} & \left(\underline{\text{Codice}}, \text{Nome}, \text{Calorie}\right); \\[0.3em]
			\textsf{COMPOSIZIONE} & \left(\underline{\text{Ricetta, Ingrediente}}, \text{Quantità}\right); \\[0.3em]
			\textsf{RICETTA} & \left(\underline{\text{CodiceRicetta}}, \text{Nome}, \text{Regione}, \text{TempoPreparazione}\right).
		\end{array}
	\end{equation*}
	N.B.: la quantità della tabella \textsf{COMPOSIZIONE} è espressa in grammi.\newline
	
	\noindent
	\emph{\textbf{Vincoli di integrità referenziale:}}
	\begin{equation*}
		\begin{array}{lll}
			\textsf{COMPOSIZIONE}.\text{Ricetta} 		&\rightarrow& \textsf{RICETTA} \\ [0.3em]
			\textsf{COMPOSIZIONE}.\text{Ingrediente} 	&\rightarrow& \textsf{INGREDIENTE}
		\end{array}
	\end{equation*}
	Data la seguente interrogazione SQL che consente di: \emph{trovare gli ingredienti usati in ricette della Regione Veneto, riportando, il codice della ricetta e il nome e le calorie dell'ingrediente}.
	\lstinputlisting[language=SQL]{code/ottimizzazione-1.sql}
	Calcolare il costo dell'interrogazione in termini di numero di accessi a memoria secondaria sotto le seguenti ipotesi:
	\begin{itemize}
		\item La selezione su ricetta richiede una scansione sequenziale della tabella \textsf{RICETTA};
		
		\item L'ordine di esecuzione del join è:
		\begin{equation*}
			\left(\textsf{RICETTA} \Join \textsf{COMPOSIZIONE}\right) \Join \textsf{INGREDIENTE}
		\end{equation*}
		
		\item Le operazioni di join vengono eseguite con la tecnica \underline{Nested Loop join} con una pagina di buffer disponibile per ogni tabella;
		
		\item Il risultato intermedio del primo join viene interamente memorizzato nel buffer;
		
		\item $\mathrm{NP}\left(\textsf{INGREDIENTE}\right) = 40$, $\mathrm{NP}\left(\textsf{COMPOSIZIONE}\right) = 200$, $\mathrm{NP}\left(\textsf{RICETTA}\right) = 12$;
		
		\item $\mathrm{NR}\left(\textsf{INGREDIENTE}\right) = 1200$, $\mathrm{NR}\left(\textsf{COMPOSIZIONE}\right) = 13000$, $\mathrm{NR}\left(\textsf{RICETTA}\right) = 260$;
		
		\item $\mathrm{VAL}\left(\text{Regione}, \textsf{RICETTA}\right) = 20$, $\mathrm{VAL}\left(\text{Ricetta}, \textsf{COMPOSIZIONE}\right) = 260$.
	\end{itemize}
	Come cambia il costo se è disponibile un indice B+-tree sull'attributo Codice della tabella INGREDIENTE che ha profondità 2.\newpage
	
	\noindent
	Il \emph{nested loop join} è un algoritmo di join che unisce due set usando due cicli nidificati. Quindi, i cicli sono:
	\begin{enumerate}
		\item Viene selezionata la regione Veneta (\textsf{WHERE R.Regione = 'Veneto'}) e successivamente eseguito il primo Join:
		\begin{equation*}
			\textsf{seleziona\_veneto}\left(\textsf{RICETTA}\right) \Join \textsf{COMPOSIZIONE}
		\end{equation*}
		Quindi il costo risulta essere una lettura della tabella esterna \textsf{RICETTA} poiché deve essere applicato una condizione \textsf{WHERE} e infine viene eseguita una moltiplicazione:
		\begin{equation*}
			\begin{array}{lll}
				\text{Costo} = & \text{NP}\left(\textsf{RICETTA}\right) + & \\
				& \mathrm{NR}\left(\textsf{RICETTA} \text{ con selezione Regione='Veneto'}\right) \times & \\
				& \mathrm{NP}\left(\textsf{COMPOSIZIONE}\right)
			\end{array}
		\end{equation*}
		
		\item Il secondo Join non prevede condizioni \textsf{WHERE} e il costo non tiene in considerazione una lettura da una tabella esterna poiché la tabella interessata è già in buffer:
		\begin{equation*}
			\begin{array}{lll}
				\text{Costo} = & \mathrm{NR}\left(\textsf{COMPOSIZIONE} \text{ per le ricette con selezione Regione='Veneto'}\right) \times & \\
				& \mathrm{NP}\left(\textsf{INGREDIENTE}\right)
			\end{array}
		\end{equation*}
	\end{enumerate}
	I dati dell'esercizio sui costi sono i seguenti:
	\begin{table}[!htp]
		\centering
		\begin{tabular}{@{} l | c c c c @{}}
			\toprule
			Tabella & NP & NR & VAL(Regione) & VAL(Ricetta) \\
			\midrule
			\textsf{Ricetta} 		& 12 & 260 & 20 & \ding{55} \\
			\textsf{Ingrediente}	& 40 & 1200 & \ding{55} & \ding{55} \\
			\textsf{Composizione}	& 200 & 13000 & \ding{55} & 260 \\
			\bottomrule
		\end{tabular}
	\end{table}
	
	\noindent
	I dati presi sono stati acquisiti dai requisiti elencati, niente di difficile. Adesso si cerca di ottenere un valore tangibile, quindi si sostituiscono i valori:
	\begin{equation*}
		\text{Costo} = 12 + \mathrm{NR}\left(\textsf{RICETTA}\right) \div \mathrm{VAL}\left(\text{Regione}, \textsf{RICETTA}\right) \times 200
	\end{equation*}
	La selezione della regione Veneto sulla tabella \textsf{RICETTA} è eseguita con la condizione:
	\begin{equation*}
		\mathrm{NR}\left(\textsf{RICETTA}\right) \div \mathrm{VAL}\left(\text{Regione}, \textsf{RICETTA}\right)
	\end{equation*}
	Andando a sostituire nuovamente i dati:
	\begin{equation*}
		\text{Costo} = 12 + 260 \div 20 \times 200 = 2'612
	\end{equation*}\newpage
	
	\noindent
	Il costo del secondo Join è possibile calcolarlo allo stesso modo. 
	\begin{equation*}
		\text{Costo} = \mathrm{NR}\left(\textsf{COMPOSIZIONE}\right) \div \mathrm{VAL}\left(\text{Ricetta}, \textsf{COMPOSIZIONE}\right) \times 13 \times 40
	\end{equation*}
	\begin{itemize}
		\item $\mathrm{NR}\left(\textsf{COMPOSIZIONE}\right) \div \mathrm{VAL}\left(\text{Ricetta}, \textsf{COMPOSIZIONE}\right)$, rappresenta il calcolo da eseguire poiché nella tabella \textsf{COMPOSIZIONE} la Ricetta è una chiave esterna;
		
		\item $13$, il valore ottenuto dal costo precedente, ovvero rappresenta il numero di regioni del veneto sulla tabella \textsf{RICETTA};
		
		\item $40$, rappresenta $\mathrm{NP}\left(\textsf{INGREDIENTE}\right)$, come da formula.
	\end{itemize}
	Quindi, il costo totale del secondo Join:
	\begin{equation*}
		\text{Costo} = 13'000 \div 260 \times 13 \times 40 = 26'000
	\end{equation*}
	E infine, il costo totale delle operazioni:
	\begin{equation*}
		\text{Costo totale} = 2'612 + 26'000 = 28'612
	\end{equation*}
	Qua di seguito, viene riportato il costo totale nel caso in cui sia disponibile un indice B+-tree sull'attributo Codice della tabella \textsf{INGREDIENTE} che ha profondità 2. Il costo sul primo Join non viene influenzato poiché non riguarda la tabella \textsf{INGREDIENTE} e quindi rimane uguale a $2'612$. Invece, il secondo Join viene influenzato poiché al posto di utilizzare NP(\textsf{INGREDIENTE}), si utilizza la profondità dell'albero, ovvero:
	\begin{equation*}
		\text{Costo secondo Join} = \left(13'000 \div 260\right) \times 13 \times \left(2+1\right) = 650 \times 3 = 1'950
	\end{equation*}
	E quindi il costo totale:
	\begin{equation*}
		\text{Costo totale} = 2'612 + 1'950 = 4'562
	\end{equation*}\newpage
	
	\subsubsection{Esercizio 2}
	
	Si consideri il seguente schema relazionale contenete le ricette di una catena di ristoranti:
	\begin{equation*}
		\begin{array}{rl}
			\textsf{RICETTA} & \left(\text{Nome}, \text{Descrizione}, \text{Regione}\right); \\[0.3em]
			\textsf{IN\_RICETTA} & \left(\text{Ricetta}, \text{Ingrediente}, \text{Quantità}\right); \\[0.3em]
			\textsf{INGREDIENTE} & \left(\text{Nome}, \text{Descrizione}\right).
		\end{array}
	\end{equation*}
	N.B.: la quantità della tabella \textsf{COMPOSIZIONE} è espressa in grammi.\newline
	
	\noindent
	\emph{\textbf{Vincoli di integrità referenziale:}}
	\begin{equation*}
		\begin{array}{lll}
			\textsf{IN\_RICETTA}.\text{Ricetta} 		&\rightarrow& \textsf{RICETTA} \\ [0.3em]
			\textsf{IN\_RICETTA}.\text{Ingrediente} 	&\rightarrow& \textsf{INGREDIENTE}
		\end{array}
	\end{equation*}
	Data la seguente interrogazione SQL:
	\lstinputlisting[language=SQL]{code/ottimizzazione-2.sql}
	Calcolare il costo dell'interrogazione in termini di numero di accessi a memoria secondaria sotto le seguenti ipotesi:
	\begin{itemize}
		\item La selezione sulla tabella \textsf{IN\_RICETTA} eseguita attraverso una scansione della tabella medesima, il risultato della selezione viene salvato in memoria secondaria in 180 pagine e riutilizzato per il join;
		
		\item La selezione delle ricette viene eseguita attraverso una scansione della tabella \textsf{RICETTA}, il risultato della selezione viene mantenuto nel buffer;
		
		\item L'ordine di esecuzione del join è:
		\begin{equation*}
			\textsf{RICETTA} \Join \textsf{IN\_RICETTA}
		\end{equation*}
		E le operazioni di join vengono eseguite con la tecnica \dquotes{Nested Loop Join} con una pagina di buffer disponibile per ogni tabella;
		
		\item $\mathrm{NP}\left(\textsf{INGREDIENTE}\right) = 30$, $\mathrm{NP}\left(\textsf{IN\_RICETTA}\right) = 780$, $\mathrm{NP}\left(\textsf{RICETTA}\right) = 150$;
		
		\item $\mathrm{NR}\left(\textsf{INGREDIENTE}\right) = 270$, $\mathrm{NR}\left(\textsf{IN\_RICETTA}\right) = 19'800$, $\mathrm{NR}\left(\textsf{RICETTA}\right) = 900$;
		
		\item $\mathrm{VAL}\left(\text{Regione}, \textsf{RICETTA}\right) = 18$, $\mathrm{VAL}\left(\text{Ricetta}, \textsf{IN\_RICETTA}\right) = 90$, \newline $\mathrm{VAL}\left(\text{Quantita}, \textsf{IN\_RICETTA}\right) = 50$.
	\end{itemize}
	Come cambia il costo se è disponibile un indice B+-tree sull'attributo Ricetta della tabella \textsf{IN\_RICETTA} che ha profondità 2.\newpage
	
	\noindent
	I dati dell'esercizio sui costi sono i seguenti:
	\begin{table}[!htp]
		\centering
		\begin{tabular}{@{} l | c c c c c @{}}
			\toprule
			Tabella & NP & NR & VAL(Regione) & VAL(Ricetta) & VAL(Quantita) \\
			\midrule
			\textsf{Ricetta} 		& 150 & 900 & 18 & \ding{55} & \ding{55} \\
			\textsf{In\_ricetta}	& 780 & 19800 & \ding{55} & 90 & 50 \\
			\textsf{Ingrediente}	& 30 & 270 & \ding{55} & \ding{55} & \ding{55} \\
			\bottomrule
		\end{tabular}
	\end{table}
	
	\noindent
	Come viene esplicitato dall'esercizio, la selezione sulla tabella \textsf{IN\_RICETTA} viene eseguita:
	\begin{enumerate}
		\item Facendo una scansione della tabella medesima:
		\begin{equation*}
			\mathrm{NP}\left(\textsf{IN\_RICETTA}\right) = 780
		\end{equation*}
		
		\item Il risultato della selezione viene salvato in memoria secondaria in 180 pagine, quindi:
		\begin{equation*}
			\mathrm{NP}\left(\textsf{IN\_RICETTA}\right) = 780 \Longrightarrow 180
		\end{equation*}
		
		\item Viene utilizzato il valore per il Join. Quindi, dato che non vi sono letture di tabelle poiché il buffer contiene già i valori d'interesse, il costo è:
		\begin{equation*}
			\begin{array}{lll}
				\text{Costo} &=& \mathrm{NR}\left(\textsf{RICETTA}\right) \div \mathrm{VAL}\left(\text{Regione}, \textsf{RICETTA}\right) \times \mathrm{NP}\left(\textsf{IN\_RICETTA}\right) \\
				&=& 900 \div 18 \times 180 \\
				&=& 50 \times 180 \\
				&=& 9'000
			\end{array}
		\end{equation*}
		Ricordando che si utilizzano 180 pagine e non 780 (punto 2).
	\end{enumerate}
	Quindi, il costo totale è dato, rispettivamente, dalla somma delle pagine di \textsf{IN\_RICETTA}, dalle pagine scritte nella memoria secondaria che si riferiscono a \textsf{IN\_RICETTA}, dal Join eseguito qua sopra e infine dalla scansione della tabella \textsf{RICETTA}:
	\begin{equation*}
		\text{Costo totale} = 780 + 180 + 9'000 + 150 = 10'110
	\end{equation*}
	Per quanto riguarda il caso in cui fosse presente un indice B+-tree, è necessario modificare il Join. Quindi diventa la formula:
	\begin{equation*}
		\begin{array}{lll}
			\text{Costo Join} &=& \mathrm{NR}\left(\text{\textsf{RICETTA} con selezione}\right) \div \mathrm{VAL}\left(\text{Regione}, \textsf{RICETTA}\right) \times \\
			&& \left(\text{Prof. Indice} + \mathrm{NR}\left(\textsf{IN\_RICETTA}\text{ con selezione}\right) \div \mathrm{VAL}\left(\text{q.ta}, \textsf{IN\_RICETTA}\right)\right) \\
			&=& 900 \div 18 \times \left(19'800 \div 50\right) \div 90 \\
			&=& 220
		\end{array}
	\end{equation*}
	Il totale quindi diventa:
	\begin{equation*}
		\text{Costo totale} = 780 + 180 + 150 + 220 = 1'330
	\end{equation*}\newpage
	
	\subsubsection{Esercizio 3}
	
	Si consideri il seguente schema relazionale:
	\begin{equation*}
		\begin{array}{ll}
			\textsf{COMUNE} 	& \left(\text{CodISTAT, Nome, Abitanti, Superficie, Prov, TipoTerritorio}\right) \\ [0.3em]
			\textsf{ADIACENTE} 	& \left(\text{Comune1, Comune2}\right) \\ [0.3em]
			\textsf{PROVINCIA} 	& \left(\text{Codice, NomeProv, SuperficieProv, Regione}\right) 
		\end{array}
	\end{equation*}
	Con il dato TipoTerritorio: \{montagna, mare, pianura, collina\}.\newline
	
	\noindent
	La tabella \textsf{ADIACENTE} rappresenta la relazione di adiacenza tra comuni: poiché la relazione è simmetrica per rappresentare l'adiacenza tra i comuni A e B si memorizza sia la tupla (A,B) sia la tupla (B,A).\newline
	
	\noindent
	Vincoli d'integrità:
	\begin{equation*}
		\begin{array}{rcl}
			\textsf{ADIACENTE}.\text{Comune1} &\rightarrow& \textsf{COMUNE} \\
			\textsf{ADIACENTE}.\text{Comune2} &\rightarrow& \textsf{COMUNE} \\
			\textsf{COMUNE}.\text{Prov} &\rightarrow& \textsf{PROVINCIA} 
		\end{array}
	\end{equation*}
	Dato lo schema dell'esercizio, si consideri la seguente interrogazione:
	\lstinputlisting[language=SQL]{code/ottimizzazione-3.sql}
	Indicare una stima del costo dell'interrogazione in termini di numero di accessi a memoria secondaria  sapendo che: (i) la selezione dei comuni viene eseguita attraverso una scansione della tabella \textsf{COMUNE} e il risultato viene mantenuto nel buffer:
	\begin{itemize}
		\item L'ordine di esecuzione del join è:
		\begin{equation*}
			\textsf{COMUNE} \Join \textsf{ADIACENTE}
		\end{equation*}
		E viene applicata la tecnica \dquotes{Nested Loop Join} e con indice B+-tree di profondità 3 sull'attributo Comune1 della tabella \textsf{ADIACENTE};
		
		\item I dati sono:
		\begin{equation*}
			\begin{array}{rll}
				\mathrm{NP}\left(\textsf{COMUNE}\right) &=& 15; \\
				\mathrm{NR}\left(\textsf{COMUNE}\right) &=& 1'900; \\
				\mathrm{NP}\left(\textsf{ADIACENTE}\right) &=& 155; \\
				\mathrm{NR}\left(\textsf{ADIACENTE}\right) &=& 38'000; \\
				\mathrm{VAL}\left(\text{TipoTerritorio}, \textsf{COMUNE}\right) &=& 4; \\
				\mathrm{VAL}\left(\text{Comune1}, \textsf{ADIACENTE}\right) &=& 1'900;
			\end{array}
		\end{equation*}
	\end{itemize}\newpage
	
	\noindent
	I dati messi a disposizione sono i seguenti:
	\begin{table}[!htp]
		\centering
		\begin{tabular}{@{} l | c c c c @{}}
			\toprule
			Tabella & NP & NR & VAL(TipoTerritorio) & VAL(Comune1) \\
			\midrule
			\textsf{COMUNE}		& 15		& 1'900		& 4			& \ding{55} \\
			\textsf{ADIACENTE}	& 155		& 38'000	& \ding{55} & 1'900		\\
			\textsf{PROVINCIA}	& \ding{55}	& \ding{55}	& \ding{55} & \ding{55} \\
			\bottomrule
		\end{tabular}
	\end{table}
	
	\noindent
	Viene richiesto il costo per:
	\begin{itemize}
		\item Eseguire una scansione della tabella \textsf{COMUNE} poiché viene eseguita una selezione dei comuni. Questo risultato è semplice poiché corrisponde a:
		\begin{equation*}
			\text{Costo} = \text{scrittura} + \text{lettura} = \mathrm{NP}\left(\textsf{COMUNE}\right) + 0 = 15
		\end{equation*}
		
		\item Eseguire il Join con la tecnica \dquotes{Nested Loop Join} senza albero B+-tree. Quindi, viene calcolato nel seguente modo il costo:
		\begin{equation*}
			\text{Costo} = 0 + \mathrm{NR}\left(\textsf{COMUNE}\right) \div \mathrm{VAL}\left(\text{TipoTerritorio}, \textsf{COMUNE}\right) \times \mathrm{NP}\left(\textsf{ADIACENTE}\right)
		\end{equation*}
		Il valore zero indica che nel buffer non viene eseguita nessuna scrittura poiché è già stata fatta in precedenza (punto precedente). Quindi, andando a sostituire i valori:
		\begin{equation*}
			\text{Costo} = 0 + 1'900 \div 4 \times 155 = 475 \times 155 = 73'625
		\end{equation*}
		
		\item Eseguire il Join con la tecnica \dquotes{Nested Loop Join} con albero B+-tree di profondità 3 sull'attributo Comune1. Quindi, viene calcolato nel seguente modo il costo:
		\begin{equation*}
			\begin{array}{lll}
				\text{Costo} &=& 0 + \mathrm{NR}\left(\textsf{COMUNE}\right) \div \mathrm{VAL}\left(\text{TipoTerritorio}, \textsf{COMUNE}\right) \times \\
				&& \left(\text{Profondità Indice } + \mathrm{NR}\left(\textsf{ADIACENTE}\right) \div \mathrm{VAL}\left(\text{Comune1}, \textsf{ADIACENTE}\right)\right) \\
				&=& 0 + 1'900 \div 4 \times \left(3 + 38'000 \div 1'900\right) \\
				&=& 475 \times 23 \\
				&=& 10'925
			\end{array}
		\end{equation*}
	\end{itemize}
	
	\newpage
	
	\subsection{XML}
	
	\subsubsection{Esercizio 1}
	
	Dato il seguente frammento XML:
	\lstinputlisting[language=XML]{code/xml-1.xml}
	\begin{itemize}
		\item Generare il corrisponde XSD, supponendo che ogni conto abbia uno o più clienti intestatari.
		
		\item Modificare la struttura dell'XML per ridurre la ridondanza e rappresentare una sola volta gli elementi ripetuti.
	\end{itemize}\newpage
	
	\noindent
	\textcolor{Green4}{\underline{\textbf{\emph{Soluzione pt.1}}}}\newline
	
	\noindent
	Qua di seguito il codice soluzione dell'esercizio supponendo che ogni conto abbia uno o più intestatari:
	\lstinputlisting[language=XML]{code/xml-3.xsd}\newpage
	
	\noindent
	\textcolor{Green4}{\underline{\textbf{\emph{Soluzione pt.2}}}}\newline
	
	\noindent
	Qua di seguito il codice soluzione dell'esercizio supponendo che ogni conto abbia uno o più intestatari:
	\lstinputlisting[language=XML]{code/xml-4.xsd}
	A seguito delle modifiche di questo schema, l'istanza del documento diventa:
	\lstinputlisting[language=XML]{code/xml-5.xml}\newpage
	
	\subsubsection{Esercizio 2}
	Dato il seguente documento XML, produrre una specifica XML schema alla quale tale documento sia conforme:
	\lstinputlisting[language=XML]{code/xml-2.xml}
	I requisiti sono i seguenti. Il tipo può assumere solo uno dei seguenti valori: \{privato, attivita commerciale, attivita professionale, servizio pubblico\}. L'attributo \textsf{id} è obbligatorio. Supponiamo per semplicità che il civico sia sempre un intero positivo. Si desuma il tipo per il numero di telefono dalle istanze nel documento XML.\newpage
	
	\noindent
	\textcolor{Green4}{\underline{\textbf{\emph{Soluzione}}}}
	\lstinputlisting[language=XML]{code/xml-6.xsd}\newpage

	\section{Domande di teoria - Terzo parziale}
	
	Domande di teoria tratte dalla terza prova intermedia dell'esame 06/2015.
	\begin{enumerate}
		\item \textbf{(\emph{3 punti})} \textcolor{Green4}{\textbf{\emph{Illustrare l'architettura di un DBMS descrivendo in particolare il modulo di gestione dei buffer; si indichi inoltre, per ogni modulo dell'architettura, quali sono le proprietà delle transazioni che contribuisce a garantire.}}}\label{dom: gestione del buffer}
		
		L'architettura di un DBMS è strutturata su 3 livelli differenti:
		\begin{itemize}
			\item Le viste, che appartengono allo schema esterno;
			\item Il modello relazionale, che appartiene allo schema logico;
			\item Le strutture fisiche, che appartengono allo schema interno.
		\end{itemize}
		Lo schema logico comunica con lo schema esterno attraverso l'indipendenza logica e con lo schema interno attraverso l'indipendenza fisica.
		\begin{figure}[!htp]
			\centering
			\includegraphics[width=\textwidth]{img/ex/arch-dbms-1.pdf}
		\end{figure}
		
		\noindent
		All'interno dello schema logico ci sono molteplici moduli, i quali garantiscono anche alcune proprietà delle transazioni:
		\begin{itemize}
			\item Gestore delle \textbf{interrogazioni};
			
			\item Gestore dell'\textbf{esecuzione concorrente}, il quale garantisce le proprietà di \underline{atomicità} e \underline{isolamento};
			
			\item Gestore dei \textbf{metodi d'accesso}, il quale garantisce la proprietà di \underline{consistenza};
			
			\item Gestore dell'\textbf{affidabilità}, il quale garantisce \underline{atomicità} e \underline{persistenza};
			
			\item Gestore del \textbf{buffer};
			
			\item Gestore della \textbf{memoria secondaria}.
		\end{itemize}
		Ricapitolando:
		\begin{table}[!htp]
			\centering
			\begin{tabular}{@{} l p{18em} @{}}
				\toprule
				Proprietà & Garantita da \\
				\midrule
				\textbf{Atomicità} 	& Gestore dell'\textbf{esecuzione concorrente} \newline Gestore dell'\textbf{affidabilità} \\ [0.5em]
				\textbf{Consistenza}	& Gestore dei \textbf{metodi d'accesso} \\ [0.5em]
				\textbf{Isolamento} 	& Gestore dell'\textbf{esecuzione concorrente} \\ [0.5em]
				\textbf{Persistenza} & Gestore dell'\textbf{affidabilità} \\
				\bottomrule
			\end{tabular}
		\end{table}
		\begin{figure}[!htp]
			\centering
			\includegraphics[width=\textwidth]{img/ex/arch-dbms-2.pdf}
		\end{figure}
		
		\noindent
		Il \textbf{modulo di gestione del buffer} è fondamentale nella \textbf{comunicazione tra memoria centrale e secondaria}. Il suo \underline{obbiettivo} è quello di \textbf{evitare accessi multipli alla memoria di massa} così da velocizzare le operazioni.\newline
		Il buffer è organizzato in pagine di dimensione pari a un numero intero di blocchi. Si occupa di caricare (lettura)/scaricare (scrittura) pagine dalle memoria centrale alla memoria di massa:
		\begin{table}[!htp]
			\centering
			\begin{tabular}{@{} l p{21em} @{}}
				\toprule
				Operazione & Descrizione \\
				\midrule
				Lettura		& Lettura dal buffer se il dato è presente, altrimenti lettura fisica dalla memoria di massa. \\ [0.5em]
				Scrittura	& Scrittura in memoria di massa se e solo se è garantita la proprietà di affidabilità del sistema, quindi se vi è la certezza che l'operazione vada a buon fine. \\
				\bottomrule
			\end{tabular}
		\end{table}
		
		\noindent
		Il modulo della gestione del buffer segue un'importante proprietà: il \textbf{principio di località dei dati}. Ovvero, \textbf{i dati referenziati di recente hanno maggior probabilità di essere referenziati nuovamente nel futuro}.\newline
		Inoltre, esiste una \textbf{legge empirica} che afferma che il 20\% dei dati è tipicamente acceduto dall'80\% delle applicazioni.
		
		Infine, il gestore del buffer memorizza le seguenti \textbf{informazioni per ogni pagina}:
		\begin{itemize}
			\item File fisico
			\item Numero di blocco
			\item (variabile di stato) Contatore, indica quanti programmi utilizzano la pagina
			\item (variabile di stato) Bit di stato, indica se la pagina è stata modificata
		\end{itemize}\newpage
		
		\begin{table}[!htp]
			\centering
			\begin{tabular}{@{} l p{23em} @{}}
				\toprule
				Concetto & Descrizione \\
				\midrule
				Architettura DBMS & L'architettura DBMS è strutturata su 3 livelli: schema esterno, logico e interno. Lo schema logico, il più importante, ha al suo interno una serie di moduli (gestori) importanti: delle interrogazioni, dell'esecuzione concorrente, dei metodi d'accesso, dell'affidabilità, del buffer, della memoria secondaria. \\ [0.5em]
				Proprietà garantite & L'atomicità è garantita dal gestore dell'esecuzione concorrente e dell'affidabilità.\newline
				La consistenza è garantita dal gestore dei metodi d'accesso.\newline
				L'isolamento è garantita dal gestore dell'esecuzione concorrente.\newline
				La persistenza è garantita dal gestore dell'affidabilità. \\ [0.5em]
				Descrizione & Modulo fondamentale nella comunicazione tra memoria centrale e di massa, ed è organizzato in pagine di dimensione pari ad un numero intero di blocchi. \\ [0.5em]
				Obbiettivo 	& Evitare accessi multipli alla memoria di massa così da velocizzare le operazioni. \\ [0.5em]
				Cosa svolge & Esegue il caricamento (lettura) e scaricamento (scrittura) delle pagine dalla memoria centrale alla memoria di massa. \\ [.5em]
				Principio importante & I dati referenziati di recente hanno maggior probabilità di essere referenziati nuovamente nel futuro. \\ [.5em]
				Legge importante & Il 20\% dei dati è acceduto dall'80\% delle applicazioni. \\
				\bottomrule
			\end{tabular}
			\caption{Riepilogo dei concetti della domanda sulla gestione del buffer.}
		\end{table}\newpage
		
		\item \textbf{(\emph{2 punti})} \textcolor{Green4}{\textbf{\emph{Si presenti in dettaglio la definizione di Conflict-Serializzabilità (CSR).}}}\label{dom: CSR - Conflict-Serializzabilità}
		
		Per dare la definizione di conflict-serializzabile, è necessario prima dare altre due definizioni. Due azioni eseguite da transazioni diverse, si dicono in \textbf{\underline{conflitto} se operano sullo stesso oggetto e almeno una di esse è in scrittura}. Quindi, le possibili combinazioni sono: lettura-scrittura, scrittura-lettura e scrittura-scrittura.\newline
		Uno schedule è definito \textbf{\underline{conflict-equivalente}} ad un altro schedule se \textbf{entrambi presentano le stesse operazioni e ogni coppia di operazioni in conflitto è nello stesso ordine nei due schedule}.\newline
		Adesso è possibile dare la definizione di conflict-serializzabile. Uno schedule è \textbf{\underline{conflict-serializzabile} se esiste uno schedule seriale a esso conflict-equivalente}. L'insieme degli schedule conflict-serializzabili si chiama CSR.
		
		\begin{table}[!htp]
			\centering
			\begin{tabular}{@{} l p{16em} @{}}
				\toprule
				Concetto & Descrizione \\
				\midrule
				Definizione di conflitto				& Due azioni di due transazioni diverse sono in conflitto se operano sullo stesso oggetto e almeno una di esse è una scrittura. \\ [.5em]
				Definizione di conflict-equivalente		& Due schedule sono conflict-equivalenti se entrambi hanno le stesse operazioni e ogni coppia di operazioni in conflitto è nello stesso ordine nei due schedule. \\ [.5em]
				Definizione di conflict-serializzabile	& Se esiste uno schedule seriale a esso conflict-equivalente. \\
				\bottomrule
			\end{tabular}
			\caption{Riepilogo dei concetti della domanda su CSR.}
		\end{table}\newpage
		
		\item \textbf{(\emph{2 punti})} \textcolor{Green4}{\textbf{\emph{Lo studente illustri la struttura di accesso ai dati denominata indice primario denso: caratteristiche della struttura, ricerca, inserimento e cancellazione di entry dall'indice.}}}\label{dom: indice primario denso}
		
		Un indice primario utilizza una chiave di ricerca che coincide con la chiave di ordinamento del file sequenziale. Esistono due varianti dell'indice primario, una di queste è l'\textbf{indice denso}: \textbf{per ogni occorrenza della chiave presente nel file esiste un corrispondente record nell'indice}.
		\begin{figure}[!htp]
			\centering
			\includegraphics[width=\textwidth]{img/ex/indice-denso-1.pdf}
		\end{figure}
		
		\noindent
		Operazione di \textbf{ricerca}: la \textbf{ricerca avviene tramite una scansione sequenziale dell'indice per trovare il record}. Una volta trovato, viene \textbf{effettuato l'accesso al file attraverso il puntatore}. Il costo è pari a:
		\begin{equation*}
			\text{Costo} = 1 \text{ accesso indice } + 1 \text{ accesso blocco dati}
		\end{equation*}
		\begin{figure}[!htp]
			\centering
			\includegraphics[width=\textwidth]{img/ex/indice-denso-2.pdf}
		\end{figure}
		
		\noindent
		Operazione di \textbf{inserimento}: l'inserimento nell'indice avviene solo se la tupla inserita nel file ha un valore di chiave K (chiave di ricerca) che non è già presente.
		
		Operazione di \textbf{cancellazione}: la cancellazione nell'indice avviene solo se la tupla cancellata nel file è l'ultima tupla con valore di chiave K (chiave di ricerca).\newpage
		
		\begin{table}[!htp]
			\centering
			\begin{tabular}{@{} l p{25em} @{}}
				\toprule
				Concetto & Definizione \\
				\midrule
				Definizione 		& Un indice primario utilizza una chiave di ricerca che coincide con la chiave di ordinamento del file sequenziale. Una variante del indice primario è l'indice primario denso: per ogni occorrenza della chiave presente nel file, esiste un corrispondente record nell'indice. \\ [.5em]
				Op. ricerca 		& Eseguita una scansione sequenziale dell'indice per trovare il record ricercato. Se viene trovato, viene effettuato l'accesso al file attraverso il puntatore. \\ [.5em]
				Op. inserimento		& L'inserimento avviene solo se la tupla inserita nel file ha un valore di chiave K (chiave di ricerca) che non è già presente. \\ [.5em]
				Op. cancellazione	& La cancellazione avviene solo se la tupla cancellata nel file è l'ultima tupla con valore di chiave K (chiave di ricerca). \\
				\bottomrule
			\end{tabular}
			\caption{Riepilogo dei concetti della domanda sull'indice denso.}
		\end{table}
	\end{enumerate}\newpage
	
	Domande di teoria tratte dalla terza prova intermedia dell'esame 07/06/2016.
	\begin{enumerate}
		\item \textbf{(\emph{3 punti})} \textcolor{Green4}{\textbf{\emph{Illustrare l'architettura di un DBMS descrivendo in particolare il modulo di gestione dei guasti (o gestore dell'affidabilità); si indichi inoltre, per ogni modulo dell'architettura, quali sono le proprietà delle transazioni che contribuisce a garantire.}}}\label{dom: gestione dei guasti}
		
		L'architettura di un DBMS è strutturata su 3 livelli differenti:
		\begin{itemize}
			\item Le viste, che appartengono allo schema esterno;
			\item Il modello relazionale, che appartiene allo schema logico;
			\item Le strutture fisiche, che appartengono allo schema interno.
		\end{itemize}
		Lo schema logico comunica con lo schema esterno attraverso l'indipendenza logica e con lo schema interno attraverso l'indipendenza fisica.
		\begin{figure}[!htp]
			\centering
			\includegraphics[width=\textwidth]{img/ex/arch-dbms-1.pdf}
		\end{figure}
		
		\noindent
		All'interno dello schema logico ci sono molteplici moduli, i quali garantiscono anche alcune proprietà delle transazioni:
		\begin{itemize}
			\item Gestore delle \textbf{interrogazioni};
			
			\item Gestore dell'\textbf{esecuzione concorrente}, il quale garantisce le proprietà di \underline{atomicità} e \underline{isolamento};
			
			\item Gestore dei \textbf{metodi d'accesso}, il quale garantisce la proprietà di \underline{consistenza};
			
			\item Gestore dell'\textbf{affidabilità}, il quale garantisce \underline{atomicità} e \underline{persistenza};
			
			\item Gestore del \textbf{buffer};
			
			\item Gestore della \textbf{memoria secondaria}.
		\end{itemize}
		Ricapitolando:
		\begin{table}[!htp]
			\centering
			\begin{tabular}{@{} l p{18em} @{}}
				\toprule
				Proprietà & Garantita da \\
				\midrule
				\textbf{Atomicità} 	& Gestore dell'\textbf{esecuzione concorrente} \newline Gestore dell'\textbf{affidabilità} \\ [0.5em]
				\textbf{Consistenza}	& Gestore dei \textbf{metodi d'accesso} \\ [0.5em]
				\textbf{Isolamento} 	& Gestore dell'\textbf{esecuzione concorrente} \\ [0.5em]
				\textbf{Persistenza} & Gestore dell'\textbf{affidabilità} \\
				\bottomrule
			\end{tabular}
		\end{table}\newpage
		\begin{figure}[!htp]
			\centering
			\includegraphics[width=\textwidth]{img/ex/arch-dbms-2.pdf}
		\end{figure}
		
		\noindent
		Il \textbf{modulo di gestione dei guasti}, o \textbf{gestore dell'affidabilità}, garantisce due \textbf{proprietà fondamentali}:
		\begin{itemize}
			\item \textbf{Atomicità}, garantire che le \textbf{transazioni non vengano lasciate incomplete};
			
			\item \textbf{Persistenza}, garantire che gli \textbf{effetti di ciascuna transazione conclusa con un \emph{commit} siano mantenuti in modo permanente}.
		\end{itemize}
		Le \textbf{proprietà} vengono \textbf{garantite grazie ai log}, ovvero un \textbf{archivio persistente su cui vengono registrate le varie azioni svolte dal DBMS}.
		
		Il gestore dell'affidabilità utilizza delle primitive (\textsf{force, fix, unfix, setDirty}) per comunicare con il gestore del buffer. Questo perché esso invia \textbf{richieste di accessi a pagine in lettura/scrittura al \emph{buffer manager}}, e genera \textbf{ulteriori richieste di lettura/scrittura necessarie a garantire la robustezza e resistenza ai guasti}.
		
		Infine, date le proprietà che deve garantire, questo \textbf{componente necessita di una memoria stabile}, \textbf{ovvero di una memoria resistente ai guasti}.
		
		\begin{table}[!htp]
			\centering
			\begin{tabular}{@{} l p{23em} @{}}
				\toprule
				Concetto & Definizione \\
				\midrule
				Architettura DBMS & L'architettura DBMS è strutturata su 3 livelli: schema esterno, logico e interno. Lo schema logico, il più importante, ha al suo interno una serie di moduli (gestori) importanti: delle interrogazioni, dell'esecuzione concorrente, dei metodi d'accesso, dell'affidabilità, del buffer, della memoria secondaria. \\ [0.5em]
				%
				Proprietà garantite & L'atomicità è garantita dal gestore dell'esecuzione concorrente e dell'affidabilità.\newline
				La consistenza è garantita dal gestore dei metodi d'accesso.\newline
				L'isolamento è garantita dal gestore dell'esecuzione concorrente.\newline
				La persistenza è garantita dal gestore dell'affidabilità. \\ [0.5em]
				%
				Approfondimento proprietà & L'atomicità garantisce che le transazioni non vengano lasciate incomplete. Invece, la persistenza garantisce che gli effetti di ciascuna transazione conclusa con un commit siano mantenuti in modo permanente. Queste proprietà sono garantite dal modulo di gestione dei guasti grazie al log, ovvero un archivio persistente su cui vengono registrate le varie azioni svolte dal DBMS. \\ [.5em]
				%
				Compito eseguito & Il modulo di gestione dei guasti invia richieste di accessi a pagine in lettura/scrittura al buffer. Inoltre, genera ulteriori richieste di lettura/scrittura necessarie a garantire la robustezza e resistenza ai guasti. \\
				\bottomrule
			\end{tabular}
			\caption{Riepilogo dei concetti della domanda sull'indice denso.}
		\end{table}\newpage
		
		\item (\emph{3 punti}) Si presenti in dettaglio la \underline{politica di concessione dei lock} applicata dal gestore dell'esecuzione concorrente secondo la tecnica detta \dquotes{Locking a due fasi}.
		
		\item (\emph{2 punti}) Lo studente illustri la struttura di accesso ai dati denominata indice primario sparso, si descrivano in particolare i seguenti punti: (i) le caratteristiche della struttura di accesso, (ii) l'algoritmo di ricerca di una tupla con chiave K usando l'indice.
		
		\item (\emph{2 punti}) Lo studente illustri le differenze tra i costrutti \textbf{elemento} e \textbf{attributo} del linguaggio XML, mostrando un esempio di documento XML dove vengono utilizzati entrambi.
	\end{enumerate}\newpage
	
	Domande di teoria tratte dalla terza prova intermedia dell'esame 21/04/2022.
	\begin{enumerate}
		\item (\emph{3 punti}) Illustrare l'architettura di un DBMS descrivendo in particolare il modulo di gestione dei buffer; si indichi inoltre quali moduli garantiscono le proprietà di persistenza e consistenza delle transazioni.
		
		\item (\emph{2 punti}) Si presenti in dettaglio la \underline{definizione di conflict-equivalenza} tra due schedule.
		
		\item (\emph{2 punti}) Lo studente illustri la struttura di accesso ai dati denominata struttura ad accesso calcolato (hashing), si descrivano in particolare i seguenti punti: (i) le caratteristiche della struttura di accesso, (ii) l'algoritmo di ricerca di una tupla con chiave K usando l'indice.
		
		\item (\emph{2 punti}) Lo studente illustri l'algoritmo di \underline{ripresa a caldo}.
	\end{enumerate}\newpage
	
	Domande di teoria tratte dalla terza prova intermedia dell'esame 22/04/2022.
	\begin{enumerate}
		\item (\emph{3 punti}) Illustrare le proprietà delle transazioni; si indichi inoltre quali moduli di un DBMS garantiscono ciascuna di tali proprietà.
		
		\item (\emph{2 punti}) Si presenti in dettaglio la \underline{definizione di view-equivalenza} tra due schedule.
		
		\item (\emph{2 punti}) Lo studente illustri la struttura di accesso ai dati denominata B+-tree, si descrivano in particolare i seguenti punti: (i) le caratteristiche della struttura di accesso, (ii) l'algoritmo di ricerca di una tupla con chiave K usando l'indice.
		
		\item (\emph{2 punti}) Illustrare il meccanismo di 2PL stretto.
	\end{enumerate}\newpage
	
	Domande di teoria tratte dalla terza prova intermedia dell'esame 10/06/2022.
	\begin{enumerate}
		\item (\emph{2 punti}) Illustrare le proprietà delle transazioni ed indicare da quali moduli del DBMS vengono garantite.
		
		\item (\emph{3 punti}) Si presenti la definizione di View-serializzabilità e la relazione tra l'insieme degli schedule VSR e l'insieme degli schedule CSR. Presentare la dimostrazione formale di tale relazione.
		
		\item (\emph{2 punti}) Lo studente illustri la struttura di accesso ai dati denominata indice secondario, si descrivano in particolare i seguenti punti: (i) le caratteristiche della struttura di accesso, (ii) l'algoritmo di ricerca di una tupla con chiave K usando l'indice.
		
		\item (\emph{2 punti}) Lo studente illustri l'algoritmo di \underline{ripresa a caldo}.
	\end{enumerate}\newpage
	
	% \section{Esami terzo parziale}
	
	% \subsection{Terzo parziale - 06/2015}
	
	% \subsection{Terzo parziale - 07/06/2016}
	
	% \subsection{Terzo parziale - 21/04/2022}
	
	% \subsection{Terzo parziale - 22/04/2022}
	
	% \subsection{Terzo parziale - 10/06/2022}\newpage
	
	\section{Indice delle domande}
	
	\subsection{Terzo parziale}
	
	\subsubsection{Domande teoriche}
	
	\begin{itemize}
		\item Illustrare l'architettura di un DBMS descrivendo in particolare il modulo di \textbf{gestione dei buffer}; si indichi inoltre, per ogni modulo dell'architettura, quali sono le proprietà delle transazioni che contribuisce a garantire.
		
		\emph{Risposta:} pagina \pageref{dom: gestione del buffer}
		
		\item Illustrare l'architettura di un DBMS descrivendo in particolare il modulo di \textbf{gestione dei guasti} (o gestore dell'affidabilità); si indichi inoltre, per ogni modulo dell'architettura, quali sono le proprietà delle transazioni che contribuisce a garantire.
		
		\emph{Risposta:} pagina \pageref{dom: gestione dei guasti}
		
		\item Si presenti in dettaglio la definizione di \textbf{Conflict-Serializzabilità (CSR)}.
		
		\emph{Risposta:} pagina \pageref{dom: CSR - Conflict-Serializzabilità}
		
		\item Lo studente illustri la struttura di accesso ai dati denominata \textbf{indice primario denso}: caratteristiche della struttura, ricerca, inserimento e cancellazione di entry dall'indice.
		
		\emph{Risposta:} pagina \pageref{dom: indice primario denso}
		
		\item 
		
		\item 
	\end{itemize}
\end{document}